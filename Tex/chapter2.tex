\documentclass{extarticle}
\sloppy
%%%%%%%%%%%%%%%%%%%%%%%%%%%%%%%%%%%%%%%%%%%%%%%%%%%%%%%%%%%%%%%%%%%%%%
% PACKAGES            																						  %
%%%%%%%%%%%%%%%%%%%%%%%%%%%%%%%%%%%%%%%%%%%%%%%%%%%%%%%%%%%%%%%%%%%%%
\usepackage[10pt]{extsizes}
\usepackage{amsfonts}
% \usepackage{amsthm}
\usepackage{amssymb}
\usepackage[shortlabels]{enumitem}
\usepackage{microtype} 
\usepackage{amsmath}
\usepackage{mathtools}
\usepackage{commath}
\usepackage{amsthm}
\usepackage{bbm}
\usepackage[colorlinks=true, allcolors=blue]{hyperref}

%%%%%%%%%%%%%%%%%%%%%%%%%%%%%%%%%%%%%%%%%%%%%%%%%%%%%%%%%%%%%%%%%%%%%%
% PROBLEM ENVIRONMENT         																			           %
%%%%%%%%%%%%%%%%%%%%%%%%%%%%%%%%%%%%%%%%%%%%%%%%%%%%%%%%%%%%%%%%%%%%%
\usepackage{tcolorbox}
\tcbuselibrary{theorems, breakable, skins}
\newtcbtheorem{prob}% environment name
              {Problem}% Title text
  {enhanced, % tcolorbox styles
  attach boxed title to top left={xshift = 4mm, yshift=-2mm},
  colback=blue!5, colframe=black, colbacktitle=blue!3, coltitle=black,
  boxed title style={size=small,colframe=gray},
  fonttitle=\bfseries,
  separator sign none
  }%
  {} 
\newenvironment{problem}[1]{\begin{prob*}{#1}{}}{\end{prob*}}

\newtcbtheorem{exer}% environment name
              {Exercise}% Title text
  {enhanced, % tcolorbox styles
  attach boxed title to top left={xshift = 4mm, yshift=-2mm},
  colback=blue!5, colframe=black, colbacktitle=blue!3, coltitle=black,
  boxed title style={size=small,colframe=gray},
  fonttitle=\bfseries,
  separator sign none
  }%
  {} 
\newenvironment{exercise}[1]{\begin{exer*}{#1}{}}{\end{exer*}}

%%%%%%%%%%%%%%%%%%%%%%%%%%%%%%%%%%%%%%%%%%%%%%%%%%%%%%%%%%%%%%%%%%%%%%
% THEOREMS/LEMMAS/ETC.         																			  %
%%%%%%%%%%%%%%%%%%%%%%%%%%%%%%%%%%%%%%%%%%%%%%%%%%%%%%%%%%%%%%%%%%%%%%
\newtheorem{thm}{Theorem}
\newtheorem*{thm-non}{Theorem}
\newtheorem{lemma}[thm]{Lemma}
\newtheorem{definition}[thm]{Definition}
\newtheorem{remark}[thm]{Remark}
\newtheorem{corollary}[thm]{Corollary}
\newtheorem{proposition}[thm]{Proposition}
% % \newtheorem*{example*}[Theorem]{Example}
% \newtheorem*{example}[thm]{Example}

\theoremstyle{remark}
\newtheorem*{example}{Example}
% \newtheorem*{example*}[thm-non]{Example}

%%%%%%%%%%%%%%%%%%%%%%%%%%%%%%%%%%%%%%%%%%%%%%%%%%%%%%%%%%%%%%%%%%%%%%
% SECTION NUMBERING																				           %
%%%%%%%%%%%%%%%%%%%%%%%%%%%%%%%%%%%%%%%%%%%%%%%%%%%%%%%%%%%%%%%%%%%%%%
\renewcommand\thesection{\Alph{section}:}

\makeatletter
\newcommand{\cbigoplus}{\DOTSB\cbigoplus@\slimits@}
\newcommand{\cbigoplus@}{\mathop{\widehat{\bigoplus}}}
\makeatother

%%%%%%%%%%%%%%%%%%%%%%%%%%%%%%%%%%%%%%%%%%%%%%%%%%%%%%%%%%%%%%%%%%%%%%
% Additional math commands																		           %
%%%%%%%%%%%%%%%%%%%%%%%%%%%%%%%%%%%%%%%%%%%%%%%%%%%%%%%%%%%%%%%%%%%%%%
% \newcommand{\E}{\mathbb{E}}
% \newcommand{\Prb}{\mathbb{P}}
% \newcommand{\R}{\mathbb{R}}
% \newcommand{\C}{\mathbb{C}}
% \newcommand{\I}{\mathbb{I}}
% \newcommand{\F}{\mathbb{F}}
% \newcommand{\vecc}{\mathbf{c}}
% \newcommand{\Z}{\mathbb{Z}}
% \newcommand{\nul}{\text{null }}
% \newcommand{\range}{\text{range }}
% \newcommand{\Span}{\text{span}}
% \newcommand{\tr}{\text{tr}}
% \newcommand{\diag}{\text{diag}}
% \newcommand{\limn}{\lim_{n \to \infty}}
% \newcommand{\zo}{\overline{z}}
% \newcommand{\wo}{\overline{w}}
% \newcommand{\ao}{\overline{a}}
% \newcommand{\vo}{\overline{v}}
% \newcommand{\glo}{\overline{\lambda}}
% \newcommand{\Var}{\text{Var}}
% \newcommand{\1}{\mathbbm{1}}
% \newcommand{\Pm}{\mathbbm{P}}

% Mathbold abbrevation
\newcommand{\ab}{\mathbf{a}}
\newcommand{\bb}{\mathbf{b}}
\newcommand{\cb}{\mathbf{c}}
\newcommand{\db}{\mathbf{d}}
\newcommand{\eb}{\mathbf{e}}
\newcommand{\fb}{\mathbf{f}}
\newcommand{\gb}{\mathbf{g}}
\newcommand{\hb}{\mathbf{h}}
\newcommand{\ib}{\mathbf{i}}
\newcommand{\jb}{\mathbf{j}}
\newcommand{\kb}{\mathbf{k}}
\newcommand{\lb}{\mathbf{l}}
\newcommand{\mb}{\mathbf{m}}
\newcommand{\nb}{\mathbf{n}}
\newcommand{\ob}{\mathbf{o}}
\newcommand{\pb}{\mathbf{p}}
\newcommand{\qb}{\mathbf{q}}
\newcommand{\rb}{\mathbf{r}}
% \newcommand{\sb}{\mathbf{s}} # already defined
\newcommand{\tb}{\mathbf{t}}
\newcommand{\ub}{\mathbf{u}}
\newcommand{\vb}{\mathbf{v}}
\newcommand{\wb}{\mathbf{w}}
\newcommand{\xb}{\mathbf{x}}
\newcommand{\yb}{\mathbf{y}}
\newcommand{\zb}{\mathbf{z}}
\newcommand{\Ab}{\mathbf{A}}
\newcommand{\Bb}{\mathbf{B}}
\newcommand{\Cb}{\mathbf{C}}
\newcommand{\Db}{\mathbf{D}}
\newcommand{\Eb}{\mathbf{E}}
\newcommand{\Fb}{\mathbf{F}}
\newcommand{\Gb}{\mathbf{G}}
\newcommand{\Hb}{\mathbf{H}}
\newcommand{\Ib}{\mathbf{I}}
\newcommand{\Jb}{\mathbf{J}}
\newcommand{\Kb}{\mathbf{K}}
\newcommand{\Lb}{\mathbf{L}}
\newcommand{\Mb}{\mathbf{M}}
\newcommand{\Nb}{\mathbf{N}}
\newcommand{\Ob}{\mathbf{O}}
\newcommand{\Pb}{\mathbf{P}}
\newcommand{\Qb}{\mathbf{Q}}
\newcommand{\Rb}{\mathbf{R}}
\newcommand{\Sb}{\mathbf{S}}
\newcommand{\Tb}{\mathbf{T}}
\newcommand{\Ub}{\mathbf{U}}
\newcommand{\Vb}{\mathbf{V}}
\newcommand{\Wb}{\mathbf{W}}
\newcommand{\Xb}{\mathbf{X}}
\newcommand{\Yb}{\mathbf{Y}}
\newcommand{\Zb}{\mathbf{Z}}


% MathCal abbrevation
\newcommand{\acal}{\mathcal{a}}
\newcommand{\bcal}{\mathcal{b}}
\newcommand{\ccal}{\mathcal{c}}
\newcommand{\dcal}{\mathcal{d}}
\newcommand{\ecal}{\mathcal{e}}
\newcommand{\fcal}{\mathcal{f}}
\newcommand{\gcal}{\mathcal{g}}
\newcommand{\hcal}{\mathcal{h}}
\newcommand{\ical}{\mathcal{i}}
\newcommand{\jcal}{\mathcal{j}}
\newcommand{\kcal}{\mathcal{k}}
\newcommand{\lcal}{\mathcal{l}}
\newcommand{\mcal}{\mathcal{m}}
\newcommand{\ncal}{\mathcal{n}}
\newcommand{\ocal}{\mathcal{o}}
\newcommand{\pcal}{\mathcal{p}}
\newcommand{\qcal}{\mathcal{q}}
\newcommand{\rcal}{\mathcal{r}}
\newcommand{\scal}{\mathcal{s}}
\newcommand{\tcal}{\mathcal{t}}
\newcommand{\ucal}{\mathcal{u}}
\newcommand{\vcal}{\mathcal{v}}
\newcommand{\wcal}{\mathcal{w}}
\newcommand{\xcal}{\mathcal{x}}
\newcommand{\ycal}{\mathcal{y}}
\newcommand{\zcal}{\mathcal{z}}
\newcommand{\Acal}{\mathcal{A}}
\newcommand{\Bcal}{\mathcal{B}}
\newcommand{\Ccal}{\mathcal{C}}
\newcommand{\Dcal}{\mathcal{D}}
\newcommand{\Ecal}{\mathcal{E}}
\newcommand{\Fcal}{\mathcal{F}}
\newcommand{\Gcal}{\mathcal{G}}
\newcommand{\Hcal}{\mathcal{H}}
\newcommand{\Ical}{\mathcal{I}}
\newcommand{\Jcal}{\mathcal{J}}
\newcommand{\Kcal}{\mathcal{K}}
\newcommand{\Lcal}{\mathcal{L}}
\newcommand{\Mcal}{\mathcal{M}}
\newcommand{\Ncal}{\mathcal{N}}
\newcommand{\Ocal}{\mathcal{O}}
\newcommand{\Pcal}{\mathcal{P}}
\newcommand{\Qcal}{\mathcal{Q}}
\newcommand{\Rcal}{\mathcal{R}}
\newcommand{\Scal}{\mathcal{S}}
\newcommand{\Tcal}{\mathcal{T}}
\newcommand{\Ucal}{\mathcal{U}}
\newcommand{\Vcal}{\mathcal{V}}
\newcommand{\Wcal}{\mathcal{W}}
\newcommand{\Xcal}{\mathcal{X}}
\newcommand{\Ycal}{\mathcal{Y}}
\newcommand{\Zcal}{\mathcal{Z}}


% MathCal abbrevation
\newcommand{\ac}{\mathcal{a}}
\newcommand{\bc}{\mathcal{b}}
% \newcommand{\cc}{\mathcal{c}}
\newcommand{\dc}{\mathcal{d}}
\newcommand{\ec}{\mathcal{e}}
\newcommand{\fc}{\mathcal{f}}
\newcommand{\gc}{\mathcal{g}}
\newcommand{\hc}{\mathcal{h}}
\newcommand{\ic}{\mathcal{i}}
\newcommand{\jc}{\mathcal{j}}
\newcommand{\kc}{\mathcal{k}}
\newcommand{\lc}{\mathcal{l}}
\newcommand{\mc}{\mathcal{m}}
\newcommand{\nc}{\mathcal{n}}
\newcommand{\oc}{\mathcal{o}}
\newcommand{\pc}{\mathcal{p}}
\newcommand{\qc}{\mathcal{q}}
\newcommand{\rc}{\mathcal{r}}
% \newcommand{\sc}{\mathcal{s}}
\newcommand{\tc}{\mathcal{t}}
\newcommand{\uc}{\mathcal{u}}
\newcommand{\vc}{\mathcal{v}}
\newcommand{\wc}{\mathcal{w}}
\newcommand{\xc}{\mathcal{x}}
\newcommand{\yc}{\mathcal{y}}
\newcommand{\zc}{\mathcal{z}}
\newcommand{\Ac}{\mathcal{A}}
\newcommand{\Bc}{\mathcal{B}}
\newcommand{\Cc}{\mathcal{C}}
\newcommand{\Dc}{\mathcal{D}}
\newcommand{\Ec}{\mathcal{E}}
\newcommand{\Fc}{\mathcal{F}}
\newcommand{\Gc}{\mathcal{G}}
\newcommand{\Hc}{\mathcal{H}}
\newcommand{\Ic}{\mathcal{I}}
\newcommand{\Jc}{\mathcal{J}}
\newcommand{\Kc}{\mathcal{K}}
\newcommand{\Lc}{\mathcal{L}}
\newcommand{\Mc}{\mathcal{M}}
\newcommand{\Nc}{\mathcal{N}}
\newcommand{\Oc}{\mathcal{O}}
\newcommand{\Pc}{\mathcal{P}}
\newcommand{\Qc}{\mathcal{Q}}
\newcommand{\Rc}{\mathcal{R}}
\newcommand{\Sc}{\mathcal{S}}
\newcommand{\Tc}{\mathcal{T}}
\newcommand{\Uc}{\mathcal{U}}
\newcommand{\Vc}{\mathcal{V}}
\newcommand{\Wc}{\mathcal{W}}
\newcommand{\Xc}{\mathcal{X}}
\newcommand{\Yc}{\mathcal{Y}}
\newcommand{\Zc}{\mathcal{Z}}

% Mathhat abbrevation
\newcommand{\ah}{\hat{a}}
\newcommand{\bh}{\hat{b}}
\newcommand{\ch}{\hat{c}}
% \newcommand{\dh}{\hat{d}}
\newcommand{\eh}{\hat{e}}
\newcommand{\fh}{\hat{f}}
\newcommand{\gh}{\hat{g}}
\newcommand{\hh}{\hat{h}}
\newcommand{\ih}{\hat{i}}
\newcommand{\jh}{\hat{j}}
\newcommand{\kh}{\hat{k}}
\newcommand{\lh}{\hat{l}}
\newcommand{\mh}{\hat{m}}
\newcommand{\nh}{\hat{n}}
\newcommand{\oh}{\hat{o}}
\newcommand{\ph}{\hat{p}}
\newcommand{\qh}{\hat{q}}
\newcommand{\rh}{\hat{r}}
\newcommand{\sh}{\hat{s}}
% \newcommand{\th}{\hat{t}}
\newcommand{\uh}{\hat{u}}
\newcommand{\vh}{\hat{v}}
\newcommand{\wh}{\hat{w}}
\newcommand{\xh}{\hat{x}}
\newcommand{\yh}{\hat{y}}
\newcommand{\zh}{\hat{z}}
\newcommand{\Ah}{\hat{A}}
\newcommand{\Bh}{\hat{B}}
\newcommand{\Ch}{\hat{C}}
\newcommand{\Dh}{\hat{D}}
\newcommand{\Eh}{\hat{E}}
\newcommand{\Fh}{\hat{F}}
\newcommand{\Gh}{\hat{G}}
\newcommand{\Hh}{\hat{H}}
\newcommand{\Ih}{\hat{I}}
\newcommand{\Jh}{\hat{J}}
\newcommand{\Kh}{\hat{K}}
\newcommand{\Lh}{\hat{L}}
\newcommand{\Mh}{\hat{M}}
\newcommand{\Nh}{\hat{N}}
\newcommand{\Oh}{\hat{O}}
\newcommand{\Ph}{\hat{P}}
\newcommand{\Qh}{\hat{Q}}
\newcommand{\Rh}{\hat{R}}
\newcommand{\Sh}{\hat{S}}
\newcommand{\Th}{\hat{T}}
\newcommand{\Uh}{\hat{U}}
\newcommand{\Vh}{\hat{V}}
\newcommand{\Wh}{\hat{W}}
\newcommand{\Xh}{\hat{X}}
\newcommand{\Yh}{\hat{Y}}
\newcommand{\Zh}{\hat{Z}}


% Mathbar abbrevation
\newcommand{\abar}{\bar{a}}
\newcommand{\bbar}{\bar{b}}
\newcommand{\cbar}{\bar{c}}
\newcommand{\dbar}{\bar{d}}
\newcommand{\ebar}{\bar{e}}
\newcommand{\fbar}{\bar{f}}
\newcommand{\gbar}{\bar{g}}
% \newcommand{\hbar}{\bar{h}}
\newcommand{\ibar}{\bar{i}}
\newcommand{\jbar}{\bar{j}}
\newcommand{\kbar}{\bar{k}}
\newcommand{\lbar}{\bar{l}}
\newcommand{\mbar}{\bar{m}}
\newcommand{\nbar}{\bar{n}}
\newcommand{\obar}{\bar{o}}
\newcommand{\pbar}{\bar{p}}
\newcommand{\qbar}{\bar{q}}
\newcommand{\rbar}{\bar{r}}
\newcommand{\sbar}{\bar{s}}
\newcommand{\tbar}{\bar{t}}
\newcommand{\ubar}{\bar{u}}
\newcommand{\vbar}{\bar{v}}
\newcommand{\wbar}{\bar{w}}
\newcommand{\xbar}{\bar{x}}
\newcommand{\ybar}{\bar{y}}
\newcommand{\zbar}{\bar{z}}
\newcommand{\Abar}{\bar{A}}
\newcommand{\Bbar}{\bar{B}}
\newcommand{\Cbar}{\bar{C}}
\newcommand{\Dbar}{\bar{D}}
\newcommand{\Ebar}{\bar{E}}
\newcommand{\Fbar}{\bar{F}}
\newcommand{\Gbar}{\bar{G}}
\newcommand{\Hbar}{\bar{H}}
\newcommand{\Ibar}{\bar{I}}
\newcommand{\Jbar}{\bar{J}}
\newcommand{\Kbar}{\bar{K}}
\newcommand{\Lbar}{\bar{L}}
\newcommand{\Mbar}{\bar{M}}
\newcommand{\Nbar}{\bar{N}}
\newcommand{\Obar}{\bar{O}}
\newcommand{\Pbar}{\bar{P}}
\newcommand{\Qbar}{\bar{Q}}
\newcommand{\Rbar}{\bar{R}}
\newcommand{\Sbar}{\bar{S}}
\newcommand{\Tbar}{\bar{T}}
\newcommand{\Ubar}{\bar{U}}
\newcommand{\Vbar}{\bar{V}}
\newcommand{\Wbar}{\bar{W}}
\newcommand{\Xbar}{\bar{X}}
\newcommand{\Ybar}{\bar{Y}}
\newcommand{\Zbar}{\bar{Z}}


% Mathtilde abbrevation
\newcommand{\at}{\tilde{a}}
\newcommand{\bt}{\tilde{b}}
\newcommand{\ct}{\tilde{c}}
\newcommand{\dt}{\tilde{d}}
\newcommand{\et}{\tilde{e}}
\newcommand{\ft}{\tilde{f}}
\newcommand{\gt}{\tilde{g}}
% \newcommand{\ht}{\tilde{h}}
% \newcommand{\it}{\tilde{i}}
\newcommand{\jt}{\tilde{j}}
\newcommand{\kt}{\tilde{k}}
\newcommand{\lt}{\tilde{l}}
\newcommand{\mt}{\tilde{m}}
\newcommand{\nt}{\tilde{n}}
\newcommand{\ot}{\tilde{o}}
\newcommand{\pt}{\tilde{p}}
\newcommand{\qt}{\tilde{q}}
\newcommand{\rt}{\tilde{r}}
\newcommand{\st}{\tilde{s}}
% \newcommand{\tt}{\tilde{t}}
\newcommand{\ut}{\tilde{u}}
\newcommand{\vt}{\tilde{v}}
\newcommand{\wt}{\tilde{w}}
\newcommand{\xt}{\tilde{x}}
\newcommand{\yt}{\tilde{y}}
\newcommand{\zt}{\tilde{z}}
\newcommand{\At}{\tilde{A}}
\newcommand{\Bt}{\tilde{B}}
\newcommand{\Ct}{\tilde{C}}
\newcommand{\Dt}{\tilde{D}}
\newcommand{\Et}{\tilde{E}}
\newcommand{\Ft}{\tilde{F}}
\newcommand{\Gt}{\tilde{G}}
\newcommand{\Ht}{\tilde{H}}
\newcommand{\It}{\tilde{I}}
\newcommand{\Jt}{\tilde{J}}
\newcommand{\Kt}{\tilde{K}}
\newcommand{\Lt}{\tilde{L}}
\newcommand{\Mt}{\tilde{M}}
\newcommand{\Nt}{\tilde{N}}
\newcommand{\Ot}{\tilde{O}}
\newcommand{\Pt}{\tilde{P}}
\newcommand{\Qt}{\tilde{Q}}
\newcommand{\Rt}{\tilde{R}}
\newcommand{\St}{\tilde{S}}
\newcommand{\Tt}{\tilde{T}}
\newcommand{\Ut}{\tilde{U}}
\newcommand{\Vt}{\tilde{V}}
\newcommand{\Wt}{\tilde{W}}
\newcommand{\Xt}{\tilde{X}}
\newcommand{\Yt}{\tilde{Y}}
\newcommand{\Zt}{\tilde{Z}}


% MathTilde abbrevation
\newcommand{\aT}{\Tilde{a}}
\newcommand{\bT}{\Tilde{b}}
\newcommand{\cT}{\Tilde{c}}
\newcommand{\dT}{\Tilde{d}}
\newcommand{\eT}{\Tilde{e}}
\newcommand{\fT}{\Tilde{f}}
\newcommand{\gT}{\Tilde{g}}
\newcommand{\hT}{\Tilde{h}}
\newcommand{\iT}{\Tilde{i}}
\newcommand{\jT}{\Tilde{j}}
\newcommand{\kT}{\Tilde{k}}
\newcommand{\lT}{\Tilde{l}}
\newcommand{\mT}{\Tilde{m}}
\newcommand{\nT}{\Tilde{n}}
\newcommand{\oT}{\Tilde{o}}
\newcommand{\pT}{\Tilde{p}}
\newcommand{\qT}{\Tilde{q}}
\newcommand{\rT}{\Tilde{r}}
\newcommand{\sT}{\Tilde{s}}
\newcommand{\tT}{\Tilde{t}}
\newcommand{\uT}{\Tilde{u}}
\newcommand{\vT}{\Tilde{v}}
\newcommand{\wT}{\Tilde{w}}
\newcommand{\xT}{\Tilde{x}}
\newcommand{\yT}{\Tilde{y}}
\newcommand{\zT}{\Tilde{z}}
\newcommand{\AT}{\Tilde{A}}
\newcommand{\BT}{\Tilde{B}}
\newcommand{\CT}{\Tilde{C}}
\newcommand{\DT}{\Tilde{D}}
\newcommand{\ET}{\Tilde{E}}
\newcommand{\FT}{\Tilde{F}}
\newcommand{\GT}{\Tilde{G}}
\newcommand{\HT}{\Tilde{H}}
\newcommand{\IT}{\Tilde{I}}
\newcommand{\JT}{\Tilde{J}}
% \newcommand{\KT}{\Tilde{K}}
\newcommand{\LT}{\Tilde{L}}
\newcommand{\MT}{\Tilde{M}}
% \newcommand{\NT}{\Tilde{N}}
\newcommand{\OT}{\Tilde{O}}
% \newcommand{\PT}{\Tilde{P}}
\newcommand{\QT}{\Tilde{Q}}
\newcommand{\RT}{\Tilde{R}}
\newcommand{\ST}{\Tilde{S}}
\newcommand{\TT}{\Tilde{T}}
\newcommand{\UT}{\Tilde{U}}
\newcommand{\VT}{\Tilde{V}}
\newcommand{\WT}{\Tilde{W}}
\newcommand{\XT}{\Tilde{X}}
\newcommand{\YT}{\Tilde{Y}}
\newcommand{\ZT}{\Tilde{Z}}



%Nested Case




% Mathbold + hat abbrevation
\newcommand{\abh}{\mathbf{\hat{a}}}
\newcommand{\bbh}{\mathbf{\hat{b}}}
\newcommand{\cbh}{\mathbf{\hat{c}}}
\newcommand{\dbh}{\mathbf{\hat{d}}}
\newcommand{\ebh}{\mathbf{\hat{e}}}
\newcommand{\fbh}{\mathbf{\hat{f}}}
\newcommand{\gbh}{\mathbf{\hat{g}}}
\newcommand{\hbh}{\mathbf{\hat{h}}}
\newcommand{\ibh}{\mathbf{\hat{i}}}
\newcommand{\jbh}{\mathbf{\hat{j}}}
\newcommand{\kbh}{\mathbf{\hat{k}}}
\newcommand{\lbh}{\mathbf{\hat{l}}}
\newcommand{\mbh}{\mathbf{\hat{m}}}
\newcommand{\nbh}{\mathbf{\hat{n}}}
\newcommand{\obh}{\mathbf{\hat{o}}}
\newcommand{\pbh}{\mathbf{\hat{p}}}
\newcommand{\qbh}{\mathbf{\hat{q}}}
\newcommand{\rbh}{\mathbf{\hat{r}}}
\newcommand{\sbh}{\mathbf{\hat{s}}}
\newcommand{\tbh}{\mathbf{\hat{t}}}
\newcommand{\ubh}{\mathbf{\hat{u}}}
\newcommand{\vbh}{\mathbf{\hat{v}}}
\newcommand{\wbh}{\mathbf{\hat{w}}}
\newcommand{\xbh}{\mathbf{\hat{x}}}
\newcommand{\ybh}{\mathbf{\hat{y}}}
\newcommand{\zbh}{\mathbf{\hat{z}}}
\newcommand{\Abh}{\mathbf{\hat{A}}}
\newcommand{\Bbh}{\mathbf{\hat{B}}}
\newcommand{\Cbh}{\mathbf{\hat{C}}}
\newcommand{\Dbh}{\mathbf{\hat{D}}}
\newcommand{\Ebh}{\mathbf{\hat{E}}}
\newcommand{\Fbh}{\mathbf{\hat{F}}}
\newcommand{\Gbh}{\mathbf{\hat{G}}}
\newcommand{\Hbh}{\mathbf{\hat{H}}}
\newcommand{\Ibh}{\mathbf{\hat{I}}}
\newcommand{\Jbh}{\mathbf{\hat{J}}}
\newcommand{\Kbh}{\mathbf{\hat{K}}}
\newcommand{\Lbh}{\mathbf{\hat{L}}}
\newcommand{\Mbh}{\mathbf{\hat{M}}}
\newcommand{\Nbh}{\mathbf{\hat{N}}}
\newcommand{\Obh}{\mathbf{\hat{O}}}
\newcommand{\Pbh}{\mathbf{\hat{P}}}
\newcommand{\Qbh}{\mathbf{\hat{Q}}}
\newcommand{\Rbh}{\mathbf{\hat{R}}}
\newcommand{\Sbh}{\mathbf{\hat{S}}}
\newcommand{\Tbh}{\mathbf{\hat{T}}}
\newcommand{\Ubh}{\mathbf{\hat{U}}}
\newcommand{\Vbh}{\mathbf{\hat{V}}}
\newcommand{\Wbh}{\mathbf{\hat{W}}}
\newcommand{\Xbh}{\mathbf{\hat{X}}}
\newcommand{\Ybh}{\mathbf{\hat{Y}}}
\newcommand{\Zbh}{\mathbf{\hat{Z}}}


% Mathbold + Tilde abbrevation
\newcommand{\abT}{\mathbf{\Tilde{a}}}
\newcommand{\bbT}{\mathbf{\Tilde{b}}}
\newcommand{\cbT}{\mathbf{\Tilde{c}}}
\newcommand{\dbT}{\mathbf{\Tilde{d}}}
\newcommand{\ebT}{\mathbf{\Tilde{e}}}
\newcommand{\fbT}{\mathbf{\Tilde{f}}}
\newcommand{\gbT}{\mathbf{\Tilde{g}}}
\newcommand{\hbT}{\mathbf{\Tilde{h}}}
\newcommand{\ibT}{\mathbf{\Tilde{i}}}
\newcommand{\jbT}{\mathbf{\Tilde{j}}}
\newcommand{\kbT}{\mathbf{\Tilde{k}}}
\newcommand{\lbT}{\mathbf{\Tilde{l}}}
\newcommand{\mbT}{\mathbf{\Tilde{m}}}
\newcommand{\nbT}{\mathbf{\Tilde{n}}}
\newcommand{\obT}{\mathbf{\Tilde{o}}}
\newcommand{\pbT}{\mathbf{\Tilde{p}}}
\newcommand{\qbT}{\mathbf{\Tilde{q}}}
\newcommand{\rbT}{\mathbf{\Tilde{r}}}
\newcommand{\sbT}{\mathbf{\Tilde{s}}}
\newcommand{\tbT}{\mathbf{\Tilde{t}}}
\newcommand{\ubT}{\mathbf{\Tilde{u}}}
\newcommand{\vbT}{\mathbf{\Tilde{v}}}
\newcommand{\wbT}{\mathbf{\Tilde{w}}}
\newcommand{\xbT}{\mathbf{\Tilde{x}}}
\newcommand{\ybT}{\mathbf{\Tilde{y}}}
\newcommand{\zbT}{\mathbf{\Tilde{z}}}
\newcommand{\AbT}{\mathbf{\Tilde{A}}}
\newcommand{\BbT}{\mathbf{\Tilde{B}}}
\newcommand{\CbT}{\mathbf{\Tilde{C}}}
\newcommand{\DbT}{\mathbf{\Tilde{D}}}
\newcommand{\EbT}{\mathbf{\Tilde{E}}}
\newcommand{\FbT}{\mathbf{\Tilde{F}}}
\newcommand{\GbT}{\mathbf{\Tilde{G}}}
\newcommand{\HbT}{\mathbf{\Tilde{H}}}
\newcommand{\IbT}{\mathbf{\Tilde{I}}}
\newcommand{\JbT}{\mathbf{\Tilde{J}}}
\newcommand{\KbT}{\mathbf{\Tilde{K}}}
\newcommand{\LbT}{\mathbf{\Tilde{L}}}
\newcommand{\MbT}{\mathbf{\Tilde{M}}}
\newcommand{\NbT}{\mathbf{\Tilde{N}}}
\newcommand{\ObT}{\mathbf{\Tilde{O}}}
\newcommand{\PbT}{\mathbf{\Tilde{P}}}
\newcommand{\QbT}{\mathbf{\Tilde{Q}}}
\newcommand{\RbT}{\mathbf{\Tilde{R}}}
\newcommand{\SbT}{\mathbf{\Tilde{S}}}
\newcommand{\TbT}{\mathbf{\Tilde{T}}}
\newcommand{\UbT}{\mathbf{\Tilde{U}}}
\newcommand{\VbT}{\mathbf{\Tilde{V}}}
\newcommand{\WbT}{\mathbf{\Tilde{W}}}
\newcommand{\XbT}{\mathbf{\Tilde{X}}}
\newcommand{\YbT}{\mathbf{\Tilde{Y}}}
\newcommand{\ZbT}{\mathbf{\Tilde{Z}}}


% Mathbold + tilde abbrevation
\newcommand{\abt}{\mathbf{\tilde{a}}}
\newcommand{\bbt}{\mathbf{\tilde{b}}}
\newcommand{\cbt}{\mathbf{\tilde{c}}}
\newcommand{\dbt}{\mathbf{\tilde{d}}}
\newcommand{\ebt}{\mathbf{\tilde{e}}}
\newcommand{\fbt}{\mathbf{\tilde{f}}}
\newcommand{\gbt}{\mathbf{\tilde{g}}}
\newcommand{\hbt}{\mathbf{\tilde{h}}}
\newcommand{\ibt}{\mathbf{\tilde{i}}}
\newcommand{\jbt}{\mathbf{\tilde{j}}}
\newcommand{\kbt}{\mathbf{\tilde{k}}}
\newcommand{\lbt}{\mathbf{\tilde{l}}}
\newcommand{\mbt}{\mathbf{\tilde{m}}}
\newcommand{\nbt}{\mathbf{\tilde{n}}}
\newcommand{\obt}{\mathbf{\tilde{o}}}
\newcommand{\pbt}{\mathbf{\tilde{p}}}
\newcommand{\qbt}{\mathbf{\tilde{q}}}
\newcommand{\rbt}{\mathbf{\tilde{r}}}
\newcommand{\sbt}{\mathbf{\tilde{s}}}
\newcommand{\tbt}{\mathbf{\tilde{t}}}
\newcommand{\ubt}{\mathbf{\tilde{u}}}
\newcommand{\vbt}{\mathbf{\tilde{v}}}
\newcommand{\wbt}{\mathbf{\tilde{w}}}
\newcommand{\xbt}{\mathbf{\tilde{x}}}
\newcommand{\ybt}{\mathbf{\tilde{y}}}
\newcommand{\zbt}{\mathbf{\tilde{z}}}
\newcommand{\Abt}{\mathbf{\tilde{A}}}
\newcommand{\Bbt}{\mathbf{\tilde{B}}}
\newcommand{\Cbt}{\mathbf{\tilde{C}}}
\newcommand{\Dbt}{\mathbf{\tilde{D}}}
\newcommand{\Ebt}{\mathbf{\tilde{E}}}
\newcommand{\Fbt}{\mathbf{\tilde{F}}}
\newcommand{\Gbt}{\mathbf{\tilde{G}}}
\newcommand{\Hbt}{\mathbf{\tilde{H}}}
\newcommand{\Ibt}{\mathbf{\tilde{I}}}
\newcommand{\Jbt}{\mathbf{\tilde{J}}}
\newcommand{\Kbt}{\mathbf{\tilde{K}}}
\newcommand{\Lbt}{\mathbf{\tilde{L}}}
\newcommand{\Mbt}{\mathbf{\tilde{M}}}
\newcommand{\Nbt}{\mathbf{\tilde{N}}}
\newcommand{\Obt}{\mathbf{\tilde{O}}}
\newcommand{\Pbt}{\mathbf{\tilde{P}}}
\newcommand{\Qbt}{\mathbf{\tilde{Q}}}
\newcommand{\Rbt}{\mathbf{\tilde{R}}}
\newcommand{\Sbt}{\mathbf{\tilde{S}}}
\newcommand{\Tbt}{\mathbf{\tilde{T}}}
\newcommand{\Ubt}{\mathbf{\tilde{U}}}
\newcommand{\Vbt}{\mathbf{\tilde{V}}}
\newcommand{\Wbt}{\mathbf{\tilde{W}}}
\newcommand{\Xbt}{\mathbf{\tilde{X}}}
\newcommand{\Ybt}{\mathbf{\tilde{Y}}}
\newcommand{\Zbt}{\mathbf{\tilde{Z}}}


% Mathbold + bar abbrevation
\newcommand{\abb}{\mathbf{\bar{a}}}
\newcommand{\bbb}{\mathbf{\bar{b}}}
\newcommand{\cbb}{\mathbf{\bar{c}}}
\newcommand{\dbb}{\mathbf{\bar{d}}}
\newcommand{\ebb}{\mathbf{\bar{e}}}
\newcommand{\fbb}{\mathbf{\bar{f}}}
\newcommand{\gbb}{\mathbf{\bar{g}}}
\newcommand{\hbb}{\mathbf{\bar{h}}}
\newcommand{\ibb}{\mathbf{\bar{i}}}
\newcommand{\jbb}{\mathbf{\bar{j}}}
\newcommand{\kbb}{\mathbf{\bar{k}}}
\newcommand{\lbb}{\mathbf{\bar{l}}}
\newcommand{\mbb}{\mathbf{\bar{m}}}
\newcommand{\nbb}{\mathbf{\bar{n}}}
\newcommand{\obb}{\mathbf{\bar{o}}}
\newcommand{\pbb}{\mathbf{\bar{p}}}
\newcommand{\qbb}{\mathbf{\bar{q}}}
\newcommand{\rbb}{\mathbf{\bar{r}}}
\newcommand{\sbb}{\mathbf{\bar{s}}}
\newcommand{\tbb}{\mathbf{\bar{t}}}
\newcommand{\ubb}{\mathbf{\bar{u}}}
\newcommand{\vbb}{\mathbf{\bar{v}}}
\newcommand{\wbb}{\mathbf{\bar{w}}}
\newcommand{\xbb}{\mathbf{\bar{x}}}
\newcommand{\ybb}{\mathbf{\bar{y}}}
\newcommand{\zbb}{\mathbf{\bar{z}}}
\newcommand{\Abb}{\mathbf{\bar{A}}}
% \newcommand{\Bbb}{\mathbf{\bar{B}}}
\newcommand{\Cbb}{\mathbf{\bar{C}}}
\newcommand{\Dbb}{\mathbf{\bar{D}}}
\newcommand{\Ebb}{\mathbf{\bar{E}}}
\newcommand{\Fbb}{\mathbf{\bar{F}}}
\newcommand{\Gbb}{\mathbf{\bar{G}}}
\newcommand{\Hbb}{\mathbf{\bar{H}}}
\newcommand{\Ibb}{\mathbf{\bar{I}}}
\newcommand{\Jbb}{\mathbf{\bar{J}}}
\newcommand{\Kbb}{\mathbf{\bar{K}}}
\newcommand{\Lbb}{\mathbf{\bar{L}}}
\newcommand{\Mbb}{\mathbf{\bar{M}}}
\newcommand{\Nbb}{\mathbf{\bar{N}}}
\newcommand{\Obb}{\mathbf{\bar{O}}}
\newcommand{\Pbb}{\mathbf{\bar{P}}}
\newcommand{\Qbb}{\mathbf{\bar{Q}}}
\newcommand{\Rbb}{\mathbf{\bar{R}}}
\newcommand{\Sbb}{\mathbf{\bar{S}}}
\newcommand{\Tbb}{\mathbf{\bar{T}}}
\newcommand{\Ubb}{\mathbf{\bar{U}}}
\newcommand{\Vbb}{\mathbf{\bar{V}}}
\newcommand{\Wbb}{\mathbf{\bar{W}}}
\newcommand{\Xbb}{\mathbf{\bar{X}}}
\newcommand{\Ybb}{\mathbf{\bar{Y}}}
\newcommand{\Zbb}{\mathbf{\bar{Z}}}


\newcommand{\E}{\mathbb{E}}
\newcommand{\Prb}{\mathbb{P}}
\newcommand{\R}{\mathbb{R}}
\newcommand{\C}{\mathbb{C}}
\newcommand{\I}{\mathbb{I}}
\newcommand{\F}{\mathbb{F}}
\newcommand{\vecc}{\mathbf{c}}
\newcommand{\Z}{\mathbb{Z}}
\newcommand{\nul}{\text{null }}
\newcommand{\range}{\text{range }}
\newcommand{\Span}{\text{span}}
\newcommand{\tr}{\text{tr}}
\newcommand{\diag}{\text{diag}}
\newcommand{\limn}{\lim_{n \to \infty}}
\newcommand{\zo}{\overline{z}}
\newcommand{\wo}{\overline{w}}
\newcommand{\ao}{\overline{a}}
\newcommand{\vo}{\overline{v}}
\newcommand{\glo}{\overline{\lambda}}
\newcommand{\Var}{\text{Var}}
\newcommand{\1}{\mathbbm{1}}
\newcommand{\Pm}{\mathbbm{P}}
\usepackage{hyperref}

\title{\vspace{-2em}Chapter 2: Finite-Dimensional Vector Space}
\author{\emph{Linear Algebra Done Right}, by Sheldon Axler}
\date{}

\begin{document}
\maketitle 
\tableofcontents
\newpage


\section*{2A: Span and Linear Independence}
\addcontentsline{toc}{section}{2A: Span and Linear Independence}

\begin{definition}[Linear Combination]
    A \emph{linear combination} of a list \(v_1, \ldots, v_m\) of vectors in \(V\)
    is a vector of the form 
    \[a_1 v_1 + \cdots + a_m v_m\]
    where \(a_1, \ldots, a_m \in \F\).
\end{definition}

\begin{definition}[Span]
    The set of all linear combinations of list of vectors \(v_1, \ldots, v_m\) 
    in \(V\) is called the \emph{span} of \(v_1, \ldots, v_m\), denoted by 
    span\((v_1, \ldots, v_m)\). In other words, 
    \[\text{span}(v_1, \ldots, v_m) = \{a_1 v_1 + \cdots a_m v_m \colon a_1, \ldots, a_m \in \F\}\]
    The span of the empty list \(()\) is defined to be \(\{0\}\).
\end{definition}


\begin{thm}
    The span of a list of vectors in \(V\) is the smallest subspace of \(V\) 
    containing all vectors in the list.
\end{thm}

\begin{definition}[Spans]
    If span\((v_1, \ldots, v_m)\) equals \(V\), we say the list \(v_1, \ldots, v_m\)
    \emph{spans} \(V\). 
\end{definition}

\begin{definition}[Finite-dimensional vector space]
    A vector space is called \emph{finite-dimensional} if some list of vectors 
    in it spans the space.
\end{definition}

\begin{definition}[polynomial]
    A function \(p \colon \F \to \F\) is called a \emph{polynomial} with coefficients
    in \(\F\) if there exist \(a_0, \ldots, a_m \in \F\) s.t. 
    \[(z) = a_0 + a_1 z + a_2 z^2 + \cdots + a_m z^m\]
    for all \(z \in \F\).

    \(\Pcal(\F)\) is the set of all polynomials with coefficients in \(\F\).
\end{definition}

\begin{definition}[Linear independence]
    A list of vectors \(v_1, \ldots, v_m\) in \(V\) is called \emph{linearly independent}
    if the only choice of \(a_1, \ldots, a_m \in \F\) that makes 
    \[a_1v_1 + \cdots + a_m v_m = 0\]
    is \(a_1 = \cdots = a_m = 0\). 

    The empty list \(()\) is also declared to be linearly independent.
\end{definition}

\begin{lemma}
    Suppose \(v_1, \ldots v_m\) is a linearly dependent list in \(V\). Then 
    there exists \(k \in \{1, 2, \ldots, m\}\) s.t. 
    \[v_k \in \text{span}(v_1, \ldots, v_{k-1})\]
    Furthermore, if \(k\) satisfies the condition above and the \(k^{\text{th}}\)
    term is removed from \(v_1, \ldots v_m\), then the span of the reamining list 
    equals \(\text{span}(v_1, \ldots, v_m)\).
\end{lemma}

\begin{lemma}[length of linearly independent list \(\leq\) length of spanning list]
    In a finite-dimensional vector space, the length of every linearly independent 
    list of vectors is less than or equal to the length of every spanning 
    list of vectors. 
\end{lemma}

\newpage 
\addcontentsline{toc}{subsection}{2A Problem Sets}
\begin{problem}{1}
    Find a list of four distinct vectors in \(\F^3\) whose span equals 
    \[\{(x, y, z) \in \F^3 \colon x + y + z =0\}\]
\end{problem}

\begin{proof}
Example: \((1, 0, 0), (0, 1, 0), (0, 0, 1), (-1, -1, -1)\). 
\end{proof}

\begin{problem}{2}
    Prove or give a counterexample: If \(v_1, v_2, v_3, v_4\) spans \(V\), then 
    the list 
    \[v_1 - v_2, v_2 - v_3, v_3 - v_4, v_4\]
    also spans \(V\).
\end{problem}

\begin{proof}
Take any \(v \in V\), then we have \(v = a_1 v_1 + a_2 v_2 + a_3 v_3 
+ a_4 v_4 = a_1(v_1 - v_2) + (a_1 + a_2) (v_2 - v_3) + (a_1 + a_2 + a_3)(
    v_3 - v_4) + (a_1 + a_2 +a_3 + a_4) v_4\). Conversely, any linear combination 
    of these vectors still belong to \(V\).
\end{proof}


\begin{problem}{3}
    Suppose \(v_1, \ldots, v_m\) is a list of vectors in \(V\). 
    For \(k \in \{1, \ldots, m\}\), let 
    \[w_k = v_1 + \cdots + v_k\]
    Show that span(\(v_1, \ldots, v_m\)) = span(\(w_1, \ldots, w_m\))
\end{problem}

\begin{proof}
Take \(v = \sum_{i=1}^m a_i v_i\) from l.h.s, then we can write 

\begin{align*}
    v 
    &= a_1 v_1 + \cdots a_m v_m \\ 
    &= (a_1 - a_m) v_1 +\cdots (a_{m-1} - a_{m}) v_{m-1} + a_m w_m \\ 
    &= (a_1 - a_m - a_{m-1}) v_1 + \cdots (a_{m - 2} - a_{m-1} - a_m) v_{m-2}
    +(a_{m-1} - a_m) w_{m-1}  + a_m w_m \\ 
    &= \sum_{i=1}^m (a_i - \sum_{j=i+1}^m a_j)w_i \in \text{r.h.s}
\end{align*}

Conversely, take \(w = \sum_{i=1}^m b_i w_i = \sum_{i=1}^m (b_i \sum_{j=1}^i c_j) v_j \in \text{l.h.s.}\).

\end{proof}

\begin{problem}{4}
    (a) Show that a list of length one in a vector space is linearly independent 
    if and only if the vector in the list is not 0. 
    
    (b) Show that a list of length two in a vector space is linearly independent 
    if and only if neither of the two vectors in the list is a scalar multiple of 
    the other.
\end{problem}

\begin{proof}
(a) In order for the only way to write \(a v = 0\) is to ensure \(v \neq 0\). 

(b) \(a v_1 + b v_2 = 0\). \(\Rightarrow\) the only solution is \(a=b=0\) so \(v_1\)
cannot be a multiple of \(v_2\); vice versa.  \(\Leftarrow\) same reason.
\end{proof}


\begin{problem}{8}
    Suppose \(v_1, v_2, v_3, v_4\) is linearly independent in \(V\). Prove that the list 
    \[v_1 - v_2, v_2 - v_3, v_3 - v_4 , v_4\]
    is also linearly independent. 
\end{problem}

\begin{proof}

    \[a_1(v_1 - v_2) + a_2 (v_2 - v_3) + a_3 (v_3 - v_4) + a_4 v_4 = a_1v_1 
    + (-a_1 + a_2)v_2 + (-a_2 + a_3)v_3 + (-a_3 + a_4) v_4\]

    The only solution is \(a_1 = 0, -a_1 + a_2 = 0, -a_2 + a_3 = 0, -a_3 + a_4 = 0\).
\end{proof}


\begin{problem}{9}
    Prove or give a counter example: If \(v_1, \ldots, v_m\) is a linearly independent 
    list of vectors in \(V\), then 
    \[5v_1 - 4v_2, v_2, v_3, \ldots, v_m\]
    is also linearly independent.
\end{problem}

\begin{proof}

\begin{align*}
    &a_1 (5v_1 - 4v_2) + a_2 v_2 + \ldots a_m v_m \\ 
    &=5a_1 v_1 + (-4a_1  + a_2)v_2 + \ldots a_m v_m
\end{align*}

The only solution is \(5a_1 = -4a_1 + a_2 = \ldots = a_m = 0\).

\end{proof}


\begin{problem}{10}
    Prove or give a counterexample: If \(v_1, \ldots, v_m\) is a linearly 
    independent list of vectors in \(V\)  and \(\lambda \in \F\) with 
    \(\lambda \neq 0\), then \(\lambda v_1, \ldots, \lambda v_m\) is linearly 
    indepedent.
\end{problem}

\begin{proof}
\[\sum_{i}^m \lambda a_i v_i = 0\]

The only solution is \(a_i = 0\) for all \(i\).
\end{proof}

\begin{problem}{12}
    Suppose \(v_1, \ldots, v_m\) is linearly independet in \(V\) 
    and \(w \in V\). Prove that if \(v_1 + w, \ldots, v_m + w\) is 
    linearly dependent, then \(w \in \text{span}(v_1, \ldots, v_m)\).
\end{problem}

\begin{proof}
We know the only solution to \(\sum_i^m a_i v_i\) is all \(a_i = 0\). Now 
we have that non-zero \(a_i\) for solving \(\sum_i^m a_i v_i + \sum_i^m a_i w = 0\).
Thus \(w = \frac{\sum_i^m a_i v_i}{\sum_i^m a_i}\) which completes the proof.  
\end{proof}

\begin{problem}{13}
    Suppose \(v_1, \ldots, v_m\) is linearly independent in \(V\) and 
    \(w \in V\). Show that 
    \[v_1, \ldots, v_m, w \text{ is linearly independent } \Leftrightarrow 
    w \notin \text{ span}(v_1, \ldots, v_m)\]
\end{problem}

\begin{proof}
\(\Rightarrow\) By contradiction if \(w\) in the span then it can be written 
as linear combination for some nonzero \(a_i\) and thus they are ld. 

\(\Leftarrow\) Similarly. 
\end{proof}


\begin{problem}{17}
Prove that \(V\) is infinite-dimensional if and only if there is a sequence 
\(v_1, v_2, \ldots\) of vectors in \(V\) such that \(v_1, \ldots, v_m\) is 
linearly independent for every positive integer \(m\). 
\end{problem}

\begin{proof}
\(\Rightarrow\) There doesn't exist any finite list of vectors that span the space. For 
the sake of contradicting assumes for every sequence  \(v_1, \ldots\) of 
vectors in \(V\) 
\(\exists\) m such that \(v_1, \ldots, v_m\)
is linearly dependent. Then this means we can construct the basis for the vector 
space as follows: select \(v_i \in V\) s.t. \(v_i\) and \(v_1, \ldots, v_{i-1}\)
are linearly independent. This means that there exits \(m\) s.t. \(v_1, \ldots, v_m\)
that spans \(V\), forming a contradiction. 

\(\Leftarrow\) Suppose for contradiction. Then there exists a span, contradicting 
the linear independence claim for every \(m\).

\end{proof}

\begin{problem}{18}
    Prove \(\F^\infty\) is infinite-dimensional. 
\end{problem}

\begin{proof}
We apply Problem 17. Construct \(v_i\) to be the vector that has 1 on the 
i-th coordinate and 0 elsewhere. clearly \(v_1, v_2, \ldots\) are s.t. 
\(v_1, \ldots, v_m\) is linearly indepedent for every positive integer \(m\).
\end{proof}


\newpage 
\addcontentsline{toc}{section}{2B: Bases}
\section*{2B: Bases}

\begin{definition}[basis]
    A \emph{basis} of \(V\) is a list of vectors in \(V\) that is linearly 
    independent and spans \(V\).
\end{definition}

\begin{thm}[criterion for basis]
    A list \(v_1, \ldots, v_m\) of vectors in \(V\) is a basis of \(V\)
    if and only if every \(v \in V\) can be written uniquely in the form 
    \[v = \sum_i^m a_i v_i\]
    where \(a_i \in \F\).
\end{thm}

\begin{lemma}[every spanning list contains a basis]
    Every spanning list in a vector space can be reduced to a basis of the 
    vector space.
\end{lemma}

\begin{lemma}
    Every finite-dimensional vector space has a basis.
\end{lemma}

\begin{lemma}
    Every linearly independent list of vectors in a finite-dimensional vector 
    space can be extended to a basis of the vector space. 
\end{lemma}

\begin{lemma}
    Suppose \(V\) is finite-dimensional and \(\Ucal\) is a subspce of \(V\). 
    Then there is a subspace \(W\) of \(V\) such that \(V = \Ucal \oplus W\).
\end{lemma}

\newpage 
\addcontentsline{toc}{subsection}{2B Problem Sets}

\begin{problem}{1}
    Find all vector spaces that have exactly one basis.
\end{problem}

\begin{proof}
The only answer is \(\{0\}\). Otherwise, for any basis \(v\) one can get 
\(av\) for \(a \neq 0, a \neq 1\).
\end{proof}

\begin{problem}{4}
    \begin{enumerate}
        \item Let \(\Ucal\) be the subspace of \(\C^5\) defined by 
        \(\Ucal = \{(z_1, z_2, z_3, z_4, z_5) \in \C^5 \colon 6z_1 = z_2
        , z_3 + 2z_4 + 3z_5 = 0\}\)
        Find a basis of \(\Ucal\).
        \item Extend the basis to a basis in \(\C^5\).
        \item Find a subspace \(\Wcal\) of \(\C^5\) s.t. \(\C^5 = \Ucal \oplus \Wcal\).
    \end{enumerate}
\end{problem}

\begin{proof}
\begin{enumerate}
    \item  \((z_1, 6z_1, -2z_4 - 3 z_5, z_4, z_5) \Rightarrow\)   \{(1,6,0,0,0), (0,0,-2,1,0), (0,0,-3,0,1)\}
    \item \(\{(1,6,0,0,0), (0,0,-2,1,0), (0,0,-3,0,1), (0,1,0,0,0), (0,0,1,0,0)\}\)
    \item \(\Wcal = \text{span}(\{(0,1,0,0,0), (0,0,1,0,0)\})\)
\end{enumerate}
\end{proof}

\begin{problem}{5}
Suppose \(V\) is finite-dimensional and \(\Ucal, \Wcal\) are subspaces of \(V\) such that \(V = \Ucal
+ \Wcal\). Prove that there exists a basis of \(V\) consisting of vectors in \(\Ucal \cup \Wcal\).    
\end{problem}

\begin{proof}
Let \(\{v_i\}_{i=1}^m\) denote the basis for the vector space \(V\). By definition we have 
\(v_i = u_i + w_i\) for some \(u_i, w_i\). Then we have the spanning set of the vector space \(V\)
\(\sum_i^m a_i (u_i + w_i)\), which can be reduced to a basis by the lemma.
\end{proof}


\begin{problem}{7}
    Suppose \(v_1, v_2, v_3, v_4\) is a basis of \(V\). Prove that 
    \[v_1 + v_2, v_2 + v_3, v_3+v_4, v_4\]
    is also a basis of \(V\). 
\end{problem}

\begin{proof}
We know \(v_1, v_2, v_3, v_4\) is linearly independent and spans \(V\). 
\[a_1(v_1 + v_2) + a_2 (v_2 + v_3) + a_3 (v_3 + v_4) + a_4 v_4 = a_1 v_1 
+ (a_1 + a_2) v_2 + (a_2 + a_3)v_3 + (a_3 + a_4)v_4\]
which shows the linear independence. For proving spanning, let \(v \in V\) then 

\[v = \sum_{i=1}^4 a_i v_i = a_1(v_1 + v_2) + (a_2 - a_1)(v_2 + v_3)
+ (a_3 - a_2)(v_3 + v_4) + (a_4 - a_3)v_4\]
\end{proof}


\begin{problem}{8}
    Prove or give a counterexample: If \(v_1, v_2, v_3, v_4\) is a basis of \(V\) and 
    \(\Ucal\) is a subspace of \(V\) such that \(v_1, v_2 \in \Ucal\) and \(v_3 \notin \Ucal\)
    and \(v_4 \notin \Ucal\), then \(v_1, v_2\) is a basis of \(\Ucal\). 
\end{problem}

\begin{proof}
Take \(V = \R^4\) and the standard basis. Consider \(\Ucal = \{(x_1, x_2, x_3, kx_3)\}\), then 
we disprove the claim. 
\end{proof}

\begin{problem}{10}
    Suppose \(\Ucal\) and \(\Wcal\) are subspaces of \(V\) s.t. \(V = \Ucal \oplus \Wcal\). Suppose 
    also that \(u_1, \ldots u_m\) is a basis of \(\Ucal\) and \(w_1, \ldots, w_n\) is a basis of 
    \(\Wcal\). Prove that 
    \[u_1, \ldots, u_m, w_1, \ldots, w_n\]
    is a basis of \(V\). 
\end{problem}


\begin{proof}
We know 
that this set is linearly independent (otherwise violating the direct sum assumption) 
so it sufficies to prove the spanning. Let \(v \in V\),
then \(v = u + w = \sum_{i=1}^m a_i u_i + \sum_{j=1}^n b_j w_j\). 

\end{proof}


\newpage 

\section*{2C: Dimension}
\addcontentsline{toc}{section}{2C: Dimension}

\begin{lemma}[basis length does not depend on basis]
    Any two bases of a finite-dimensional vector space have the same length. 
\end{lemma}

\begin{definition}[dimension]
    \begin{itemize}
        \item The \emph{dimension} of a finite-dimensional vector space is the length of 
        any basis of the vector space. 
        \item The dimension of a finite-dimensional vector space \(V\) is denoted by dim \(V\).
    \end{itemize}
\end{definition}

\begin{corollary}
    If \(V\) is finite-dimensional and \(\Ucal\) is a subspace of \(V\), then 
    \(\dim \Ucal \leq \dim V\). 
\end{corollary}

\begin{corollary}
    Suppose \(V\) is finite-dimensional. Then every linearly independent list of vectors 
    in \(V\) of length \(\dim V\) is a basis of \(V\). 
\end{corollary}

\begin{corollary}
    Suppose that \(V\) is finite-dimensional and \(\Ucal\) is a subspace of \(V\) such that 
    \(\dim \Ucal = \dim V\). Then \(\Ucal = V\).
\end{corollary}

\begin{corollary}
    Suppose that \(V\) is finite-dimensional. Then every spanning list of vectors in \(V\)
    of length dim \(V\) is a basis of \(V\). 
\end{corollary}

\begin{thm}[dimension of a sum]
    If \(V_1\) and \(V_2\) are subspaces of a finite-dimensional vector space, then 
    \[\dim(V_1 + V_2) = \dim V_1 + \dim V_ 2 - \dim (V_1 \cap V_2)\]
\end{thm}

\newpage 
\addcontentsline{toc}{subsection}{2C Problem Sets}


\begin{problem}{1}
    Show that the subspaces of \(\R^2\) are precisely \(\{0\}\), all lines in \(\R^2\)
    containing the origin and \(\R^2\).
\end{problem}

\begin{proof}
We know \(\dim (\R^2) = 2\) so the subspace dimension is either 0 (\(\{0\}\)) or 
1 (then this means it has to be lines crossing the origin). 
\end{proof}



\begin{problem}{5}
    (a) Let \(\Ucal = \{p \in \Pcal_4 (\F) \colon p(2) = p(5)\}\). Find a basis of \(\Ucal\).

    (b) Extend the basis in (a) to a basis of \(\Pcal_4 (\F)\). 

    (c) Find a subspace \(\Wcal\) of \(\Pcal_4 (\F)\) s.t. \(\Pcal_4(\F) = \Ucal \oplus \Wcal\).
\end{problem}

\begin{proof}
(a)  This means that 
\[a_0 + 2a_1 + 4a_2 + 8a_3 + 16a_4 = a_0 + 5a_1 + 25a_2 + 125 a_3 + 625a_4\]

Solving this gives that \(a_1 = -7a_2 - 39a_3 - 203 a_4\). So we can write that 

\begin{align*}
    p(x)
    &= a_0 + (-7a_2 - 39a_3 - 203 a_4)x + a_2 x^2 + a_3 x^3 + a_4 x^4 \\ 
    &= a_0 + a_2(x^2 - 7x) + a_3(x^3 - 39x) + a_4 (x^4 - 203x)
\end{align*}

The basis now becomes \(\{1, x^2 - 7x, x^3 - 39x, x^4 - 203x\}\)

(b) \(\{1, x, x^2 - 7x, x^3 - 39x, x^4 - 203x\}\)

(c) \(\Wcal = \{x\}\)

\end{proof}

\begin{problem}{8}
    Suppose \(v_1, \ldots, v_m\) is linearly independent in \(V\) and \(w \in V\). Prove that 

    \[\dim \text{span} (v_1 + w, \ldots, v_m + w) \geq m - 1\]
\end{problem}

\begin{proof}
We claim that 
\[v_{i+1} - v_i \in \text{span}(v_1 + w, \ldots, v_m + w) \text{ for all } m \geq i \geq 2\]

as \(v_{i+1} - v_i = (v_{i+1} + w) - (v_i + w)\). Thus \(v_2 - v_1, \ldots, v_m - v_{m-1}\) is 
in the \(\text{span}(v_1 + w, \ldots, v_m + w)\). We've proved that \(v_2 - v_1, \ldots, v_m - v_{m-1}\)
is linearly independent and this list has \(m - 1\) vectors and thus we've proved the claim.

\end{proof}


\begin{problem}{9}
    Suppose \(m\) is a positive integer and \(p_0, p_1, \ldots, p_m \in \Pcal(\F)\) are 
    such that each \(p_k\) has degree \(k\). Prove that \(p_0, p_1, \ldots, p_m\) 
    is a basis of \(\Pcal_m (\F)\).
\end{problem}

\begin{proof}
It's easy to see that this list is linearly independent. Take any element in \(\Pcal_m (\F)\),
we can decompose that by degrees and get each component to be some multiple of \(p_i's\).
\end{proof}

\begin{problem}{10}
    Suppose \(m\) is a positive integer. For \(0 \leq k \leq m\), let 
    \[p_k(x) = x^k (1 - x)^{m - k}\]
    Show that \(p_0, \ldots, p_m\) is a basis of \(\Pcal_m(\F)\). 
\end{problem}

\begin{proof}
It suffices to prove that \(p_k(x)\) are linearly indepedent. We know from binomial theorem 
that 
\[(x+y)^m = \sum_{i=0}^{m} {m \choose i} x^i y^{m-i} \]

Applying this identity here we get that 
\[p_k(x) = x^k \sum_{i=0}^{m-k} {m-k \choose i} 1^{m-i} (-1)^{i} x^{i} = \sum_{i=0}^{m-k}
a_i x^{k + i}\]

where \(a_i = {m - k \choose i} (-1)^i\). For the polynomial to be identically 0 (\(\sum_{k=0}^{m}c_k p_k (x) = 0\)), for each 
power \(x^i\) we need the coefficient to be 0:

\[\sum_{k=0}^{\min(i, m)} c_k {m - k \choose i - k} (-1)^{i - k} = 0\]

We can prove the only solution for this is all \(c_k = 0\) by induction on \(m\). 

The base case is trivial. Assume the statement holds for \(k = m - 1\), then we try 
to prove for \(p_m(x)\), where we have \(P(x) - c_m p_m (x) = \sum_{k=0}^{m-1}c_kp_k(x)=0\). 
This means that \(c_m = 0\) and we've proved the linear independence. 

\end{proof}

\begin{problem}{11}
    Suppose \(\Ucal\) and \(\Wcal\) are both four-dimensional subspaces of \(\C^6\). Prove that 
    there exist two vectors in \(\Ucal \cap \Wcal\) such that neither of these vectors is a 
    scalar multiple of the other. 
\end{problem}

\begin{proof}
\(\dim (\Ucal \cap \Wcal) = \dim(\Ucal) + \dim (\Wcal) - \dim(\Ucal + \Wcal) \geq 8-6= 2\)

So there must exist two linearly independent vectors in the intersection. 
\end{proof}

\begin{problem}{12}
    Suppose that \(\Ucal\) and \(\Wcal\) are subspaces of \(\R^8\) such that 
    \(\dim \Ucal = 3, \dim \Wcal = 5\) and \(\Ucal + \Wcal = \R^8\). Prove that 
    \(\R^8 = \Ucal \oplus \Wcal\)
\end{problem}

\begin{proof}
Similar to problem 11, we can get tht \(\dim (\Ucal \cap \Wcal) = 0\). 
\end{proof}


\begin{problem}{14}
    Suppose \(V\) is a ten-dimensional vector space and \(V_1, V_2, V_3\) are subspaces 
    of \(V\) with \(\dim V_1 = \dim V_2 = \dim V_3 = 7\). Prove that \(V_1 \cap V_2 
    \cap V_3 \neq \{0\}\). 
\end{problem}

\begin{proof}
\[\dim ((V_1 \cap V_2) + V_3) = \dim (V_1 \cap V_2) + \dim V_3 - \dim (V_1 \cap V_2 \cap V_3) \]
and also that 
\[\dim (V_1 + V_2) = \dim V_1 + \dim V_2 - \dim(V_1 \cap V_2)\]

This gives that 

\[\dim (V_1 \cap V_2 \cap V_3) = \dim V_1 + \dim V_2 + \dim V_3 - \dim (V_1 + V_2) - 
\dim ((V_1 \cap V_2) + V_3)\]

Note that from question we know \(\dim V_1 + \dim V_2 + \dim V_3 = 21 > 20 = 2\dim V\)

Hence we know 

\[\dim (V_1 \cap V_2 \cap V_3) > (\dim(V) - \dim(V_1 + V_2)) + (\dim(V) - \dim(
    (V_1 \cap V_2) + V_3
)) > 0\]
We've thus proved the claim.

\end{proof}



\begin{problem}{16}
    Suppose \(V\) is finite-dimensional and \(\Ucal\) is a subspace of \(V\)
    with \(\Ucal \neq V\). Let \(n = \dim V\) and \(m = \dim \Ucal\). Prove that 
    there exist \(n - m\) subspaces of \(V\), each of dimension \(n -1\), whose 
    intersection equals \(\Ucal\). 
\end{problem}


\begin{proof}

    To show existence, we can start with the basis for \(\Ucal: u_1, \ldots, u_m\). 
    We extend the basis to \(V: \{ u_1, \ldots, u_m, v_1, \ldots, v_{n-m}\} \coloneqq K\). Construct 
    the subspace \(V_i = \text{span}\{K \backslash \{v_i\}\}\). Then we know that 
    \(\bigcap_i V_i = \text{span}\{\{u_1, \ldots, u_m\}\}\). 

\end{proof}

\begin{problem}{17}
    Suppose that \(V_1, \ldots, V_m\) are finite-dimensional subspaces of \(V\). Prove that 
    \(V_1 + \cdots + V_m\) is finite-dimensional and 
    \[\dim(V_1 + \cdots + V_m) \leq \dim V_1 + \cdots + \dim V_m\]
\end{problem}

\begin{proof}
We prove by induction on \(m\). The base case is trivial. Asume the statement hold for \(k\),
then for \(k+1\), we have that (Denote \(V_1 + \ldots + V_k = M_k\))

\begin{align*}
    \dim(M_k + V_{k+1}) \leq \dim(V_1) + \cdots +\dim (V_{k+1}) 
\end{align*}

which is finite.

\end{proof}

\begin{problem}{18}
    Suppose \(V\) is finite-dimensional with \(\dim V = n \geq 1\). Prove that there exist 
    one-dimensional subspaces \(V_1, \ldots, V_n\) of \(V\) such that 
    \[V = V_1 \oplus \cdots V_n\]
\end{problem}

\begin{proof}
We know there are basis \(\{v_1, \ldots, v_n\}\) for \(V\). Hence we can construct 
\[V_i = \text{span}\{v_i\}\]
\end{proof}


\begin{problem}{19}
    Prove or give a counter example:

    \begin{align*}
        \dim(V_1 + V_2 + V_3) = &\dim V_1 + \dim V_2 + \dim V_3\\ 
        &-\dim(V_1 \cap V_2) - \dim (V_1 \cap V_3) - \dim (V_2 \cap V_3) \\ 
        &+\dim(V_1 \cap V_2 \cap V_3)
    \end{align*}
\end{problem}

\begin{proof}
We know that 

\begin{align*}
&\dim ((V_1 + V_2) + V_3) = \dim(V_1 + V_2) + \dim(V_3) - \dim((V_1 + V_2)\cap V_3) \\ 
&= \dim (V_1) + \dim (V_2) + \dim (V_3) - \dim(V_1 \cap V_2) - \dim((V_1 + V_2)\cap V_3)
\end{align*}

Here we can get that 

\begin{align*}
    \dim((V_1 + V_2)\cap V_3) &= \dim((V_1 \cap V_3) + (V_2 \cap V_3)) \\ 
    &= \dim(V_1 \cap V_3) + \dim(V_2 \cap V_3) - \dim(V_1 \cap V_2 \cap V_3)
\end{align*}

Substitute this back we can get the desired solution. 


\end{proof}



\end{document}