\documentclass{extarticle}
\sloppy
%%%%%%%%%%%%%%%%%%%%%%%%%%%%%%%%%%%%%%%%%%%%%%%%%%%%%%%%%%%%%%%%%%%%%%
% PACKAGES            																						  %
%%%%%%%%%%%%%%%%%%%%%%%%%%%%%%%%%%%%%%%%%%%%%%%%%%%%%%%%%%%%%%%%%%%%%
\usepackage[10pt]{extsizes}
\usepackage{amsfonts}
% \usepackage{amsthm}
\usepackage{amssymb}
\usepackage[shortlabels]{enumitem}
\usepackage{microtype} 
\usepackage{amsmath}
\usepackage{mathtools}
\usepackage{commath}
\usepackage{amsthm}
\usepackage{bbm}
\usepackage[colorlinks=true, allcolors=blue]{hyperref}

%%%%%%%%%%%%%%%%%%%%%%%%%%%%%%%%%%%%%%%%%%%%%%%%%%%%%%%%%%%%%%%%%%%%%%
% PROBLEM ENVIRONMENT         																			           %
%%%%%%%%%%%%%%%%%%%%%%%%%%%%%%%%%%%%%%%%%%%%%%%%%%%%%%%%%%%%%%%%%%%%%
\usepackage{tcolorbox}
\tcbuselibrary{theorems, breakable, skins}
\newtcbtheorem{prob}% environment name
              {Problem}% Title text
  {enhanced, % tcolorbox styles
  attach boxed title to top left={xshift = 4mm, yshift=-2mm},
  colback=blue!5, colframe=black, colbacktitle=blue!3, coltitle=black,
  boxed title style={size=small,colframe=gray},
  fonttitle=\bfseries,
  separator sign none
  }%
  {} 
\newenvironment{problem}[1]{\begin{prob*}{#1}{}}{\end{prob*}}

\newtcbtheorem{exer}% environment name
              {Exercise}% Title text
  {enhanced, % tcolorbox styles
  attach boxed title to top left={xshift = 4mm, yshift=-2mm},
  colback=blue!5, colframe=black, colbacktitle=blue!3, coltitle=black,
  boxed title style={size=small,colframe=gray},
  fonttitle=\bfseries,
  separator sign none
  }%
  {} 
\newenvironment{exercise}[1]{\begin{exer*}{#1}{}}{\end{exer*}}

%%%%%%%%%%%%%%%%%%%%%%%%%%%%%%%%%%%%%%%%%%%%%%%%%%%%%%%%%%%%%%%%%%%%%%
% THEOREMS/LEMMAS/ETC.         																			  %
%%%%%%%%%%%%%%%%%%%%%%%%%%%%%%%%%%%%%%%%%%%%%%%%%%%%%%%%%%%%%%%%%%%%%%
\newtheorem{thm}{Theorem}
\newtheorem*{thm-non}{Theorem}
\newtheorem{lemma}[thm]{Lemma}
\newtheorem{definition}[thm]{Definition}
\newtheorem{remark}[thm]{Remark}
\newtheorem{corollary}[thm]{Corollary}
\newtheorem{proposition}[thm]{Proposition}
% % \newtheorem*{example*}[Theorem]{Example}
% \newtheorem*{example}[thm]{Example}

\theoremstyle{remark}
\newtheorem*{example}{Example}
% \newtheorem*{example*}[thm-non]{Example}

%%%%%%%%%%%%%%%%%%%%%%%%%%%%%%%%%%%%%%%%%%%%%%%%%%%%%%%%%%%%%%%%%%%%%%
% SECTION NUMBERING																				           %
%%%%%%%%%%%%%%%%%%%%%%%%%%%%%%%%%%%%%%%%%%%%%%%%%%%%%%%%%%%%%%%%%%%%%%
\renewcommand\thesection{\Alph{section}:}

\makeatletter
\newcommand{\cbigoplus}{\DOTSB\cbigoplus@\slimits@}
\newcommand{\cbigoplus@}{\mathop{\widehat{\bigoplus}}}
\makeatother

%%%%%%%%%%%%%%%%%%%%%%%%%%%%%%%%%%%%%%%%%%%%%%%%%%%%%%%%%%%%%%%%%%%%%%
% Additional math commands																		           %
%%%%%%%%%%%%%%%%%%%%%%%%%%%%%%%%%%%%%%%%%%%%%%%%%%%%%%%%%%%%%%%%%%%%%%
% \newcommand{\E}{\mathbb{E}}
% \newcommand{\Prb}{\mathbb{P}}
% \newcommand{\R}{\mathbb{R}}
% \newcommand{\C}{\mathbb{C}}
% \newcommand{\I}{\mathbb{I}}
% \newcommand{\F}{\mathbb{F}}
% \newcommand{\vecc}{\mathbf{c}}
% \newcommand{\Z}{\mathbb{Z}}
% \newcommand{\nul}{\text{null }}
% \newcommand{\range}{\text{range }}
% \newcommand{\Span}{\text{span}}
% \newcommand{\tr}{\text{tr}}
% \newcommand{\diag}{\text{diag}}
% \newcommand{\limn}{\lim_{n \to \infty}}
% \newcommand{\zo}{\overline{z}}
% \newcommand{\wo}{\overline{w}}
% \newcommand{\ao}{\overline{a}}
% \newcommand{\vo}{\overline{v}}
% \newcommand{\glo}{\overline{\lambda}}
% \newcommand{\Var}{\text{Var}}
% \newcommand{\1}{\mathbbm{1}}
% \newcommand{\Pm}{\mathbbm{P}}

% Mathbold abbrevation
\newcommand{\ab}{\mathbf{a}}
\newcommand{\bb}{\mathbf{b}}
\newcommand{\cb}{\mathbf{c}}
\newcommand{\db}{\mathbf{d}}
\newcommand{\eb}{\mathbf{e}}
\newcommand{\fb}{\mathbf{f}}
\newcommand{\gb}{\mathbf{g}}
\newcommand{\hb}{\mathbf{h}}
\newcommand{\ib}{\mathbf{i}}
\newcommand{\jb}{\mathbf{j}}
\newcommand{\kb}{\mathbf{k}}
\newcommand{\lb}{\mathbf{l}}
\newcommand{\mb}{\mathbf{m}}
\newcommand{\nb}{\mathbf{n}}
\newcommand{\ob}{\mathbf{o}}
\newcommand{\pb}{\mathbf{p}}
\newcommand{\qb}{\mathbf{q}}
\newcommand{\rb}{\mathbf{r}}
% \newcommand{\sb}{\mathbf{s}} # already defined
\newcommand{\tb}{\mathbf{t}}
\newcommand{\ub}{\mathbf{u}}
\newcommand{\vb}{\mathbf{v}}
\newcommand{\wb}{\mathbf{w}}
\newcommand{\xb}{\mathbf{x}}
\newcommand{\yb}{\mathbf{y}}
\newcommand{\zb}{\mathbf{z}}
\newcommand{\Ab}{\mathbf{A}}
\newcommand{\Bb}{\mathbf{B}}
\newcommand{\Cb}{\mathbf{C}}
\newcommand{\Db}{\mathbf{D}}
\newcommand{\Eb}{\mathbf{E}}
\newcommand{\Fb}{\mathbf{F}}
\newcommand{\Gb}{\mathbf{G}}
\newcommand{\Hb}{\mathbf{H}}
\newcommand{\Ib}{\mathbf{I}}
\newcommand{\Jb}{\mathbf{J}}
\newcommand{\Kb}{\mathbf{K}}
\newcommand{\Lb}{\mathbf{L}}
\newcommand{\Mb}{\mathbf{M}}
\newcommand{\Nb}{\mathbf{N}}
\newcommand{\Ob}{\mathbf{O}}
\newcommand{\Pb}{\mathbf{P}}
\newcommand{\Qb}{\mathbf{Q}}
\newcommand{\Rb}{\mathbf{R}}
\newcommand{\Sb}{\mathbf{S}}
\newcommand{\Tb}{\mathbf{T}}
\newcommand{\Ub}{\mathbf{U}}
\newcommand{\Vb}{\mathbf{V}}
\newcommand{\Wb}{\mathbf{W}}
\newcommand{\Xb}{\mathbf{X}}
\newcommand{\Yb}{\mathbf{Y}}
\newcommand{\Zb}{\mathbf{Z}}


% MathCal abbrevation
\newcommand{\acal}{\mathcal{a}}
\newcommand{\bcal}{\mathcal{b}}
\newcommand{\ccal}{\mathcal{c}}
\newcommand{\dcal}{\mathcal{d}}
\newcommand{\ecal}{\mathcal{e}}
\newcommand{\fcal}{\mathcal{f}}
\newcommand{\gcal}{\mathcal{g}}
\newcommand{\hcal}{\mathcal{h}}
\newcommand{\ical}{\mathcal{i}}
\newcommand{\jcal}{\mathcal{j}}
\newcommand{\kcal}{\mathcal{k}}
\newcommand{\lcal}{\mathcal{l}}
\newcommand{\mcal}{\mathcal{m}}
\newcommand{\ncal}{\mathcal{n}}
\newcommand{\ocal}{\mathcal{o}}
\newcommand{\pcal}{\mathcal{p}}
\newcommand{\qcal}{\mathcal{q}}
\newcommand{\rcal}{\mathcal{r}}
\newcommand{\scal}{\mathcal{s}}
\newcommand{\tcal}{\mathcal{t}}
\newcommand{\ucal}{\mathcal{u}}
\newcommand{\vcal}{\mathcal{v}}
\newcommand{\wcal}{\mathcal{w}}
\newcommand{\xcal}{\mathcal{x}}
\newcommand{\ycal}{\mathcal{y}}
\newcommand{\zcal}{\mathcal{z}}
\newcommand{\Acal}{\mathcal{A}}
\newcommand{\Bcal}{\mathcal{B}}
\newcommand{\Ccal}{\mathcal{C}}
\newcommand{\Dcal}{\mathcal{D}}
\newcommand{\Ecal}{\mathcal{E}}
\newcommand{\Fcal}{\mathcal{F}}
\newcommand{\Gcal}{\mathcal{G}}
\newcommand{\Hcal}{\mathcal{H}}
\newcommand{\Ical}{\mathcal{I}}
\newcommand{\Jcal}{\mathcal{J}}
\newcommand{\Kcal}{\mathcal{K}}
\newcommand{\Lcal}{\mathcal{L}}
\newcommand{\Mcal}{\mathcal{M}}
\newcommand{\Ncal}{\mathcal{N}}
\newcommand{\Ocal}{\mathcal{O}}
\newcommand{\Pcal}{\mathcal{P}}
\newcommand{\Qcal}{\mathcal{Q}}
\newcommand{\Rcal}{\mathcal{R}}
\newcommand{\Scal}{\mathcal{S}}
\newcommand{\Tcal}{\mathcal{T}}
\newcommand{\Ucal}{\mathcal{U}}
\newcommand{\Vcal}{\mathcal{V}}
\newcommand{\Wcal}{\mathcal{W}}
\newcommand{\Xcal}{\mathcal{X}}
\newcommand{\Ycal}{\mathcal{Y}}
\newcommand{\Zcal}{\mathcal{Z}}


% MathCal abbrevation
\newcommand{\ac}{\mathcal{a}}
\newcommand{\bc}{\mathcal{b}}
% \newcommand{\cc}{\mathcal{c}}
\newcommand{\dc}{\mathcal{d}}
\newcommand{\ec}{\mathcal{e}}
\newcommand{\fc}{\mathcal{f}}
\newcommand{\gc}{\mathcal{g}}
\newcommand{\hc}{\mathcal{h}}
\newcommand{\ic}{\mathcal{i}}
\newcommand{\jc}{\mathcal{j}}
\newcommand{\kc}{\mathcal{k}}
\newcommand{\lc}{\mathcal{l}}
\newcommand{\mc}{\mathcal{m}}
\newcommand{\nc}{\mathcal{n}}
\newcommand{\oc}{\mathcal{o}}
\newcommand{\pc}{\mathcal{p}}
\newcommand{\qc}{\mathcal{q}}
\newcommand{\rc}{\mathcal{r}}
% \newcommand{\sc}{\mathcal{s}}
\newcommand{\tc}{\mathcal{t}}
\newcommand{\uc}{\mathcal{u}}
\newcommand{\vc}{\mathcal{v}}
\newcommand{\wc}{\mathcal{w}}
\newcommand{\xc}{\mathcal{x}}
\newcommand{\yc}{\mathcal{y}}
\newcommand{\zc}{\mathcal{z}}
\newcommand{\Ac}{\mathcal{A}}
\newcommand{\Bc}{\mathcal{B}}
\newcommand{\Cc}{\mathcal{C}}
\newcommand{\Dc}{\mathcal{D}}
\newcommand{\Ec}{\mathcal{E}}
\newcommand{\Fc}{\mathcal{F}}
\newcommand{\Gc}{\mathcal{G}}
\newcommand{\Hc}{\mathcal{H}}
\newcommand{\Ic}{\mathcal{I}}
\newcommand{\Jc}{\mathcal{J}}
\newcommand{\Kc}{\mathcal{K}}
\newcommand{\Lc}{\mathcal{L}}
\newcommand{\Mc}{\mathcal{M}}
\newcommand{\Nc}{\mathcal{N}}
\newcommand{\Oc}{\mathcal{O}}
\newcommand{\Pc}{\mathcal{P}}
\newcommand{\Qc}{\mathcal{Q}}
\newcommand{\Rc}{\mathcal{R}}
\newcommand{\Sc}{\mathcal{S}}
\newcommand{\Tc}{\mathcal{T}}
\newcommand{\Uc}{\mathcal{U}}
\newcommand{\Vc}{\mathcal{V}}
\newcommand{\Wc}{\mathcal{W}}
\newcommand{\Xc}{\mathcal{X}}
\newcommand{\Yc}{\mathcal{Y}}
\newcommand{\Zc}{\mathcal{Z}}

% Mathhat abbrevation
\newcommand{\ah}{\hat{a}}
\newcommand{\bh}{\hat{b}}
\newcommand{\ch}{\hat{c}}
% \newcommand{\dh}{\hat{d}}
\newcommand{\eh}{\hat{e}}
\newcommand{\fh}{\hat{f}}
\newcommand{\gh}{\hat{g}}
\newcommand{\hh}{\hat{h}}
\newcommand{\ih}{\hat{i}}
\newcommand{\jh}{\hat{j}}
\newcommand{\kh}{\hat{k}}
\newcommand{\lh}{\hat{l}}
\newcommand{\mh}{\hat{m}}
\newcommand{\nh}{\hat{n}}
\newcommand{\oh}{\hat{o}}
\newcommand{\ph}{\hat{p}}
\newcommand{\qh}{\hat{q}}
\newcommand{\rh}{\hat{r}}
\newcommand{\sh}{\hat{s}}
% \newcommand{\th}{\hat{t}}
\newcommand{\uh}{\hat{u}}
\newcommand{\vh}{\hat{v}}
\newcommand{\wh}{\hat{w}}
\newcommand{\xh}{\hat{x}}
\newcommand{\yh}{\hat{y}}
\newcommand{\zh}{\hat{z}}
\newcommand{\Ah}{\hat{A}}
\newcommand{\Bh}{\hat{B}}
\newcommand{\Ch}{\hat{C}}
\newcommand{\Dh}{\hat{D}}
\newcommand{\Eh}{\hat{E}}
\newcommand{\Fh}{\hat{F}}
\newcommand{\Gh}{\hat{G}}
\newcommand{\Hh}{\hat{H}}
\newcommand{\Ih}{\hat{I}}
\newcommand{\Jh}{\hat{J}}
\newcommand{\Kh}{\hat{K}}
\newcommand{\Lh}{\hat{L}}
\newcommand{\Mh}{\hat{M}}
\newcommand{\Nh}{\hat{N}}
\newcommand{\Oh}{\hat{O}}
\newcommand{\Ph}{\hat{P}}
\newcommand{\Qh}{\hat{Q}}
\newcommand{\Rh}{\hat{R}}
\newcommand{\Sh}{\hat{S}}
\newcommand{\Th}{\hat{T}}
\newcommand{\Uh}{\hat{U}}
\newcommand{\Vh}{\hat{V}}
\newcommand{\Wh}{\hat{W}}
\newcommand{\Xh}{\hat{X}}
\newcommand{\Yh}{\hat{Y}}
\newcommand{\Zh}{\hat{Z}}


% Mathbar abbrevation
\newcommand{\abar}{\bar{a}}
\newcommand{\bbar}{\bar{b}}
\newcommand{\cbar}{\bar{c}}
\newcommand{\dbar}{\bar{d}}
\newcommand{\ebar}{\bar{e}}
\newcommand{\fbar}{\bar{f}}
\newcommand{\gbar}{\bar{g}}
% \newcommand{\hbar}{\bar{h}}
\newcommand{\ibar}{\bar{i}}
\newcommand{\jbar}{\bar{j}}
\newcommand{\kbar}{\bar{k}}
\newcommand{\lbar}{\bar{l}}
\newcommand{\mbar}{\bar{m}}
\newcommand{\nbar}{\bar{n}}
\newcommand{\obar}{\bar{o}}
\newcommand{\pbar}{\bar{p}}
\newcommand{\qbar}{\bar{q}}
\newcommand{\rbar}{\bar{r}}
\newcommand{\sbar}{\bar{s}}
\newcommand{\tbar}{\bar{t}}
\newcommand{\ubar}{\bar{u}}
\newcommand{\vbar}{\bar{v}}
\newcommand{\wbar}{\bar{w}}
\newcommand{\xbar}{\bar{x}}
\newcommand{\ybar}{\bar{y}}
\newcommand{\zbar}{\bar{z}}
\newcommand{\Abar}{\bar{A}}
\newcommand{\Bbar}{\bar{B}}
\newcommand{\Cbar}{\bar{C}}
\newcommand{\Dbar}{\bar{D}}
\newcommand{\Ebar}{\bar{E}}
\newcommand{\Fbar}{\bar{F}}
\newcommand{\Gbar}{\bar{G}}
\newcommand{\Hbar}{\bar{H}}
\newcommand{\Ibar}{\bar{I}}
\newcommand{\Jbar}{\bar{J}}
\newcommand{\Kbar}{\bar{K}}
\newcommand{\Lbar}{\bar{L}}
\newcommand{\Mbar}{\bar{M}}
\newcommand{\Nbar}{\bar{N}}
\newcommand{\Obar}{\bar{O}}
\newcommand{\Pbar}{\bar{P}}
\newcommand{\Qbar}{\bar{Q}}
\newcommand{\Rbar}{\bar{R}}
\newcommand{\Sbar}{\bar{S}}
\newcommand{\Tbar}{\bar{T}}
\newcommand{\Ubar}{\bar{U}}
\newcommand{\Vbar}{\bar{V}}
\newcommand{\Wbar}{\bar{W}}
\newcommand{\Xbar}{\bar{X}}
\newcommand{\Ybar}{\bar{Y}}
\newcommand{\Zbar}{\bar{Z}}


% Mathtilde abbrevation
\newcommand{\at}{\tilde{a}}
\newcommand{\bt}{\tilde{b}}
\newcommand{\ct}{\tilde{c}}
\newcommand{\dt}{\tilde{d}}
\newcommand{\et}{\tilde{e}}
\newcommand{\ft}{\tilde{f}}
\newcommand{\gt}{\tilde{g}}
% \newcommand{\ht}{\tilde{h}}
% \newcommand{\it}{\tilde{i}}
\newcommand{\jt}{\tilde{j}}
\newcommand{\kt}{\tilde{k}}
\newcommand{\lt}{\tilde{l}}
\newcommand{\mt}{\tilde{m}}
\newcommand{\nt}{\tilde{n}}
\newcommand{\ot}{\tilde{o}}
\newcommand{\pt}{\tilde{p}}
\newcommand{\qt}{\tilde{q}}
\newcommand{\rt}{\tilde{r}}
\newcommand{\st}{\tilde{s}}
% \newcommand{\tt}{\tilde{t}}
\newcommand{\ut}{\tilde{u}}
\newcommand{\vt}{\tilde{v}}
\newcommand{\wt}{\tilde{w}}
\newcommand{\xt}{\tilde{x}}
\newcommand{\yt}{\tilde{y}}
\newcommand{\zt}{\tilde{z}}
\newcommand{\At}{\tilde{A}}
\newcommand{\Bt}{\tilde{B}}
\newcommand{\Ct}{\tilde{C}}
\newcommand{\Dt}{\tilde{D}}
\newcommand{\Et}{\tilde{E}}
\newcommand{\Ft}{\tilde{F}}
\newcommand{\Gt}{\tilde{G}}
\newcommand{\Ht}{\tilde{H}}
\newcommand{\It}{\tilde{I}}
\newcommand{\Jt}{\tilde{J}}
\newcommand{\Kt}{\tilde{K}}
\newcommand{\Lt}{\tilde{L}}
\newcommand{\Mt}{\tilde{M}}
\newcommand{\Nt}{\tilde{N}}
\newcommand{\Ot}{\tilde{O}}
\newcommand{\Pt}{\tilde{P}}
\newcommand{\Qt}{\tilde{Q}}
\newcommand{\Rt}{\tilde{R}}
\newcommand{\St}{\tilde{S}}
\newcommand{\Tt}{\tilde{T}}
\newcommand{\Ut}{\tilde{U}}
\newcommand{\Vt}{\tilde{V}}
\newcommand{\Wt}{\tilde{W}}
\newcommand{\Xt}{\tilde{X}}
\newcommand{\Yt}{\tilde{Y}}
\newcommand{\Zt}{\tilde{Z}}


% MathTilde abbrevation
\newcommand{\aT}{\Tilde{a}}
\newcommand{\bT}{\Tilde{b}}
\newcommand{\cT}{\Tilde{c}}
\newcommand{\dT}{\Tilde{d}}
\newcommand{\eT}{\Tilde{e}}
\newcommand{\fT}{\Tilde{f}}
\newcommand{\gT}{\Tilde{g}}
\newcommand{\hT}{\Tilde{h}}
\newcommand{\iT}{\Tilde{i}}
\newcommand{\jT}{\Tilde{j}}
\newcommand{\kT}{\Tilde{k}}
\newcommand{\lT}{\Tilde{l}}
\newcommand{\mT}{\Tilde{m}}
\newcommand{\nT}{\Tilde{n}}
\newcommand{\oT}{\Tilde{o}}
\newcommand{\pT}{\Tilde{p}}
\newcommand{\qT}{\Tilde{q}}
\newcommand{\rT}{\Tilde{r}}
\newcommand{\sT}{\Tilde{s}}
\newcommand{\tT}{\Tilde{t}}
\newcommand{\uT}{\Tilde{u}}
\newcommand{\vT}{\Tilde{v}}
\newcommand{\wT}{\Tilde{w}}
\newcommand{\xT}{\Tilde{x}}
\newcommand{\yT}{\Tilde{y}}
\newcommand{\zT}{\Tilde{z}}
\newcommand{\AT}{\Tilde{A}}
\newcommand{\BT}{\Tilde{B}}
\newcommand{\CT}{\Tilde{C}}
\newcommand{\DT}{\Tilde{D}}
\newcommand{\ET}{\Tilde{E}}
\newcommand{\FT}{\Tilde{F}}
\newcommand{\GT}{\Tilde{G}}
\newcommand{\HT}{\Tilde{H}}
\newcommand{\IT}{\Tilde{I}}
\newcommand{\JT}{\Tilde{J}}
% \newcommand{\KT}{\Tilde{K}}
\newcommand{\LT}{\Tilde{L}}
\newcommand{\MT}{\Tilde{M}}
% \newcommand{\NT}{\Tilde{N}}
\newcommand{\OT}{\Tilde{O}}
% \newcommand{\PT}{\Tilde{P}}
\newcommand{\QT}{\Tilde{Q}}
\newcommand{\RT}{\Tilde{R}}
\newcommand{\ST}{\Tilde{S}}
\newcommand{\TT}{\Tilde{T}}
\newcommand{\UT}{\Tilde{U}}
\newcommand{\VT}{\Tilde{V}}
\newcommand{\WT}{\Tilde{W}}
\newcommand{\XT}{\Tilde{X}}
\newcommand{\YT}{\Tilde{Y}}
\newcommand{\ZT}{\Tilde{Z}}



%Nested Case




% Mathbold + hat abbrevation
\newcommand{\abh}{\mathbf{\hat{a}}}
\newcommand{\bbh}{\mathbf{\hat{b}}}
\newcommand{\cbh}{\mathbf{\hat{c}}}
\newcommand{\dbh}{\mathbf{\hat{d}}}
\newcommand{\ebh}{\mathbf{\hat{e}}}
\newcommand{\fbh}{\mathbf{\hat{f}}}
\newcommand{\gbh}{\mathbf{\hat{g}}}
\newcommand{\hbh}{\mathbf{\hat{h}}}
\newcommand{\ibh}{\mathbf{\hat{i}}}
\newcommand{\jbh}{\mathbf{\hat{j}}}
\newcommand{\kbh}{\mathbf{\hat{k}}}
\newcommand{\lbh}{\mathbf{\hat{l}}}
\newcommand{\mbh}{\mathbf{\hat{m}}}
\newcommand{\nbh}{\mathbf{\hat{n}}}
\newcommand{\obh}{\mathbf{\hat{o}}}
\newcommand{\pbh}{\mathbf{\hat{p}}}
\newcommand{\qbh}{\mathbf{\hat{q}}}
\newcommand{\rbh}{\mathbf{\hat{r}}}
\newcommand{\sbh}{\mathbf{\hat{s}}}
\newcommand{\tbh}{\mathbf{\hat{t}}}
\newcommand{\ubh}{\mathbf{\hat{u}}}
\newcommand{\vbh}{\mathbf{\hat{v}}}
\newcommand{\wbh}{\mathbf{\hat{w}}}
\newcommand{\xbh}{\mathbf{\hat{x}}}
\newcommand{\ybh}{\mathbf{\hat{y}}}
\newcommand{\zbh}{\mathbf{\hat{z}}}
\newcommand{\Abh}{\mathbf{\hat{A}}}
\newcommand{\Bbh}{\mathbf{\hat{B}}}
\newcommand{\Cbh}{\mathbf{\hat{C}}}
\newcommand{\Dbh}{\mathbf{\hat{D}}}
\newcommand{\Ebh}{\mathbf{\hat{E}}}
\newcommand{\Fbh}{\mathbf{\hat{F}}}
\newcommand{\Gbh}{\mathbf{\hat{G}}}
\newcommand{\Hbh}{\mathbf{\hat{H}}}
\newcommand{\Ibh}{\mathbf{\hat{I}}}
\newcommand{\Jbh}{\mathbf{\hat{J}}}
\newcommand{\Kbh}{\mathbf{\hat{K}}}
\newcommand{\Lbh}{\mathbf{\hat{L}}}
\newcommand{\Mbh}{\mathbf{\hat{M}}}
\newcommand{\Nbh}{\mathbf{\hat{N}}}
\newcommand{\Obh}{\mathbf{\hat{O}}}
\newcommand{\Pbh}{\mathbf{\hat{P}}}
\newcommand{\Qbh}{\mathbf{\hat{Q}}}
\newcommand{\Rbh}{\mathbf{\hat{R}}}
\newcommand{\Sbh}{\mathbf{\hat{S}}}
\newcommand{\Tbh}{\mathbf{\hat{T}}}
\newcommand{\Ubh}{\mathbf{\hat{U}}}
\newcommand{\Vbh}{\mathbf{\hat{V}}}
\newcommand{\Wbh}{\mathbf{\hat{W}}}
\newcommand{\Xbh}{\mathbf{\hat{X}}}
\newcommand{\Ybh}{\mathbf{\hat{Y}}}
\newcommand{\Zbh}{\mathbf{\hat{Z}}}


% Mathbold + Tilde abbrevation
\newcommand{\abT}{\mathbf{\Tilde{a}}}
\newcommand{\bbT}{\mathbf{\Tilde{b}}}
\newcommand{\cbT}{\mathbf{\Tilde{c}}}
\newcommand{\dbT}{\mathbf{\Tilde{d}}}
\newcommand{\ebT}{\mathbf{\Tilde{e}}}
\newcommand{\fbT}{\mathbf{\Tilde{f}}}
\newcommand{\gbT}{\mathbf{\Tilde{g}}}
\newcommand{\hbT}{\mathbf{\Tilde{h}}}
\newcommand{\ibT}{\mathbf{\Tilde{i}}}
\newcommand{\jbT}{\mathbf{\Tilde{j}}}
\newcommand{\kbT}{\mathbf{\Tilde{k}}}
\newcommand{\lbT}{\mathbf{\Tilde{l}}}
\newcommand{\mbT}{\mathbf{\Tilde{m}}}
\newcommand{\nbT}{\mathbf{\Tilde{n}}}
\newcommand{\obT}{\mathbf{\Tilde{o}}}
\newcommand{\pbT}{\mathbf{\Tilde{p}}}
\newcommand{\qbT}{\mathbf{\Tilde{q}}}
\newcommand{\rbT}{\mathbf{\Tilde{r}}}
\newcommand{\sbT}{\mathbf{\Tilde{s}}}
\newcommand{\tbT}{\mathbf{\Tilde{t}}}
\newcommand{\ubT}{\mathbf{\Tilde{u}}}
\newcommand{\vbT}{\mathbf{\Tilde{v}}}
\newcommand{\wbT}{\mathbf{\Tilde{w}}}
\newcommand{\xbT}{\mathbf{\Tilde{x}}}
\newcommand{\ybT}{\mathbf{\Tilde{y}}}
\newcommand{\zbT}{\mathbf{\Tilde{z}}}
\newcommand{\AbT}{\mathbf{\Tilde{A}}}
\newcommand{\BbT}{\mathbf{\Tilde{B}}}
\newcommand{\CbT}{\mathbf{\Tilde{C}}}
\newcommand{\DbT}{\mathbf{\Tilde{D}}}
\newcommand{\EbT}{\mathbf{\Tilde{E}}}
\newcommand{\FbT}{\mathbf{\Tilde{F}}}
\newcommand{\GbT}{\mathbf{\Tilde{G}}}
\newcommand{\HbT}{\mathbf{\Tilde{H}}}
\newcommand{\IbT}{\mathbf{\Tilde{I}}}
\newcommand{\JbT}{\mathbf{\Tilde{J}}}
\newcommand{\KbT}{\mathbf{\Tilde{K}}}
\newcommand{\LbT}{\mathbf{\Tilde{L}}}
\newcommand{\MbT}{\mathbf{\Tilde{M}}}
\newcommand{\NbT}{\mathbf{\Tilde{N}}}
\newcommand{\ObT}{\mathbf{\Tilde{O}}}
\newcommand{\PbT}{\mathbf{\Tilde{P}}}
\newcommand{\QbT}{\mathbf{\Tilde{Q}}}
\newcommand{\RbT}{\mathbf{\Tilde{R}}}
\newcommand{\SbT}{\mathbf{\Tilde{S}}}
\newcommand{\TbT}{\mathbf{\Tilde{T}}}
\newcommand{\UbT}{\mathbf{\Tilde{U}}}
\newcommand{\VbT}{\mathbf{\Tilde{V}}}
\newcommand{\WbT}{\mathbf{\Tilde{W}}}
\newcommand{\XbT}{\mathbf{\Tilde{X}}}
\newcommand{\YbT}{\mathbf{\Tilde{Y}}}
\newcommand{\ZbT}{\mathbf{\Tilde{Z}}}


% Mathbold + tilde abbrevation
\newcommand{\abt}{\mathbf{\tilde{a}}}
\newcommand{\bbt}{\mathbf{\tilde{b}}}
\newcommand{\cbt}{\mathbf{\tilde{c}}}
\newcommand{\dbt}{\mathbf{\tilde{d}}}
\newcommand{\ebt}{\mathbf{\tilde{e}}}
\newcommand{\fbt}{\mathbf{\tilde{f}}}
\newcommand{\gbt}{\mathbf{\tilde{g}}}
\newcommand{\hbt}{\mathbf{\tilde{h}}}
\newcommand{\ibt}{\mathbf{\tilde{i}}}
\newcommand{\jbt}{\mathbf{\tilde{j}}}
\newcommand{\kbt}{\mathbf{\tilde{k}}}
\newcommand{\lbt}{\mathbf{\tilde{l}}}
\newcommand{\mbt}{\mathbf{\tilde{m}}}
\newcommand{\nbt}{\mathbf{\tilde{n}}}
\newcommand{\obt}{\mathbf{\tilde{o}}}
\newcommand{\pbt}{\mathbf{\tilde{p}}}
\newcommand{\qbt}{\mathbf{\tilde{q}}}
\newcommand{\rbt}{\mathbf{\tilde{r}}}
\newcommand{\sbt}{\mathbf{\tilde{s}}}
\newcommand{\tbt}{\mathbf{\tilde{t}}}
\newcommand{\ubt}{\mathbf{\tilde{u}}}
\newcommand{\vbt}{\mathbf{\tilde{v}}}
\newcommand{\wbt}{\mathbf{\tilde{w}}}
\newcommand{\xbt}{\mathbf{\tilde{x}}}
\newcommand{\ybt}{\mathbf{\tilde{y}}}
\newcommand{\zbt}{\mathbf{\tilde{z}}}
\newcommand{\Abt}{\mathbf{\tilde{A}}}
\newcommand{\Bbt}{\mathbf{\tilde{B}}}
\newcommand{\Cbt}{\mathbf{\tilde{C}}}
\newcommand{\Dbt}{\mathbf{\tilde{D}}}
\newcommand{\Ebt}{\mathbf{\tilde{E}}}
\newcommand{\Fbt}{\mathbf{\tilde{F}}}
\newcommand{\Gbt}{\mathbf{\tilde{G}}}
\newcommand{\Hbt}{\mathbf{\tilde{H}}}
\newcommand{\Ibt}{\mathbf{\tilde{I}}}
\newcommand{\Jbt}{\mathbf{\tilde{J}}}
\newcommand{\Kbt}{\mathbf{\tilde{K}}}
\newcommand{\Lbt}{\mathbf{\tilde{L}}}
\newcommand{\Mbt}{\mathbf{\tilde{M}}}
\newcommand{\Nbt}{\mathbf{\tilde{N}}}
\newcommand{\Obt}{\mathbf{\tilde{O}}}
\newcommand{\Pbt}{\mathbf{\tilde{P}}}
\newcommand{\Qbt}{\mathbf{\tilde{Q}}}
\newcommand{\Rbt}{\mathbf{\tilde{R}}}
\newcommand{\Sbt}{\mathbf{\tilde{S}}}
\newcommand{\Tbt}{\mathbf{\tilde{T}}}
\newcommand{\Ubt}{\mathbf{\tilde{U}}}
\newcommand{\Vbt}{\mathbf{\tilde{V}}}
\newcommand{\Wbt}{\mathbf{\tilde{W}}}
\newcommand{\Xbt}{\mathbf{\tilde{X}}}
\newcommand{\Ybt}{\mathbf{\tilde{Y}}}
\newcommand{\Zbt}{\mathbf{\tilde{Z}}}


% Mathbold + bar abbrevation
\newcommand{\abb}{\mathbf{\bar{a}}}
\newcommand{\bbb}{\mathbf{\bar{b}}}
\newcommand{\cbb}{\mathbf{\bar{c}}}
\newcommand{\dbb}{\mathbf{\bar{d}}}
\newcommand{\ebb}{\mathbf{\bar{e}}}
\newcommand{\fbb}{\mathbf{\bar{f}}}
\newcommand{\gbb}{\mathbf{\bar{g}}}
\newcommand{\hbb}{\mathbf{\bar{h}}}
\newcommand{\ibb}{\mathbf{\bar{i}}}
\newcommand{\jbb}{\mathbf{\bar{j}}}
\newcommand{\kbb}{\mathbf{\bar{k}}}
\newcommand{\lbb}{\mathbf{\bar{l}}}
\newcommand{\mbb}{\mathbf{\bar{m}}}
\newcommand{\nbb}{\mathbf{\bar{n}}}
\newcommand{\obb}{\mathbf{\bar{o}}}
\newcommand{\pbb}{\mathbf{\bar{p}}}
\newcommand{\qbb}{\mathbf{\bar{q}}}
\newcommand{\rbb}{\mathbf{\bar{r}}}
\newcommand{\sbb}{\mathbf{\bar{s}}}
\newcommand{\tbb}{\mathbf{\bar{t}}}
\newcommand{\ubb}{\mathbf{\bar{u}}}
\newcommand{\vbb}{\mathbf{\bar{v}}}
\newcommand{\wbb}{\mathbf{\bar{w}}}
\newcommand{\xbb}{\mathbf{\bar{x}}}
\newcommand{\ybb}{\mathbf{\bar{y}}}
\newcommand{\zbb}{\mathbf{\bar{z}}}
\newcommand{\Abb}{\mathbf{\bar{A}}}
% \newcommand{\Bbb}{\mathbf{\bar{B}}}
\newcommand{\Cbb}{\mathbf{\bar{C}}}
\newcommand{\Dbb}{\mathbf{\bar{D}}}
\newcommand{\Ebb}{\mathbf{\bar{E}}}
\newcommand{\Fbb}{\mathbf{\bar{F}}}
\newcommand{\Gbb}{\mathbf{\bar{G}}}
\newcommand{\Hbb}{\mathbf{\bar{H}}}
\newcommand{\Ibb}{\mathbf{\bar{I}}}
\newcommand{\Jbb}{\mathbf{\bar{J}}}
\newcommand{\Kbb}{\mathbf{\bar{K}}}
\newcommand{\Lbb}{\mathbf{\bar{L}}}
\newcommand{\Mbb}{\mathbf{\bar{M}}}
\newcommand{\Nbb}{\mathbf{\bar{N}}}
\newcommand{\Obb}{\mathbf{\bar{O}}}
\newcommand{\Pbb}{\mathbf{\bar{P}}}
\newcommand{\Qbb}{\mathbf{\bar{Q}}}
\newcommand{\Rbb}{\mathbf{\bar{R}}}
\newcommand{\Sbb}{\mathbf{\bar{S}}}
\newcommand{\Tbb}{\mathbf{\bar{T}}}
\newcommand{\Ubb}{\mathbf{\bar{U}}}
\newcommand{\Vbb}{\mathbf{\bar{V}}}
\newcommand{\Wbb}{\mathbf{\bar{W}}}
\newcommand{\Xbb}{\mathbf{\bar{X}}}
\newcommand{\Ybb}{\mathbf{\bar{Y}}}
\newcommand{\Zbb}{\mathbf{\bar{Z}}}


\newcommand{\E}{\mathbb{E}}
\newcommand{\Prb}{\mathbb{P}}
\newcommand{\R}{\mathbb{R}}
\newcommand{\C}{\mathbb{C}}
\newcommand{\I}{\mathbb{I}}
\newcommand{\F}{\mathbb{F}}
\newcommand{\vecc}{\mathbf{c}}
\newcommand{\Z}{\mathbb{Z}}
\newcommand{\nul}{\text{null }}
\newcommand{\range}{\text{range }}
\newcommand{\Span}{\text{span}}
\newcommand{\tr}{\text{tr}}
\newcommand{\diag}{\text{diag}}
\newcommand{\limn}{\lim_{n \to \infty}}
\newcommand{\zo}{\overline{z}}
\newcommand{\wo}{\overline{w}}
\newcommand{\ao}{\overline{a}}
\newcommand{\vo}{\overline{v}}
\newcommand{\glo}{\overline{\lambda}}
\newcommand{\Var}{\text{Var}}
\newcommand{\1}{\mathbbm{1}}
\newcommand{\Pm}{\mathbbm{P}}
\usepackage{hyperref}


\title{\vspace{-2em}Chapter 8: Operators on Complex Vector Spaces}
\author{\emph{Linear Algebra Done Right}, by Sheldon Axler}
\date{}

\begin{document}
\maketitle 
\tableofcontents

\newpage 


%%%%%%%%%%%%%%%%%%%%%%%%%%%%%%%%%%%%%%%%%%%%%%%%%%%%%%%%%%%%
%%%%%%%%%%%%%%%%%%%%%%%%%%%% 8A %%%%%%%%%%%%%%%%%%%%%%%%%%%%
%%%%%%%%%%%%%%%%%%%%%%%%%%%%%%%%%%%%%%%%%%%%%%%%%%%%%%%%%%%%


\newpage 
\section*{8A: Generalized Eigenvectors and Nilpotent Operators}
\addcontentsline{toc}{section}{8A: Generalized Eigenvectors and Nilpotent Operators}


\begin{lemma}[sequence of increasing null spaces]
    Suppose \(T \in \Lc(V)\). Then 
    \[\{0\} = \nul T^0 \subseteq \nul T^1 
    \subseteq \nul T^2 \cdots \subseteq \nul T^k 
    \subseteq \nul T^{k+1} \cdots\]
\end{lemma}

\begin{lemma}[equality in the sequence of null spaces]
    Suppose \(T \in \Lc(V)\) and \(m\) is a nonnegative 
    integer such that 
    \[\nul T^m = \nul T^{m+1}\]
    Then 
    \[\nul T^{m} = \nul T^{m+1} = \nul T^{m+2} = \cdots\]
\end{lemma}

\begin{lemma}[null space stop growing]
    Suppose \(T \in \Lc(V)\). Then 
    \[\nul T^{\dim V} = \nul T^{\dim V + 1} = \nul T^{\dim V + 2} = \cdots\]
\end{lemma}

\begin{thm}[\(V\) is the direct sum of \(\nul T^{\dim V}\) and \(\range T^{\dim V}\)]
    Suppose \(T \in \Lc(V)\). Then 
    \[V = \nul T^{\dim V} \oplus \range T^{\dim V}\]
\end{thm}

\begin{definition}[generalized eigenvector]
    Suppose \(T \in \Lc(V)\) and \(\lambda\) is an eigenvalue 
    of \(T\). A vector \(v \in T\) is called a 
    \textbf{generalized eigenvector} of \(T\) corresponding
    to \(\lambda\) if \(v \neq 0\) and 
    \[(T - \lambda I)^k v = 0\]
    for some positive integer \(k\).
\end{definition}

\begin{remark}
    There is no notion of ``generalized eigenvalues'' since 
    we do not create new eigenvalues. 
\end{remark}

\begin{remark}
    A nonzero vector \(v \in V\) is a generalized eigenvector 
    of \(T\) if and only if \((T - \lambda I)^{\dim V} v = 0\)
\end{remark}

\begin{thm}[a basis of generalized eigenvectors]
    Suppose \(\F = \C\) and \(T \in \Lc(V)\). Then 
    there is a basis of \(V\) consisting of generalized 
    eigenvectors of \(T\).    
\end{thm}

\begin{proposition}
    Suppose \(T \in \Lc(V)\). Then each generalized 
    eigenvector of \(T\) only corresponds to one eigenvalue 
    of \(T\).
\end{proposition}

\begin{proposition}
    Suppose that \(T \in \Lc(V)\). Then every list of 
    generalized eigenvectors of \(T\) corresponding to 
    distinct eigenvalues are linearly independent.
\end{proposition}

\begin{definition}[nilpotent]
    An operator is called \textbf{nilpotent} if some powers
    of it equals 0.    
\end{definition}

\begin{remark}
    An operator is nilpotent if every nonzero vector in 
    \(V\) is a generalized eigenvector of \(T\) corresponding 
    to eigenvalue 0. 
\end{remark}

\begin{corollary}
    Suppose \(T \in \Lc(V)\) is nilpotent. Then \(T^{\dim V} = 0\).
\end{corollary}

\begin{thm}[eigenvalues of nilpotent operator]
    Suppose \(T \in \Lc(V)\). 
    \begin{enumerate}[label=(\alph*)]
        \item If \(T\) is nilpotent, then \(0\) is an eigenvalue of \(T\) and \(T\) has no other 
        eigenvalues. 
        \item If \(\F = \C\) and \(0\) is the only eigenvalue of \(T\), then \(T\) is nilpotent.
    \end{enumerate}
\end{thm}

\begin{thm}[minimal polynomial and upper-triangular matrix of nilpotent operator]
    Suppose \( T \in \mathcal{L}(V) \). Then the following are equivalent.
    \begin{itemize}
        \item[(a)] \( T \) is nilpotent.
        \item[(b)] The minimal polynomial of \( T \) is \( z^m \) for some positive integer \( m \).
        \item[(c)] There is a basis of \( V \) with respect to which the matrix of \( T \) has the form
        \[
        \begin{pmatrix}
        0 & * & \cdots & * \\
        0 & 0 & \ddots & \vdots \\
        \vdots & \vdots & \ddots & * \\
        0 & 0 & \cdots & 0
        \end{pmatrix},
        \]
        where all entries on and below the diagonal equal 0.
    \end{itemize}
\end{thm}




%%%%%%%%%%%%%%%%%%%%%%%%%%%%%%%%%%%%%%%%%%%%%%%%%%%%%%%%%%%%%%%
%%%%%%%%%%%%%%%%%%%%%%%%%%%% 8A PS %%%%%%%%%%%%%%%%%%%%%%%%%%%%
%%%%%%%%%%%%%%%%%%%%%%%%%%%%%%%%%%%%%%%%%%%%%%%%%%%%%%%%%%%%%%%

\newpage 
\addcontentsline{toc}{subsection}{8A Problem Sets}

\begin{problem}{1}
    SUppose \(T \in \Lc(V)\). Prove that if \(\dim \nul T^4 = 8\) and \(\dim 
    \nul T^6 = 9\), then \(\dim \nul T^m = 9\) for all  integers \(m \geq 5\).
\end{problem}

\begin{proof}
Suppose not, then \(\dim \nul T^5 = 8 = \dim \nul T^6\), forming a contradiction. Therefore, 
the statement holds. 
\end{proof}


\begin{problem}{2}
    Suppose \(T \in \Lc(V)\), \(m\) is a positive integer, \(v \in V\), and \(T^{m-1}v  \neq 0\) 
    but \(T^m v = 0\). Prove that \(v, Tv, T^2v, \ldots, T^{m-1}v\) is linearly independent. 
\end{problem}

\begin{proof}
Consider 
\[a_0 v + a_1 Tv + \cdots + a_{m-1}T^{m-1}v = 0\]
Apply \(T^{m-1}\) on both sides yields that 
\[a_0 T^{m-1} v = 0\]
and therefore \(a_0 = 0\). Note that \(v \neq \nul T^{m-1}\) and therefore 
\(v \neq \nul T^j\) for \(j \leq m-1\). Hence continuing apply the argument above will 
gets that all \(a_i = 0\).
\end{proof}

\begin{problem}{3}
    Suppose \(T \in \Lc(V)\). Prove that 
    \[V = \nul T \oplus \range T \Longleftrightarrow \nul T^2 = \nul T\]
\end{problem}

\begin{proof}
\(\Rightarrow\) We know that \(\nul T \subseteq \nul T^2\). Take \(v \in \nul T^2\), then 
\[T^2 v = 0 = T(Tv)\]
Therefore \(Tv \in \nul T\), but \(Tv \in \range T\) so \(Tv = 0\), which gives that 
\(v \in \nul T\).

\(\Leftarrow\) Let \(v \in (\nul T) \cap (\range T)\). Then there exists \(u\) s.t. 
\(Tu = v\) and \(Tv = 0\). Therefore \(T^2u = Tv = 0\) and thus \(u \in \nul T^2 
=\nul T\). So \(v = T0  = 0\). We've proved the claim.
\end{proof}


\begin{problem}{6}
    Suppose \(T \in \Lc(V)\). Show that 
    \[V = \range T^0 \supseteq \range T^1 \supseteq \cdots \supseteq \range T^k \supseteq T^{k+1} \supseteq \cdots\]
\end{problem}

\begin{proof}
Take \(v \in \range T^{k+1}\), then we know that ther exists \(u \in V\) s.t. 
\(v = T^{k+1} u = T^{k}(Tu)\), therefore \(v \in \range T^k\). 
\end{proof}

\begin{problem}{9}
    Suppose \(T \in \Lc(V)\) and \(m\) is a nonnegative integer. Prove that 
    \[\nul T^m = \nul T^{m+1} \Longleftrightarrow \range T^m = \range T^{m+1}\]
\end{problem}

\begin{proof}
We know that 
\[\dim V = \dim \nul T^m + \dim \range T^m = \dim \nul T^{m+1} + \dim \range T^{m+1}\]
Therefore 
\[\nul T^m = \nul T^{m+1} \Longleftrightarrow \range T^m = \range T^{m+1}\]
\end{proof}

\begin{problem}{12}
    Suppose \(T \in \Lc(V)\) is such that every vector in \(V\) is a generalized eigenvector of \(T\). 
    Prove that there exists \(\lambda \in \F\) such that \(T - \lambda I\) is nilpotent.
\end{problem}

\begin{proof}
If \(T\) has only one eigenvalue, then it is easy to tell that \(T - \lambda I\) is nilpotent 
for the only eigenvalue \(\lambda\). Suppose for the contradiction that it has multiple 
distinct eigenvalues. Then we know that for \(v_1 \in G(\lambda_1, T)\) and \(v_2 \in G(\lambda_2, T)\)
are both invariant under \(T\), but \(v = v_1 + v_2 \in G(\lambda, T)\) is also invariant under 
\(T\). If \(\lambda = \lambda_1\) or 
\(\lambda_2\), then this contradicts that \(\lambda_1 \neq \lambda_2\). If \(\lambda \neq \lambda_1\)
and \(\lambda \neq \lambda_2\), then this contradicts that \(G(\lambda_1, T) \cap G(\lambda, T) = \{0\}\).
Therefore there is only one eigenvalue and thus \(T - \lambda I\) is nilpotent for the only 
eigenvaue \(\lambda\). 

\end{proof}


\begin{problem}{13}
    Suppose \(S, T \in \Lc(V)\) and \(ST\) is nilpotent. Prove that \(TS\) is nilpotent. 
\end{problem}

\begin{proof}
We know \((ST)^k = 0\) for some \(k\). Then 
\[(TS)^{k+1} = T(ST)^kS = 0\]
\end{proof}

\begin{problem}{14}
    Suppose \(T \in \Lc(V)\) is nilpotent and \(T \neq 0\). Prove that \(T\) is not 
    diagonalizable.
\end{problem}

\begin{proof}
\(0\) is the only eigenvalue of \(T\) and any nonzero \(v \in V\) cannot be represented by an eigenbasis.
\end{proof}



\begin{problem}{22}
    Suppose \(T \in \Lc(\C^5)\) is such that \(\range T^4 \neq \range T^5\). Prove that 
    \(T\) is nilpotent. 
\end{problem}

\begin{proof}
By Problem 9 we have that \(\nul T^4 \neq \nul T^5 \) and therefore \(\dim \nul T^4 < 
\dim \nul T^5 = 5\) where \(\dim \C^5 =5 \). Hence \(T\) is nilpotent. 
\end{proof}



%%%%%%%%%%%%%%%%%%%%%%%%%%%%%%%%%%%%%%%%%%%%%%%%%%%%%%%%%%%%
%%%%%%%%%%%%%%%%%%%%%%%%%%%% 8B %%%%%%%%%%%%%%%%%%%%%%%%%%%%
%%%%%%%%%%%%%%%%%%%%%%%%%%%%%%%%%%%%%%%%%%%%%%%%%%%%%%%%%%%%


\newpage 
\section*{8B: Generalized Eigenspace Decomposition}
\addcontentsline{toc}{section}{8B: Generalized Eigenspace Decomposition}


\begin{definition}[generalized eigenspace, \(G(\lambda, T)\)]
    Suppose \(T \in \Lc(V)\) and \(\lambda \in \F\). The \textbf{generalized eigenspace} of 
    \(T\) corresponding to \(\lambda\), denoted by \(G(\lambda, T)\), is defined bby 
    \[G(\lambda, T) = \{v \in V \colon (T - \lambda I)^k v = 0 \text{ for some positive integer }k\}.\]
    Thus \(G(\lambda, T)\) is the set of generalized eigenvectors of \(T\) corresponding to \(\lambda\), 
    along with the 0 vector. 
\end{definition}

\begin{remark}
    \(E(\lambda, T) \subseteq G(\lambda, T)\) as each eigenvector is a generalized eigenvector.
\end{remark}


\begin{corollary}[description of generalized eigenspaces]
    Suppose \(T \in \Lc(V)\) and \(\lambda \in \F\). Then \(G(\lambda, T) = \nul (T - \lambda I)^{\dim V}\).
\end{corollary}


\begin{thm}[generalized eigenspace decomposition]
    Suppose \(\F = \C\) and \(T \in \Lc(V)\). Let \(\lambda_1, \ldots, \lambda_m\) be the distinct 
    eigenvalues of \(T\). Then 
    \begin{enumerate}[label=(\alph*)]
        \item \(G(\lambda_k, T)\) is invariant under \(T\) for each \(k = 1, \ldots, m\); 
        \item \((T - \lambda_k I)|_{G(\lambda_k, T)}\) is nilpotent for each \(k = 1, \ldots, m\); 
        \item \(V = G(\lambda_1, T) \oplus \cdots \oplus G(\lambda_m, T)\).
    \end{enumerate}
\end{thm}

\begin{definition}[multiplicity]
    Suppose \(T \in \Lc(V)\). The \textbf{multiplicity} of an eigenvalue \(\lambda\) of 
    \(T\) is defined to be the dimension of the corresponding generalized eigenspace 
    \(G(\lambda, T)\). In other words, the multiplicity of an eigenvalue \(\lambda\) of \(T\)
    equals 
    \[\dim \nul (T -\lambda I)^{\dim V}\]
\end{definition}


\begin{corollary}
    Suppose \(\F = \C\) and \(T \in \Lc(V)\). Then the sum of the multiplicities of all 
    eigenvalue of \(T\) equals \(\dim V\).
\end{corollary}

\begin{remark}
    We may know the term \textbf{algebraic multiplicity} and \textbf{geometric multiplicity} 
    in some books. We have 
    \begin{align*}
        \text{algebraic multiplicity of } \lambda = \dim \nul (T - \lambda I)^{\dim V} = \dim G(\lambda, T). \\ 
        \text{geometric multiplicity of } \lambda = \dim \nul (T - \lambda I) = \dim E(\lambda, T).
    \end{align*}
\end{remark}

\begin{remark}
    If \(V\) is an inner product space, \(T \in \Lc(V)\) is normal, and \(\lambda\) is an eigenvalue 
    of \(T\), then the algebraic multiplicity of \(\lambda\) equals the geometric multiplicity of 
    \(\lambda\) (i.e. every eigenvector is a generalized eigenvector).
\end{remark}

\begin{definition}[\textcolor{red}{characteristic polynomial}]
    Suppose \(\F = \C\) and \(T \in \Lc(V)\). Let \(\lambda_1, \ldots, \lambda_m\) denote the 
    distinct eigenvalues of \(T\), with multiplicities \(d_1, \ldots, d_m\). The polynomial 
    \[(z - \lambda_1)^{d_1} \cdots (z - \lambda_m)^{d_m}\]
    is called the \textbf{characteristic polynomial} of \(T\).
\end{definition}

\begin{corollary}
    Suppose \(\F = \C\) and \(T \in \Lc(V)\). Then 
    \begin{enumerate}[label=(\alph*)]
        \item the characteristic polynomial of \(T\) has degree \(\dim V\); 
        \item the zeros of the characteristic polynomial of \(T\) are the eigenvalues of \(T\).
    \end{enumerate}
\end{corollary}

\begin{thm}[Cayley-Hamilton theorem]
    Suppose \(\F = \C, T \in \Lc(V)\), and \(q\) is the characteristic polynomial of \(T\). 
    Then \(q(T) = 0\).
\end{thm}


\begin{corollary}
    Suppose \(\F = \C\) and \(T \in \Lc(V)\). Then the characteristic polynomial of \(T\) is 
    a polynomial multiple of the minimal polynomial of \(T\).
\end{corollary}


\begin{thm}
    Suppose \(\F = \C\) and \(T \in \Lc(V)\). Suppose \(v_1, \ldots, v_n\) is a basis of 
    \(V\) such that \(\Mc(T, (v_1, \ldots, v_n))\) is upper triangular. Then the number 
    of times each eigenvalue \(\lambda\) of \(T\) appears on the diagonal of 
    \(\Mc(T, (v_1, \ldots, v_n))\) equals the multiplicity of \(\lambda\) as an eigenvalue 
    of \(T\).
\end{thm}

\begin{definition}[block diagonal matrix]
    A \textbf{block diagonal matrix} is a square matrix of the form 
    \[\begin{pmatrix}
        A_1 & & 0 \\ 
        & \ddots &  \\ 
        0 & & A_m 
    \end{pmatrix}\]
    where \(A_1, \ldots, A_m\) are square matrices lying along the diagonal and all other 
    entries of the matrix equal 0.
\end{definition}

\begin{remark}
    wrt. an appropriate basis, every operator on a finite-dimensional complex vector space 
    has a matrix of the form. 
\end{remark}

\begin{thm}[block diagonal matrix with upper-triangular blocks]
    Suppose \(\F = \C\) and \(T \in \Lc(V)\). Let \(\lambda_1, \ldots, \lambda_m\) be the 
    distinct eigenvalues of \(T\), with multiplicities \(d_1, \ldots, d_m\). Then there is a basis 
    of \(V\) with respect to which \(T\) has a block diagonal matrix of the form
    \[\begin{pmatrix}
        A_1 & & 0 \\ 
        & \ddots & \\ 
        0 & & A_m 
    \end{pmatrix},\]
    where each \(A_k\) is a \(d_k\)-by-\(d_k\) upper-triangular matrix of the form 
    \[ 
        A_k = \begin{pmatrix}
            \lambda_k & & * \\ 
            & \ddots & \\ 
            0 & & \lambda_k
        \end{pmatrix}  \]
\end{thm}



%%%%%%%%%%%%%%%%%%%%%%%%%%%%%%%%%%%%%%%%%%%%%%%%%%%%%%%%%%%%%%%
%%%%%%%%%%%%%%%%%%%%%%%%%%%% 8B PS %%%%%%%%%%%%%%%%%%%%%%%%%%%%
%%%%%%%%%%%%%%%%%%%%%%%%%%%%%%%%%%%%%%%%%%%%%%%%%%%%%%%%%%%%%%%

\newpage 
\addcontentsline{toc}{subsection}{8B Problem Sets}


\begin{problem}{1}
    Define \(T \in \Lc(\C^2)\) by \(T(w, z) = (-z, w)\). Find the generalized eigenspaces 
    corresponding to the distinct eigenvalues of \(T\).
\end{problem}

\begin{proof}
We have the matrix of \(T\) to be 
\[\begin{pmatrix}
    0 & -1 \\ 
    1 & 0
\end{pmatrix}\]
The eigenvalues for \(T\) are \(\pm i\). For \(\lambda_1 = i\), we have \(v_1 = (i, 1)\); for 
\(\lambda_2 = - i\), we have \(v_2 = (-i, 1)\). There eigenspace is therefore:
\[E_{i} = \Span\{(i, 1)\} \ \ \ E_{-i} = \Span\{(-i, 1)\}\]
\end{proof}

\begin{problem}{2}
    Suppose \(T \in \Lc(V)\). Prove that \(G(\lambda, T) = G(\frac{1}{\lambda}, T^{-1})\) for 
    every \(\lambda \in \F\) with \(\lambda \neq 0\). 
\end{problem}

\begin{proof}
WLOG let \(v \in G(\lambda, T)\). Then we have that for some \(k\)
\[0 = (T - \lambda I)^k\]
We have 
\[(\lambda^{-1})^k (T^{-1})^k (T - \lambda I)^k 
= (\lambda^{-1}T^{-1}(T - \lambda I))^k = (\lambda^{-1} I - T^{-1})^k = 0\]
which shows that \( v \in G(\frac{1}{\lambda}, T)\). The other direction follows accordingly. 
\end{proof}


\begin{problem}{3}
    Suppose \(T \in \Lc(V)\). Suppose \(S \in \Lc(V)\) is invertible. Prove that \(T\)
    and \(S^{-1}TS\) have the same eigenvalues with the same multiplicities.
\end{problem}

\begin{proof}
Let \(\lambda\) be an eigenvalue of \(T\) with multiplicity \(d\). Then we know 
\[(T - \lambda I)^d = 0\]
Therefore 
\[(S^{-1})^d (T - \lambda I)^d S^d = (S^{-1}(T - \lambda I) S)^d = (S^{-1}TS - \lambda I)^d = 0\]
The converse is proved identically. 
\end{proof}


\begin{problem}{5}
    Suppose \(T \in \Lc(V)\) and 3 and 8 are eigenvalues of \(T\). Let \(n = \dim V\). Prove that 
    \(V = (\nul T^{n-2}) \oplus (\range T^{n-2})\). 
\end{problem}


\begin{proof}
This means that the minimal polynomial of \(T\) can be written as 
\[m_T(x) = (x-3)(x-8)q(x)\]
with \(\max \deg q(x) \leq n-2\). Hence we have that \(\nul T^{n} = \nul T^{n-1} = \nul T^{n-2}\) and 
\(\range T^{n} = \range T^{n-1} = \range T^{n-2}\). Applying P3 from section 8A solves the problem. 
\end{proof}

\begin{problem}{10}
    Suppose \(V\) is a complex inner product space, \(e_1, \ldots, e_n\) is an orthonormal basis of \(T\), 
    and \(T \in \Lc(V)\). Let \(\lambda_1, \ldots, \lambda_n\) be the eigenvalues of \(T\), each included 
    as many times as its multiplicity. Prove that 
    \[|\lambda_1|^2 + \cdots + |\lambda_n|^2 \leq \norm{T e_1}^2 + \cdots + \norm{T e_n}^2\]
\end{problem}

\begin{proof}

\begin{align*}
    \sum_{i=1}^{n}\norm{T e_i}^2 
    &= \sum_{i=1}^{n}\norm{U \Sigma V^* e_i}^2  \\ 
    &= \sum_{i=1}^{n}\norm{U \Sigma f_i}^2 \\ 
    &= \sum_{i=1}^{n}\norm{\Sigma f_i}^2 \\  
    &= \sum_{i=1}^{n} \norm{\lambda_i f_i}^2 \\ 
    &\geq \sum_{i=1}^{n} |\lambda_i|^2
\end{align*}
by Bessel's inequality at last step. 
\end{proof}


\begin{problem}{14}
    Give an example of an operator on \(\C^4\) whose characteristic polynomial equals 
    \(z(z-1)^2 (z-3)\) and whose minimal polynomial equals \(z(z - 1) (z-3)\).
\end{problem}

\begin{proof}
Consider 
\[
J = \begin{pmatrix}
0 & 0 & 0 & 0 \\
0 & 1 & 1 & 0 \\
0 & 0 & 1 & 0 \\
0 & 0 & 0 & 3
\end{pmatrix}
\]
\end{proof}


\begin{problem}{17}
    Suppose \(\F = \C\) and \(P \in \Lc(V)\) is such that \(P^2 = P\). Prove that the characteristic 
    polynomial of \(P\) is \(z^m (z-1)^n\), where \(m = \dim \nul P\) and \(n = \dim \range P\).
\end{problem}

\begin{proof}
We know that the projection operator \(P\) has eigenvalue \(0\) and \(1\) (from definition). By many of 
our prior exericises, we know that the (generalized eigenspace of) eigenvalue 0 partitions the null space 
and nonzero ones partitions the range space. You may verify it by yourselves.
\end{proof}






%%%%%%%%%%%%%%%%%%%%%%%%%%%%%%%%%%%%%%%%%%%%%%%%%%%%%%%%%%%%
%%%%%%%%%%%%%%%%%%%%%%%%%%%% 8C %%%%%%%%%%%%%%%%%%%%%%%%%%%%
%%%%%%%%%%%%%%%%%%%%%%%%%%%%%%%%%%%%%%%%%%%%%%%%%%%%%%%%%%%%


\newpage 
\section*{8C: Consequences of Generalized Eigenspace Decomposition}
\addcontentsline{toc}{section}{8C: Consequences of Generalized Eigenspace Decomposition}


\begin{lemma}
    Suppose \(T \in \Lc(V)\) is nilpotent. Then \(I + T\) has a square root.
\end{lemma}

\begin{remark}
    This lemma holds on both real and complex vector spaces. 
\end{remark}

\begin{lemma}
    Suppose \(V\) is a complex vector space and \(T \in \Lc(V)\) is invertible. Then \(T\) has a square root.
\end{lemma}

\begin{definition}[Jordan basis]
    Suppose \(T \in \Lc(V)\). A basis of \(V\) is called a \emph{Jordan basis} for \(T\) if with respect 
    to this basis \(T\) has a block diagonal matrix 
    \[\begin{pmatrix}
        A_1 & & 0 \\ 
        & \ddots & \\ 
        0 & & A_p
    \end{pmatrix}\]
    in which each \(A_k\) is an upper-triangular matrix of the form 
    \[A_k = \begin{pmatrix}
        \lambda_k & 1 & & 0 \\ 
        & \ddots & \ddots & \\ 
        & & \ddots & 1  \\ 
        0 & & & \lambda_k 
    \end{pmatrix}\]
\end{definition}

\begin{thm}[every nilpotent operator has a Jordan basis]
    Suppose \(T \in \Lc(V)\) is nilpotent. Then there is a basis of \(V\) that is a Jordan basis for \(T\).
\end{thm}


\begin{corollary}[Jordan form]
    Suppose \(\F = \C\) and \(T \in \Lc(V)\). Then there is a basis of \(V\) that is a Jordan basis for 
    \(T\). 
\end{corollary}



%%%%%%%%%%%%%%%%%%%%%%%%%%%%%%%%%%%%%%%%%%%%%%%%%%%%%%%%%%%%%%%
%%%%%%%%%%%%%%%%%%%%%%%%%%%% 8C PS %%%%%%%%%%%%%%%%%%%%%%%%%%%%
%%%%%%%%%%%%%%%%%%%%%%%%%%%%%%%%%%%%%%%%%%%%%%%%%%%%%%%%%%%%%%%

\newpage 
\addcontentsline{toc}{subsection}{8C Problem Sets}


\begin{problem}{1}
    Suppose \(T \in \Lc(\C^3)\) is the oprator defined by \(T(z_1, z_2, z_3) = (z_2, z_3, 0)\). Prove that 
    \(T\) does not have a square root. 
\end{problem}

\begin{proof}
Note that for eigenvalue 
\[(z_2, z_3, 0) = \lambda(z_1, z_2, z_3)\]
the only solution is \(\lambda = 0\) with multiplicity 3. Suppose for contradiction that \(S^2 = T\). So 
any eigenvalue \(\lambda\) of \(S\), \(\lambda^2\) will be the eigenvalue of \(T\), so \(S\) also only 
has \(\lambda=0\) as its only eigenvalue, indicating that \(S\) is nilpotent and \(S^3 = 0\). This gives 
that \(T^2 = SS^3 = 0\). However, we in fact have that 
\[T^2 (z_1, z_2, z_3) = (z_3, 0, 0) \neq 0\]
reaching a contradiction.
\end{proof}

\begin{problem}{6}
    Find a basis of \(\mathcal{P}_4 (\R)\) that is a Jordan basis for the differentiation operator 
    \(D\) on \(\mathcal{P}_4(\R)\) defined by \(D p = p'\).
\end{problem}

\begin{proof}
Note that the goal here is to find linearly independent \(v_1, \ldots, v_5\) s.t. \(D(v_1) = 0\) and 
\(D(v_{i}) = v_{i-1}\). This gives that 
\[\{1, x, \frac{1}{2}x^2, \frac{1}{6}x^3, \frac{1}{24}x^4\}\]
\end{proof}


Skip the rest of questions.
% \begin{problem}{8}
%     Suppose \(T \in \Lc(V)\) and \(v_1, \ldots, v_n\) is a basis of \(V\) that is a Jordan basis for \(T\). 
%     Describe the matrix of \(T^2\) wrt. this basis.
% \end{problem}


%%%%%%%%%%%%%%%%%%%%%%%%%%%%%%%%%%%%%%%%%%%%%%%%%%%%%%%%%%%%
%%%%%%%%%%%%%%%%%%%%%%%%%%%% 8D %%%%%%%%%%%%%%%%%%%%%%%%%%%%
%%%%%%%%%%%%%%%%%%%%%%%%%%%%%%%%%%%%%%%%%%%%%%%%%%%%%%%%%%%%


\newpage 
\section*{8D: Trace: A Connection Between Matrices and Operators}
\addcontentsline{toc}{section}{8D: Trace: A Connection Between Matrices and Operators}

\begin{definition}[trace of a matrix]
    Suppose \(A\) is a square matrix with entires in \(\F\). The \textbf{trace} of \(A\), denoted by 
    \(\tr A\), is defined to be the diagonal entries of \(A\).
\end{definition}

\begin{proposition}[trace of \(AB\) equals trace of \(BA\)]
    Suppose \(A\) is an m-by-n matrix and \(B\) is an n-by-m matrix. Then 
    \[\tr (AB) = \tr (BA)\]
\end{proposition}

\begin{lemma}
    Suppose \(T \in \Lc(V)\). Suppose \(u_1, \ldots, u_n\) and \(v_1, \ldots, v_n\) are bases of \(V\). Then 
    \[\tr \Mc \left( T, \left( u_1, \ldots, u_n \right) \right) 
    = \tr \Mc \left( T, \left( v_1, \ldots, v_n \right) \right)\]
\end{lemma}


\begin{definition}[trace of an operator]
    Suppose \(T \in \Lc(V)\). The \textbf{trace} of \(T\), denoted by \(\tr T\), is defined by 
    \[\tr \ T = \tr \ \Mc \left( T, \left( v_1, \ldots, v_n \right) \right)\]
    where \(v_1, \ldots, v_n\) is any basis of \(V\).
\end{definition}


\begin{corollary}
    Suppose \(\F = \C\) and \(T \in \Lc(V)\). Then \(\tr \ T\) equals the sum of the eigenvalues of \(T\), with 
    each eigenvalue included as many times as its multiplicity. 
\end{corollary}

\begin{corollary}
    Suppose \(\F = \C\) and \(T \in \Lc(V)\). Let \(n = \dim V\). Then \(\tr \ T\) equals the negative 
    of the coefficient of \(z^{n-1}\) in the characteristic polynomial of \(T\).
\end{corollary}

\begin{corollary}
    Suppose \(V\) is an inner product space, \(T \in \Lc(V)\), and \(e_1, \ldots, e_n\) is an orthonormal 
    basis of \(V\). Then 
    \[\tr \ T = \langle Te_1,e_1 \rangle + \cdots + \langle Te_n,e_n \rangle\]
\end{corollary}

\begin{thm}[trace is linear]
    The function \(\tr \ \colon \Lc(V) \to \F\) is a linear functional on \(\Lc(V)\) such that 
    \[\tr \ (ST) = \tr \ (TS)\]
    for all \(S, T \in \Lc(V)\).
\end{thm}


\begin{corollary}
    There do not exist operators \(S, T \in \Lc(V)\) such that \(ST - TS = I\).
\end{corollary}


%%%%%%%%%%%%%%%%%%%%%%%%%%%%%%%%%%%%%%%%%%%%%%%%%%%%%%%%%%%%%%%
%%%%%%%%%%%%%%%%%%%%%%%%%%%% 8D PS %%%%%%%%%%%%%%%%%%%%%%%%%%%%
%%%%%%%%%%%%%%%%%%%%%%%%%%%%%%%%%%%%%%%%%%%%%%%%%%%%%%%%%%%%%%%

\newpage 
\addcontentsline{toc}{subsection}{8D Problem Sets}

\begin{problem}{1}
    Suppose \(V\) is an inner product space and \(v, w \in V\). Define an operator \(T \in \Lc(V)\)
    by \(Tu = \langle u,v \rangle w\). Find a formula for \(\tr \ T\).
\end{problem}

\begin{proof}
Let \(e_1, \ldots, e_n\) be the standard orthonormal basis of \(V\). Then we have that 
\begin{align*}
    \tr \ T 
    &= \sum_{i=1}^{n} \langle T e_i, e_i \rangle \\ 
    &= \sum_{i=1}^{n} \langle \langle e_i,v \rangle w, e_i \rangle \\ 
    &= \sum_{i=1}^{n} \langle e_i,v \rangle \langle w,e_i \rangle \\ 
    &= \sum_{i=1}^{n} v_i w_i \\ 
    &=  v \cdot w
\end{align*}
\end{proof}


\begin{problem}{2}
    Suppose \(P \in \Lc(V)\) satisfies \(P^2 = P\). Prove that 
    \[\tr \ P = \dim \range P\]
\end{problem}

\begin{proof}
Note that \(\tr P = \sum_{i=1}^{n} \lambda_i\) where \(\lambda_i = 1 \) or \(0\). The multiplicity of 
\(\lambda_i = 1\) determines the \(\dim \range P\) and thus gives the desired conclusion.  
\end{proof}


\begin{problem}{5}
    Suppose \(V\) is an inner product space. Suppose \(T \in \Lc(V)\) is a positive operator and 
    \(\tr \ T = 0\). Prove that \(T = 0\).
\end{problem}

\begin{proof}
We know that \(\lambda_i \geq 0\) for all \(i\) for positive \(T\). Since \(\tr = 0\), all eigenvalues 
are 0 and thus \(T = 0\) (as it's self-adjoint by positvitiy).
\end{proof}

\begin{problem}{9}
    Suppose \(T \in \Lc(V)\) is such that \(\tr \ (ST) = 0\) for all \(S \in \Lc(V)\) Prove that \(T = 0\).
\end{problem}

\begin{proof}
Consider orthonormal basis \(e_1, \ldots, e_n\) and define \(S_{ij}\) to be such that 
maps \(e_j\) to \(e_i\) while keep all other zero. Therefore \(\tr \ (S_{ij}T) = T_{ij} = 0\) 
for all \(i, j\). Hence \(T = 0\).
\end{proof}


\begin{problem}{11}
    Suppose \(V\) and \(W\) are inner product spaces and \(T \in \Lc(V, W)\). Prove that if \(e_1, \ldots, e_n\)
    is an orthonormal basis of \(V\) and \(f_1, \ldots, f_m\) is an orthonormal basis of \(W\), then 
    \[\tr \ (T^* T) = \sum_{k=1}^{n} \sum_{j=1}^m |\langle T e_k,f_j \rangle|^2\]
\end{problem}

\begin{proof}
We have that 
\begin{align*}
    \tr \ (T^* T) 
    &= \sum_{k=1}^{n} \langle T^* T e_k,e_k \rangle \\ 
    &= \sum_{k=1}^{n} \langle T e_k, T e_k \rangle \\ 
    &= \sum_{k=1}^{n} \langle T e_k, \sum_{j=1}^{m} \langle Te_k,f_j \rangle f_j \rangle \\ 
    &= \sum_{k=1}^{n} \sum_{j=1}^{m} |\langle T e_k,f_j \rangle|^2
\end{align*}
\end{proof}


\begin{problem}{12}
    Suppose \(V\) and \(W\) are finite-dimensional inner product spaces. 
    \begin{enumerate}[label=(\alph*)]
        \item Prove that \(\langle S,T \rangle = \tr \ (T^* S)\) defines an inner product on \(\Lc(V, W)\). 
        \item Suppose \(e_1, \ldots, e_n\) is an orthonormal basis of \(V\) and \(f_1, \ldots, f_m\) 
        is an orthonormal basis of \(W\). Show that the inner product on \(\Lc(V, W)\) from (a) is the same 
        as the standard inner product on \(\F^{mn}\), where we identify each element of \(\Lc(V, W)\) with 
        its matrix (with repsect to the bases just mentioned) and then with an element of \(\F^{mn}\). 
    \end{enumerate}
\end{problem}

\begin{remark}
    The norm from (a) is called the Frbenius norm or the Hilbert-Schmidt norm.
\end{remark}

\begin{proof}
(a) We check each condition manually:
\begin{itemize}
    \item Positivity: \(\langle S,S \rangle = \tr (S^*S) = \sum_{i=1}^{n} (A^*A)_{ii} 
    = \sum_{i=1}^{n} \sum_{j=1}^{m} A^*_{ij} A_{ji} 
    = \sum_{i=1}^{n} \sum_{j=1}^{m} |A_{ij}|^2 \geq 0\) with equality 
    iff \(S = 0\). 
    \item Linearity in first slot: \(\langle \lambda S_1 + S_2, T \rangle 
    = \tr (T^* (\lambda S_1 + S_2)) 
    = \lambda\tr (T^* S_1) + \tr (T^* S_2) =\lambda \langle S_1,T \rangle + \langle S_2,T \rangle\)
    \item Conjugate symmetry: \( \overline{ \langle S,T \rangle } =\overline{\tr (T^* S)}  
    = \tr (\overline{T^* S}) = \tr (S^* T) = \langle T,S \rangle\) 


    (b) The standard inner product on \(\F^{mn}\) for the two matrices \(A, B\) is 
    \[\langle A,B \rangle = \sum_{i=1}^{n} \sum_{j=1}^{m} A_{ij} \overline{B}_{ij}\] 
    which is exactly how we define in (a).
\end{itemize}
\end{proof}


\end{document}

