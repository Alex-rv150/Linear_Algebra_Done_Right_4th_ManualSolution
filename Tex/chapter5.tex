\documentclass{extarticle}
\sloppy
%%%%%%%%%%%%%%%%%%%%%%%%%%%%%%%%%%%%%%%%%%%%%%%%%%%%%%%%%%%%%%%%%%%%%%
% PACKAGES            																						  %
%%%%%%%%%%%%%%%%%%%%%%%%%%%%%%%%%%%%%%%%%%%%%%%%%%%%%%%%%%%%%%%%%%%%%
\usepackage[10pt]{extsizes}
\usepackage{amsfonts}
% \usepackage{amsthm}
\usepackage{amssymb}
\usepackage[shortlabels]{enumitem}
\usepackage{microtype} 
\usepackage{amsmath}
\usepackage{mathtools}
\usepackage{commath}
\usepackage{amsthm}
\usepackage{bbm}
\usepackage[colorlinks=true, allcolors=blue]{hyperref}

%%%%%%%%%%%%%%%%%%%%%%%%%%%%%%%%%%%%%%%%%%%%%%%%%%%%%%%%%%%%%%%%%%%%%%
% PROBLEM ENVIRONMENT         																			           %
%%%%%%%%%%%%%%%%%%%%%%%%%%%%%%%%%%%%%%%%%%%%%%%%%%%%%%%%%%%%%%%%%%%%%
\usepackage{tcolorbox}
\tcbuselibrary{theorems, breakable, skins}
\newtcbtheorem{prob}% environment name
              {Problem}% Title text
  {enhanced, % tcolorbox styles
  attach boxed title to top left={xshift = 4mm, yshift=-2mm},
  colback=blue!5, colframe=black, colbacktitle=blue!3, coltitle=black,
  boxed title style={size=small,colframe=gray},
  fonttitle=\bfseries,
  separator sign none
  }%
  {} 
\newenvironment{problem}[1]{\begin{prob*}{#1}{}}{\end{prob*}}

\newtcbtheorem{exer}% environment name
              {Exercise}% Title text
  {enhanced, % tcolorbox styles
  attach boxed title to top left={xshift = 4mm, yshift=-2mm},
  colback=blue!5, colframe=black, colbacktitle=blue!3, coltitle=black,
  boxed title style={size=small,colframe=gray},
  fonttitle=\bfseries,
  separator sign none
  }%
  {} 
\newenvironment{exercise}[1]{\begin{exer*}{#1}{}}{\end{exer*}}

%%%%%%%%%%%%%%%%%%%%%%%%%%%%%%%%%%%%%%%%%%%%%%%%%%%%%%%%%%%%%%%%%%%%%%
% THEOREMS/LEMMAS/ETC.         																			  %
%%%%%%%%%%%%%%%%%%%%%%%%%%%%%%%%%%%%%%%%%%%%%%%%%%%%%%%%%%%%%%%%%%%%%%
\newtheorem{thm}{Theorem}
\newtheorem*{thm-non}{Theorem}
\newtheorem{lemma}[thm]{Lemma}
\newtheorem{definition}[thm]{Definition}
\newtheorem{remark}[thm]{Remark}
\newtheorem{corollary}[thm]{Corollary}
\newtheorem{proposition}[thm]{Proposition}
% % \newtheorem*{example*}[Theorem]{Example}
% \newtheorem*{example}[thm]{Example}

\theoremstyle{remark}
\newtheorem*{example}{Example}
% \newtheorem*{example*}[thm-non]{Example}

%%%%%%%%%%%%%%%%%%%%%%%%%%%%%%%%%%%%%%%%%%%%%%%%%%%%%%%%%%%%%%%%%%%%%%
% SECTION NUMBERING																				           %
%%%%%%%%%%%%%%%%%%%%%%%%%%%%%%%%%%%%%%%%%%%%%%%%%%%%%%%%%%%%%%%%%%%%%%
\renewcommand\thesection{\Alph{section}:}

\makeatletter
\newcommand{\cbigoplus}{\DOTSB\cbigoplus@\slimits@}
\newcommand{\cbigoplus@}{\mathop{\widehat{\bigoplus}}}
\makeatother

%%%%%%%%%%%%%%%%%%%%%%%%%%%%%%%%%%%%%%%%%%%%%%%%%%%%%%%%%%%%%%%%%%%%%%
% Additional math commands																		           %
%%%%%%%%%%%%%%%%%%%%%%%%%%%%%%%%%%%%%%%%%%%%%%%%%%%%%%%%%%%%%%%%%%%%%%
% \newcommand{\E}{\mathbb{E}}
% \newcommand{\Prb}{\mathbb{P}}
% \newcommand{\R}{\mathbb{R}}
% \newcommand{\C}{\mathbb{C}}
% \newcommand{\I}{\mathbb{I}}
% \newcommand{\F}{\mathbb{F}}
% \newcommand{\vecc}{\mathbf{c}}
% \newcommand{\Z}{\mathbb{Z}}
% \newcommand{\nul}{\text{null }}
% \newcommand{\range}{\text{range }}
% \newcommand{\Span}{\text{span}}
% \newcommand{\tr}{\text{tr}}
% \newcommand{\diag}{\text{diag}}
% \newcommand{\limn}{\lim_{n \to \infty}}
% \newcommand{\zo}{\overline{z}}
% \newcommand{\wo}{\overline{w}}
% \newcommand{\ao}{\overline{a}}
% \newcommand{\vo}{\overline{v}}
% \newcommand{\glo}{\overline{\lambda}}
% \newcommand{\Var}{\text{Var}}
% \newcommand{\1}{\mathbbm{1}}
% \newcommand{\Pm}{\mathbbm{P}}

% Mathbold abbrevation
\newcommand{\ab}{\mathbf{a}}
\newcommand{\bb}{\mathbf{b}}
\newcommand{\cb}{\mathbf{c}}
\newcommand{\db}{\mathbf{d}}
\newcommand{\eb}{\mathbf{e}}
\newcommand{\fb}{\mathbf{f}}
\newcommand{\gb}{\mathbf{g}}
\newcommand{\hb}{\mathbf{h}}
\newcommand{\ib}{\mathbf{i}}
\newcommand{\jb}{\mathbf{j}}
\newcommand{\kb}{\mathbf{k}}
\newcommand{\lb}{\mathbf{l}}
\newcommand{\mb}{\mathbf{m}}
\newcommand{\nb}{\mathbf{n}}
\newcommand{\ob}{\mathbf{o}}
\newcommand{\pb}{\mathbf{p}}
\newcommand{\qb}{\mathbf{q}}
\newcommand{\rb}{\mathbf{r}}
% \newcommand{\sb}{\mathbf{s}} # already defined
\newcommand{\tb}{\mathbf{t}}
\newcommand{\ub}{\mathbf{u}}
\newcommand{\vb}{\mathbf{v}}
\newcommand{\wb}{\mathbf{w}}
\newcommand{\xb}{\mathbf{x}}
\newcommand{\yb}{\mathbf{y}}
\newcommand{\zb}{\mathbf{z}}
\newcommand{\Ab}{\mathbf{A}}
\newcommand{\Bb}{\mathbf{B}}
\newcommand{\Cb}{\mathbf{C}}
\newcommand{\Db}{\mathbf{D}}
\newcommand{\Eb}{\mathbf{E}}
\newcommand{\Fb}{\mathbf{F}}
\newcommand{\Gb}{\mathbf{G}}
\newcommand{\Hb}{\mathbf{H}}
\newcommand{\Ib}{\mathbf{I}}
\newcommand{\Jb}{\mathbf{J}}
\newcommand{\Kb}{\mathbf{K}}
\newcommand{\Lb}{\mathbf{L}}
\newcommand{\Mb}{\mathbf{M}}
\newcommand{\Nb}{\mathbf{N}}
\newcommand{\Ob}{\mathbf{O}}
\newcommand{\Pb}{\mathbf{P}}
\newcommand{\Qb}{\mathbf{Q}}
\newcommand{\Rb}{\mathbf{R}}
\newcommand{\Sb}{\mathbf{S}}
\newcommand{\Tb}{\mathbf{T}}
\newcommand{\Ub}{\mathbf{U}}
\newcommand{\Vb}{\mathbf{V}}
\newcommand{\Wb}{\mathbf{W}}
\newcommand{\Xb}{\mathbf{X}}
\newcommand{\Yb}{\mathbf{Y}}
\newcommand{\Zb}{\mathbf{Z}}


% MathCal abbrevation
\newcommand{\acal}{\mathcal{a}}
\newcommand{\bcal}{\mathcal{b}}
\newcommand{\ccal}{\mathcal{c}}
\newcommand{\dcal}{\mathcal{d}}
\newcommand{\ecal}{\mathcal{e}}
\newcommand{\fcal}{\mathcal{f}}
\newcommand{\gcal}{\mathcal{g}}
\newcommand{\hcal}{\mathcal{h}}
\newcommand{\ical}{\mathcal{i}}
\newcommand{\jcal}{\mathcal{j}}
\newcommand{\kcal}{\mathcal{k}}
\newcommand{\lcal}{\mathcal{l}}
\newcommand{\mcal}{\mathcal{m}}
\newcommand{\ncal}{\mathcal{n}}
\newcommand{\ocal}{\mathcal{o}}
\newcommand{\pcal}{\mathcal{p}}
\newcommand{\qcal}{\mathcal{q}}
\newcommand{\rcal}{\mathcal{r}}
\newcommand{\scal}{\mathcal{s}}
\newcommand{\tcal}{\mathcal{t}}
\newcommand{\ucal}{\mathcal{u}}
\newcommand{\vcal}{\mathcal{v}}
\newcommand{\wcal}{\mathcal{w}}
\newcommand{\xcal}{\mathcal{x}}
\newcommand{\ycal}{\mathcal{y}}
\newcommand{\zcal}{\mathcal{z}}
\newcommand{\Acal}{\mathcal{A}}
\newcommand{\Bcal}{\mathcal{B}}
\newcommand{\Ccal}{\mathcal{C}}
\newcommand{\Dcal}{\mathcal{D}}
\newcommand{\Ecal}{\mathcal{E}}
\newcommand{\Fcal}{\mathcal{F}}
\newcommand{\Gcal}{\mathcal{G}}
\newcommand{\Hcal}{\mathcal{H}}
\newcommand{\Ical}{\mathcal{I}}
\newcommand{\Jcal}{\mathcal{J}}
\newcommand{\Kcal}{\mathcal{K}}
\newcommand{\Lcal}{\mathcal{L}}
\newcommand{\Mcal}{\mathcal{M}}
\newcommand{\Ncal}{\mathcal{N}}
\newcommand{\Ocal}{\mathcal{O}}
\newcommand{\Pcal}{\mathcal{P}}
\newcommand{\Qcal}{\mathcal{Q}}
\newcommand{\Rcal}{\mathcal{R}}
\newcommand{\Scal}{\mathcal{S}}
\newcommand{\Tcal}{\mathcal{T}}
\newcommand{\Ucal}{\mathcal{U}}
\newcommand{\Vcal}{\mathcal{V}}
\newcommand{\Wcal}{\mathcal{W}}
\newcommand{\Xcal}{\mathcal{X}}
\newcommand{\Ycal}{\mathcal{Y}}
\newcommand{\Zcal}{\mathcal{Z}}


% MathCal abbrevation
\newcommand{\ac}{\mathcal{a}}
\newcommand{\bc}{\mathcal{b}}
% \newcommand{\cc}{\mathcal{c}}
\newcommand{\dc}{\mathcal{d}}
\newcommand{\ec}{\mathcal{e}}
\newcommand{\fc}{\mathcal{f}}
\newcommand{\gc}{\mathcal{g}}
\newcommand{\hc}{\mathcal{h}}
\newcommand{\ic}{\mathcal{i}}
\newcommand{\jc}{\mathcal{j}}
\newcommand{\kc}{\mathcal{k}}
\newcommand{\lc}{\mathcal{l}}
\newcommand{\mc}{\mathcal{m}}
\newcommand{\nc}{\mathcal{n}}
\newcommand{\oc}{\mathcal{o}}
\newcommand{\pc}{\mathcal{p}}
\newcommand{\qc}{\mathcal{q}}
\newcommand{\rc}{\mathcal{r}}
% \newcommand{\sc}{\mathcal{s}}
\newcommand{\tc}{\mathcal{t}}
\newcommand{\uc}{\mathcal{u}}
\newcommand{\vc}{\mathcal{v}}
\newcommand{\wc}{\mathcal{w}}
\newcommand{\xc}{\mathcal{x}}
\newcommand{\yc}{\mathcal{y}}
\newcommand{\zc}{\mathcal{z}}
\newcommand{\Ac}{\mathcal{A}}
\newcommand{\Bc}{\mathcal{B}}
\newcommand{\Cc}{\mathcal{C}}
\newcommand{\Dc}{\mathcal{D}}
\newcommand{\Ec}{\mathcal{E}}
\newcommand{\Fc}{\mathcal{F}}
\newcommand{\Gc}{\mathcal{G}}
\newcommand{\Hc}{\mathcal{H}}
\newcommand{\Ic}{\mathcal{I}}
\newcommand{\Jc}{\mathcal{J}}
\newcommand{\Kc}{\mathcal{K}}
\newcommand{\Lc}{\mathcal{L}}
\newcommand{\Mc}{\mathcal{M}}
\newcommand{\Nc}{\mathcal{N}}
\newcommand{\Oc}{\mathcal{O}}
\newcommand{\Pc}{\mathcal{P}}
\newcommand{\Qc}{\mathcal{Q}}
\newcommand{\Rc}{\mathcal{R}}
\newcommand{\Sc}{\mathcal{S}}
\newcommand{\Tc}{\mathcal{T}}
\newcommand{\Uc}{\mathcal{U}}
\newcommand{\Vc}{\mathcal{V}}
\newcommand{\Wc}{\mathcal{W}}
\newcommand{\Xc}{\mathcal{X}}
\newcommand{\Yc}{\mathcal{Y}}
\newcommand{\Zc}{\mathcal{Z}}

% Mathhat abbrevation
\newcommand{\ah}{\hat{a}}
\newcommand{\bh}{\hat{b}}
\newcommand{\ch}{\hat{c}}
% \newcommand{\dh}{\hat{d}}
\newcommand{\eh}{\hat{e}}
\newcommand{\fh}{\hat{f}}
\newcommand{\gh}{\hat{g}}
\newcommand{\hh}{\hat{h}}
\newcommand{\ih}{\hat{i}}
\newcommand{\jh}{\hat{j}}
\newcommand{\kh}{\hat{k}}
\newcommand{\lh}{\hat{l}}
\newcommand{\mh}{\hat{m}}
\newcommand{\nh}{\hat{n}}
\newcommand{\oh}{\hat{o}}
\newcommand{\ph}{\hat{p}}
\newcommand{\qh}{\hat{q}}
\newcommand{\rh}{\hat{r}}
\newcommand{\sh}{\hat{s}}
% \newcommand{\th}{\hat{t}}
\newcommand{\uh}{\hat{u}}
\newcommand{\vh}{\hat{v}}
\newcommand{\wh}{\hat{w}}
\newcommand{\xh}{\hat{x}}
\newcommand{\yh}{\hat{y}}
\newcommand{\zh}{\hat{z}}
\newcommand{\Ah}{\hat{A}}
\newcommand{\Bh}{\hat{B}}
\newcommand{\Ch}{\hat{C}}
\newcommand{\Dh}{\hat{D}}
\newcommand{\Eh}{\hat{E}}
\newcommand{\Fh}{\hat{F}}
\newcommand{\Gh}{\hat{G}}
\newcommand{\Hh}{\hat{H}}
\newcommand{\Ih}{\hat{I}}
\newcommand{\Jh}{\hat{J}}
\newcommand{\Kh}{\hat{K}}
\newcommand{\Lh}{\hat{L}}
\newcommand{\Mh}{\hat{M}}
\newcommand{\Nh}{\hat{N}}
\newcommand{\Oh}{\hat{O}}
\newcommand{\Ph}{\hat{P}}
\newcommand{\Qh}{\hat{Q}}
\newcommand{\Rh}{\hat{R}}
\newcommand{\Sh}{\hat{S}}
\newcommand{\Th}{\hat{T}}
\newcommand{\Uh}{\hat{U}}
\newcommand{\Vh}{\hat{V}}
\newcommand{\Wh}{\hat{W}}
\newcommand{\Xh}{\hat{X}}
\newcommand{\Yh}{\hat{Y}}
\newcommand{\Zh}{\hat{Z}}


% Mathbar abbrevation
\newcommand{\abar}{\bar{a}}
\newcommand{\bbar}{\bar{b}}
\newcommand{\cbar}{\bar{c}}
\newcommand{\dbar}{\bar{d}}
\newcommand{\ebar}{\bar{e}}
\newcommand{\fbar}{\bar{f}}
\newcommand{\gbar}{\bar{g}}
% \newcommand{\hbar}{\bar{h}}
\newcommand{\ibar}{\bar{i}}
\newcommand{\jbar}{\bar{j}}
\newcommand{\kbar}{\bar{k}}
\newcommand{\lbar}{\bar{l}}
\newcommand{\mbar}{\bar{m}}
\newcommand{\nbar}{\bar{n}}
\newcommand{\obar}{\bar{o}}
\newcommand{\pbar}{\bar{p}}
\newcommand{\qbar}{\bar{q}}
\newcommand{\rbar}{\bar{r}}
\newcommand{\sbar}{\bar{s}}
\newcommand{\tbar}{\bar{t}}
\newcommand{\ubar}{\bar{u}}
\newcommand{\vbar}{\bar{v}}
\newcommand{\wbar}{\bar{w}}
\newcommand{\xbar}{\bar{x}}
\newcommand{\ybar}{\bar{y}}
\newcommand{\zbar}{\bar{z}}
\newcommand{\Abar}{\bar{A}}
\newcommand{\Bbar}{\bar{B}}
\newcommand{\Cbar}{\bar{C}}
\newcommand{\Dbar}{\bar{D}}
\newcommand{\Ebar}{\bar{E}}
\newcommand{\Fbar}{\bar{F}}
\newcommand{\Gbar}{\bar{G}}
\newcommand{\Hbar}{\bar{H}}
\newcommand{\Ibar}{\bar{I}}
\newcommand{\Jbar}{\bar{J}}
\newcommand{\Kbar}{\bar{K}}
\newcommand{\Lbar}{\bar{L}}
\newcommand{\Mbar}{\bar{M}}
\newcommand{\Nbar}{\bar{N}}
\newcommand{\Obar}{\bar{O}}
\newcommand{\Pbar}{\bar{P}}
\newcommand{\Qbar}{\bar{Q}}
\newcommand{\Rbar}{\bar{R}}
\newcommand{\Sbar}{\bar{S}}
\newcommand{\Tbar}{\bar{T}}
\newcommand{\Ubar}{\bar{U}}
\newcommand{\Vbar}{\bar{V}}
\newcommand{\Wbar}{\bar{W}}
\newcommand{\Xbar}{\bar{X}}
\newcommand{\Ybar}{\bar{Y}}
\newcommand{\Zbar}{\bar{Z}}


% Mathtilde abbrevation
\newcommand{\at}{\tilde{a}}
\newcommand{\bt}{\tilde{b}}
\newcommand{\ct}{\tilde{c}}
\newcommand{\dt}{\tilde{d}}
\newcommand{\et}{\tilde{e}}
\newcommand{\ft}{\tilde{f}}
\newcommand{\gt}{\tilde{g}}
% \newcommand{\ht}{\tilde{h}}
% \newcommand{\it}{\tilde{i}}
\newcommand{\jt}{\tilde{j}}
\newcommand{\kt}{\tilde{k}}
\newcommand{\lt}{\tilde{l}}
\newcommand{\mt}{\tilde{m}}
\newcommand{\nt}{\tilde{n}}
\newcommand{\ot}{\tilde{o}}
\newcommand{\pt}{\tilde{p}}
\newcommand{\qt}{\tilde{q}}
\newcommand{\rt}{\tilde{r}}
\newcommand{\st}{\tilde{s}}
% \newcommand{\tt}{\tilde{t}}
\newcommand{\ut}{\tilde{u}}
\newcommand{\vt}{\tilde{v}}
\newcommand{\wt}{\tilde{w}}
\newcommand{\xt}{\tilde{x}}
\newcommand{\yt}{\tilde{y}}
\newcommand{\zt}{\tilde{z}}
\newcommand{\At}{\tilde{A}}
\newcommand{\Bt}{\tilde{B}}
\newcommand{\Ct}{\tilde{C}}
\newcommand{\Dt}{\tilde{D}}
\newcommand{\Et}{\tilde{E}}
\newcommand{\Ft}{\tilde{F}}
\newcommand{\Gt}{\tilde{G}}
\newcommand{\Ht}{\tilde{H}}
\newcommand{\It}{\tilde{I}}
\newcommand{\Jt}{\tilde{J}}
\newcommand{\Kt}{\tilde{K}}
\newcommand{\Lt}{\tilde{L}}
\newcommand{\Mt}{\tilde{M}}
\newcommand{\Nt}{\tilde{N}}
\newcommand{\Ot}{\tilde{O}}
\newcommand{\Pt}{\tilde{P}}
\newcommand{\Qt}{\tilde{Q}}
\newcommand{\Rt}{\tilde{R}}
\newcommand{\St}{\tilde{S}}
\newcommand{\Tt}{\tilde{T}}
\newcommand{\Ut}{\tilde{U}}
\newcommand{\Vt}{\tilde{V}}
\newcommand{\Wt}{\tilde{W}}
\newcommand{\Xt}{\tilde{X}}
\newcommand{\Yt}{\tilde{Y}}
\newcommand{\Zt}{\tilde{Z}}


% MathTilde abbrevation
\newcommand{\aT}{\Tilde{a}}
\newcommand{\bT}{\Tilde{b}}
\newcommand{\cT}{\Tilde{c}}
\newcommand{\dT}{\Tilde{d}}
\newcommand{\eT}{\Tilde{e}}
\newcommand{\fT}{\Tilde{f}}
\newcommand{\gT}{\Tilde{g}}
\newcommand{\hT}{\Tilde{h}}
\newcommand{\iT}{\Tilde{i}}
\newcommand{\jT}{\Tilde{j}}
\newcommand{\kT}{\Tilde{k}}
\newcommand{\lT}{\Tilde{l}}
\newcommand{\mT}{\Tilde{m}}
\newcommand{\nT}{\Tilde{n}}
\newcommand{\oT}{\Tilde{o}}
\newcommand{\pT}{\Tilde{p}}
\newcommand{\qT}{\Tilde{q}}
\newcommand{\rT}{\Tilde{r}}
\newcommand{\sT}{\Tilde{s}}
\newcommand{\tT}{\Tilde{t}}
\newcommand{\uT}{\Tilde{u}}
\newcommand{\vT}{\Tilde{v}}
\newcommand{\wT}{\Tilde{w}}
\newcommand{\xT}{\Tilde{x}}
\newcommand{\yT}{\Tilde{y}}
\newcommand{\zT}{\Tilde{z}}
\newcommand{\AT}{\Tilde{A}}
\newcommand{\BT}{\Tilde{B}}
\newcommand{\CT}{\Tilde{C}}
\newcommand{\DT}{\Tilde{D}}
\newcommand{\ET}{\Tilde{E}}
\newcommand{\FT}{\Tilde{F}}
\newcommand{\GT}{\Tilde{G}}
\newcommand{\HT}{\Tilde{H}}
\newcommand{\IT}{\Tilde{I}}
\newcommand{\JT}{\Tilde{J}}
% \newcommand{\KT}{\Tilde{K}}
\newcommand{\LT}{\Tilde{L}}
\newcommand{\MT}{\Tilde{M}}
% \newcommand{\NT}{\Tilde{N}}
\newcommand{\OT}{\Tilde{O}}
% \newcommand{\PT}{\Tilde{P}}
\newcommand{\QT}{\Tilde{Q}}
\newcommand{\RT}{\Tilde{R}}
\newcommand{\ST}{\Tilde{S}}
\newcommand{\TT}{\Tilde{T}}
\newcommand{\UT}{\Tilde{U}}
\newcommand{\VT}{\Tilde{V}}
\newcommand{\WT}{\Tilde{W}}
\newcommand{\XT}{\Tilde{X}}
\newcommand{\YT}{\Tilde{Y}}
\newcommand{\ZT}{\Tilde{Z}}



%Nested Case




% Mathbold + hat abbrevation
\newcommand{\abh}{\mathbf{\hat{a}}}
\newcommand{\bbh}{\mathbf{\hat{b}}}
\newcommand{\cbh}{\mathbf{\hat{c}}}
\newcommand{\dbh}{\mathbf{\hat{d}}}
\newcommand{\ebh}{\mathbf{\hat{e}}}
\newcommand{\fbh}{\mathbf{\hat{f}}}
\newcommand{\gbh}{\mathbf{\hat{g}}}
\newcommand{\hbh}{\mathbf{\hat{h}}}
\newcommand{\ibh}{\mathbf{\hat{i}}}
\newcommand{\jbh}{\mathbf{\hat{j}}}
\newcommand{\kbh}{\mathbf{\hat{k}}}
\newcommand{\lbh}{\mathbf{\hat{l}}}
\newcommand{\mbh}{\mathbf{\hat{m}}}
\newcommand{\nbh}{\mathbf{\hat{n}}}
\newcommand{\obh}{\mathbf{\hat{o}}}
\newcommand{\pbh}{\mathbf{\hat{p}}}
\newcommand{\qbh}{\mathbf{\hat{q}}}
\newcommand{\rbh}{\mathbf{\hat{r}}}
\newcommand{\sbh}{\mathbf{\hat{s}}}
\newcommand{\tbh}{\mathbf{\hat{t}}}
\newcommand{\ubh}{\mathbf{\hat{u}}}
\newcommand{\vbh}{\mathbf{\hat{v}}}
\newcommand{\wbh}{\mathbf{\hat{w}}}
\newcommand{\xbh}{\mathbf{\hat{x}}}
\newcommand{\ybh}{\mathbf{\hat{y}}}
\newcommand{\zbh}{\mathbf{\hat{z}}}
\newcommand{\Abh}{\mathbf{\hat{A}}}
\newcommand{\Bbh}{\mathbf{\hat{B}}}
\newcommand{\Cbh}{\mathbf{\hat{C}}}
\newcommand{\Dbh}{\mathbf{\hat{D}}}
\newcommand{\Ebh}{\mathbf{\hat{E}}}
\newcommand{\Fbh}{\mathbf{\hat{F}}}
\newcommand{\Gbh}{\mathbf{\hat{G}}}
\newcommand{\Hbh}{\mathbf{\hat{H}}}
\newcommand{\Ibh}{\mathbf{\hat{I}}}
\newcommand{\Jbh}{\mathbf{\hat{J}}}
\newcommand{\Kbh}{\mathbf{\hat{K}}}
\newcommand{\Lbh}{\mathbf{\hat{L}}}
\newcommand{\Mbh}{\mathbf{\hat{M}}}
\newcommand{\Nbh}{\mathbf{\hat{N}}}
\newcommand{\Obh}{\mathbf{\hat{O}}}
\newcommand{\Pbh}{\mathbf{\hat{P}}}
\newcommand{\Qbh}{\mathbf{\hat{Q}}}
\newcommand{\Rbh}{\mathbf{\hat{R}}}
\newcommand{\Sbh}{\mathbf{\hat{S}}}
\newcommand{\Tbh}{\mathbf{\hat{T}}}
\newcommand{\Ubh}{\mathbf{\hat{U}}}
\newcommand{\Vbh}{\mathbf{\hat{V}}}
\newcommand{\Wbh}{\mathbf{\hat{W}}}
\newcommand{\Xbh}{\mathbf{\hat{X}}}
\newcommand{\Ybh}{\mathbf{\hat{Y}}}
\newcommand{\Zbh}{\mathbf{\hat{Z}}}


% Mathbold + Tilde abbrevation
\newcommand{\abT}{\mathbf{\Tilde{a}}}
\newcommand{\bbT}{\mathbf{\Tilde{b}}}
\newcommand{\cbT}{\mathbf{\Tilde{c}}}
\newcommand{\dbT}{\mathbf{\Tilde{d}}}
\newcommand{\ebT}{\mathbf{\Tilde{e}}}
\newcommand{\fbT}{\mathbf{\Tilde{f}}}
\newcommand{\gbT}{\mathbf{\Tilde{g}}}
\newcommand{\hbT}{\mathbf{\Tilde{h}}}
\newcommand{\ibT}{\mathbf{\Tilde{i}}}
\newcommand{\jbT}{\mathbf{\Tilde{j}}}
\newcommand{\kbT}{\mathbf{\Tilde{k}}}
\newcommand{\lbT}{\mathbf{\Tilde{l}}}
\newcommand{\mbT}{\mathbf{\Tilde{m}}}
\newcommand{\nbT}{\mathbf{\Tilde{n}}}
\newcommand{\obT}{\mathbf{\Tilde{o}}}
\newcommand{\pbT}{\mathbf{\Tilde{p}}}
\newcommand{\qbT}{\mathbf{\Tilde{q}}}
\newcommand{\rbT}{\mathbf{\Tilde{r}}}
\newcommand{\sbT}{\mathbf{\Tilde{s}}}
\newcommand{\tbT}{\mathbf{\Tilde{t}}}
\newcommand{\ubT}{\mathbf{\Tilde{u}}}
\newcommand{\vbT}{\mathbf{\Tilde{v}}}
\newcommand{\wbT}{\mathbf{\Tilde{w}}}
\newcommand{\xbT}{\mathbf{\Tilde{x}}}
\newcommand{\ybT}{\mathbf{\Tilde{y}}}
\newcommand{\zbT}{\mathbf{\Tilde{z}}}
\newcommand{\AbT}{\mathbf{\Tilde{A}}}
\newcommand{\BbT}{\mathbf{\Tilde{B}}}
\newcommand{\CbT}{\mathbf{\Tilde{C}}}
\newcommand{\DbT}{\mathbf{\Tilde{D}}}
\newcommand{\EbT}{\mathbf{\Tilde{E}}}
\newcommand{\FbT}{\mathbf{\Tilde{F}}}
\newcommand{\GbT}{\mathbf{\Tilde{G}}}
\newcommand{\HbT}{\mathbf{\Tilde{H}}}
\newcommand{\IbT}{\mathbf{\Tilde{I}}}
\newcommand{\JbT}{\mathbf{\Tilde{J}}}
\newcommand{\KbT}{\mathbf{\Tilde{K}}}
\newcommand{\LbT}{\mathbf{\Tilde{L}}}
\newcommand{\MbT}{\mathbf{\Tilde{M}}}
\newcommand{\NbT}{\mathbf{\Tilde{N}}}
\newcommand{\ObT}{\mathbf{\Tilde{O}}}
\newcommand{\PbT}{\mathbf{\Tilde{P}}}
\newcommand{\QbT}{\mathbf{\Tilde{Q}}}
\newcommand{\RbT}{\mathbf{\Tilde{R}}}
\newcommand{\SbT}{\mathbf{\Tilde{S}}}
\newcommand{\TbT}{\mathbf{\Tilde{T}}}
\newcommand{\UbT}{\mathbf{\Tilde{U}}}
\newcommand{\VbT}{\mathbf{\Tilde{V}}}
\newcommand{\WbT}{\mathbf{\Tilde{W}}}
\newcommand{\XbT}{\mathbf{\Tilde{X}}}
\newcommand{\YbT}{\mathbf{\Tilde{Y}}}
\newcommand{\ZbT}{\mathbf{\Tilde{Z}}}


% Mathbold + tilde abbrevation
\newcommand{\abt}{\mathbf{\tilde{a}}}
\newcommand{\bbt}{\mathbf{\tilde{b}}}
\newcommand{\cbt}{\mathbf{\tilde{c}}}
\newcommand{\dbt}{\mathbf{\tilde{d}}}
\newcommand{\ebt}{\mathbf{\tilde{e}}}
\newcommand{\fbt}{\mathbf{\tilde{f}}}
\newcommand{\gbt}{\mathbf{\tilde{g}}}
\newcommand{\hbt}{\mathbf{\tilde{h}}}
\newcommand{\ibt}{\mathbf{\tilde{i}}}
\newcommand{\jbt}{\mathbf{\tilde{j}}}
\newcommand{\kbt}{\mathbf{\tilde{k}}}
\newcommand{\lbt}{\mathbf{\tilde{l}}}
\newcommand{\mbt}{\mathbf{\tilde{m}}}
\newcommand{\nbt}{\mathbf{\tilde{n}}}
\newcommand{\obt}{\mathbf{\tilde{o}}}
\newcommand{\pbt}{\mathbf{\tilde{p}}}
\newcommand{\qbt}{\mathbf{\tilde{q}}}
\newcommand{\rbt}{\mathbf{\tilde{r}}}
\newcommand{\sbt}{\mathbf{\tilde{s}}}
\newcommand{\tbt}{\mathbf{\tilde{t}}}
\newcommand{\ubt}{\mathbf{\tilde{u}}}
\newcommand{\vbt}{\mathbf{\tilde{v}}}
\newcommand{\wbt}{\mathbf{\tilde{w}}}
\newcommand{\xbt}{\mathbf{\tilde{x}}}
\newcommand{\ybt}{\mathbf{\tilde{y}}}
\newcommand{\zbt}{\mathbf{\tilde{z}}}
\newcommand{\Abt}{\mathbf{\tilde{A}}}
\newcommand{\Bbt}{\mathbf{\tilde{B}}}
\newcommand{\Cbt}{\mathbf{\tilde{C}}}
\newcommand{\Dbt}{\mathbf{\tilde{D}}}
\newcommand{\Ebt}{\mathbf{\tilde{E}}}
\newcommand{\Fbt}{\mathbf{\tilde{F}}}
\newcommand{\Gbt}{\mathbf{\tilde{G}}}
\newcommand{\Hbt}{\mathbf{\tilde{H}}}
\newcommand{\Ibt}{\mathbf{\tilde{I}}}
\newcommand{\Jbt}{\mathbf{\tilde{J}}}
\newcommand{\Kbt}{\mathbf{\tilde{K}}}
\newcommand{\Lbt}{\mathbf{\tilde{L}}}
\newcommand{\Mbt}{\mathbf{\tilde{M}}}
\newcommand{\Nbt}{\mathbf{\tilde{N}}}
\newcommand{\Obt}{\mathbf{\tilde{O}}}
\newcommand{\Pbt}{\mathbf{\tilde{P}}}
\newcommand{\Qbt}{\mathbf{\tilde{Q}}}
\newcommand{\Rbt}{\mathbf{\tilde{R}}}
\newcommand{\Sbt}{\mathbf{\tilde{S}}}
\newcommand{\Tbt}{\mathbf{\tilde{T}}}
\newcommand{\Ubt}{\mathbf{\tilde{U}}}
\newcommand{\Vbt}{\mathbf{\tilde{V}}}
\newcommand{\Wbt}{\mathbf{\tilde{W}}}
\newcommand{\Xbt}{\mathbf{\tilde{X}}}
\newcommand{\Ybt}{\mathbf{\tilde{Y}}}
\newcommand{\Zbt}{\mathbf{\tilde{Z}}}


% Mathbold + bar abbrevation
\newcommand{\abb}{\mathbf{\bar{a}}}
\newcommand{\bbb}{\mathbf{\bar{b}}}
\newcommand{\cbb}{\mathbf{\bar{c}}}
\newcommand{\dbb}{\mathbf{\bar{d}}}
\newcommand{\ebb}{\mathbf{\bar{e}}}
\newcommand{\fbb}{\mathbf{\bar{f}}}
\newcommand{\gbb}{\mathbf{\bar{g}}}
\newcommand{\hbb}{\mathbf{\bar{h}}}
\newcommand{\ibb}{\mathbf{\bar{i}}}
\newcommand{\jbb}{\mathbf{\bar{j}}}
\newcommand{\kbb}{\mathbf{\bar{k}}}
\newcommand{\lbb}{\mathbf{\bar{l}}}
\newcommand{\mbb}{\mathbf{\bar{m}}}
\newcommand{\nbb}{\mathbf{\bar{n}}}
\newcommand{\obb}{\mathbf{\bar{o}}}
\newcommand{\pbb}{\mathbf{\bar{p}}}
\newcommand{\qbb}{\mathbf{\bar{q}}}
\newcommand{\rbb}{\mathbf{\bar{r}}}
\newcommand{\sbb}{\mathbf{\bar{s}}}
\newcommand{\tbb}{\mathbf{\bar{t}}}
\newcommand{\ubb}{\mathbf{\bar{u}}}
\newcommand{\vbb}{\mathbf{\bar{v}}}
\newcommand{\wbb}{\mathbf{\bar{w}}}
\newcommand{\xbb}{\mathbf{\bar{x}}}
\newcommand{\ybb}{\mathbf{\bar{y}}}
\newcommand{\zbb}{\mathbf{\bar{z}}}
\newcommand{\Abb}{\mathbf{\bar{A}}}
% \newcommand{\Bbb}{\mathbf{\bar{B}}}
\newcommand{\Cbb}{\mathbf{\bar{C}}}
\newcommand{\Dbb}{\mathbf{\bar{D}}}
\newcommand{\Ebb}{\mathbf{\bar{E}}}
\newcommand{\Fbb}{\mathbf{\bar{F}}}
\newcommand{\Gbb}{\mathbf{\bar{G}}}
\newcommand{\Hbb}{\mathbf{\bar{H}}}
\newcommand{\Ibb}{\mathbf{\bar{I}}}
\newcommand{\Jbb}{\mathbf{\bar{J}}}
\newcommand{\Kbb}{\mathbf{\bar{K}}}
\newcommand{\Lbb}{\mathbf{\bar{L}}}
\newcommand{\Mbb}{\mathbf{\bar{M}}}
\newcommand{\Nbb}{\mathbf{\bar{N}}}
\newcommand{\Obb}{\mathbf{\bar{O}}}
\newcommand{\Pbb}{\mathbf{\bar{P}}}
\newcommand{\Qbb}{\mathbf{\bar{Q}}}
\newcommand{\Rbb}{\mathbf{\bar{R}}}
\newcommand{\Sbb}{\mathbf{\bar{S}}}
\newcommand{\Tbb}{\mathbf{\bar{T}}}
\newcommand{\Ubb}{\mathbf{\bar{U}}}
\newcommand{\Vbb}{\mathbf{\bar{V}}}
\newcommand{\Wbb}{\mathbf{\bar{W}}}
\newcommand{\Xbb}{\mathbf{\bar{X}}}
\newcommand{\Ybb}{\mathbf{\bar{Y}}}
\newcommand{\Zbb}{\mathbf{\bar{Z}}}


\newcommand{\E}{\mathbb{E}}
\newcommand{\Prb}{\mathbb{P}}
\newcommand{\R}{\mathbb{R}}
\newcommand{\C}{\mathbb{C}}
\newcommand{\I}{\mathbb{I}}
\newcommand{\F}{\mathbb{F}}
\newcommand{\vecc}{\mathbf{c}}
\newcommand{\Z}{\mathbb{Z}}
\newcommand{\nul}{\text{null }}
\newcommand{\range}{\text{range }}
\newcommand{\Span}{\text{span}}
\newcommand{\tr}{\text{tr}}
\newcommand{\diag}{\text{diag}}
\newcommand{\limn}{\lim_{n \to \infty}}
\newcommand{\zo}{\overline{z}}
\newcommand{\wo}{\overline{w}}
\newcommand{\ao}{\overline{a}}
\newcommand{\vo}{\overline{v}}
\newcommand{\glo}{\overline{\lambda}}
\newcommand{\Var}{\text{Var}}
\newcommand{\1}{\mathbbm{1}}
\newcommand{\Pm}{\mathbbm{P}}
\usepackage{hyperref}


\title{\vspace{-2em}Chapter 5: Eigenvalues and Eigenvectors}
\author{\emph{Linear Algebra Done Right}, by Sheldon Axler}
\date{}

\begin{document}
\maketitle 
\tableofcontents

\newpage 

%%%%%%%%%%%%%%%%%%%%%%%%%%%%%%%%%%%%%%%%%%%%%%%%%%%%%%%%%%%%
%%%%%%%%%%%%%%%%%%%%%%%%%%%% 5A %%%%%%%%%%%%%%%%%%%%%%%%%%%%
%%%%%%%%%%%%%%%%%%%%%%%%%%%%%%%%%%%%%%%%%%%%%%%%%%%%%%%%%%%%

\section*{5A: Invariant Subspaces}
\addcontentsline{toc}{section}{5A: Invariant Subspaces}

\begin{definition}[operator]
    A linear map from a vector space to itself is called an \textbf{operator}. 
\end{definition}

\begin{definition}[invariant subspace]
    Suppose \(T \in \Lc((V))\). A subspace \(U\) of \(V\) is called \textbf{invariant}
    under \(T\) if \(Tu \in U\) for every \(u \in U\).
\end{definition}

\begin{remark}
    Four common invairant subspaces of \(T \in \Lc(V)\) are: \\ 
    \(\{0\}, V, \nul T, \range T\). To discover invariant subspaces 
    other than \(\{0\}, V\) (the simplest possible nontrivial 
    one is the invariant subspace of dimension one) 
    motivates the concept 
    of eigenspaces. 
\end{remark}


\begin{definition}[eigenvalue]
    Suppose \(T \in \Lc(V)\). A number \(\lambda \in \F\) is called an \textbf{eigenvalue}
    of \(T\) if there exists \(v \in V\) such that \(v \neq 0\) and \(Tv=\lambda v\). 
\end{definition}

\begin{remark}
    Note that here we can think of \(U = \{\lambda v \colon \lambda \in \F\} = \text{span}(v)\)
    as the one-dimensional invariant subspace spanned by the eigenvector \(v\). 
\end{remark}


\begin{lemma}[equivalent conditions to be an eigenvalue]
    Suppose \(V\) is finite-dimensional, \(T \in \Lc(V)\) and \(\lambda \in \F\). Then 
    the following are eequivalent:
    \begin{enumerate}[label=(\alph*)]
        \item \(\lambda\) is an eigenvalue of \(T\).
        \item \(T - \lambda I\) is not injective. 
        \item \(T - \lambda I\) is not surjective. 
        \item \(T - \lambda I\) is not invertible. 
    \end{enumerate}
\end{lemma}



\begin{definition}[eigenvector]
    Suppose \(T \in \Lc(V)\) and \(\lambda \in \F\) is an eigenvalue of \(T\). A vector 
    \(v \in V\) is called an \textbf{eigenvector} of \(T\) corresponding to \(\lambda\)
    if \(v \neq 0\) and \(Tv = \lambda v\). 
\end{definition}

\begin{remark}
    A vector \(v \in V\) with \(v \neq 0\) is an eigenvector of \(T\) corresponding to \(\lambda\)
    if and only if \(v \in \nul (T - \lambda I)\). 

    \[Tv = \lambda v \ \ \Leftrightarrow \ \ (T - \lambda I) v = 0\]
\end{remark}


\begin{lemma}[linearly independent eigenvectors]
    Suppose \(T \in \Lc(V)\). Then every list of eigenvectors of \(T\) corresponding to 
    distinct eigenvalues of \(T\) is linearly independent.
\end{lemma}


\begin{thm}[operator cannot have more eigenvalues than dimension of vector space]
    Suppose \(V\) is finite-dimensional. Then each operator on \(V\) has at most 
    \(\dim V\) distinct eigenvalues. 
\end{thm}


\begin{definition}[\(T^m\)]
    Suppose \(T \in \Lc(V)\) and \(m\) is a postive integer. 
    \begin{itemize}
        \item \(T^m \in \Lc(V)\) is defined by \(T^m = \underbrace{T\cdots T}_{m \text{ times}}\).
        \item \(T^0\) is defined to be the identity operator \(I\) on \(V\). 
        \item If \(T\) is invertible with inverse \(T^{-1}\), then \(T^{-m} \in \Lc(V)\) is defined by 
        \[T^{-m} = (T^{-1})^m.\]
    \end{itemize}
\end{definition}



\begin{definition}[\(p(T)\)]
    Suppose \(T \in \Lc(V)\) and \(p \in \Pc(\F)\) is a polynomial given by 
    \[p(z) = a_0 + a_1 z + a_2 z^2 + \cdots + a_m z^m\]
    for all \(z \in \F\). Then \(p(T)\) is the operator on \(V\) defined by 
    \[p(T) = a_0I + a_1 T + a_2 T^2 + \cdots + a_m T^m.\]
\end{definition}

\begin{remark}
    If we fix an operator \(T \in \Lc(V)\), then the function from \(\Pc(\F)\) to \(\Lc(V)\)
    given by \(p \mapsto p(T)\) is linear. 
\end{remark}


\begin{definition}[product of polynomials]
    If \(p, q \in \Pc(\F)\), then \(pq \in \Pc(\F)\) is the polynomial defined by 
    \[(pq)(z) = p(z)q(z)\]
    for all \(z \in \F\).
\end{definition}

\begin{thm}
    Suppose \(p, q \in \Pc(\F)\) and \(T \in \Lc(V)\). Then 
    \begin{enumerate}[label=(\alph*)]
        \item \((pq)(T) = p(T) q(T)\);
        \item \(p(T)q(T) = q(T)p(T)\).
    \end{enumerate}
\end{thm}

\begin{definition}[null space and range of \(p(T)\) are invariant under \(T\)]
    Suppose \(T \in \Lc(V)\) and \(p \in \Pc(\F)\). Then \(\nul p(T)\) and \(\range p(T)\)
    are invariant under \(T\).
\end{definition}






%%%%%%%%%%%%%%%%%%%%%%%%%%%%%%%%%%%%%%%%%%%%%%%%%%%%%%%%%%%%%%%
%%%%%%%%%%%%%%%%%%%%%%%%%%%% 5A PS %%%%%%%%%%%%%%%%%%%%%%%%%%%%
%%%%%%%%%%%%%%%%%%%%%%%%%%%%%%%%%%%%%%%%%%%%%%%%%%%%%%%%%%%%%%%

\newpage 
\addcontentsline{toc}{subsection}{5A Problem Sets}

\begin{problem}{1}
    Suppose \(T \in \Lc(V)\) and \(U\) is a subspace of \(V\). 
    \begin{enumerate}[label=(\alph*)]
        \item Prove that if \(U \subseteq \nul T\), then \(U\) is invariant under \(T\). 
        \item Prove that if \(\range T \subseteq U\), then \(U\) is invariant under \(T\).
    \end{enumerate}
\end{problem}

\begin{proof}
\begin{enumerate}[label=(\alph*)]
    \item  Suppose for contradiction that \(U\) is not invariant under \(T\). Then this means 
    that there exists \(u \in U\) such that \(T u \notin U\). We know that \(Tu = 0 \in \nul T\). 
    As \(U \in \nul T\), this forms a contradiction. 
    
    \item Take \(u \in U\), then \(Tu \in \range T \subseteq U\), and thus \(Tu \in U\).
\end{enumerate}
\end{proof}

\begin{problem}{3}
    Suppose \(T \in \Lc(V)\). Prove that the intersection of every collection of 
    subspaces of \(V\) invariant under \(T\) is invariant under \(T\).
\end{problem}

\begin{proof}
Suppose \(\{V_i\}_{i=1}^m\) for positive integer \(m\) is a collection of invariant subspaces. 
We hope to prove that \(\bigcap_{i=1}^m V_i\) is invariant. 

Take any \(v \in \bigcap_{i=1}^m V_i\), then we know that \(Tv \in V_i\) for each \(i\)
and thus \(Tv \in \bigcap_{i=1}^m V_i\).
\end{proof}

\begin{problem}{5}
    Suppose \(T \in \Lc(\R^2)\) is defined by \(T(x, y) = (-3y, x)\). Find the eigenvalues 
    of \(T\).
\end{problem}

\begin{proof}
We have that 
\[T(x, y) = (-3 y, x) = \lambda(x, y)\]

Then this means that \(-3y = \lambda x, x= \lambda y\). Solving this gives that \(\lambda^2 = -3\)
and thus \(\lambda = \pm \sqrt{3}i\). 
\end{proof}

\begin{problem}{7}
    Define \(T \in \Lc(\F^3)\) by \(T(z_1, z_2, z_3) = (2z_2, 0, 5z_3)\). Find all eigenvalues 
    and eigenvectors of \(T\).
\end{problem}


\begin{proof}
\[T(z_1, z_2, z_3) = (2z_2, 0, 5z_3) = \lambda(z_1, z_2, z_3)\]

Then \(\lambda z_1 = 2z_2, \lambda z_2 = 0, \lambda z_3 = 5 z_3\). This gives that \(\lambda=5 \text{ or } 0\).
For \(\lambda = 0\), the eigenvector would be \(\{(z_1, 0, z_3) \colon z_1, z_3 \in \F\}\). 
For \(\lambda = 5\), we have that \(\{(0, 0, z_3) \colon z_3 \in \F\}\). 


\end{proof}


\begin{problem}{9}
    Define \(T \colon \Pc(\R) \to \Pc(\R)\) by \(Tp = p'\). Find all eigenvalues and 
    eigenvectors of \(T\).
\end{problem}


\begin{proof}
\[Tp = p' = \lambda p\]
This means that \(\deg p' = \deg p\) so \(\deg p = 0\). In this case, the only satisifying 
solution is \(\lambda = 0\) and correspondingly \(v = \{(c) \colon c \in \F\}\).
\end{proof}


\begin{problem}{11}
    Suppose \(V\) is finite-dimensional, \(T \in \Lc(V)\), and \(\alpha \in \F\). Prove that 
    there exists \(\delta > 0\) such that \(T - \lambda I\) is invertible for all \(\lambda \in \F\)
    such that \(0 < |\alpha - \lambda| < \delta\).
\end{problem}

\begin{proof}
Let \(\lambda^*\) be the eigenvalue of \(T\) which is cloest to \(\alpha\), i.e. \(\lambda^* = \min_\lambda\{
    |\lambda - \alpha|\}\). We can take \(\delta = |\alpha - \lambda^*|/2\), then we know that 
    under the condition \(T - \lambda I\) is injective and thus reaching the desired result. 
\end{proof}

\begin{problem}{13}
    Suppose \(T \in \Lc(V)\). Suppose \(S \in \Lc(V)\) is invertible. 
    \begin{enumerate}[label=(\alph*)]
        \item Prove that \(T\) and \(S^{-1}TS\) have the same eigenvalues. 
        \item What is the relationship between the eigenvectors of \(T\) and the eigenvectors of \(S^{-1}TS\)?
    \end{enumerate}
\end{problem}

\begin{proof}
\begin{enumerate}[label=(\alph*)]
    \item Notice that \(S^{-1}v\) is an eigenvector of \(S^{-1}TS\) for \(v \in V\). For the
    forward direction, take \(\lambda\) to be the eigenvalue of \(T\). Then we know that \(Tv = 
    \lambda v\) for all \(v \in V\). Let's consider \(S^{-1}TS (S^{-1}u) = S^{-1}Tu = S^{-1} \lambda u 
    = \lambda (S^{-1} u)\). Therefore, \(\lambda\) is also an eigenvalue for \(S^{-1}TS\). 
    Conversely, take \(\lambda\) to be the eigenvalue of \(S^{-1}TS\). Then we have that 
    \(S^{-1}TS (S^{-1}v) = S^{-1}Tv = \lambda S^{-1}v\). Multiplying \(S\) on both sides yield 
    that \(Tv = \lambda v\). Thus \(\lambda\) is also an eigenvalue of \(T\). 
    
    \item  \(v\) is an eigenvector of \(S^{-1}TS\) iff \(u = Sv\) is an eigenvector of \(T\).
\end{enumerate}
\end{proof}

\begin{problem}{15}
    Suppose \(V\) is finite-dimensional, \(T \in \Lc(V)\), and \(\lambda \in \F\). Show that 
    \(\lambda\) is an eigenvalue of \(T\) if and only if \(\lambda\) is an eigenvalue of 
    the dual operator \(T' \in \Lc(V')\).
\end{problem}


\begin{proof}
Forward direction, take \(\lambda\) to be an eigenvalue of \(T\), then we have that \(Tv = \lambda v\). 
Then take any \(\varphi \in V'\), we have that \(T'(\varphi) (v)= \varphi \circ Tv 
= \lambda \varphi (v)\). 

Backward direction, take \(\lambda\) to be an eigenvalue of \(T'\), then we know there exists 
nonzero \(\varphi \in V'\) such that 
\(T' \varphi (v) = \varphi T(v) = \lambda \varphi (v)\) 
This implies that 
\[\varphi(Tv - \lambda v) = 0\]

Suppose for contradiction that \(\lambda\) is not an eigenvalue of \(T\), then this means that 
\(T - \lambda I\) is invertible and therefore its image equals the whole space \(V\). Then this 
means that \(\varphi\) is a zero functional, forming a contradiction. Hence, \(Tv = \lambda v\)
and thus \(\lambda\) is an eigenvalue of \(T\) as well. 

% Since \(\varphi \neq 0\), we have that \(Tv - \lambda v = 0\) and thus \(\lambda\) is also an 
% eigenvalue of \(T'\).

\end{proof}


\begin{problem}{16}
    Supose \(v_1, \ldots, v_n\) is a basis of \(V\) and \(T \in \Lc(V)\). Prove that if 
    \(\lambda\) is an eigenvalue of \(T\), then 
    \[|\lambda| \leq n \max \{|\Mc(T)_{j,k}|\colon 1 \leq j,k \leq n\},\]
    where \(\Mc(T)_{j,k}\) denotes the entry in row \(j\), column \(k\) of the matrix of 
    \(T\) with respect to the basis \(v_1, \ldots, v_n\). 
\end{problem}

\begin{proof}
We know that 
\[T v_k = \sum_{j=1}^{n} \Mc(T)_{j, k} v_j = \lambda v_k\]
and that by Triangular inequality we have 
\[|Tv_k| \leq \sum_{j=1}^{n} |\Mc(T)_{j, k} v_j| \leq M \sum_{j=1}^{n} |v_j|\]
where we take \(M = \max \{|\Mc(T)_{j,k}|\colon 1 \leq j,k \leq n\}\). Here we take 
\(k\) to be the index such that maximizes the norm, i.e. \(k = \max_k |v_k|\). Then we have that 
\[|T v_k| = |\lambda| |v_k| \leq nM |v_k|\]
which gives us that 
\[|\lambda| \leq nM\]
\end{proof}


\begin{problem}{18}
    Suppose \(\F = \R, T \in \Lc(V)\), and \(\lambda \in \C\). Prove that \(\lambda\) is 
    an eigenvalue of the complexification \(T_C\) if and only if \(\overline{\lambda}\) 
    is an eigenvalue of \(T_C\), where \(T_C \colon \Lc(V_C)\) is defined by 
    \[T_C(u + iv) = Tu + iTv\]
    for all \(u, v \in V\).
\end{problem}

\begin{proof} Let \(v_1, \ldots, v_n\) be basis of \(V_C\).
Suppose \(\lambda\) is an eigenvalue of \(T_C\) and \(v = \sum_{i=1}^{n}a_i v_i\) is the 
corresponding eigenvector. Then we have that 
\[\lambda v = Tv = \sum_{i=1}^{n}a_i T v_i\] 

Taking conjugates yields that 

\[\glo \vo = \overline{\sum_{i=1}^{n} a_i T v_i} = \sum_{i=1}^{n} \ao_i \overline{T v_i}\]
Let \(M\) be the matrix of \(T\) with real entries, and then we have 
\[\overline{Tv_i} = \overline{\sum_{j=1}^{n}M_{j, i} v_j} = \sum_{j=1}^{n}M_{j, i} \vo_j 
= T \vo_i\]

Substitue this back to previous equation gives that 
\[\glo \vo = T(\overline{\sum_{i=1}^{n} a_i v_i}) = T \vo \]

Hence \(\glo\) is also an eigenvector. 
\end{proof}


\begin{problem}{19}
    Show that the forward shift operator \(T \in \Lc(\F^\infty)\) defined by 
    \[T(z_1, z_2, \cdots) = (0, z_1, z_2, \cdots)\]
    has no eigenvalues. 
\end{problem}

\begin{proof}
We have that 
\[(0, z_1, z_2, \cdots) = \lambda (z_1, z_2, \cdots)\]
So we either have \(\lambda =0\) or \(z_1 = 0 = z_2 = \cdots\). In the first case, we have 
that \(z_1 = 0 = z_2 = \cdots\) and thus there is no nonzero eigenvector. In the second case, 
there is also no nonzero eigenvectors. 
\end{proof}



\begin{problem}{21}
    Suppose \(T \in \Lc(V)\) is invertible. 
    \begin{enumerate}[label=(\alph*)]
        \item Suppose \(\lambda \in \F\) with \(\lambda \neq 0\). Prove that 
        \(\lambda\) is an eigenvalue of \(T\) if and only if \(\frac{1}{\lambda}\) 
        is an eigenvalue of \(T^{-1}\). 

        \item Prove that \(T\) and \(T^{-1}\) have the same eigenvectors. 
    \end{enumerate}
\end{problem}

\begin{proof}
%  Let \(\lambda\) be eigenvalue and \(v\) be eigenvector of \(T\). 
\begin{align*}
    Tv  = \lambda v  \Longleftrightarrow T^{-1} T v = T^{-1} \lambda v \Longleftrightarrow 
    \frac{1}{\lambda} v = T^{-1} v   
\end{align*}
\end{proof}


\begin{problem}{23}
    Suppose \(V\) is finite-dimensional and \(S, T \in \Lc(V)\). Prove that \(ST\) and \(TS\)
    have the same eigenvalues. 
\end{problem}

\begin{proof}
\(\Rightarrow\) Let \(\lambda, v\) be the eigenvalue and eigenvector of \(ST\). Then take 
\(u = Tv\), we have that 
\[TS(u) = TS(Tv) = T(ST)(v) = T (\lambda v) = \lambda (Tv) = \lambda u\]
Thus \(\lambda\) is also the eigenvalue of \(TS\). 

Conversely, let \(\lambda, v\) be eigenvalue and eigenvector of \(TS\). Then take 
\(u = Sv\), we have that 
\[ST(u) = ST(Sv) = S(TS) (v) = S(\lambda v) = \lambda (SV) = \lambda u\]
Thus \(\lambda\) is also the eigenvalue of \(ST\). 
\end{proof}


\begin{problem}{24}
    Suppose \(A\) is an n-by-n matrix with entries in \(\F\). Define 
    \(T \in \Lc(\F^n)\) by \(Tx = Ax\), where elements of \(\F^n\) are thought of 
    as n-by-1 column vectors. 
    \begin{enumerate}[label=(\alph*)]
        \item Suppose the sum of the entries in each row of \(A\) equals 1. Prove that 
        1 is an eigenvalue of \(T\).
        \item Suppose the sum of entries in each column of \(A\) equals 1. Prove that 
        1 is an eigenvalue of \(T\).
    \end{enumerate}
\end{problem}

\begin{proof}
Let \(x = (1, \ldots, 1)^\top\) be the all-one vector. Then we have that 
\[(Tx)_{i, \colon} = (Ax)_{i, \colon} = \sum_{j=1}^{n} A_{i, j} x_j = \sum_{j=1}^{n} A_{i, j} = 1\]
Therefore, \(Tx = x\) and thus 1 is an eigenvalue of \(T\). 

For the column case, we can simply take \(A = A^\top\) and we know the eigenvalues of \(A\) equal 
\(A^\top\). 

\end{proof}



\begin{problem}{25}
    Suppose \(T \in \Lc(V)\) and \(u, w\) are eigenvectors of \(T\) such that 
    \(u + w\) is also an eigenvector of \(T\). Prove that \(u\) and \(w\) are eigenvectors 
    of \(T\) corresponding to the same eigenvalue.  
\end{problem}


\begin{proof}
Suppose not. Then there exists \(\lambda_u\) and \(\lambda_w\) such that 
\[Tu = \lambda_u u \ \ \ Tw = \lambda_w w\]

We also know that 
\[T(u+w) = Tu + Tw = \lambda_u u + \lambda_w w = \lambda(u + w)\]

for some \(\lambda\). Solving this gives that \(\lambda = \lambda_u = \lambda_w\).

\end{proof}


\begin{problem}{26}
    Suppose \(T \in \Lc(V)\) is such that every nonzero vector in \(V\) is an eigenvector of 
    \(T\). Prove that \(T\) is a scalar multiple of the identity operator. 
\end{problem}

\begin{proof}
We claim that \(Tv = \lambda v\) for all nonzero \(v \in V\), i.e. \(T = \lambda I\). Let's 
get \(v_1, v_2 \in V\). First we consider the case that they are linearly dependent, meaning 
that \(v_1 = c v_2\) for some \(c\). We have that 
\[\lambda_1 v_1 = Tv_1 = c Tv_2 = c \lambda_2 v_2 = \lambda_2 v_1\]
Therefore \(\lambda_1 = \lambda_2\). Next we consider the case that they are linearly independent, 
then we have that 
\[ \lambda_{1+2} (v_1 + v_2) = T (v_1 + v_2) = Tv_1 + Tv_2 = \lambda_1 v_1 + \lambda_2 v_2 \]
Therefore \(\lambda_{1+2} = \lambda_1  = \lambda_2\), completing the proof. 
\end{proof}


\begin{problem}{28}
    Suppose \(V\) is finite-dimensional and \(T \in \Lc(V)\). Prove that \(T\) has at most 
    \(1 + \dim \range T\) distinct eigenvalues. 
\end{problem}


\begin{proof}
Suppose \(T\) has \(m\) distinct eigenvalues. For nonzero \(\lambda_k\), we have that 
\[T(\frac{1}{\lambda_k} v_k) = v_k\]
Thus at most \(m-1\) distinct eigenvectors are in \(\range T\), so \(m - 1 \leq \dim \range T\)
and \(m = 1 + \dim \range T\) at most. 
\end{proof}


% \begin{problem}{32}
%     Suppose \(T \in \Lc(V)\) has no eigenvalues and \(T^4 = I\). Prove that \(T^2 = -I\).
% \end{problem}


% \begin{proof}

% \end{proof}


\begin{problem}{33}
    Suppose \(T \in \Lc(V)\) and \(m\) is a positive integer. 
    \begin{enumerate}[label=(\alph*)]
        \item Prove that \(T\) is injective if and only if \(T^m\) is injective. 
        \item Prove that \(T\) is surjective if and only if \(T^m\) is surjective.
    \end{enumerate}
\end{problem}

\begin{proof}
(a) \(\Rightarrow\) Given \(T\) is injective, take \(v \in \nul T^m\). Then we know that 
\(T^m(v) = T(T^{m-1}v) = 0\), which implies that \(T^{m-1}v = 0\). This can be recursively 
deduced to that \(Tv = 0\) and thus \(v = 0\). Therefore \(T^m\) is injective. 

\(\Leftarrow\) Conversely, take \(v \in \nul T\), then we have \(T^m(v)
=T^{m-1}(Tv) = T^{m-1}(0) = 0\). Since \(T^m\) is injective, \(v=0\). 

(b) \(\Rightarrow\) Given \(T\) is surjective, then \(\dim range T = \dim V\). We also have 
\(\dim \range T^2 = \dim V\) as \(T\) is surjective. Therefore, \(T^m\) is surjective. 

\(\Leftarrow\) Let \(w \in V\), then there exists \(v \in V\) such that \(T^m(v) = w\). Then 
we have that let \(u = T^{m-1}(v)\), then \(Tu = w\). Thus \(T\) is also surjective.  
\end{proof}



\begin{problem}{34}
    Suppose \(V\) is finite-dimensional and \(v_1, \ldots, v_m \in V\). Prove that 
    the list \(v_1, \ldots, v_m\) is linearly independent if and only if there 
    exists \(T \in \Lc(V)\) such that \(v_1, \ldots, v_m\) are eigenvectors of \(T\)
    corresponding to distinct eigenvalues. 
\end{problem}

\begin{proof}
\(\Rightarrow\) We can define \(Tv_k = k v_k\) and extend \(T\) to \(V\). 

\(\Leftarrow\) This is proved by Theorem in the book.
\end{proof}

\begin{problem}{35}
    Suppose that \(\lambda_1, \ldots, \lambda_n\) is a list of distinct real numbers. Prove that 
    the list \(e^{\lambda_1 x}, \ldots, e^{\lambda_n x}\) is linearly independent in the vector 
    space of real-valued functions on \(\R\).
\end{problem}

\begin{proof}
Following the hint, let \(V = \text{span}(e^{\lambda_1 x}, \ldots e^{\lambda_n x})\). Define 
an operator \(D \in \Lc(V)\) by \(Df = f'\). Then we have that 
\[D e^{\lambda_k x} = \lambda_k e^{\lambda_k x}\]

Here naturally \(\lambda_k\) is a distinct eigenvalue with corresponding eigenvector \(e^{\lambda_k x}\). 
By P34, we complete the proof. 
\end{proof}


\begin{problem}{37}
    Suppose \(V\) is finite-dimensional and \(T \in \Lc(V)\). Define \(\Ac \in \Lc(\Lc(V))\) by 
    \[\Ac(S) = TS\]
    for each \(S \in \Lc(V)\). Prove that the set of eigenvalues of \(T\) equals the set of 
    eigenvalues of \(\Ac\).
\end{problem}

\begin{proof}
    \(\Rightarrow\) Let \(\lambda\) be an eigenvalue of \(T\) and \(v\) be the corresponding 
    eigenvector of \(T\).  Suppose \(v_1, \ldots, v_n\) is the basis of \(V\), we can define 
    \(S \in \Lc(V)\) such that 
    \[S v_i = v\]
    for all \(i\). Here we have that 
    \begin{align*}
        \Ac(S)(v) 
        &= TS(v) \\ 
        &= TS \sum_{i=1}^{n} a_i v_i \\ 
        &= T\sum_{i=1}^{n} a_i Sv_i \\ 
        &= \sum_{i=1}^{n} a_i Tv \\ 
        &= \lambda \sum_{i=1}^{n} a_i v  \\ 
        &= \lambda \sum_{i=1}^{n} S (a_i v_i) \\ 
        &= (\lambda S) (v) 
    \end{align*}
    Hence \(\lambda\) is also an eigenvalue of \(\Ac(S)\).


    \(\Leftarrow\) Consider 
    \[\Ac(S) = TS = \lambda S\] 
    Applying \(v\) on both sides yields that 
    \[T(Sv) = \lambda(Sv)\]
    Therefore \(\lambda\) is also an eigenvalue of \(T\).
\end{proof}


\begin{problem}{38}
    Suppose \(V\) is finite-dimensional, \(T \in \Lc(V)\), and \(U\) is a subspace 
    of \(V\) invariant under \(T\). The \emph{quotient operator} \(T/U \in \Lc(V/U)\) 
    is defined by 
    \[(T/U)(v + U) = Tv + U\]
    for each \(v \in V\). 
    \begin{enumerate}[label=(\alph*)]
        \item Show that the definition of \(T/U\) makes sense and show that \(T/U\) is an operator 
        on \(V/U\).
        \item Show that each eigenvalue of \(T/U\) is an eigenvalue of \(T\). 
    \end{enumerate}
\end{problem}

\begin{proof}
(a) Suppose \(v - w \in U\). As \(U\) is invairant under \(T\), \(T(v - w) = Tv - Tw \in U\). 
So we have that \(Tv + U = Tw + U\) and thus it is well-defined. To show that \(T/U\) is an operator, 
we can just check its linearity. \((T/U)( (\lambda v_1 + U) + (v_2 + U)  ) 
= (T/U)((\lambda v_1 + v_2) + U) = \lambda_1 Tv_1 + U + T v_2 + U = \lambda_1 T/U(v_1 + U) 
+ T/U (v_2 + U)\). 

(b) Let \(\lambda\) be the eigenvalue of \(T/U\), then we have 
\[(T/U)(v+U) = Tv + U = \lambda v + U\]
This means that 
\[Tv = \lambda v + u\]
for some \(u \in U\). Now suppose we also have \(u' \in U\), then 
\[T(v + u') = \lambda v + u + Tu'\]

We hope to find \(u'\) s.t. \(u + Tu' = \lambda u'\), equivalently \((\lambda I - T)u' = u\).

Since \(U\) is invariant under \(T\), it is also invariant under \(T - \lambda I\). Here 
if \(T - \lambda I\) is not invertible on \(U\), then we are done (\(\lambda\) would be 
an eigenvalue). If it is, we can define \(u' = -(T - \lambda I)^{-1}u\), completing the proof.

\end{proof}


\begin{problem}{39}
    Suppose \(V\) is finite-dimensional and \(T \in \Lc(V)\). Prove that \(T\) has an eigenvalue 
    if and only if there exists a subspace of \(V\) of dimension \(\dim V -1\) that is invariant 
    under \(T\).
\end{problem}


\begin{proof}

% We know that \(T\) has an eigenvalue if and only if the quotient operator \(T/U\) has an eigenvalue. Then 
% we can now prove that \(T/U\) has an eigenvalue if and only if there exists a subspace of \(V\) of 
% dimension \(\dim V - 1\) that is invariant under \(T/U\)

\(\Rightarrow\) By the hypothesis, ther exists \(\lambda \in \F\), \(v \in V\) with \(v \neq 0\) s.t. 
\(Tv = \lambda v\). We claim that 
\[v^\perp = \{u \in V \colon v^\top u = 0\}\]

is invairant under \(T\) of dimension \(n - 1\). Take any \(u \in v^\perp\), then 
\(v^\top T(u) = (v^\top T)(u) = \lambda v^\top u = 0\). Hence \(T(u) \in v^\perp\). This set has 
dimension \(\dim V - 1\) as it is the orthogonal complement of a dimensional 1 subspace. 

\(\Leftarrow\) Let \(U\) be the the invariant subspace of \(\dim V - 1\) under \(T\). Let 
\(\{w_1, \ldots, w_{n-1}\}\) be its basis. Then we can extend the basis to \(\{w_1, \ldots, w_{n-1}, v\}\)
for some \(v \in V \backslash U\). Hence, we have that 
\[T(v) = \sum_{i=1}^{n-1} a_i w_i + c v\]
for some scalar \(c \in \F\). We claim that \(c\) is the eigenvalue of \(T\). Suppose not, then we can get that 
\[(T - cI)(v) = \sum_{i=1}^{n-1} a_i w_i\]
Then this means that \(v \in U\), reaching a contradiction.

\end{proof}


\begin{problem}{40}
    Suppose \(S, T \in \Lc(V)\) and \(S\) is invertile. Suppose \(p \in \Pc(\F)\) is a polynomial. Prove that 
    \[p(STS^{-1}) = S p(T)S^{-1}\]
\end{problem}

\begin{proof}
We note that 
\[(STS^{-1})^m = (STS^{-1})(STS^{-1}) \cdots (STS^{-1}) = ST^m S^{-1}\]

Therefore, 
\begin{align*}
    p(STS^{-1})
    &= a_0I + a_1(STS^{-1}) + \cdots + a_m(STS^{-1})^m \\ 
    &= a_0I + a_1(STS^{_1}) + \cdots + a_m(ST^mS^{-1}) \\ 
    &= S(a_0 I + a_1 T + \cdots + a_m T^m)S^{-1} \\ 
    &= Sp(T)S^{-1}
\end{align*}
\end{proof}


\begin{problem}{41}
    Suppose \(T \in \Lc(V)\) and \(U\) is a subspace of \(V\) invariant under \(T\). Prove that \(U\) is invariant 
    under \(p(T)\) for every polynomial \(p \in \Lc(\F)\).
\end{problem}

\begin{proof}
Take \(u \in U\), then 
\begin{align*}
    p(T)(u) 
    &= a_0I(u) + a_1T(u) + \cdots + a_m T^m(u)
\end{align*}

Now it suffices to prove that \(T^m(u) \in U\) for all positive integer. To see this, we can make an inductive 
argument on \(m\) and gets the desired conclusion.

\end{proof}


% \begin{problem}{43}
%     Suppose \(V\) is finite-dimensional, \(\dim V > 1\), and \(T \in \Lc(V)\). Prove that 
%     \(\{p(T) \colon p \in \Pc(\F)\} \neq \Lc(V)\). 
% \end{problem}

% \begin{proof}

% \end{proof}


%%%%%%%%%%%%%%%%%%%%%%%%%%%%%%%%%%%%%%%%%%%%%%%%%%%%%%%%%%%%
%%%%%%%%%%%%%%%%%%%%%%%%%%%% 5B %%%%%%%%%%%%%%%%%%%%%%%%%%%%
%%%%%%%%%%%%%%%%%%%%%%%%%%%%%%%%%%%%%%%%%%%%%%%%%%%%%%%%%%%%


\newpage 
\section*{5B: The Minimal Polynomial}
\addcontentsline{toc}{section}{5B: The Minimal Polynomial}

\begin{thm}[existence of eigenvalues]
    Every operator on a finite-dimensional nonzero complex vector space has an eigenvalue. 
\end{thm}


\begin{definition}[monic polynomial]
    A \textbf{monic polynomial} is a polynomial whose highest-degree coefficient 
    equals \(1\). 
\end{definition}

\begin{thm}[existence, uniqueness, and degree of minimal polynomial]
    Suppose \(V\) is finite-dimensional and \(T \in \Lc(V)\). Then there is a unique 
    monic polynomial \(p \in \Pc(\F)\) of smallest degree such that \(p(T) = 0\). 
    Furthermore, \(\deg p \leq \dim V\). 
\end{thm}


\begin{definition}[minimal polynomial]
    Suppose \(V\) is finite-dimensional and \(T \in \Lc(V)\). Then the \textbf{minimal polynomial} of \(T\)
    is the unique monic polynomial \(p \in \Pc(\F)\) of smallest degree such that \(p(T) = 0\).
\end{definition}

\begin{remark}
    This means to find the smallest positive integer \(m\) such that 
    \[c_0 I + c_1 T + \cdots + c_{m-1}T^{m-1} = -T^{m}\]
    has a solution \(c_0, c_1, \ldots, c_{m-1} \in \F\). A way to solve this is to pick \(v \in V\) 
    with \(v \neq 0\) and consider that 
    \[c_0 v + c_1 Tv + \cdots + c_{\dim V - 1} T^{\dim V - 1} v = - T^{\dim V}v.\]
\end{remark}






\begin{thm}[eigenvalues are the zeros of the minimal polynomial]
    Suppose \(V\) is finite-dimensional and \(T \in \Lc(V)\). 
    \begin{enumerate}[label=(\alph*)]
        \item The zeros of the minimal polynomial of \(T\) are the eigenvalues of \(T\). 
        \item If \(V\) is a complex vector space, then the minimal polynomial of \(T\) has the 
        form 
        \[(z - \lambda_1) \cdots (z-\lambda_m).\]
        where \(\lambda_1, \ldots, \lambda_m\) is a list of all eigenvalues of \(T\), possibly with repetitions.
    \end{enumerate}
\end{thm}


\begin{thm}[\(q(T) = 0 \Leftrightarrow q\) is a polynomial multiple of the minimal polynomial]
    Suppose \(V\) is finite-dimensional, \(T \in \Lc(V)\), and \(q \in \Pc(\F)\). Then 
    \(q(T) = 0\) if and only if \(q\) is a polynomial multiple of the minimal polynomial of \(T\).
\end{thm}


\begin{thm}[minimal polynomial of a restriction operator]
    Suppose \(V\) is finite-dimensional, \(T \in \Lc(V)\), and \(U\) is a subspace of \(V\) that is invariant 
    under \(T\). Then the minimal polynomial of \(T\) is a polynomial multiple of the minimal polynomial of 
    \(T |_U\).
\end{thm}

\begin{corollary}
    Suppose \(V\) is finite-dimensional and \(T \in \Lc(V)\). Then \(T\) is not invertible if and only if the 
    constant term of the minimal polynomial of \(T\) is 0.
\end{corollary}

\begin{thm}[even-dimensional null space]
    Suppose \(\F = \R\) and \(V\) is finite-dimensional. Suppose also that \(T \in \Lc(V)\) and 
    \(b, c \in \R\) with \(b^2 < 4c\). Then \(\dim \nul(T^2 + bT +cI)\) is an even number.
\end{thm}

\begin{thm}
    Every operator on an odd-dimensional vector space has an eigenvalue. 
\end{thm}


%%%%%%%%%%%%%%%%%%%%%%%%%%%%%%%%%%%%%%%%%%%%%%%%%%%%%%%%%%%%%%%
%%%%%%%%%%%%%%%%%%%%%%%%%%%% 5B PS %%%%%%%%%%%%%%%%%%%%%%%%%%%%
%%%%%%%%%%%%%%%%%%%%%%%%%%%%%%%%%%%%%%%%%%%%%%%%%%%%%%%%%%%%%%%

\newpage 
\addcontentsline{toc}{subsection}{5B Problem Sets}


\begin{problem}{1}
    Suppose \(T \in \Lc(V)\). Prove that 9 is an eigenvalue of \(T^2\) if and only if 3 or -3 is an eigenvalue 
    of \(T\). 
\end{problem}

\begin{proof}
\(\Rightarrow\) Given 9 is an eigenvalue of \(T^2\), then this means that \((T^2 - 9I) = (T - 3I)(T+3I) = 0\). 
So either 3 or -3 is an eigenvalue of \(T\). 

\(\Leftarrow\) We know there exists nonzero \(v \in V\) s.t. \(Tv = \pm 3v\), so \(T(Tv) = T( \pm 3v) 
= \pm 3 T(v) = (\pm 3)^2 v = 9v\).

\end{proof}

\begin{problem}{2}
    Suppose \(V\) is a complex vector space and \(T \in \Lc(V)\) has no eigenvalues. Prove that every subspace 
    of \(V\) invariant under \(T\) is either \(\{0\}\) or infinite-dimensional. 
\end{problem}

\begin{proof}
Suppose for contradiction that there exists finite-dimensional nontrivial subpace \(U\) of \(V\) such that is invariant 
under \(T\). So we know that by theorem 5.31 and 5.22 \(T|_U\) has eigenvalue and so does \(T\). This forms a contradiction 
to the hypothesis.
\end{proof}


\begin{problem}{3}
    Suppose \(n\) is a positive integer and \(T \in \Lc(\F^n)\) is defined by 
    \[T(x_1, \ldots, x_n) = (x_1 + \cdots + x_n, \ldots, x_1 + \cdots + x_n)\]
    \begin{enumerate}[label=(\alph*)]
        \item Find all eigenvalues and eigenvectors of \(T\). 
        \item Find the minimal polynomial of \(T\).
    \end{enumerate}
\end{problem}


\begin{proof}
(a)  Let \(v\) be such that 
\[(x_1 + \cdots + x_n, \cdots, x_1 + \cdots + x_n) = (\lambda x_1, \ldots, \lambda x_n)\]
This means \(\sum_{i=1}^{n} x_i = \lambda x_j = \lambda x_k\) for all \(j, k\). If \(\lambda = 0\), then 
\(\sum_{i=1}^{n} x_i = 0\) and the eigenvector would be \(\{(x_1, \ldots, x_n) \colon \sum_{i=1}^{n}x_i = 0\}\). 
If \(\lambda \neq 0\), then we have that \(x_i = x_j\) for all \(i, j\) and thus \(\lambda = n\), the corresponding 
eigenvector is that \(\{(a, \ldots, a) \colon a \in \F\}\). 

(b) We know the eigenvalues are \(0; n\), so it must be of the form 
\[p(\lambda) = \lambda (\lambda - n)\]


\end{proof}


\begin{problem}{4}
    Suppose \(\F = \C, T \in \Lc(V), p \in \Pc(\C)\), and \(\alpha \in \C\). Prove that \(\alpha\) is an 
    eigenvalue of \(p(T)\) if and only if \(\alpha = p(\lambda)\) for some eigenvalue \(\lambda\) of \(T\).
\end{problem}

\begin{proof}
\(\Rightarrow\) Given \(\alpha\) is an eigenvalue of \(p(T)\), then we know that  
\[p(z)  - \alpha = c(z - \lambda_1) \cdots (z - \lambda_m)\]
for some \(\lambda_i\)'s.  Note that since \(p\) is nonconstant \(c \neq 0\). Then this means that 
one of \(z - \lambda_i = 0 \Leftrightarrow T - \lambda_i I = 0\). Then this means \(p(\lambda_i) - \alpha = 0\)
so \(p(\lambda_i) = \alpha\). 

\(\Leftarrow\) Conversely, we have that for some nonzero \(v\) to be the eigenvector of \(T\)
\begin{align*}
    p(T)(v) 
    &= (a_0 I + a_1 T + \cdots + a_m T^m)(v) \\ 
    &= (a_0I + a_1 \lambda + \cdots + a_m \lambda^m)(v) \\ 
    &= \alpha(v)
\end{align*}
\end{proof}


\begin{problem}{6}
    Suppose \(T \in \Lc(\F^2)\) is defined by \(T(w, z) = (-z, w)\). Find the minimal polynomial of \(T\).
\end{problem}

\begin{proof}
Pick \(v = e_1\) then \(Tv = (0, 1)\) and \(T^2(v) = (-1, 0)\). We have that  
\[(1 + T^2)(v) = 0\]
\end{proof}


\begin{problem}{8}
    Suppose \(T \in \Lc(\R^2)\) is the operator of counterclockwise rotation by \(1^o\). Find the minimal 
    polynomial of \(T\). 
\end{problem}


\begin{proof}
We know that the matrix represented by the operator is that 
\[T = \begin{bmatrix}
    \cos(1^\circ)    &  -\sin(1^\circ) \\ 
    \sin(1^\circ)    &  \cos(1^\circ)
\end{bmatrix}\]

Let \(v = e_1, Tv = (\cos1^\circ, \sin 1^\circ)\) and 
\(T^2v = (\cos^2 1^\circ - \sin^21^\circ, 2\cos 1^\circ \sin 1^\circ)\). Hence we have that 

\begin{align*}
    (-1)1 + (2 \cos 1^\circ)\cos 1^\circ &= \cos^2 1^\circ - \sin^2 1^\circ  \\ 
    (-1)0 + (2 \cos 1^\circ)\sin 1^\circ &= 2 \cos 1^\circ \sin 1^\circ 
\end{align*}

Then the minimal polynomial is 
\[p(T) = T^2 - 2 \cos 1^\circ T - I\]


\end{proof}




\begin{problem}{9}
    Suppose \(T \in \Lc(V)\) is such that with respect to some basis of \(V\), all entries of the matrix 
    of \(T\) are rational numbers. Explain why all coefficients of the minimal polynomial of 
    \(T\) are rational numbers. 
\end{problem}

\begin{proof}
Let \(M\) be the matrix of \(T\) and \(d\) be the degree of its minimal polynomial. Consider 
\[\Ab = \begin{pmatrix}
    \text{vec}(I), \text{vec}(M), \ldots, \text{vec}(M^d)
\end{pmatrix}\]

Then we know that \(\Ab\) is rational and we can partition \(\Ab\) into first \(d\) column and last \(1\) 
column where the last column is a linear combination of the first \(d\) linearly independent ones. This further 
implies that we can find a solution using Gaussian elimination, where its entries are therefore all rational 
as well, and it is indeed the coefficients of the minimal polynomial. 
\end{proof}

\begin{problem}{10}
    Suppose \(V\) is finite-dimensional, \(T \in \Lc(V)\), and \(v \in V\). Prove that 
    \[\text{span}(v, Tv, \ldots, T^m v) = \text{span}(v, Tv, \ldots, T^{\dim V - 1}v)\]
    for all integers \(m \geq \dim V - 1\).
\end{problem}


\begin{proof}
    Denote \(A = \text{span}(v, Tv, \ldots, T^m v)\) and \(B = \text{span}(v, Tv, \ldots, T^{\dim V - 1}v)\).

\(\Rightarrow\) Since the list \(A\) has length greater than \(\dim V\), the elements after \(T^{\dim V - 1}v\)
can be written as linear combination of the previous terms (aka terms in \(B\)). Therefore any element in \(A\) 
can be written as linear combination of elements in \(B\) so \(A \subseteq B\).

\(\Leftarrow\) This direction is trivial. 
\end{proof}


\begin{problem}{12}
    Define \(T \in \Lc(\F^n)\) by \(T(x_1, x_2, x_3, \ldots, x_n) = (x_1, 2x_2, 3x_3, \ldots, n x_n)\). 
    Find the minimal polynomial of \(T\).
\end{problem}

\begin{proof}
Let \(v = (1, 0, 0, \ldots, 0)\). Then \(T^k v = e_1\) for all \(1 \leq k \leq n\). Thus we have that 
\[p(T) = T^n - I\]
\end{proof}

\begin{problem}{13}
    Suppose \(T \in \Lc(V)\) and \(p \in \Pc(\F)\). Prove that there exists a unique \(r \in \Pc(\F)\)
    such that \(p(T) = r(T)\) and \(\deg r\) is less than the degree of the minimal polynomial 
    of \(T\). 
\end{problem}


\begin{proof}
Let \(m_T(x)\) be the minimal polynomial of \(T\). By the division algorithm, there exists polynomial 
\(q(x)\) and \(r(x)\) with \(\deg r < \deg m_T\) such that 
\[p(x) = q(x)m_T(x) + r(x)\]

Evaluating at \(T\) then yields that 
\[p(T) = r(T)\]

Here we prove the uniqueness of \(r\). Suppose for contradiction that there exists another \(r'(x)\) such 
that 
\[p(T) = r(T) = r'(T)\]
If we now define \(r''(x) = r(x) - r'(x) = 0 = r''(T)\), then it must be a multiple of \(m_T(x)\) by the definition 
of the minimal polynomial. However, as its degree is less than \(m_T\), this forms a contradiction, 
completing the proof. 

\end{proof}


\begin{problem}{15}
    Suppose \(V\) is a finite-dimensional complex vector space with \(\dim V > 0\) and \(T \in \Lc(V)\). 
    Define \(f \colon \C \to \R\) by 
    \[f(\lambda) = \dim \range(T - \lambda I).\]
    Prove that \(f\) is not a continuous function. 
\end{problem}

\begin{proof}
Suppose \(\lambda\) is an eigenvalue of \(T\) and consider a sequence \(\{\lambda_k\}\) converging to \(\lambda\), 
where \(\lambda_k\)'s are not eigenvalues of \(T\) (you may verify such a sequence exists). For \(\lambda_k\) near \(\lambda\), we have that 
\(f(\lambda_k) = \dim V - \dim \nul (T - \lambda_k I) = \dim V\), where \(f(\lambda) < \dim V\). Hence, \(f\) 
is discontinuous at \(\lambda\) and therefore not a continuous function. 
\end{proof}


\begin{problem}{17}
    Suppose \(V\) is finite-dimensional, \(T \in \Lc(V)\), and \(p\) is the minimal polynomial of \(T\). 
    Suppose \(\lambda \in \F\). Show that the minimal polynomial of \(T - \lambda I\) is the polynomial 
    \(q\) defined by \(q(z) = p(z + \lambda)\). 
\end{problem}

\begin{proof}
First we can see that \(q(T - \lambda I) = p(T) = 0\). Now it suffices to prove the minimality condition. 
Suppose there exists another monic polynomial \(r(z)\) of lesser degree than \(q(z)\) such that 
\(r(T - \lambda I) = 0\). We can now define \(s(x) = r(x - \lambda)\) and then 
\(s(T) = r(T - \lambda I) = 0\) with \(\deg s < \deg p\), contradicting the minimal polynomial assumption 
of \(p\). Therefore we complete the proof. 
\end{proof}

\begin{problem}{19}
    Suppose \(V\) is finite-dimensional and \(T \in \Lc(V)\). Let \(\Ec\) be the subspace of 
    \(\Lc(V)\) defined by 
    \[\Ec = \{q(T) \colon q \in \Pc(\F)\}.\]
    Prove that \(\dim \Ec\) equals the degree of the minimal polynomial of \(T\).
\end{problem}

\begin{proof}
Note that \(\{a_0 I, a_1 T, a_2 T^2, a_3 T^3, \ldots, a_{n-1} T^{n-1}\} \coloneqq A\) is a basis of \(\Ec\) with dimension \(n\). 
Suppose \(\deg m_t = m\) for notational purposes. We have that 
\begin{align*}
    A \text{ is linearly independent } &\Leftrightarrow \text{Express } T^n \text{ as linear combination of terms in } A \\ 
    &\Leftrightarrow \text{Minimal degree of } m_T \text{ is } n 
\end{align*}



% Suppose for contradiction that the statement does not hold, so either \(\dim \Ec < \deg m_T\) or 
% \(\dim \Ec > \deg m_T\). Suppose \(\deg m_t = m\) for notational purposes. In the first case, we have 
% \(n < m\), so we know that \(a_0I, a_1 T, \ldots, a_{n-1} T^{n-1}\) is linearly independent, then we can express 
% \(a_{n}T^{n}\) as a linear combination of the previous terms, then 
\end{proof}


\begin{problem}{21}
    Suppose \(V\) is finite-dimensional and \(T \in \Lc(V)\). Prove that the minimal polynomial has 
    degree at most \(1 + \dim \range T\).
\end{problem}

\begin{proof}

Let \(k = \dim \range T, n = \dim V\). We note that \(T(\range T) \subseteq \range T\), so \(T\)
naturally induces a well-defined linear operator \(S \in \Lc(\range T)\) by restriction (\(Sv=Tv \forall  v 
\in \range T\)). Consider the minimal polynomial \(m_S(x)\) of \(S\). We know that \(\deg m_S \leq \dim 
\range T = k\). Under the construction, we also have \(T^2 = TS\), which implies that \(T\) behaves like 
\(S\) on \(\range T\). Note for any \(v \in \nul T, T(v) = 0\). 

Now we can construct a polynomial that annihilates \(T\) on \(V\). Define \(q(x) = x \cdot m_S(x)\). We 
claim that \(q(T) = 0\). This is true since for \(v \in \nul T, q(T)(v) = T \cdot m_S(T)(v) = T(0) = 0\). 
For \(v \in \range T\), we have \(m_S(T)(v) = m_S(S)(v) = 0\) and therefore \(q(T)(v) = 0\). Note that 
under this construction we have \(\deg q \leq k + 1\). Since the minimal polynomial divides \(p\), its 
degree is also upper bounded by \(1 + \dim \range T\). 


% We first knnow that there exists a minimal polynomial \(p\) for \(T\) such that \(\deg p \leq \dim V\). 
% We know that there exists \(S \in \Lc(\range T)\) such that \(T^2 = TS\) (from exericise in ch3), so 
% the minimal polynomial of \(S \)


% Suppose \(\dim \range T = n\) and for contradiction that the minimal polynomial of \(T\) has degree 
% more than \(1 + n\). This means that for \(m \geq n+2\).
% \[p(z) = (z - \lambda_1)\cdots(z - \lambda_{m} ).\]




\end{proof}



\begin{problem}{22}
    Suppose \(V\) is finite-dimensional and \(T \in \Lc(V)\). Prove that \(T\) is invertible if and only 
    if \(I \in \text{span}(T, T^2, \ldots, T^{\dim V})\). 
\end{problem}

\begin{proof}
\(\Rightarrow\) This means that 
\[I = a_0 T + a_1 T^2 + \cdots + a_n T^n\]
with \(n = \dim V\). Rewriting the equation gives that (for the largest nonzero degree), we have 
\[-I + \frac{a_0}{a_m}T + \frac{a_1}{a_m}T^2 + \cdots + T^m = 0\]
Therefore the constant term of minimal polynomial of \(T\) is not 0 and thus \(T\) is invertible. 



% Given \(T\) is invertible, then consider the list 
% \[\{T, T^2, \ldots, T^{n-1}, T^n\}\]
% We claim that this list is linearly independent and thus \(I\) is in the span of 
% the list. To prove the claim, suppose that it does not hold. Then there exists some 
% polynomial such that \(p(T) = 0\), contradicting that \(T\) is invertible. 

\(\Leftarrow\) We know that 
\[I = a_0 T + a_1 T^2 + \ldots + a_n T^n = T(a_0 I + a_1 T + \ldots + a_nT^{n-1})\]
Define \(S = (a_0 I + a_1 T + \ldots + a_nT^{n-1})\) so \(TS = I\). At the same time,
\[ST = (a_0 I + a_1 T + \ldots + a_n T^{n-1})T = a_0 T + a_1 T^2 + \ldots + a_n T^n 
= I\]
So \(T\) is invertible. 
\end{proof}

\begin{problem}{23}
    Suppose \(V\) is finite-dimensional and \(T \in \Lc(V)\). Let \(n = \dim V\). Prove that 
    if \(v \in V\), then \(\text{span}(v, Tv, \ldots, T^{n-1}v)\) is invariant under \(T\).
\end{problem}

\begin{proof}
We know that \(Tv, \ldots, T^{n-1}v \in \text{span}(v, Tv, \ldots, T^{n-1}v)\), so it suffices 
to prove that \(T^n v \in \text{span}(v, Tv, \ldots, T^{n-1}v)\). To see this, notice that the list 
\[v, Tv, \ldots, T^{n-1}v, T^n v\]
has length more than \(n\) and can be reduced to at least \(n\) terms. Therefore, we complete the proof.
\end{proof}


\begin{problem}{28}
    Suppose \(V\) is finite-dimensional and \(T \in \Lc(V)\). Prove that the minimal polynomial of 
    \(T' \in \Lc(V')\) equals the minimal polynomial of \(T\).
\end{problem}

Let \(p'\) and \(p\) be the minimal polynomial of \(T'\) and \(T\) respectively. We know that \(T'(\varphi)
= \varphi \circ T\). We have that  
\[p'(\varphi) = (c_0 I + c_1 T' + \cdots c_k(T')^k)(\varphi) 
= \varphi \circ (c_0 I + c_1 T + \cdots + c_k T^k) = \varphi \circ p = 0\]

which essentially shows that the minimality condition is iff for \(p'\) and \(p\), otherwise it would reach 
a contradiction that one is the minimal polynomial.



\begin{problem}{29}
    Show that every operator on a finite-dimensional vector space of dimension at least two has an 
    invairant subspace of dimension two. 
\end{problem}

\begin{proof}
We prove this claim through the induction on \(n = \dim V\). For base case, we have \(\dim V = 2\), then as 
\(V\) is invariant, the claim is satified. 

For inductive case, suppose the statement holds for \(V\) with dimension \(n-1\). Then consider \(V\) with 
dimension \(n\). Take any nonzero \(v \in V\) and consider \(V \backslash \text{span}(v)\), which is a space of 
dimension \(n - 1\), so it has an invariant subspace of dimension two, which is still a subspace of \(V\), 
completing the proof.
\end{proof}



%%%%%%%%%%%%%%%%%%%%%%%%%%%%%%%%%%%%%%%%%%%%%%%%%%%%%%%%%%%%
%%%%%%%%%%%%%%%%%%%%%%%%%%%% 5C %%%%%%%%%%%%%%%%%%%%%%%%%%%%
%%%%%%%%%%%%%%%%%%%%%%%%%%%%%%%%%%%%%%%%%%%%%%%%%%%%%%%%%%%%

\newpage 
\section*{5C: Upper-Triangular Matrices}
\addcontentsline{toc}{section}{5C: Upper-Triangular Matrices}

\begin{definition}[matrix of an operator, \(\Mc(T)\)]
    Suppose \(T \in \Lc(V)\). The \textbf{matrix of} \(T\) with respect to a basis 
    \(v_1, \ldots, v_n\) of \(V\) is the n-by-n matrix 
    \[\Mc(T) = \begin{bmatrix}
        A_{1,1} & \cdots & A_{1, n} \\ 
        \vdots &  & \vdots  \\ 
        A_{n, 1} & \cdots & A_{n,n}
    \end{bmatrix}\]

    whose entries \(A_{j, k}\) are defined by 
    \[Tv_k = A_{1, k}v_1 + \cdots + A_{n, k} v_n .\]
    The notation \(\Mc(T, (v_1, \ldots, v_n))\) is used if the basis is not clear from the context. 
\end{definition}

\begin{remark}
    Operators have square matrices.
\end{remark}

\begin{definition}[diagnoal of a matrix]
    The \textbf{diagonal} of a square matrix consists of the entries on the line from the upper left 
    corner to the bottom right corner. 
\end{definition}

\begin{definition}[upper-triangular matrix]
    A square matrix is called \textbf{upper triangular} if all entries below the diagnoal are 0.
\end{definition}

\begin{thm}[conditions for upper-triangular matrix]
    Suppose \(T \in \Lc(V)\) and \(v_1, \ldots, v_n\) is a basis of \(V\). Then the following are 
    equivalent. 
    \begin{enumerate}[label=(\alph*)]
        \item The matrix of \(T\) with respect to \(v_1, \ldots, v_n\) is upper triangular. 
        \item \(\text{span}(v_1, \ldots, v_k)\) is invariant under \(T\) for each \(k = 1, \ldots, n\). 
        \item \(Tv_k \in \text{span}(v_1, \ldots, v_k)\) for each \(k = 1, \ldots, n\). 
    \end{enumerate}
\end{thm}

\begin{lemma}
    Suppose \(T \in \Lc(V)\) and \(V\) has a basis with respect to which \(T\) has an upper-triangular matrix 
    with diagnoal matrix with diagnoal entries \(\lambda_1, \ldots, \lambda_n\). Then 
    \[(T - \lambda_1 I) \cdots (T - \lambda_n I) = 0\]
\end{lemma}

\begin{thm}[determination of eigenvalues from upper-triangular matrix]
    Suppose \(T \in \Lc(V)\) has an upper-triangular matrix with respect to some 
    basis of \(V\). Then the eigenvalues of \(T\) are precisely the entries on the diagonal 
    of that upper-triangular matrix. 
\end{thm}

\begin{remark}
    Main proof technique: \((T - \lambda_k)I v_k \in \text{span}(v_1, \ldots, v_{k-1})\) for the matrix 
    of \(T\) to be upper-triangular. 
\end{remark}


\begin{lemma}[necessary and sufficient condition to have an upper-triangular matrix]
    Suppose \(V\) is finite-dimensional and \(T \in \Lc(V)\). Then \(T\) has an upper-triangular matrix 
    with respect to some basis of \(V\) if and only if the minimal polynomial equals 
    \((z - \lambda_1)\cdots(z - \lambda_m)\) for some \(\lambda_1, \ldots, \lambda_m \in \F\).
\end{lemma}

\begin{thm}[if \(\F = \C\), then every operator on \(V\) has an upper-triangular matrix]
    Suppose \(V\) is a finite-dimensional complex vector space and \(T \in \Lc(V)\). Then \(T\) 
    has an upper-triangular matrix with respect to some basis of \(V\).
\end{thm}









%%%%%%%%%%%%%%%%%%%%%%%%%%%%%%%%%%%%%%%%%%%%%%%%%%%%%%%%%%%%%%%
%%%%%%%%%%%%%%%%%%%%%%%%%%%% 5C PS %%%%%%%%%%%%%%%%%%%%%%%%%%%%
%%%%%%%%%%%%%%%%%%%%%%%%%%%%%%%%%%%%%%%%%%%%%%%%%%%%%%%%%%%%%%%

\newpage 
\addcontentsline{toc}{subsection}{5C Problem Sets}

\begin{problem}{1}
    Prove or give a counter example: If \(T \in \Lc(V)\) and \(T^2\) has an upper-triangular matrix with 
    respect to some basis of \(V\), then \(T\) has an upper-triangular matrix with respect to some basis 
    of \(V\).
\end{problem}

\begin{proof}

Since \(T^2\) has an upper triangular matrix wrt some basis of \(V\), its minimal polynomial 
equals \(p_{T^2}(x) = (z - \lambda_1)\cdots(z - \lambda_m)\) for some \(\lambda_1, \ldots, \lambda_m \in \F\). 
This means that \(p_{T^2}(T^2) = (T^2 - \lambda_1 I) \cdots (T^2 - \lambda_m I) = (T - \sqrt{\lambda_1}I) 
(T + \sqrt{\lambda_1}I)\cdots(T - \sqrt{\lambda_m}I)(T + \sqrt{\lambda_m}I) = 0\). This means that the minimal polynomial 
of \(T\) can also be in the form of an upper-triangular matrix.  

% We can try to prove the contrapositive: suppose \(T\) does not have upper-triangular matrix with respect 
% to any basis, then \(T^2\) does not as well. We can prove this from the perspective of matrix multiplication.
% Let \(\Ab\) be the matrix of \(T\), then we know that there exists nonzero entry \((i, j)\) with 
% \(i > j\). 

    % By theorem 5.44, this means that the minimal polynomial of \(T^2\) equals \((z^2 - \lambda_1)\cdots(z^2 - \lambda_m)\)
% for some \(\lambda_1, \ldots, \lambda_m \in \F\). Naturally for \(T\) we have that 
% \((z - \sqrt{\lambda_1})(z - (- \sqrt{\lambda_1})) \cdots (z - \sqrt{\lambda_m})(z-(-\sqrt{\lambda_m}))\). 

% We prove this statement through induction on the dimension of \(V\), denoted by \(n\). 

% For the base case where \(n = 1\), this statement naturally holds. 

% For the inductive case suppose it holds for \(n < k\), so we know that there exists 
% \(v_1, \ldots, v_{k-1}\) basis of \(V\) such that if the matrix of \(T^2\) wrt to them is upper-triangular. 
% When \(n = k\), extend the basis to \(v_1, \ldots, v_{k-1}, v_k\). We know that \(Tv_j \in \text{span}(v_1, \ldots, v_{j})\)
% for \(j = 1, \ldots, k-1\). Now we consider \(Tv_k\), it naturally holds that \(Tv_k \in \text{span} 
% (v_1, \ldots, v_k)\) since \(Tv_k \in V\). Therefore, \(T\) 
\end{proof}


\begin{problem}{3}
    Suppose \(T \in \Lc(V)\) is invertile and \(v_1, \ldots, v_n\) is a basis of \(V\) with respect 
    to which the matrix of \(T\) is upper triangular, with \(\lambda_1, \ldots, \lambda_n\) on the 
    diagnoal. Show that the matrix of \(T^{-1}\) is also upper triangular with respect to the basis 
    \(v_1, \ldots, v_n\), with 
    \[\frac{1}{\lambda_1}, \ldots, \frac{1}{\lambda_n}\]
    on the diagonal. 
\end{problem}

\begin{proof}
We first show that \(T^{-1}\) is also upper triangular wrt to the same basis. To see this, we know 
that \(T v_k \in \text{span}(v_1, \ldots, v_k)\) for each \(k\), or equivalently, 
\[T(v_k) = \lambda_k v_k + \sum_{j=1}^{k-1}A_{j,k}v_j\]
Let \(w_k = T^{-1}v_k\), then we know that \(T w_k = v_k \in \text{span}(v_1, \ldots, v_k)\). So this 
implies that \(w_k \in \text{span}(v_1, \ldots, v_k)\) by property of \(T\). Thus this completes the 
first part of the proof. For the second part, notice that \(TT^{-1} = I\) and since the two matrices 
are both upper-triangular, by rule of matrix multiplication, \(T_{ii} T^{-1}_{ii} = 1\) and thus 
the entries on the diagonal of \(T^{-1}\) is the inverse of \(T\)'s, completing the proof.  

\end{proof}

\begin{problem}{6}
    Suppose \(\F = \C\), \(V\) is finite-dimensional, and \(T \in \Lc(V)\). Prove that if 
    \(k \in \{1, \ldots, \dim V\}\), then \(V\) has a \(k-\)dimensional subspace 
    invariant under \(T\).
\end{problem}

\begin{proof}
We have proved in the book that \(T\) has an upper-triangular matrix with respect to some basis 
of \(V\). Then by equivalent conditions of upper-triangular matrix, we know that for some basis 
\(v_1, \ldots, v_n\), the \(\text{span}(v_1, \ldots, v_k)\) is invariant under \(T\) for each 
\(k = 1, \ldots, n\), completing the proof. 
\end{proof}


\begin{problem}{7}
    Suppose \(V\) is finite-dimensional, \(T \in \Lc(V)\), and \(v \in V\). 
    \begin{enumerate}[label=(\alph*)]
        \item Prove that there exists a unique monic polynomial \(p_v\) of smallest degree such that 
        \(p_v(T)v = 0\).
        \item Prove that the minimal polynomial of \(T\) is a polynomial multiple of \(p_v\).
    \end{enumerate}
\end{problem}

\begin{proof}
(a) First we prove the existence. Consider the list 
\[v, Tv, \ldots, T^n v\] 
which is a linearly dependent list in \(\dim V = n\) space. Therefore we can always reduce this list 
to a smallest degree such that 
\[T^m v = \sum_{k=0}^{m-1}\frac{a_k}{a_m}T^m v\]
Next we prove uniqueness. Suppose there exists another monic polynomial of minimal degree 
\(s\) such that \(s(T)(v) = 0\). Then we can consider 
\[0 = p(T)(v) - s(T)(v) = (p - s)(T)(v)\]
So this polynomial \((p - s)\) also satifies the required condition and it has smaller degree 
than \(p\) or \(s\), reaching a contradiction. 


(b) Let \(m\) be the minimal polynomial of \(T\). By the division algorithm, we have 
\[m(z) = q(z)p(z) + r(z)\]
for some polynomial \(q\) and \(r\) with \(\deg r < \deg p\). We claim that \(r = 0\). This is true as 
\[0 = m(T)(v) = qp(T)(v) + r(T)(v) = r(T)(v)\]
Thus \(r(T)(v) = 0\). If \(r\) is non-zero map, then this means there exists a monic polynomial of even 
lesser degree than \(p\) that also maps \(v\) to 0, 
contradicting that \(p\) is the unique smallest monic polynomial. Therefore we complete the proof.
\end{proof}



\begin{problem}{9}
    Suppose \(B\) is a square matrix with complex entries. Prove that there exists an invertible square 
    matrix \(A\) with complex entries such that \(A^{-1}BA\) is an upper-triangular matrix.
\end{problem}

\begin{proof}
Consider the linear map \(T\) represented by the matrix \(B\). Since the field is taken over \(\C\), we know 
that \(T\) has an upper-triangular matrix with respect to some basis of \(V\). Then let \(A\) be such 
change-of-basis matrix we can indeed find the desired matrix. 
\end{proof}


\begin{problem}{10}
    Suppose \(T \in \Lc(V)\) and \(v_1, \ldots, v_n\) is a basis of \(V\). Show that the following are 
    equivalent. 

    \begin{enumerate}[label=(\alph*)]
        \item The matrix of \(T\) with respect to \(v_1, \ldots, v_n\) is lower triangular. 
        \item \(\text{span}(v_k, \ldots, v_n)\) is invairant under \(T\) for each \(k = 1, \ldots, n\). 
        \item \(Tv_k \in \text{span}(v_k, \ldots, v_n)\) for each \(k = 1, \ldots, n\).
    \end{enumerate}
\end{problem}


\begin{proof}
\((a) \Rightarrow (b)\) This means that \(Tv_j \in \text{span}(v_j, \ldots, v_n)\). If \(j \geq k\), then 
we have that 
\[T v_j \in \text{span}(v_k, \ldots, v_n)\]

\((b) \Rightarrow (c)\) This holds by definition. 

\((c) \Rightarrow (a)\) This means when writing each \(Tv_k\) as a linear combination of the basis vectors 
\(v_1, \ldots, v_n\), we need to use only the vectors \(v_k, \ldots, v_n\). Hence all entries above the 
diagnoal of \(\Mc(T)\) are 0 and thus it is an lower-triangular matrix.
\end{proof}


\begin{problem}{12}
    Suppose \(V\) is finite-dimensional, \(T \in \Lc(V)\) has an upper-triangular matrix with respect 
    to some basis of \(V\), and \(U\) is a subspace of \(V\) that is invariant under \(T\). 
    \begin{enumerate}[label=(\alph*)]
        \item Prove that \(T|_U\) has an upper-triangular matrix with respect to some basis of \(U\). 
        \item Prove that the quotient operator \(T/U\) has an upper-triangular matrix with respect 
        to some basis of \(V/U\).
    \end{enumerate}
\end{problem}

\begin{proof} We know that there exists a basis \(v_1, \ldots, v_n\) 
    such that \(T v_k \in \text{span}(v_1, \ldots, v_k)\) for each \(k\). 

(a) We can always find a subset of the basis such that \(v_1, \ldots, v_m\) is the basis of 
\(U\) (WLOG we make the numbering easy). Then for the list we always have that \(T v_k \in 
\text{span}(v_1, \ldots, v_k)\) for each \(k\) and thus \(T|_U\) has an upper-triangular matrix. 

(b) Consider \(\{v_{m+1} + U, \ldots, v_n + U\}\), which is the basis of \(V/U\). Note that 
we still have \(T(v_k + U) \in \text{span}(v_{m+1} + U, \ldots, v_{k} +U)\), hence \(T/U\) 
also has an upper-triangular matrix. 
\end{proof}


\begin{problem}{14}
    Suppose \(V\) is finite-dimensional and \(T \in \Lc(V)\). Prove that \(T\) has an upper-triangular 
    matrix with respect to some basis of \(V\) if and only if the dual operator \(T'\) has an upper-triangular 
    matrix with respect to some basis of the dual space. 
\end{problem}

\begin{proof}
From 5B problem 28 we know that the minimal polynomial of \(T'\) equals the minimal polynomial of \(T\). 
Then this means that they have the same roots and therefore the same form of \((z - \lambda_1) 
\cdots (z - \lambda_m)\) if they having upper-triangular matrix. The iff is directly deduced from the iff 
of the relation between each one's minimal polynomial.
\end{proof}








%%%%%%%%%%%%%%%%%%%%%%%%%%%%%%%%%%%%%%%%%%%%%%%%%%%%%%%%%%%%
%%%%%%%%%%%%%%%%%%%%%%%%%%%% 5D %%%%%%%%%%%%%%%%%%%%%%%%%%%%
%%%%%%%%%%%%%%%%%%%%%%%%%%%%%%%%%%%%%%%%%%%%%%%%%%%%%%%%%%%%


\newpage 
\section*{5D: Diagonalizable Operators}
\addcontentsline{toc}{section}{5D: Diagonalizable Operators}


\begin{definition}[diagonal matrix]
    A \textbf{diagonal matrix} is a square matrix that is 0 everywhere except possibly on the 
    diagonal.    
\end{definition}


\begin{remark}
    The entris on the diagonal are precisely the eigenvalue of the operator.
\end{remark}

\begin{definition}[diagonalizable]
    An oeprator on \(V\) is called \textbf{diagonalizable} if the operator has a diagonal matrix 
    with respect to some basis of \(V\).
\end{definition}


\begin{remark}
    Diagonalization may require a different basis.
\end{remark}

\begin{definition}[eigenspace, \(E(\lambda, T)\)]
    Suppose \(T \in \Lc(V)\) and \(\lambda \in \F\). The \textbf{eigenspace} of \(T\) corresponding to 
    \(\lambda\) is the subspace \(E (\lambda, T)\) of \(V\) defined by 
    \[E(\lambda, T) = \nul (T - \lambda I) = \{v \in V \colon Tv = \lambda v\}.\]
    Hence \(E(\lambda, T)\) is the set of all eigenvectors of \(T\) corresponding to \(\lambda\), along 
    with the 0 vector. 
\end{definition}

\begin{remark}
    \(\lambda\) is an eigenvalue of \(T\) if and only if \(E(\lambda, T) \neq \{0\}\).
\end{remark}

\begin{thm}
    Suppose \(T \in \Lc(V)\) and \(\lambda_1, \ldots, \lambda_m\) are distinct eigenvalues of \(T\). Then 
    \[E(\lambda_1, T) + \cdots + E(\lambda_m, T)\]
    is a direct sum. Furthermore, if \(V\) is finite-dimensional, then 
    \[\dim E(\lambda_1, T) + \cdots + \dim (\lambda_m, T) \leq \dim V\]
\end{thm}


\begin{thm}[conditions equivalent to diagonalizability]
    Suppose \(V\) is finite-dimensional and \(T \in \Lc(V)\). Let \(\lambda_1, \ldots, \lambda_m\) denote 
    the distinct eigenvalues of \(T\). Then the following are equivalent.
    \begin{enumerate}[label=(\alph*)]
        \item \(T\) is diagonalizable. 
        \item \(V\) has a basis consisting of eigenvectors of \(T\). 
        \item \(V = E(\lambda_1, T) \oplus \cdots \oplus E(\lambda_m, T)\). 
        \item \(\dim V = \dim E(\lambda_1, T) + \cdots + \dim E(\lambda_m, T)\).
    \end{enumerate}
\end{thm}

\begin{corollary}[enough eigenvalues implies diagonalizability]
    Suppose \(V\) is finite-dimensional and \(T \in \Lc(V)\) has \(\dim V\) distinct eigenvalues. 
    Then \(T\) is diagonalizable.
\end{corollary}


\begin{thm}[necessary and sufficient condition for diagonalizability]
    Suppose \(V\) is finite-dimensional and \(T \in \Lc(V)\). Then \(T\) is diagonalizable if and only if 
    the minimal polynomial of \(T\) equals \((z - \lambda_1)\cdots(z - \lambda_m)\) for some 
    list of \textbf{distinct} numbers \(\lambda_1, \ldots, \lambda_m \in \F\).
\end{thm}

\begin{corollary}
    Suppose \(T \in \Lc(V)\) is diagonalizable and \(U\) is a subspace of \(V\) that is invariant 
    under \(T\). Then \(T |_U\) is a diagonalizable operator on \(U\). 
\end{corollary}


\begin{definition}[Gershgorin disks]
    Suppose \(T \in \Lc(V)\) and \(v_1, \ldots, v_n\) is a basis of \(V\). Let \(A\) denote the matrix 
    of \(T\) with respect to this basis. A \textbf{Gershgorin disk} of \(T\) with respect to the basis 
    \(v_1, \ldots, v_n\) is a set of the form 
    \[\{z \in \F \colon |z - A_{j,j}| \leq \sum_{k=1 k \neq j}^{n} |A_{j,k}| \},\]
    where \(j \in \{1, \ldots, n\}.\)
\end{definition}

\begin{remark}
    Intuition: if the nondiagonal entries of \(A\) are small, then each eigenvalue of \(T\) is near 
    a diagonal entry of \(A\).
\end{remark}

\begin{thm}[Gershgorin disk theorem]
    Suppose \(T \in \Lc(V)\) and \(v_1, \ldots, v_n\) is a basis of \(V\). Then each eigenvalue of \(T\)
    is contained in some Gershgorin disk of \(T\) with respect to the basis \(v_1, \ldots, v_n\).
\end{thm}







%%%%%%%%%%%%%%%%%%%%%%%%%%%%%%%%%%%%%%%%%%%%%%%%%%%%%%%%%%%%%%%
%%%%%%%%%%%%%%%%%%%%%%%%%%%% 5D PS %%%%%%%%%%%%%%%%%%%%%%%%%%%%
%%%%%%%%%%%%%%%%%%%%%%%%%%%%%%%%%%%%%%%%%%%%%%%%%%%%%%%%%%%%%%%

\newpage 
\addcontentsline{toc}{subsection}{5D Problem Sets}


\begin{problem}{1}
    Suppose \(V\) is a finite-dimensional complex vetor space and \(T \in \Lc(V)\). 
    \begin{enumerate}[label=(\alph*)]
        \item Prove that if \(T^4 = I\), then \(T\) is diagonalizable. 
        \item Prove that if \(T^4 = T\), then \(T\) is diagonalizable. 
        \item Give an example of an operator \(T \in \Lc(\C^2)\) such that \(T^4 = T^2\) and \(T\) is not 
        diagnoalizable.
    \end{enumerate} 
\end{problem}

\begin{proof}
(a) The polynomial \(p(x) = x^4 - 1\) annihilates \(T\) (\(p(T) = 0\)) has four distinct roots. The minimal 
polynomial of \(T\) divides \(p\) and therefore \(T\) is diagnoalizable. 

(b) The polynomial \(p(x) = x^4 - x = x(x-1)(x^2 +1)\) annihilates \(T\) has four distinct roots. The mininmal 
polynomial of \(T\) divides \(p\) and therefore \(T\) is diagnoalizable. 

(c) Consider 
\[T = \begin{bmatrix}
    0 & 1 \\ 
    0 & 0
\end{bmatrix}\]
Then we have that 
\[T^2 = \begin{bmatrix}
    0 & 0 \\ 
    0 & 0
\end{bmatrix} =  T^4 \]
\end{proof}

\begin{problem}{3}
    Suppose \(V\) is finite-dimensional and \(T \in \Lc(V)\). Prove that if the operator \(T\) is 
    diagnoalizable, then \(V = \nul T \oplus \range T\).
\end{problem}

\begin{proof}
By equivalent characterizations, there exists \(v_1, \ldots, v_n\) to be the eigenbasis of \(V\). Let 
\(\lambda_1, \ldots, \lambda_n\) be the corresponding eigenvalues. We can partition these vectors into corresponding 
basis for \(\nul T\) and \(\range T\). Naturally, we can partition these eigenbasis into two parts: let 
\(\lambda_1, \ldots, \lambda_m\) denotes the eigenbasis with 0 eigenvalues and \(\lambda_{m+1}, \ldots, 
\lambda_n\) denotes the eigenbasis with nonzero eigenvalues. We relabel the eigenvectors accordingly. 
We claim that \(\nul T = \text{span}(v_1, 
\ldots, v_m)\) and \(\range T = \text{span}(v_{m+1}, \ldots, v_n)\). The proof is complete once 
we complete proving our claim. 

For proving the null space, \(\Rightarrow\) take \(v \in \nul T\), then we have that 
\begin{align*}
    Tv 
    &= T \sum_{i=1}^{m}a_i v_i + T \sum_{i=m+1}^{n} a_i v_i \\ 
    &= \sum_{i=m+1}^{n} (a_i \lambda_i) v_i = 0 
\end{align*}

By linear independence of eigenbasis (as \(\lambda_i \neq 0 \) for \(m+1 \leq i \leq n\)), we have that 
\(a_i = 0\) for all \(m+1 \leq i \leq n\). Thus \(v \in \text{span}(v_1, \ldots, v_m)\). \(\Leftarrow\) 
for the other direction, it follows naturally by construction. 

For proving the range space, \(\Rightarrow\) take \(v \in \range T\), then there exists \(u \in V\) 
s.t. 
\[v = Tu  = \sum_{i=m+1}^{n} (a_i \lambda_i)v_i\]
Therefore \(v \in \text{span}(v_{m+1}, \ldots, v_n)\). \(\Leftarrow\) for the other direction, it follows 
by construction. 

\end{proof}

\begin{problem}{4}
    Suppose \(V\) is finite-dimensional and \(T \in \Lc(V)\). Prove that the following are equivalent.
    \begin{enumerate}[label=(\alph*)]
        \item \(V = \nul T \oplus \range T\). 
        \item \(V = \nul T + \range T\). 
        \item \(\nul T \cap \range T = \{0\}\).
    \end{enumerate}
\end{problem}


\begin{proof}
\((a) \Rightarrow (b)\) Trivial. 

\((b) \Rightarrow (c)\) We know that 
\[\dim V = \dim (\nul T  + \range T) = \dim \nul T + \dim \range T - \dim \nul T \cap \range T\]
At the same time, \(\dim V = \dim \nul T + \dim \range T\), hence 
\[\dim \nul T \cap \range T = 0\]

\((c) \Rightarrow (a)\) This can be proved similarily as above. 
\end{proof}


\begin{problem}{5}
    Suppose \(V\) is a finite-dimensional complex vector space and \(T \in \Lc(V)\). Prove that 
    \(T\) is diagonalizable if and only if 
    \[V= \nul (T - \lambda I) \oplus \range (T - \lambda I)\]
    for every \(\lambda \in \C\).
\end{problem}

\begin{proof}
\(\Rightarrow\) Given \(T\) is diagnoalizable, then we know that \(T - \lambda I\) is also 
diagnoalizable (the matrix of that is diagonal). Hence by P3 we get the desired result. 

\(\Leftarrow\) The minimal polynomial of \(T\) can be written as 
\((z - \lambda_1)^{n_1}\ldots(z - \lambda_m)^{n_m}\) for some distinct distinct \(\lambda_1, \ldots, 
\lambda_m \in \C\). If \(n_1 = \cdots = n_m = 1\), then we are done. WLOG suppose \(n_1 > 1\). Then take 
arbitrary \(v \in V\), we have 
\[\prod_{k=1}^m (T - \lambda_k I)^{n_k} (v) = (T - \lambda_1 I)(T - \lambda_1 I)^{n_1 - 1} 
\prod_{k=2}^m (T - \lambda_k I)^{n_k}(v) = 0\]

This implies that \((T - \lambda_1 I)^{n_1 - 1} \prod_{k=2}^m (T - \lambda_k I)^{n_k} \in \nul (T - \lambda_1 I) 
\cap \range (T - \lambda_1 I) = \{0\}\). Therefore, this reach a contradiction to the minimal polynomial we 
previously get.

\end{proof}


\begin{problem}{6}
    Suppose \(T \in \Lc(\F^5)\) and \(\dim E(8, T) = 4\). Prove that \(T - 2 I\) or \(T - 6I\) is invertible. 
\end{problem}

\begin{proof}
We know that 
\[\dim E(8, T) + \dim E(2, T) + \dim E(6, T) \leq 5\]
Since \(\dim (8, T) = 4\), then we have that 
\[\dim E(2,T) = 0 \ \text{or} \ \dim E(6, T) = 0\]
In other words, \(T - 2I\) or \(T - 6I\) is invertible. 

\end{proof}


\begin{problem}{7}
    Suppose \(T \in \Lc(V)\) is invertible. Prove that 
    \[E(\lambda, T) = E(\frac{1}{\lambda}, T^{-1})\]
    for every \(\lambda \in \F\) with \(\lambda \neq 0\). 
\end{problem}

\begin{proof}
We know that 
\begin{align*}
    Tv = \lambda v \Leftrightarrow v = T^{-1} \lambda v  
    \Leftrightarrow \frac{1}{\lambda}v = T^{-1} v
\end{align*}
\end{proof}

\begin{problem}{8}
    Suppose \(V\) is finite-dimensional and \(T \in \Lc(V)\). Let \(\lambda_1, \ldots, \lambda_m\) denote 
    the distinct nonzero eigenvalues of \(T\). Prove that 
    \[\dim E(\lambda_1, T) + \cdots + \dim E(\lambda_m, T) \leq \dim \range T.\]
\end{problem}

\begin{proof}
It suffices to show that 
\[E' = E(\lambda_1, T) + \cdots + E(\lambda_m, T) = E(\lambda_1, T) \oplus \cdots \oplus E(\lambda_m, T) 
\subseteq \range (T)\]
To see this, let \(v \in E(\lambda_k, T)\), then we know that \(Tv = \lambda_k v\) and that 
\(v = T \frac{1}{\lambda_k}v  \in \range T\). Hence we complete the proof. 
\end{proof}



\begin{problem}{9}
    Suppose \(R, T \in \Lc(\F^3)\) each have 2,6,7 as eigenvalues. Prove that there exists an invertible 
    operator \(S \in \Lc(\F^3)\) such that \(R = S^{-1}TS\).
\end{problem}

\begin{proof}
This means that \(R, T\) are both diagnoalizable and thus invertible. Thus we can define a change-of-basis 
operator \(S\) that is invertible (one may provide more details here).
\end{proof}

\begin{problem}{12}
    Suppose \(T \in \Lc(\C^3)\) is such that 6 and 7 are eigenvalues of \(T\). Furthermore, suppose \(T\) 
    does not have a diagonal matrix with respect to any basis of \(\C^3\). Prove that there exists 
    \((z_1, z_2, z_3) \in \C^3\) such that 
    \[T(z_1, z_2, z_3) = (6 + 8 z_1, 7 + 8z_2, 13 + 8 z_3).\]
\end{problem}

\begin{proof}
This means that \(T\) only has 6, 7 as the eigenvalues and so 8 is not an eigenvalue of \(T\). Equivalently, 
\(T - 8I\) is invertible and thus there exists \((z_1, z_2, z_3) \in \C^3\) such that \((T - 8I) 
(z_1, z_2, z_3) = (6, 7, 13)\). Rewriting the equation gives the desired result. 
\end{proof}

\begin{problem}{13}
    Suppose \(A\) is a diagonal matrix with distinct entries on the diagonal and \(B\) is a matrix of the same 
    size as \(A\). Show that \(AB = BA\) if and only if \(B\) is a diagonal matrix. 
\end{problem}

\begin{proof}
\(\Rightarrow\) We can simply examine each entry of \(AB\) and \(BA\), note that 
\[(AB)_{ij} = \sum_{k=1}^{n}A_{i, k}B_{k, j} = A_{i, i} B_{i, j}\]
and 
\[(BA)_{ij} = \sum_{k=1}^{n}B_{i, k}A_{k, j} = A_{j, j} B_{i, j}\]
Since \(A_{i, i} \neq A_{j, j}\), \(B_{i, j} = 0\) for \(i \neq j\). Hence we have \(B\) is diagonal. 

\(\Leftarrow\) This naturally holds by matrix multiplication. 
\end{proof}

\begin{problem}{14}
    \begin{enumerate}[label=(\alph*)]
        \item Give an example of a finite-dimensional complex vector space and an operato \(T\) on 
        that vector space such that \(T^2\) is diagnoalizable but \(T\) is not diagnoalizable. 
        \item Suppose \(\F = \C\), \(k\) is a positive integer, and \(T \in \Lc(V)\) is invertible. 
        Prove that \(T\) is diagnoalizable if and only if \(T^k\) is diagnoalizable.
    \end{enumerate}
\end{problem}


\begin{proof}
(a) Consider the matrix of \(T\) to be 
\[\begin{bmatrix}
    0 & 1 \\ 
    0 & 0
\end{bmatrix}\]
and let \(\F = \R\). Then \(T^2 = 0\) which is diagnoalizable.

(b) \(\Rightarrow\) By P13, we know that since \(T\) is diagonal, \(T^2\) is diagonal, and 
\(T(T^2)\) is diagnoal. Recursively applying the argument yields that \(T^m\) is diagonal and therefore 
diagonalizable. 

\(\Leftarrow\) We know that \(p(x) = (x^n - \lambda_1)\cdots(x^n - \lambda_k)\) annihilates \(T\) where 
\(\lambda_1, \ldots, \lambda_k\) are distinct eigenvalues of \(T^m\). Since all \(\lambda_i\) are distinct, 
\(x^n - \lambda_i\) and \(x^n - \lambda_j\) do not share any common root for \(i \neq j\). Then since the 
minimal polynomial of \(T\) divides \(p(x)\), it also does not have any repeated roots and therefore \(T\) 
is also diagonalizable. 
\end{proof}


\begin{problem}{16}
    Suppose \(T \in \Lc(V)\) is diagnoalizable. Let \(\lambda_1, \ldots, \lambda_m\) denote the distinct 
    eigenvalues of \(T\). Prove that a subspace \(U\) of \(V\) is invariant under \(T\) if and only if 
    there exist subspaces \(U_1, \ldots, U_m\) of \(V\) such that \(U_k \subseteq E(\lambda_k, T)\) for 
    each \(k\) and \(U = U_1 \oplus \cdots \oplus U_m\). 
\end{problem}

\begin{proof}
\(\Leftarrow\) Take any \(u = u_1 + \cdots + u_m \in U\), then \(Tu = Tu_1 + \cdots + Tu_m 
= \lambda_1 u_1 + \cdots + \lambda_m u_m \in U\). 

\(\Rightarrow\) Define \(U_k = E(\lambda_k, T) \cap U\). Clearly \(U_k \subseteq E(\lambda_k, T)\). 
Since \(U\) is invariant under \(T\), \({U_k}\) is also invariant. By 5.65 we know that 
\(T |_{U_k}\) is also diagnoalizable. Since we know that \(E(\lambda_i, T) \cap E(\lambda_j, T) = \{0\}\) for \(i \neq j\), 
\(U_j \cap U_i = \{0\}\) for \(i \neq j\) as well. Now it suffices to show that 
\(U \subseteq U_1 \oplus \cdots \oplus U_m\). Take \(u \in U \subset V\), since we know that 
\(V = E(\lambda_1, T) \oplus \cdots \oplus E(\lambda_m, T)\), \(u \in E_k\) for some \(k\), therefore 
completing the proof. 
\end{proof}


\begin{problem}{18}
    Suppose that \(T \in \Lc(V)\) is diagonalizable and \(U\) is a subspace of \(V\) that is invariant under 
    \(T\). Prove that the quotient operator \(T/U\) is a diagonalizable operator on \(V/U\). 
\end{problem}

\begin{proof}
First we know that there exists eigenbasis \(v_1, \ldots, v_n\). We can first partition these basis into 
\(v_1, \ldots, v_m\) for \(U\) and then \(v_{m+1} + U, \ldots, v_n + U\) would be the basis for 
\(V/U\). Notice that here we have \((T/U)(v_j + U) = T(v_j) + U + \lambda_j v_j + U\), with the same eigenbasis 
so preserving the diagonalizability.
\end{proof}


\begin{problem}{20}
    Suppose \(V\) is finite-dimensional and \(T \in \Lc(V)\). Prove that \(T\) is diagnoalizable if 
    and only if the dual operator \(T'\) is diagnoalizable. 
\end{problem}

\begin{proof}
WLOG we go from the forward direction \(\Rightarrow\). Let \(v_1, \ldots, v_n\) be the eigenbasis and 
\(\varphi_1, \ldots, \varphi_n\) be the corresponding dual basis. Then we have that 
\[(T' \varphi_i)(v_j) = \varphi_i \circ Tv_j = \lambda_j \varphi_i(v_j) = \lambda_i \delta_{ij}\]

Hence we have that \(T' \varphi_i = \lambda_i \varphi_i\) for each \(i\) and thus \(T'\) is diagnoalizable.

\end{proof}


\begin{problem}{22}
    Suppose \(T \in \Lc(V)\) and \(A\) is an n-by-n matrix that is the matrix of \(T\) with respect to some 
    basis of \(V\). Prove that if 
    \[|A_{j,j}| > \sum_{k=1, k\neq j}^{n} |A_{j,k}|\] 
    for each \(j \in \{1, \ldots, n\}\), then \(T\) is invertible. 

    In other words, the implication is that if the diagonal entries of the matrix of \(T\) are large enough 
    compred to non-diagonal ones, then \(T\) is invertible. 
\end{problem}

\begin{proof}
Equivalently, we aim to prove that 0 is not an eigenvalue of \(A\). From the Gershgorin disk theorem, we know 
that every eigenvalue of \(A\) lies within at least one of the Gersshgorin disk 
\[\{z \in \F \colon |z - A_{j,j}| \leq \sum_{k=1 k \neq j }^{n} |A_{j,k}|\}\]
for \(j \in \{1, \ldots, n\}\). From the question we can see that none of the eigenvalues are 0 and therefore 
\(A\) is strictly diagonally dominant and hence invertible. 

\end{proof}





%%%%%%%%%%%%%%%%%%%%%%%%%%%%%%%%%%%%%%%%%%%%%%%%%%%%%%%%%%%%
%%%%%%%%%%%%%%%%%%%%%%%%%%%% 5E %%%%%%%%%%%%%%%%%%%%%%%%%%%%
%%%%%%%%%%%%%%%%%%%%%%%%%%%%%%%%%%%%%%%%%%%%%%%%%%%%%%%%%%%%


\newpage 
\section*{5E: Commuting Operators}
\addcontentsline{toc}{section}{5E: Commuting Operators}


\begin{definition}[commute]
    \begin{itemize}
        \item Two operators \(S\) and \(T\) on the same vector space \textbf{commute} if \(ST = TS\).
        \item Two square matrices \(A\) and \(B\) of the same size \textbf{commute} if \(AB = BA\).
    \end{itemize}
\end{definition}

\begin{lemma}[commuting operators correspond to commuting matrices]
    Suppose \(S, T \in \Lc(V)\) and \(v_1, \ldots, v_n\) is a basis of \(V\). Then \(S\) and \(T\)
    commute if and only if \(\Mc(S, (v_1, \ldots, v_n))\) and \(\Mc(T, (v_1, \ldots, v_n))\) commute.
\end{lemma}


\begin{lemma}[eigenspace is invariant under commuting operator]
    Suppose \(S, T \in \Lc(V)\) commute and \(\lambda \in \F\). Then 
    \(E(\lambda, S)\) is invariant under \(T\).
\end{lemma}

\begin{thm}[simultaenous diagonalizablity \(\Longleftrightarrow\) commutativity]
    Two diagonalizable operators on the same vector space have diagonal matrices with 
    respect to the same basis if and only if the two operators commute.
\end{thm}


\begin{lemma}[common eigenvector for commuting operators]
    Every pairs of commuting operators on a finite-dimensional nonzero complex vector space has 
    a common eigenvector.
\end{lemma}


\begin{lemma}[commuting operators are simultaenously upper triangular]
    Suppose \(V\) is a finite-dimensional complex vector space and \(S, T\) are commuting operators on 
    \(V\). Then there is a basis of \(V\) with respect to which both 
    \(S\) and \(T\) have upper-triangular matrices. 
\end{lemma}


\begin{thm}[eigenvalues of sum and product of commuting operators]
    Suppose \(V\) is a finite-dimensional complex vector space and \(S, T\) are commuting operators 
    on \(V\). Then 
    \begin{itemize}
        \item Every eigenvalue of \(S + T\) is an eigenvalue of \(S\) plus an eigenvalue of \(T\), 
        \item Every eigenvalue of \(ST\) is an eigenvalue of \(S\) times an eigenvalue of \(T\).
    \end{itemize}
\end{thm}





%%%%%%%%%%%%%%%%%%%%%%%%%%%%%%%%%%%%%%%%%%%%%%%%%%%%%%%%%%%%%%%
%%%%%%%%%%%%%%%%%%%%%%%%%%%% 5E PS %%%%%%%%%%%%%%%%%%%%%%%%%%%%
%%%%%%%%%%%%%%%%%%%%%%%%%%%%%%%%%%%%%%%%%%%%%%%%%%%%%%%%%%%%%%%

\newpage 
\addcontentsline{toc}{subsection}{5E Problem Sets}


\begin{problem}{2}
    Suppose \(\Ec\) is a subset of \(\Lc(V)\) and every element of \(\Ec\) is diagonalizable. Prove that 
    there exists a basis of \(V\) with respect to which every element of \(\Ec\) has a diagonal matrix 
    if and only if every pair of element of \(\Ec\) commutes.
\end{problem}

\begin{proof}
\(\Rightarrow\) This follows by products of diagonal matrices.

\(\Leftarrow\) We are given that every pair of element of \(\Ec\) commutes. Pick any \(A \in \Ec\) and 
let \(E(\lambda) = \{v \in V \colon Av = \lambda v\}\) be the \(\lambda\)-eigenspace of \(A\). We know 
that as \(A\) is diagonalizable, 
\[V = \bigoplus_{\lambda \in \F} E(\lambda)\] 

Take another element \(B \in \Ec\), then we know that \(E(\lambda)\) is invairant under \(B\) and thus 
\(B|_{E(\lambda)}\) is diagnoalizable by 5.75. This means that we can further get that \(E(\lambda, \mu) 
= \{v \in E(\lambda) \colon Bv = \mu v\} = \{v \in V \colon Av=\lambda v, Bv = \mu v\}\), and we have that 
\[V = \bigoplus_{\lambda, \mu \in \F} E(\lambda, \mu)\]
Applying the argument recursively yields that 
\[V = \bigoplus_{\{\lambda_i\}_{i=1}^\infty \colon \lambda_i \in \F} E(\{\lambda_i\}_{i=1}^\infty) \]
where \(E(\{\lambda_i\}_{i=1}^\infty)\) consits of all vectors \(v\) such that 
\(T_i v = \lambda_i v\) for commuting operators \(T_i \in \Ec\). Note that even notionally we make 
\(\infty\) this is still a finite set since the collection of finite(ly nonzero eigenvalues) is still 
finite. Hence we derive a common eigenbasis and finish the proof.
\end{proof}


\begin{problem}{3}
    Suppose \(S, T \in \Lc(V)\) are such that \(ST = TS\). Suppose \(p \in \Pc(\F)\). 
    \begin{enumerate}[label=(\alph*)]
        \item Prove that \(\nul p(S)\) is invariant under \(T\). 
        \item Prove that \(\range p(S)\) is invairant under \(T\).
    \end{enumerate}
\end{problem}


\begin{proof}
(a) Let \(p(S) = a_0I + a_1 S + a_2 S^2 + \cdots + a_m S^m\). Take \(v \in \nul p(S)\), then this means that 
\(p(S)(v) = 0\). We have that 
\begin{align*}
    p(S)(Tv) = T(p(S)(v)) = T(0) = 0 
\end{align*}
Hence \(Tv \in \nul p(S)\). 

(b) Take \(v \in \range p(S)\), then this means there exists \(u \in V\) such that \(p(S)(u) = v\). Consider that 
\[p(S)(Tu) = T(p(S)u) = Tv\]
Therefore \(Tv \in \range p(S)\).
\end{proof}



\begin{problem}{5}
    Prove that a pair of operators on a finite-dimensional vector space commute if and only if 
    their dual operators commute.
\end{problem}

\begin{proof}
Take \(\phi \in V'\) and \(v \in V\). Then \(\Rightarrow\) assume \(TS = ST\), we have that 
\[(T^* S^* \phi)(v) = S^* \phi(Tv) = \phi (STv)\]
At the same time 
\[(S^*T^* \phi)(v) = T^* \phi (SV) = \phi (TSv)\]
Hence we have that \[T^*S^* = S^*T^*\]. 

\(\Leftarrow\) Assume \(T^* S^* = S^* T^*\), then 
\[(T^* S^* \phi)(v) = \phi(STv) = \phi (TSv) = (S^*T^* \phi)(v)\]
Since this holds for all \(\phi \in V', v \in V\), \(ST = TS\).
\end{proof}


\begin{problem}{6}
    Supose \(V\) is a complex vector space, \(S,T \in \Lc(V)\) commute. Prove that there exist 
    \(\alpha, \lambda \in \C\) such that 
    \[\range (S - \alpha I) + \range (T - \lambda I) \neq V.\]
\end{problem}

\begin{proof}
Let \(v_1, \ldots, v_n\) be the same basis such that \(\Mc(S), \Mc(T)\) are diagnoal. Then we let 
\(\alpha = \Mc(S)_{1,1}\) and \(\lambda = \Mc(T)_{1, 1}\). Then this means that 
\[\range (S - \lambda I) + \range (T - \lambda I) = V \backslash\text{span}(v_1) \neq V.\]
\end{proof}








\end{document}

