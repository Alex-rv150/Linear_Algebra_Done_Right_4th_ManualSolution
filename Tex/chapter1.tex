\documentclass{extarticle}
\sloppy
%%%%%%%%%%%%%%%%%%%%%%%%%%%%%%%%%%%%%%%%%%%%%%%%%%%%%%%%%%%%%%%%%%%%%%
% PACKAGES            																						  %
%%%%%%%%%%%%%%%%%%%%%%%%%%%%%%%%%%%%%%%%%%%%%%%%%%%%%%%%%%%%%%%%%%%%%
\usepackage[10pt]{extsizes}
\usepackage{amsfonts}
% \usepackage{amsthm}
\usepackage{amssymb}
\usepackage[shortlabels]{enumitem}
\usepackage{microtype} 
\usepackage{amsmath}
\usepackage{mathtools}
\usepackage{commath}
\usepackage{amsthm}
\usepackage{bbm}
\usepackage[colorlinks=true, allcolors=blue]{hyperref}

%%%%%%%%%%%%%%%%%%%%%%%%%%%%%%%%%%%%%%%%%%%%%%%%%%%%%%%%%%%%%%%%%%%%%%
% PROBLEM ENVIRONMENT         																			           %
%%%%%%%%%%%%%%%%%%%%%%%%%%%%%%%%%%%%%%%%%%%%%%%%%%%%%%%%%%%%%%%%%%%%%
\usepackage{tcolorbox}
\tcbuselibrary{theorems, breakable, skins}
\newtcbtheorem{prob}% environment name
              {Problem}% Title text
  {enhanced, % tcolorbox styles
  attach boxed title to top left={xshift = 4mm, yshift=-2mm},
  colback=blue!5, colframe=black, colbacktitle=blue!3, coltitle=black,
  boxed title style={size=small,colframe=gray},
  fonttitle=\bfseries,
  separator sign none
  }%
  {} 
\newenvironment{problem}[1]{\begin{prob*}{#1}{}}{\end{prob*}}

\newtcbtheorem{exer}% environment name
              {Exercise}% Title text
  {enhanced, % tcolorbox styles
  attach boxed title to top left={xshift = 4mm, yshift=-2mm},
  colback=blue!5, colframe=black, colbacktitle=blue!3, coltitle=black,
  boxed title style={size=small,colframe=gray},
  fonttitle=\bfseries,
  separator sign none
  }%
  {} 
\newenvironment{exercise}[1]{\begin{exer*}{#1}{}}{\end{exer*}}

%%%%%%%%%%%%%%%%%%%%%%%%%%%%%%%%%%%%%%%%%%%%%%%%%%%%%%%%%%%%%%%%%%%%%%
% THEOREMS/LEMMAS/ETC.         																			  %
%%%%%%%%%%%%%%%%%%%%%%%%%%%%%%%%%%%%%%%%%%%%%%%%%%%%%%%%%%%%%%%%%%%%%%
\newtheorem{thm}{Theorem}
\newtheorem*{thm-non}{Theorem}
\newtheorem{lemma}[thm]{Lemma}
\newtheorem{definition}[thm]{Definition}
\newtheorem{remark}[thm]{Remark}
\newtheorem{corollary}[thm]{Corollary}
\newtheorem{proposition}[thm]{Proposition}
% % \newtheorem*{example*}[Theorem]{Example}
% \newtheorem*{example}[thm]{Example}

\theoremstyle{remark}
\newtheorem*{example}{Example}
% \newtheorem*{example*}[thm-non]{Example}

%%%%%%%%%%%%%%%%%%%%%%%%%%%%%%%%%%%%%%%%%%%%%%%%%%%%%%%%%%%%%%%%%%%%%%
% SECTION NUMBERING																				           %
%%%%%%%%%%%%%%%%%%%%%%%%%%%%%%%%%%%%%%%%%%%%%%%%%%%%%%%%%%%%%%%%%%%%%%
\renewcommand\thesection{\Alph{section}:}

\makeatletter
\newcommand{\cbigoplus}{\DOTSB\cbigoplus@\slimits@}
\newcommand{\cbigoplus@}{\mathop{\widehat{\bigoplus}}}
\makeatother

%%%%%%%%%%%%%%%%%%%%%%%%%%%%%%%%%%%%%%%%%%%%%%%%%%%%%%%%%%%%%%%%%%%%%%
% Additional math commands																		           %
%%%%%%%%%%%%%%%%%%%%%%%%%%%%%%%%%%%%%%%%%%%%%%%%%%%%%%%%%%%%%%%%%%%%%%
% \newcommand{\E}{\mathbb{E}}
% \newcommand{\Prb}{\mathbb{P}}
% \newcommand{\R}{\mathbb{R}}
% \newcommand{\C}{\mathbb{C}}
% \newcommand{\I}{\mathbb{I}}
% \newcommand{\F}{\mathbb{F}}
% \newcommand{\vecc}{\mathbf{c}}
% \newcommand{\Z}{\mathbb{Z}}
% \newcommand{\nul}{\text{null }}
% \newcommand{\range}{\text{range }}
% \newcommand{\Span}{\text{span}}
% \newcommand{\tr}{\text{tr}}
% \newcommand{\diag}{\text{diag}}
% \newcommand{\limn}{\lim_{n \to \infty}}
% \newcommand{\zo}{\overline{z}}
% \newcommand{\wo}{\overline{w}}
% \newcommand{\ao}{\overline{a}}
% \newcommand{\vo}{\overline{v}}
% \newcommand{\glo}{\overline{\lambda}}
% \newcommand{\Var}{\text{Var}}
% \newcommand{\1}{\mathbbm{1}}
% \newcommand{\Pm}{\mathbbm{P}}

% Mathbold abbrevation
\newcommand{\ab}{\mathbf{a}}
\newcommand{\bb}{\mathbf{b}}
\newcommand{\cb}{\mathbf{c}}
\newcommand{\db}{\mathbf{d}}
\newcommand{\eb}{\mathbf{e}}
\newcommand{\fb}{\mathbf{f}}
\newcommand{\gb}{\mathbf{g}}
\newcommand{\hb}{\mathbf{h}}
\newcommand{\ib}{\mathbf{i}}
\newcommand{\jb}{\mathbf{j}}
\newcommand{\kb}{\mathbf{k}}
\newcommand{\lb}{\mathbf{l}}
\newcommand{\mb}{\mathbf{m}}
\newcommand{\nb}{\mathbf{n}}
\newcommand{\ob}{\mathbf{o}}
\newcommand{\pb}{\mathbf{p}}
\newcommand{\qb}{\mathbf{q}}
\newcommand{\rb}{\mathbf{r}}
% \newcommand{\sb}{\mathbf{s}} # already defined
\newcommand{\tb}{\mathbf{t}}
\newcommand{\ub}{\mathbf{u}}
\newcommand{\vb}{\mathbf{v}}
\newcommand{\wb}{\mathbf{w}}
\newcommand{\xb}{\mathbf{x}}
\newcommand{\yb}{\mathbf{y}}
\newcommand{\zb}{\mathbf{z}}
\newcommand{\Ab}{\mathbf{A}}
\newcommand{\Bb}{\mathbf{B}}
\newcommand{\Cb}{\mathbf{C}}
\newcommand{\Db}{\mathbf{D}}
\newcommand{\Eb}{\mathbf{E}}
\newcommand{\Fb}{\mathbf{F}}
\newcommand{\Gb}{\mathbf{G}}
\newcommand{\Hb}{\mathbf{H}}
\newcommand{\Ib}{\mathbf{I}}
\newcommand{\Jb}{\mathbf{J}}
\newcommand{\Kb}{\mathbf{K}}
\newcommand{\Lb}{\mathbf{L}}
\newcommand{\Mb}{\mathbf{M}}
\newcommand{\Nb}{\mathbf{N}}
\newcommand{\Ob}{\mathbf{O}}
\newcommand{\Pb}{\mathbf{P}}
\newcommand{\Qb}{\mathbf{Q}}
\newcommand{\Rb}{\mathbf{R}}
\newcommand{\Sb}{\mathbf{S}}
\newcommand{\Tb}{\mathbf{T}}
\newcommand{\Ub}{\mathbf{U}}
\newcommand{\Vb}{\mathbf{V}}
\newcommand{\Wb}{\mathbf{W}}
\newcommand{\Xb}{\mathbf{X}}
\newcommand{\Yb}{\mathbf{Y}}
\newcommand{\Zb}{\mathbf{Z}}


% MathCal abbrevation
\newcommand{\acal}{\mathcal{a}}
\newcommand{\bcal}{\mathcal{b}}
\newcommand{\ccal}{\mathcal{c}}
\newcommand{\dcal}{\mathcal{d}}
\newcommand{\ecal}{\mathcal{e}}
\newcommand{\fcal}{\mathcal{f}}
\newcommand{\gcal}{\mathcal{g}}
\newcommand{\hcal}{\mathcal{h}}
\newcommand{\ical}{\mathcal{i}}
\newcommand{\jcal}{\mathcal{j}}
\newcommand{\kcal}{\mathcal{k}}
\newcommand{\lcal}{\mathcal{l}}
\newcommand{\mcal}{\mathcal{m}}
\newcommand{\ncal}{\mathcal{n}}
\newcommand{\ocal}{\mathcal{o}}
\newcommand{\pcal}{\mathcal{p}}
\newcommand{\qcal}{\mathcal{q}}
\newcommand{\rcal}{\mathcal{r}}
\newcommand{\scal}{\mathcal{s}}
\newcommand{\tcal}{\mathcal{t}}
\newcommand{\ucal}{\mathcal{u}}
\newcommand{\vcal}{\mathcal{v}}
\newcommand{\wcal}{\mathcal{w}}
\newcommand{\xcal}{\mathcal{x}}
\newcommand{\ycal}{\mathcal{y}}
\newcommand{\zcal}{\mathcal{z}}
\newcommand{\Acal}{\mathcal{A}}
\newcommand{\Bcal}{\mathcal{B}}
\newcommand{\Ccal}{\mathcal{C}}
\newcommand{\Dcal}{\mathcal{D}}
\newcommand{\Ecal}{\mathcal{E}}
\newcommand{\Fcal}{\mathcal{F}}
\newcommand{\Gcal}{\mathcal{G}}
\newcommand{\Hcal}{\mathcal{H}}
\newcommand{\Ical}{\mathcal{I}}
\newcommand{\Jcal}{\mathcal{J}}
\newcommand{\Kcal}{\mathcal{K}}
\newcommand{\Lcal}{\mathcal{L}}
\newcommand{\Mcal}{\mathcal{M}}
\newcommand{\Ncal}{\mathcal{N}}
\newcommand{\Ocal}{\mathcal{O}}
\newcommand{\Pcal}{\mathcal{P}}
\newcommand{\Qcal}{\mathcal{Q}}
\newcommand{\Rcal}{\mathcal{R}}
\newcommand{\Scal}{\mathcal{S}}
\newcommand{\Tcal}{\mathcal{T}}
\newcommand{\Ucal}{\mathcal{U}}
\newcommand{\Vcal}{\mathcal{V}}
\newcommand{\Wcal}{\mathcal{W}}
\newcommand{\Xcal}{\mathcal{X}}
\newcommand{\Ycal}{\mathcal{Y}}
\newcommand{\Zcal}{\mathcal{Z}}


% MathCal abbrevation
\newcommand{\ac}{\mathcal{a}}
\newcommand{\bc}{\mathcal{b}}
% \newcommand{\cc}{\mathcal{c}}
\newcommand{\dc}{\mathcal{d}}
\newcommand{\ec}{\mathcal{e}}
\newcommand{\fc}{\mathcal{f}}
\newcommand{\gc}{\mathcal{g}}
\newcommand{\hc}{\mathcal{h}}
\newcommand{\ic}{\mathcal{i}}
\newcommand{\jc}{\mathcal{j}}
\newcommand{\kc}{\mathcal{k}}
\newcommand{\lc}{\mathcal{l}}
\newcommand{\mc}{\mathcal{m}}
\newcommand{\nc}{\mathcal{n}}
\newcommand{\oc}{\mathcal{o}}
\newcommand{\pc}{\mathcal{p}}
\newcommand{\qc}{\mathcal{q}}
\newcommand{\rc}{\mathcal{r}}
% \newcommand{\sc}{\mathcal{s}}
\newcommand{\tc}{\mathcal{t}}
\newcommand{\uc}{\mathcal{u}}
\newcommand{\vc}{\mathcal{v}}
\newcommand{\wc}{\mathcal{w}}
\newcommand{\xc}{\mathcal{x}}
\newcommand{\yc}{\mathcal{y}}
\newcommand{\zc}{\mathcal{z}}
\newcommand{\Ac}{\mathcal{A}}
\newcommand{\Bc}{\mathcal{B}}
\newcommand{\Cc}{\mathcal{C}}
\newcommand{\Dc}{\mathcal{D}}
\newcommand{\Ec}{\mathcal{E}}
\newcommand{\Fc}{\mathcal{F}}
\newcommand{\Gc}{\mathcal{G}}
\newcommand{\Hc}{\mathcal{H}}
\newcommand{\Ic}{\mathcal{I}}
\newcommand{\Jc}{\mathcal{J}}
\newcommand{\Kc}{\mathcal{K}}
\newcommand{\Lc}{\mathcal{L}}
\newcommand{\Mc}{\mathcal{M}}
\newcommand{\Nc}{\mathcal{N}}
\newcommand{\Oc}{\mathcal{O}}
\newcommand{\Pc}{\mathcal{P}}
\newcommand{\Qc}{\mathcal{Q}}
\newcommand{\Rc}{\mathcal{R}}
\newcommand{\Sc}{\mathcal{S}}
\newcommand{\Tc}{\mathcal{T}}
\newcommand{\Uc}{\mathcal{U}}
\newcommand{\Vc}{\mathcal{V}}
\newcommand{\Wc}{\mathcal{W}}
\newcommand{\Xc}{\mathcal{X}}
\newcommand{\Yc}{\mathcal{Y}}
\newcommand{\Zc}{\mathcal{Z}}

% Mathhat abbrevation
\newcommand{\ah}{\hat{a}}
\newcommand{\bh}{\hat{b}}
\newcommand{\ch}{\hat{c}}
% \newcommand{\dh}{\hat{d}}
\newcommand{\eh}{\hat{e}}
\newcommand{\fh}{\hat{f}}
\newcommand{\gh}{\hat{g}}
\newcommand{\hh}{\hat{h}}
\newcommand{\ih}{\hat{i}}
\newcommand{\jh}{\hat{j}}
\newcommand{\kh}{\hat{k}}
\newcommand{\lh}{\hat{l}}
\newcommand{\mh}{\hat{m}}
\newcommand{\nh}{\hat{n}}
\newcommand{\oh}{\hat{o}}
\newcommand{\ph}{\hat{p}}
\newcommand{\qh}{\hat{q}}
\newcommand{\rh}{\hat{r}}
\newcommand{\sh}{\hat{s}}
% \newcommand{\th}{\hat{t}}
\newcommand{\uh}{\hat{u}}
\newcommand{\vh}{\hat{v}}
\newcommand{\wh}{\hat{w}}
\newcommand{\xh}{\hat{x}}
\newcommand{\yh}{\hat{y}}
\newcommand{\zh}{\hat{z}}
\newcommand{\Ah}{\hat{A}}
\newcommand{\Bh}{\hat{B}}
\newcommand{\Ch}{\hat{C}}
\newcommand{\Dh}{\hat{D}}
\newcommand{\Eh}{\hat{E}}
\newcommand{\Fh}{\hat{F}}
\newcommand{\Gh}{\hat{G}}
\newcommand{\Hh}{\hat{H}}
\newcommand{\Ih}{\hat{I}}
\newcommand{\Jh}{\hat{J}}
\newcommand{\Kh}{\hat{K}}
\newcommand{\Lh}{\hat{L}}
\newcommand{\Mh}{\hat{M}}
\newcommand{\Nh}{\hat{N}}
\newcommand{\Oh}{\hat{O}}
\newcommand{\Ph}{\hat{P}}
\newcommand{\Qh}{\hat{Q}}
\newcommand{\Rh}{\hat{R}}
\newcommand{\Sh}{\hat{S}}
\newcommand{\Th}{\hat{T}}
\newcommand{\Uh}{\hat{U}}
\newcommand{\Vh}{\hat{V}}
\newcommand{\Wh}{\hat{W}}
\newcommand{\Xh}{\hat{X}}
\newcommand{\Yh}{\hat{Y}}
\newcommand{\Zh}{\hat{Z}}


% Mathbar abbrevation
\newcommand{\abar}{\bar{a}}
\newcommand{\bbar}{\bar{b}}
\newcommand{\cbar}{\bar{c}}
\newcommand{\dbar}{\bar{d}}
\newcommand{\ebar}{\bar{e}}
\newcommand{\fbar}{\bar{f}}
\newcommand{\gbar}{\bar{g}}
% \newcommand{\hbar}{\bar{h}}
\newcommand{\ibar}{\bar{i}}
\newcommand{\jbar}{\bar{j}}
\newcommand{\kbar}{\bar{k}}
\newcommand{\lbar}{\bar{l}}
\newcommand{\mbar}{\bar{m}}
\newcommand{\nbar}{\bar{n}}
\newcommand{\obar}{\bar{o}}
\newcommand{\pbar}{\bar{p}}
\newcommand{\qbar}{\bar{q}}
\newcommand{\rbar}{\bar{r}}
\newcommand{\sbar}{\bar{s}}
\newcommand{\tbar}{\bar{t}}
\newcommand{\ubar}{\bar{u}}
\newcommand{\vbar}{\bar{v}}
\newcommand{\wbar}{\bar{w}}
\newcommand{\xbar}{\bar{x}}
\newcommand{\ybar}{\bar{y}}
\newcommand{\zbar}{\bar{z}}
\newcommand{\Abar}{\bar{A}}
\newcommand{\Bbar}{\bar{B}}
\newcommand{\Cbar}{\bar{C}}
\newcommand{\Dbar}{\bar{D}}
\newcommand{\Ebar}{\bar{E}}
\newcommand{\Fbar}{\bar{F}}
\newcommand{\Gbar}{\bar{G}}
\newcommand{\Hbar}{\bar{H}}
\newcommand{\Ibar}{\bar{I}}
\newcommand{\Jbar}{\bar{J}}
\newcommand{\Kbar}{\bar{K}}
\newcommand{\Lbar}{\bar{L}}
\newcommand{\Mbar}{\bar{M}}
\newcommand{\Nbar}{\bar{N}}
\newcommand{\Obar}{\bar{O}}
\newcommand{\Pbar}{\bar{P}}
\newcommand{\Qbar}{\bar{Q}}
\newcommand{\Rbar}{\bar{R}}
\newcommand{\Sbar}{\bar{S}}
\newcommand{\Tbar}{\bar{T}}
\newcommand{\Ubar}{\bar{U}}
\newcommand{\Vbar}{\bar{V}}
\newcommand{\Wbar}{\bar{W}}
\newcommand{\Xbar}{\bar{X}}
\newcommand{\Ybar}{\bar{Y}}
\newcommand{\Zbar}{\bar{Z}}


% Mathtilde abbrevation
\newcommand{\at}{\tilde{a}}
\newcommand{\bt}{\tilde{b}}
\newcommand{\ct}{\tilde{c}}
\newcommand{\dt}{\tilde{d}}
\newcommand{\et}{\tilde{e}}
\newcommand{\ft}{\tilde{f}}
\newcommand{\gt}{\tilde{g}}
% \newcommand{\ht}{\tilde{h}}
% \newcommand{\it}{\tilde{i}}
\newcommand{\jt}{\tilde{j}}
\newcommand{\kt}{\tilde{k}}
\newcommand{\lt}{\tilde{l}}
\newcommand{\mt}{\tilde{m}}
\newcommand{\nt}{\tilde{n}}
\newcommand{\ot}{\tilde{o}}
\newcommand{\pt}{\tilde{p}}
\newcommand{\qt}{\tilde{q}}
\newcommand{\rt}{\tilde{r}}
\newcommand{\st}{\tilde{s}}
% \newcommand{\tt}{\tilde{t}}
\newcommand{\ut}{\tilde{u}}
\newcommand{\vt}{\tilde{v}}
\newcommand{\wt}{\tilde{w}}
\newcommand{\xt}{\tilde{x}}
\newcommand{\yt}{\tilde{y}}
\newcommand{\zt}{\tilde{z}}
\newcommand{\At}{\tilde{A}}
\newcommand{\Bt}{\tilde{B}}
\newcommand{\Ct}{\tilde{C}}
\newcommand{\Dt}{\tilde{D}}
\newcommand{\Et}{\tilde{E}}
\newcommand{\Ft}{\tilde{F}}
\newcommand{\Gt}{\tilde{G}}
\newcommand{\Ht}{\tilde{H}}
\newcommand{\It}{\tilde{I}}
\newcommand{\Jt}{\tilde{J}}
\newcommand{\Kt}{\tilde{K}}
\newcommand{\Lt}{\tilde{L}}
\newcommand{\Mt}{\tilde{M}}
\newcommand{\Nt}{\tilde{N}}
\newcommand{\Ot}{\tilde{O}}
\newcommand{\Pt}{\tilde{P}}
\newcommand{\Qt}{\tilde{Q}}
\newcommand{\Rt}{\tilde{R}}
\newcommand{\St}{\tilde{S}}
\newcommand{\Tt}{\tilde{T}}
\newcommand{\Ut}{\tilde{U}}
\newcommand{\Vt}{\tilde{V}}
\newcommand{\Wt}{\tilde{W}}
\newcommand{\Xt}{\tilde{X}}
\newcommand{\Yt}{\tilde{Y}}
\newcommand{\Zt}{\tilde{Z}}


% MathTilde abbrevation
\newcommand{\aT}{\Tilde{a}}
\newcommand{\bT}{\Tilde{b}}
\newcommand{\cT}{\Tilde{c}}
\newcommand{\dT}{\Tilde{d}}
\newcommand{\eT}{\Tilde{e}}
\newcommand{\fT}{\Tilde{f}}
\newcommand{\gT}{\Tilde{g}}
\newcommand{\hT}{\Tilde{h}}
\newcommand{\iT}{\Tilde{i}}
\newcommand{\jT}{\Tilde{j}}
\newcommand{\kT}{\Tilde{k}}
\newcommand{\lT}{\Tilde{l}}
\newcommand{\mT}{\Tilde{m}}
\newcommand{\nT}{\Tilde{n}}
\newcommand{\oT}{\Tilde{o}}
\newcommand{\pT}{\Tilde{p}}
\newcommand{\qT}{\Tilde{q}}
\newcommand{\rT}{\Tilde{r}}
\newcommand{\sT}{\Tilde{s}}
\newcommand{\tT}{\Tilde{t}}
\newcommand{\uT}{\Tilde{u}}
\newcommand{\vT}{\Tilde{v}}
\newcommand{\wT}{\Tilde{w}}
\newcommand{\xT}{\Tilde{x}}
\newcommand{\yT}{\Tilde{y}}
\newcommand{\zT}{\Tilde{z}}
\newcommand{\AT}{\Tilde{A}}
\newcommand{\BT}{\Tilde{B}}
\newcommand{\CT}{\Tilde{C}}
\newcommand{\DT}{\Tilde{D}}
\newcommand{\ET}{\Tilde{E}}
\newcommand{\FT}{\Tilde{F}}
\newcommand{\GT}{\Tilde{G}}
\newcommand{\HT}{\Tilde{H}}
\newcommand{\IT}{\Tilde{I}}
\newcommand{\JT}{\Tilde{J}}
% \newcommand{\KT}{\Tilde{K}}
\newcommand{\LT}{\Tilde{L}}
\newcommand{\MT}{\Tilde{M}}
% \newcommand{\NT}{\Tilde{N}}
\newcommand{\OT}{\Tilde{O}}
% \newcommand{\PT}{\Tilde{P}}
\newcommand{\QT}{\Tilde{Q}}
\newcommand{\RT}{\Tilde{R}}
\newcommand{\ST}{\Tilde{S}}
\newcommand{\TT}{\Tilde{T}}
\newcommand{\UT}{\Tilde{U}}
\newcommand{\VT}{\Tilde{V}}
\newcommand{\WT}{\Tilde{W}}
\newcommand{\XT}{\Tilde{X}}
\newcommand{\YT}{\Tilde{Y}}
\newcommand{\ZT}{\Tilde{Z}}



%Nested Case




% Mathbold + hat abbrevation
\newcommand{\abh}{\mathbf{\hat{a}}}
\newcommand{\bbh}{\mathbf{\hat{b}}}
\newcommand{\cbh}{\mathbf{\hat{c}}}
\newcommand{\dbh}{\mathbf{\hat{d}}}
\newcommand{\ebh}{\mathbf{\hat{e}}}
\newcommand{\fbh}{\mathbf{\hat{f}}}
\newcommand{\gbh}{\mathbf{\hat{g}}}
\newcommand{\hbh}{\mathbf{\hat{h}}}
\newcommand{\ibh}{\mathbf{\hat{i}}}
\newcommand{\jbh}{\mathbf{\hat{j}}}
\newcommand{\kbh}{\mathbf{\hat{k}}}
\newcommand{\lbh}{\mathbf{\hat{l}}}
\newcommand{\mbh}{\mathbf{\hat{m}}}
\newcommand{\nbh}{\mathbf{\hat{n}}}
\newcommand{\obh}{\mathbf{\hat{o}}}
\newcommand{\pbh}{\mathbf{\hat{p}}}
\newcommand{\qbh}{\mathbf{\hat{q}}}
\newcommand{\rbh}{\mathbf{\hat{r}}}
\newcommand{\sbh}{\mathbf{\hat{s}}}
\newcommand{\tbh}{\mathbf{\hat{t}}}
\newcommand{\ubh}{\mathbf{\hat{u}}}
\newcommand{\vbh}{\mathbf{\hat{v}}}
\newcommand{\wbh}{\mathbf{\hat{w}}}
\newcommand{\xbh}{\mathbf{\hat{x}}}
\newcommand{\ybh}{\mathbf{\hat{y}}}
\newcommand{\zbh}{\mathbf{\hat{z}}}
\newcommand{\Abh}{\mathbf{\hat{A}}}
\newcommand{\Bbh}{\mathbf{\hat{B}}}
\newcommand{\Cbh}{\mathbf{\hat{C}}}
\newcommand{\Dbh}{\mathbf{\hat{D}}}
\newcommand{\Ebh}{\mathbf{\hat{E}}}
\newcommand{\Fbh}{\mathbf{\hat{F}}}
\newcommand{\Gbh}{\mathbf{\hat{G}}}
\newcommand{\Hbh}{\mathbf{\hat{H}}}
\newcommand{\Ibh}{\mathbf{\hat{I}}}
\newcommand{\Jbh}{\mathbf{\hat{J}}}
\newcommand{\Kbh}{\mathbf{\hat{K}}}
\newcommand{\Lbh}{\mathbf{\hat{L}}}
\newcommand{\Mbh}{\mathbf{\hat{M}}}
\newcommand{\Nbh}{\mathbf{\hat{N}}}
\newcommand{\Obh}{\mathbf{\hat{O}}}
\newcommand{\Pbh}{\mathbf{\hat{P}}}
\newcommand{\Qbh}{\mathbf{\hat{Q}}}
\newcommand{\Rbh}{\mathbf{\hat{R}}}
\newcommand{\Sbh}{\mathbf{\hat{S}}}
\newcommand{\Tbh}{\mathbf{\hat{T}}}
\newcommand{\Ubh}{\mathbf{\hat{U}}}
\newcommand{\Vbh}{\mathbf{\hat{V}}}
\newcommand{\Wbh}{\mathbf{\hat{W}}}
\newcommand{\Xbh}{\mathbf{\hat{X}}}
\newcommand{\Ybh}{\mathbf{\hat{Y}}}
\newcommand{\Zbh}{\mathbf{\hat{Z}}}


% Mathbold + Tilde abbrevation
\newcommand{\abT}{\mathbf{\Tilde{a}}}
\newcommand{\bbT}{\mathbf{\Tilde{b}}}
\newcommand{\cbT}{\mathbf{\Tilde{c}}}
\newcommand{\dbT}{\mathbf{\Tilde{d}}}
\newcommand{\ebT}{\mathbf{\Tilde{e}}}
\newcommand{\fbT}{\mathbf{\Tilde{f}}}
\newcommand{\gbT}{\mathbf{\Tilde{g}}}
\newcommand{\hbT}{\mathbf{\Tilde{h}}}
\newcommand{\ibT}{\mathbf{\Tilde{i}}}
\newcommand{\jbT}{\mathbf{\Tilde{j}}}
\newcommand{\kbT}{\mathbf{\Tilde{k}}}
\newcommand{\lbT}{\mathbf{\Tilde{l}}}
\newcommand{\mbT}{\mathbf{\Tilde{m}}}
\newcommand{\nbT}{\mathbf{\Tilde{n}}}
\newcommand{\obT}{\mathbf{\Tilde{o}}}
\newcommand{\pbT}{\mathbf{\Tilde{p}}}
\newcommand{\qbT}{\mathbf{\Tilde{q}}}
\newcommand{\rbT}{\mathbf{\Tilde{r}}}
\newcommand{\sbT}{\mathbf{\Tilde{s}}}
\newcommand{\tbT}{\mathbf{\Tilde{t}}}
\newcommand{\ubT}{\mathbf{\Tilde{u}}}
\newcommand{\vbT}{\mathbf{\Tilde{v}}}
\newcommand{\wbT}{\mathbf{\Tilde{w}}}
\newcommand{\xbT}{\mathbf{\Tilde{x}}}
\newcommand{\ybT}{\mathbf{\Tilde{y}}}
\newcommand{\zbT}{\mathbf{\Tilde{z}}}
\newcommand{\AbT}{\mathbf{\Tilde{A}}}
\newcommand{\BbT}{\mathbf{\Tilde{B}}}
\newcommand{\CbT}{\mathbf{\Tilde{C}}}
\newcommand{\DbT}{\mathbf{\Tilde{D}}}
\newcommand{\EbT}{\mathbf{\Tilde{E}}}
\newcommand{\FbT}{\mathbf{\Tilde{F}}}
\newcommand{\GbT}{\mathbf{\Tilde{G}}}
\newcommand{\HbT}{\mathbf{\Tilde{H}}}
\newcommand{\IbT}{\mathbf{\Tilde{I}}}
\newcommand{\JbT}{\mathbf{\Tilde{J}}}
\newcommand{\KbT}{\mathbf{\Tilde{K}}}
\newcommand{\LbT}{\mathbf{\Tilde{L}}}
\newcommand{\MbT}{\mathbf{\Tilde{M}}}
\newcommand{\NbT}{\mathbf{\Tilde{N}}}
\newcommand{\ObT}{\mathbf{\Tilde{O}}}
\newcommand{\PbT}{\mathbf{\Tilde{P}}}
\newcommand{\QbT}{\mathbf{\Tilde{Q}}}
\newcommand{\RbT}{\mathbf{\Tilde{R}}}
\newcommand{\SbT}{\mathbf{\Tilde{S}}}
\newcommand{\TbT}{\mathbf{\Tilde{T}}}
\newcommand{\UbT}{\mathbf{\Tilde{U}}}
\newcommand{\VbT}{\mathbf{\Tilde{V}}}
\newcommand{\WbT}{\mathbf{\Tilde{W}}}
\newcommand{\XbT}{\mathbf{\Tilde{X}}}
\newcommand{\YbT}{\mathbf{\Tilde{Y}}}
\newcommand{\ZbT}{\mathbf{\Tilde{Z}}}


% Mathbold + tilde abbrevation
\newcommand{\abt}{\mathbf{\tilde{a}}}
\newcommand{\bbt}{\mathbf{\tilde{b}}}
\newcommand{\cbt}{\mathbf{\tilde{c}}}
\newcommand{\dbt}{\mathbf{\tilde{d}}}
\newcommand{\ebt}{\mathbf{\tilde{e}}}
\newcommand{\fbt}{\mathbf{\tilde{f}}}
\newcommand{\gbt}{\mathbf{\tilde{g}}}
\newcommand{\hbt}{\mathbf{\tilde{h}}}
\newcommand{\ibt}{\mathbf{\tilde{i}}}
\newcommand{\jbt}{\mathbf{\tilde{j}}}
\newcommand{\kbt}{\mathbf{\tilde{k}}}
\newcommand{\lbt}{\mathbf{\tilde{l}}}
\newcommand{\mbt}{\mathbf{\tilde{m}}}
\newcommand{\nbt}{\mathbf{\tilde{n}}}
\newcommand{\obt}{\mathbf{\tilde{o}}}
\newcommand{\pbt}{\mathbf{\tilde{p}}}
\newcommand{\qbt}{\mathbf{\tilde{q}}}
\newcommand{\rbt}{\mathbf{\tilde{r}}}
\newcommand{\sbt}{\mathbf{\tilde{s}}}
\newcommand{\tbt}{\mathbf{\tilde{t}}}
\newcommand{\ubt}{\mathbf{\tilde{u}}}
\newcommand{\vbt}{\mathbf{\tilde{v}}}
\newcommand{\wbt}{\mathbf{\tilde{w}}}
\newcommand{\xbt}{\mathbf{\tilde{x}}}
\newcommand{\ybt}{\mathbf{\tilde{y}}}
\newcommand{\zbt}{\mathbf{\tilde{z}}}
\newcommand{\Abt}{\mathbf{\tilde{A}}}
\newcommand{\Bbt}{\mathbf{\tilde{B}}}
\newcommand{\Cbt}{\mathbf{\tilde{C}}}
\newcommand{\Dbt}{\mathbf{\tilde{D}}}
\newcommand{\Ebt}{\mathbf{\tilde{E}}}
\newcommand{\Fbt}{\mathbf{\tilde{F}}}
\newcommand{\Gbt}{\mathbf{\tilde{G}}}
\newcommand{\Hbt}{\mathbf{\tilde{H}}}
\newcommand{\Ibt}{\mathbf{\tilde{I}}}
\newcommand{\Jbt}{\mathbf{\tilde{J}}}
\newcommand{\Kbt}{\mathbf{\tilde{K}}}
\newcommand{\Lbt}{\mathbf{\tilde{L}}}
\newcommand{\Mbt}{\mathbf{\tilde{M}}}
\newcommand{\Nbt}{\mathbf{\tilde{N}}}
\newcommand{\Obt}{\mathbf{\tilde{O}}}
\newcommand{\Pbt}{\mathbf{\tilde{P}}}
\newcommand{\Qbt}{\mathbf{\tilde{Q}}}
\newcommand{\Rbt}{\mathbf{\tilde{R}}}
\newcommand{\Sbt}{\mathbf{\tilde{S}}}
\newcommand{\Tbt}{\mathbf{\tilde{T}}}
\newcommand{\Ubt}{\mathbf{\tilde{U}}}
\newcommand{\Vbt}{\mathbf{\tilde{V}}}
\newcommand{\Wbt}{\mathbf{\tilde{W}}}
\newcommand{\Xbt}{\mathbf{\tilde{X}}}
\newcommand{\Ybt}{\mathbf{\tilde{Y}}}
\newcommand{\Zbt}{\mathbf{\tilde{Z}}}


% Mathbold + bar abbrevation
\newcommand{\abb}{\mathbf{\bar{a}}}
\newcommand{\bbb}{\mathbf{\bar{b}}}
\newcommand{\cbb}{\mathbf{\bar{c}}}
\newcommand{\dbb}{\mathbf{\bar{d}}}
\newcommand{\ebb}{\mathbf{\bar{e}}}
\newcommand{\fbb}{\mathbf{\bar{f}}}
\newcommand{\gbb}{\mathbf{\bar{g}}}
\newcommand{\hbb}{\mathbf{\bar{h}}}
\newcommand{\ibb}{\mathbf{\bar{i}}}
\newcommand{\jbb}{\mathbf{\bar{j}}}
\newcommand{\kbb}{\mathbf{\bar{k}}}
\newcommand{\lbb}{\mathbf{\bar{l}}}
\newcommand{\mbb}{\mathbf{\bar{m}}}
\newcommand{\nbb}{\mathbf{\bar{n}}}
\newcommand{\obb}{\mathbf{\bar{o}}}
\newcommand{\pbb}{\mathbf{\bar{p}}}
\newcommand{\qbb}{\mathbf{\bar{q}}}
\newcommand{\rbb}{\mathbf{\bar{r}}}
\newcommand{\sbb}{\mathbf{\bar{s}}}
\newcommand{\tbb}{\mathbf{\bar{t}}}
\newcommand{\ubb}{\mathbf{\bar{u}}}
\newcommand{\vbb}{\mathbf{\bar{v}}}
\newcommand{\wbb}{\mathbf{\bar{w}}}
\newcommand{\xbb}{\mathbf{\bar{x}}}
\newcommand{\ybb}{\mathbf{\bar{y}}}
\newcommand{\zbb}{\mathbf{\bar{z}}}
\newcommand{\Abb}{\mathbf{\bar{A}}}
% \newcommand{\Bbb}{\mathbf{\bar{B}}}
\newcommand{\Cbb}{\mathbf{\bar{C}}}
\newcommand{\Dbb}{\mathbf{\bar{D}}}
\newcommand{\Ebb}{\mathbf{\bar{E}}}
\newcommand{\Fbb}{\mathbf{\bar{F}}}
\newcommand{\Gbb}{\mathbf{\bar{G}}}
\newcommand{\Hbb}{\mathbf{\bar{H}}}
\newcommand{\Ibb}{\mathbf{\bar{I}}}
\newcommand{\Jbb}{\mathbf{\bar{J}}}
\newcommand{\Kbb}{\mathbf{\bar{K}}}
\newcommand{\Lbb}{\mathbf{\bar{L}}}
\newcommand{\Mbb}{\mathbf{\bar{M}}}
\newcommand{\Nbb}{\mathbf{\bar{N}}}
\newcommand{\Obb}{\mathbf{\bar{O}}}
\newcommand{\Pbb}{\mathbf{\bar{P}}}
\newcommand{\Qbb}{\mathbf{\bar{Q}}}
\newcommand{\Rbb}{\mathbf{\bar{R}}}
\newcommand{\Sbb}{\mathbf{\bar{S}}}
\newcommand{\Tbb}{\mathbf{\bar{T}}}
\newcommand{\Ubb}{\mathbf{\bar{U}}}
\newcommand{\Vbb}{\mathbf{\bar{V}}}
\newcommand{\Wbb}{\mathbf{\bar{W}}}
\newcommand{\Xbb}{\mathbf{\bar{X}}}
\newcommand{\Ybb}{\mathbf{\bar{Y}}}
\newcommand{\Zbb}{\mathbf{\bar{Z}}}


\newcommand{\E}{\mathbb{E}}
\newcommand{\Prb}{\mathbb{P}}
\newcommand{\R}{\mathbb{R}}
\newcommand{\C}{\mathbb{C}}
\newcommand{\I}{\mathbb{I}}
\newcommand{\F}{\mathbb{F}}
\newcommand{\vecc}{\mathbf{c}}
\newcommand{\Z}{\mathbb{Z}}
\newcommand{\nul}{\text{null }}
\newcommand{\range}{\text{range }}
\newcommand{\Span}{\text{span}}
\newcommand{\tr}{\text{tr}}
\newcommand{\diag}{\text{diag}}
\newcommand{\limn}{\lim_{n \to \infty}}
\newcommand{\zo}{\overline{z}}
\newcommand{\wo}{\overline{w}}
\newcommand{\ao}{\overline{a}}
\newcommand{\vo}{\overline{v}}
\newcommand{\glo}{\overline{\lambda}}
\newcommand{\Var}{\text{Var}}
\newcommand{\1}{\mathbbm{1}}
\newcommand{\Pm}{\mathbbm{P}}

\title{\vspace{-2em}Chapter 1: Vector Spaces}
\author{\emph{Linear Algebra Done Right (4th Edition)}, by Sheldon Axler}
\date{Last updated: \today}

\begin{document}
\maketitle 
\tableofcontents
\newpage

\section*{1A: $\R^n$ and $\C^n$}
\addcontentsline{toc}{section}{1A: $\R^n$ and $\C^n$}

\begin{problem}{1}
    Show that \(\alpha + \beta = \beta + \alpha\) for all \(\alpha, 
    \beta \in \C\). 
\end{problem}

\begin{proof}
    Let \(\alpha = a + bi, \beta = c + di\). Then we have that 
    \begin{align*}
        \alpha + \beta 
        &= (a + bi) + (c + di) \\ 
        &= (c + di) + (a + bi) \\ 
        &= \beta + \alpha
    \end{align*}
\end{proof}

\begin{problem}{3}
    Show that \((\alpha \beta) \gamma = \alpha(\beta \gamma)\) for all \(\alpha,
    \beta, \gamma \in \C\). 
\end{problem}

\begin{proof}
Choose arbitrary \(\alpha, \beta, \gamma \in \C\). Denote \(\alpha = a + bi,
\beta=c + di, \gamma = e + fi\). Then we have that 

\begin{align*}
    (\alpha \beta) \gamma 
    &= ((ac - bd) + (ad + bc)i ) (e + fi) \\ 
    &=( ace - bde - adf - bcf ) + (ade + bce + acf - bdf)i
\end{align*}

At the same time, we have 

\begin{align*}
    \alpha (\beta \gamma) 
    &= (a + bi)((ce - df) + (cf + de)i) \\ 
    &= (ace - adf - bcf - bde) + (ade + acf + bce - bdf)i
\end{align*}

Hence, \((\alpha \beta)\gamma = \alpha (\beta \gamma)\). 
\end{proof}

\begin{problem}{5}
    Show that for any \(\alpha \in \C\), there exists a unique \(\beta \in \C\) such
    that \(\alpha + \beta = 0\). 
\end{problem}

\begin{proof}
Denote \(\alpha = a + bi\). By the property of a field, we know there exists unique 
\(c = -a\) and \(d = -b\) such that \(\beta = c + di\) and \(\alpha + \beta=0\). Suppose
for the sake of contradiction that \(\beta\) is not unique, then there exists 
\(c' + d'i\) such that \((a + c') + (b + d')i = 0\) while \(c' \neq a\) or \(d' \neq d\),
contradicting the uniqueness of additive inverse property.
\end{proof}


\begin{problem}{8}
    Find two distinct squared roots of \(i\). 
\end{problem}

\begin{proof}
Suppose \(\alpha = a + bi\)'s square equals one. 

\begin{equation*}
    (a + bi)^2 = a^2 - b^2 + 2abi = 1
\end{equation*}

Then \(|a| = |b|, 2ab = 1 \). So we get the solution to be 
\[\frac{1}{\sqrt{2}} + \frac{1}{\sqrt{2}}, -\frac{1}{\sqrt{2}} - \frac{1}{\sqrt{2}}i\] 
\end{proof}


\begin{problem}{10}
    Show that \((\xb + \yb) + \zb = \xb + (\yb + \zb) \ \forall  \xb, \yb, \zb \in \F^n\). 
\end{problem}

\begin{proof}
\begin{align*}
    (\xb + \yb) + \zb 
    &= (x_1 + y_1, \ldots, x_n + y_n) + (z_1 , \ldots, z_n) \\ 
    &= (x_1, \ldots, x_n) + (y_1 + z_1, \ldots, y_n + z_n) \\ 
    &= \xb + (\yb + \zb)
\end{align*}
\end{proof}

\begin{problem}{14}
    Show that \(\gamma(\xb + \yb) = \gamma \xb + \gamma \yb \ \forall \gamma \in \F, \xb, \yb \in \F^n\).
\end{problem}

\begin{proof}
\begin{align*}
    \gamma(\xb + \yb)
    &= \gamma (x_1 + y_1, \ldots, x_n + y_n) \\ 
    &= (\gamma(x_1 + y_1), \ldots, \gamma(x_n + y_n)) \\ 
    &= (\gamma x_1 + \gamma y_1, \ldots, \gamma x_n + \gamma y_n) \\ 
    &= \gamma (x_1, \ldots, x_n) + \gamma(y_1, \ldots, y_n) \\ 
    &= \gamma \xb + \gamma \yb 
\end{align*}
\end{proof}

\newpage 

\section*{1B: Definition of Vector Space}
\addcontentsline{toc}{section}{1B: Definition of Vector Space}

\begin{thm}
    A vector space is a set that is closed under \emph{vector addition} and \emph{scalar multiplication}. 
    It also has the following properties:
    \begin{itemize}
        \item commutativity 
        \item associativity 
        \item additive identity 
        \item additive inverse 
        \item multiplicative identity
        \item (scalar) distributive 
    \end{itemize}
\end{thm}

Notation: \(\F^S\). 

Explanation: If \(S\) is a set, then \(\F^S\) denotes the set of functions from \(S\) to \(\F\) (scalar
function). e.g. \(f \in \F^S\). 

Comment: \(\F^S\) is a vector space; One can think of \(f \in \F^n\) as \(f:
\{1, 2,\ldots, n\} \to \F\). 

\begin{problem}{1}
    Prove that \(-(-\vb) = \vb\) for all \(\vb \in V\). 
\end{problem}

\begin{proof}
We know \(- (-\vb)\) is the unique additive inverse of \(-\vb\). At same time by definition,
\(\vb + (-\vb) = 0\) and thus by commutativity \((-\vb) + \vb = 0\). This shows that \(\vb\)
is the unique additive inverse of \((-\vb)\), and such that \(-(-\vb) = \vb\). 
\end{proof}

\begin{problem}{2}
    Suppose \(a \in \F, \vb \in V\), and \(a \vb = 0\). Prove that \(a = 0\)
    or \(\vb = 0\). 
\end{problem}

\begin{proof}
Suppose for the sake of contradiction that \(a \neq 0\) and \(\vb \neq 0\) but 
\(a \vb = 0\).  
\begin{align*}
   \vb &= 1 \vb \\ 
   \vb &= \frac{1}{a} \cdot a \vb \\ 
   \vb &= \frac{1}{a} 0  
\end{align*}
This forms a contradiction. 
\end{proof}

\begin{problem}{3}
    Suppose \(\vb, \wb \in V\). Explain why there exists a unique \(\xb \in V\) such 
    that \(\vb + 3\xb = \wb\). 
\end{problem}

\begin{proof}
Suppose there exists \(\xb'\) which also satisfies the condition. Then we have
\begin{equation}
    \vb + 3\xb = \wb \ \  \ \ \vb + 3 \xb' = \wb 
\end{equation}

This gives that \(\xb = (\wb - \vb) / 3 = \xb'\) which shows \(\xb\) is unique. 
\end{proof}

\begin{problem}{4}
    The empty set is not a vector space, why? 
\end{problem}

\begin{proof}
There is no additive identity in the empty set. 
\end{proof}

\begin{problem}{7}
    Suppose \(S\) is a nonempty set. Let \(V^S\) denotes the set of functions from \(S\)
    to \(V\). Define a natural addition and scalar multiplication on \(V^S\), and show 
    that \(V^S\) is a vector space with these definitions. 
\end{problem}

\begin{proof}
Let \(f, g \in V^S \colon S \to V\). Define the addition and multiplication to be that 
\[f + g (x) = f(x) + g(x)\]

We have that 
\begin{itemize}
    \item commutativity: \(f + g(x) = f(x) + g(x) = g(x) + f(x) = g+f (x)\)
    \item associativity: \((f + g)+h (x) = f(x) + g(x) + h(x) = f + (g + h)(x)\)
    \item additive identity: Define \(0\colon S \to 0 \in V\), then \(f+0(x) = 0 + f(x) = f(x)\)
    \item additive inverse: for every \(f \in V^S\), define \(g(x) = -f(x)\) which exists
    by the property of vector space and thus we have that \(g + f = 0\) for every \(x\)
    and thus that it exists. 
    \item multiplicative identity: same as above 
    \item (scalar) distributive: \(a(f+g)(x) = a(f(x) + g(x)) = af(x)+g(x)\)
\end{itemize}
\end{proof}

\newpage 
\section*{1C: Subspaces}
\addcontentsline{toc}{section}{1C: Subspaces}

\begin{definition}[subspace]
    A subset \(\Ucal\) of \(V\) is called \emph{subspace} of \(V\) if \(\Ucal\)
    is also a vector space with the same additive identity, addition, and 
    scalar multiplication as on \(V\). 
\end{definition}

\begin{remark}
    The set \(\{0\}\) is the smallest subspace of \(V\), and \(V\) itself 
    is the largest subspace of \(V\). 
\end{remark}

\begin{remark}
    The subspace of \(\R^2\) are precisely \(\{0\}\), all lines in \(\R^2\)
    containing the origin, \(\R^2\). 
\end{remark}

\begin{definition}[Sum of subspace] 
    Suppose \(V_1, \cdots, V_m\) are subspaces. The \emph{sum} of them, denoted
    by \(V_1 + \cdots + V_m\), is the set of all possible sums of element 
    of \(V_1, \cdots, V_m\). Specifically,

    \[V_1 + \cdots + V_m = \{v_1 + \cdots + v_m \colon v_1 \in V_1, \cdots, v_m \in V_m\}\]
\end{definition}

\begin{lemma}
    Suppose \(V_1, \cdots, V_m\) are subspaces of \(V\). Then \(V_1 + \cdots +V_m\)
    is the smallest subspace of \(V\) containing \(V_1, \cdots, V_m\).
\end{lemma}

\begin{definition}[Direct Sum]
    Suppose \(V_1, \cdots, V_m\) are subspaces of \(V\). 
    \begin{itemize}
        \item The sum \(V_1 + \cdots + V_m\) is called a \emph{direct sum} if 
        each element of \(V_1 + \cdots + V_m\) can be written only as a sum 
        of \(v_1, \cdots, v_m\), where each \(v_k \in V_k\). 
        \item If \(V_1 + \cdots + V_m\) is a direct sum, then \(V_1 \oplus \cdots \oplus V_m\)
        denotes \(V_1 + \cdots + V_m\), with \(\oplus\) serving as the 
        indication that this is a direct sum. 
    \end{itemize}
\end{definition}

\begin{example}
    Suppose \(V_k\) is a subspace of \(\F^n\) of those vectors whose coordinates
are all zero but \(k\)-th coordinate. Then we have 
\[\F^n = V_1 \oplus \cdots + V_m\]
\end{example}

\begin{example}[Sum that is not a direct sum]
    Suppose 
    \begin{align*}
        & V_1 = \{(x, y, 0) \in \F^3 \colon  x, y \in \F\} \\ 
        & V_2 = \{(0, 0, z) \in \F^3 \colon z \in \F\} \\ 
        & V_3 = \{(0, y, y) \in \F^3 \colon y \in \F\}
    \end{align*}

    Then \(F^3 = V_1 + V_2 + V_3\) because for every \((x, y, z) \in \F^3\), 

    \[(x, y, z) = (x, y, 0) + (0, 0, z) + (0, 0, 0)\]

    However, \(F^3 \neq V_1 \oplus V_2 \oplus V_3\) since 
    \begin{align*}
        (0, 0, 0) &= (0, -1, 0) + (0, 0, -1) + (0, 1, 1) \\ 
        &= (0, 0, 0) + (0, 0, 0) + (0, 0, 0)
    \end{align*}
\end{example}

\begin{thm}
    Suppose \(V_1, \ldots, V_m\) are subspaces of \(V\). Then \(V_1 + \cdots + V_m\) 
    is a direct sum if and only if the only way to write 0 as a sum of \(v_1 + \cdots + v_m\),
    where \(v_k \in V_k\), is by taking each \(v_k\) to equal 0. 
\end{thm}

\begin{thm}
    Suppose that \(\Ucal\) and \(\Wcal\) are subspaces of \(V\). Then 
    \[\Ucal + \Wcal \text{ is a direct sum} \Longleftrightarrow \Ucal \cap \Wcal = \{0\}\]
\end{thm}

\begin{problem}{1}
Verify the following examples to be valid subspaces:
\begin{enumerate}
    \item If \(b \in \F\), then 
    \[\{(x_1, x_2, x_3, x_4) \in \F^4 \colon x_3 = 5x_4 + b\} \]
    is a subspace of \(\F^4\) if and only if \(b = 0\).
    \item The set of continuous real-valued functions on the interval \([0, 1]\) is 
    a subspace of \(\R^{[0,1]}\). 
    \item The set of differential real-valued functions on \(\R\) is a subspace of \(\R^\R\).
    \item The set of differentiable real-valued functions \(f\) on the interval \((0,3)\)
    such that \(f'(2) = b\) is a subspace of \(\R^{(0,3)}\) if and only if \(b=0\).
    \item The set of all sequences of complex numbers with limit 0 is a subspace of \(\C^\infty\).
\end{enumerate}
\end{problem}

\begin{proof}
\begin{enumerate}
    \item \(\Rightarrow\) \((0, 0, 0, 0)\) is an element and thus \(0 = 0 + b\), b = 0 
    \(\Leftarrow\) Easy to verify. 
    \item 0 function is cts; cts functions are closed under addition and scalar multiplication.
    \item 0 function is differentiable; differentiable functions are closed under addition and 
    scalar multiplication. 
    \item For this to be closed under addition, one needs to restrict that \(f'(2) + g'(2) = b + b = b\) and thus 
    \(b = 0\).
    \item \(\limn a(S_1 + S_2) = \limn a S_1 + \limn aS_2 = 0 + 0 = 0\). At the same time, 
    the 0 sequence has limit 0.
\end{enumerate}
\end{proof}

\begin{problem}{4}
    Suppose \(b \in \R\). Show that the set of continuous real-valued functions \(f\)
    on the interval \([0, 1]\) such that \(\int_0^1 f = b\) is a subspace of \(\R^{[0,1]}\)
    if and only if \(b = 0\). 
\end{problem}

\begin{proof}
\(\Rightarrow\) \(\int_0^1 f + g = \int_0^1 f + \int_0^1 g = 2b = b\) so \(b =0\). 

\(\Leftarrow\) 0 is in the set; closed under addition/multiplication.
\end{proof}

\begin{problem}{5}
    Prove that \(\R^2\) is not a subspace of \(\C^2\) over the field \(\C\). 
\end{problem}

\begin{proof}

This does not hold for scalar multiplication since we've defined scalar to be complex 
numbers. To see this, Let \(a = (x + yi) \in \C\) and \(\zb = (z_1, z_2) \in \R^2\). We 
have that \(a\zb = (z_1(x+yi), z_2 (x+yi)) \notin \R^2\). 

\end{proof}

\begin{problem}{7}
    Prove or disprove: If \(\Ucal\) is a nonempty subset of \(\R^2\) such that \(\Ucal\) is closed 
    under addition and under taking additive inverse (\(-u \in \Ucal\)), then 
    \(\Ucal\) is a subspace in \(\R^2\).
\end{problem}

\begin{proof}
No. \(\Ucal = \{(x_1, x_2) \colon x_1, x_2 \in \Z\}\). Then \(\frac{1}{2}(1, 1) \notin \Ucal\).
\end{proof}

\begin{problem}{9}
    A function \(f\colon \R \to \R\) is called \emph{periodic} if there exists a 
    positive number \(p\) s.t. \(f(x) = f(x + p)\) for all \(x \in \R\). Is the 
    set of periodic functions from \(\R\) to \(\R\) a subspace of \(\R^\R\)?
\end{problem}

\begin{proof}
No, problem occurs with the addition. Suppose we have \(f(x) = f(x + p)\) and 
\(g(x) = g(x + q)\). Then \((f + g)(x) = f(x) + g(x) = f(x + p) + g(x + q) \neq 
(f + g)(x + l)\) for some fixed \(l\) for all \(p,q\). 
\end{proof}

\begin{problem}{11}
    Prove that the intersection of every collection of subspaces of \(V\) is a subspace 
    of \(V\). 
\end{problem}

\begin{proof}
Let \(\bigcap_i V_i\) denote the collection of subspaces of \(V\). Then we know 
\(0 \in bigcup_i V_i\). Let \(a \in \F, \xb, \yb \in \bigcap_i V_i\). We have that
\(a(\xb + \yb) \in \bigcap_i V_i\) and thus finish the proof.
\end{proof}

\begin{problem}{12}
    Prove that the union of two subspaces of \(V\) is a subspace of \(V\) if 
    and only if one of the subspaces is contained in the other. 
\end{problem}

\begin{proof}
Let \(V_1, V_2\) be two subspaces of \(V\).

\(\Rightarrow\) Suppose for the sake of contradiction that 
there exists \(v_1 \in V_1\) s.t. \(v_1 \notin V_2\) and \(v_2 \in V_2\) s.t. 
\(v_2 \notin V_1\). Then by assumption we have that \(v_1 + v_2 \in V_1 \cup V_2\).
Here we can also show that \(v_1 + v_2 \notin V_1\) because if it does, \(v_1 +
v_2 + (-v_1) = v_2 \in V_1\). Similarly, \(v_1 + v_2 \notin V_2\). Thus we've reached
a contradiction. 

\(\Leftarrow\) This direction is trivial. 
\end{proof}

% \begin{problem}{13}
%     Prove that the union of three subspaces of \(V\) if and only if one of 
%     the subspaces contains the other two. 
% \end{problem}

% \begin{proof}
% Let \(V_1, V_2, V_3\) be three subspaces of \(V\). 

% \(\Rightarrow\) Suppose for the sake of contradiction that none of the subspace 
% contains the other two. Then there exists \(v_1 \in V_1\) s.t. 
% \(v_1 \notin V_2 \), \(v_2 \in V_2\) s.t. \(v_2 \notin V_1 \), and
% \(v_3 \in V_3\) s.t. \(v_3 \notin V_1\) (or it recovers the ). This means that \(av_1 + v_2 \notin V_1 \cup V_2\)
% and \((a-1)v_1 + v_2 \notin V_1 \cup V_2\) for some \(a \in \F\). Then \(av_1 + v_2 \in V_3\) and 
% \((a - 1) v_1 + v_2 \in V_3\) and thus \(v_1 \in V_3\)

% % This means that \(v_1 + v_2 + v_3 \notin V_1 \cup V_2 \cup V_3\) because if it does, then 
% % assume \(v_1 + v_2 + v_3 \in V_1\), \(v_2\)

% \(\Leftarrow\) This direction is trivial. 
% \end{proof}

\begin{problem}{14}

    Suppose 
    \[\Ucal = \{(x, -x, 2x)\in \F^3 \colon x \in \F\} \text{ and } \Wcal =\{(x, x, 2x) \in \F^3 \colon x \in \F\}\]

Describe \(\Ucal + \Wcal\).
    
\end{problem}

\begin{proof}
\[(x, -x, 2x) + (y, y, 2y) = (x + y, -x + y, 2(x + y))\]

One can think of this as \(\Ucal + \Wcal = \{(a, b, 2a) \colon a,b \in \F\}\). 
\end{proof}

\begin{problem}{15}
    Suppose \(\Ucal\) is a subspace of \(V\), what is \(\Ucal + \Ucal\)?
\end{problem}

\begin{proof}
\(\Ucal + \Ucal = \Ucal\). 

Take \(\ub_1 + \ub_2 \in \Ucal + \Ucal\), then \(\ub_1 + \ub_2 \in \Ucal\). Conversely,
take \(\ub \in \Ucal\), then \(\ub = \ub + 0 \in \Ucal + \Ucal\).
\end{proof}

\begin{problem}{16}
    Is the operation of addition on the subspaces of \(V\) commutative (\(\Ucal +
    \Wcal = \Wcal + \Ucal\))? 
\end{problem}

\begin{proof}
Take \(u \in \Ucal, w \in \Wcal\), then \(u + w = w + u\), implying the conclusion. 
\end{proof}

\begin{problem}{18}
    Does the operation of addition on the subspaces of \(V\) have an additive 
    identity? Which subspaces have additive inverses?
\end{problem}

\begin{proof}
Yes, the zero subspace \emph{i.e.} \(\{0\}\) is the additive identity. The subspace
that have additive inverses is only \(\{0\}\). 
\end{proof}

\begin{problem}{19}
    Prove or disprove: If \(V_1, V_2, \Ucal\) are subspaces of \(V\) such that 
    \[V_1 + \Ucal = V_2 + \Ucal\]
    then \(V_1 = V_2\).
\end{problem}

\begin{proof}
Counterexample: Consider when \(\Ucal = V_1 \bigsqcup V_2\), then the relation holds 
while \(V_1 \neq V_2\).
\end{proof}

\begin{problem}{20}
    Suppose 
    \[\Ucal = \{(x,x,y,y) \in \F^4 \colon x, y \in \F\}\]
    Find a subspace \(\Wcal \in\) \(\F^4\) s.t. \(\F^4  =\Ucal \oplus \Wcal\).
\end{problem}

\begin{proof}
Define \(\Wcal = \{(0, a, b, 0) \in \F^4 \colon a,b \in \F\}\) to be a subspace 
of \(\F^4\).

Then first \(\Wcal + \Ucal \subseteq \F^4\). Take \((q,w,e,r) \in \F^4\), we 
can have \((q,w,e,r) = (q,q,r,r) + (0,w-q,e-r,r) \in \Ucal + \Wcal\). We have 
\(\F^4 = \Ucal + \Wcal\). Furthermore, take \((x,x,y,y) \in \Ucal, (0,a,b,0) \in 
\Wcal\). For the element to be in the intersection, we need to have 
\((x,x,y,y)=(0,a,b,0)\) which implies that \(x=y=a=b=0\) and thus \(\Wcal 
\cap \Ucal = \{0\}\). 
\end{proof}

\begin{problem}{21}
    Suppose 
    \[\Ucal = \{x, y, x+y, x-y, 2x\} \in \F^5 \colon x,y \in \F\]
    Find a subspace \(\Wcal \in \F^5\) s.t. \(\F^5 = \Ucal \oplus \Wcal\). 
\end{problem}

\begin{proof}
Define 
\[\Wcal = \{(0,0,m,n,z) \colon m,n,z \in \F\}\]

Then \((a,b,c,d,e) = (a, b, a+b, a-b, 2a) + (0,0, c-(a+b), d-(a - b), e-2a)\). The 
rest follows exactly as in P20.
\end{proof}

\begin{problem}{23}
    Prove or disprove: If \(V_1, V_2, \Ucal\) are subspaces of \(V\) s.t. 
    \[V = V_1 \oplus \Ucal \text{ and } V = V_2 \oplus \Ucal ,\] 
    then \(V_1 = V_2\).
\end{problem}

\begin{proof}
Counterexample: Let \(\Ucal = \{(x,x) \colon x \in \F\}, V_1 = \{(x,0)\colon x \in \F\},
V_2 = \{(0, x) \colon x \in \F\}\). 
\end{proof}

\begin{problem}{24}
    A function \(f \colon \R \to \R\) is called \emph{even} if \(f(-x)=f(x)\) and 
    \emph{odd} if \(f(-x) = -f(x)\) for all \(x \in \R\). Let \(V_e\) denote the 
    set of real-valued even functions on \(\R\) and let \(V_o\) denote the 
    set of real-valued odd functions on \(\R\). Show that \(\R^\R = V_e \oplus V_o\).
\end{problem}

\begin{proof}
\(\Leftarrow\) Trivial direction. 

\(\Rightarrow\) \(f(x) = \frac{f(x) + f(-x)}{2} + \frac{f(x) - f(-x)}{2}\)
\end{proof}


\end{document}