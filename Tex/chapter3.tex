\documentclass{extarticle}
\sloppy
%%%%%%%%%%%%%%%%%%%%%%%%%%%%%%%%%%%%%%%%%%%%%%%%%%%%%%%%%%%%%%%%%%%%%%
% PACKAGES            																						  %
%%%%%%%%%%%%%%%%%%%%%%%%%%%%%%%%%%%%%%%%%%%%%%%%%%%%%%%%%%%%%%%%%%%%%
\usepackage[10pt]{extsizes}
\usepackage{amsfonts}
% \usepackage{amsthm}
\usepackage{amssymb}
\usepackage[shortlabels]{enumitem}
\usepackage{microtype} 
\usepackage{amsmath}
\usepackage{mathtools}
\usepackage{commath}
\usepackage{amsthm}
\usepackage{bbm}
\usepackage[colorlinks=true, allcolors=blue]{hyperref}

%%%%%%%%%%%%%%%%%%%%%%%%%%%%%%%%%%%%%%%%%%%%%%%%%%%%%%%%%%%%%%%%%%%%%%
% PROBLEM ENVIRONMENT         																			           %
%%%%%%%%%%%%%%%%%%%%%%%%%%%%%%%%%%%%%%%%%%%%%%%%%%%%%%%%%%%%%%%%%%%%%
\usepackage{tcolorbox}
\tcbuselibrary{theorems, breakable, skins}
\newtcbtheorem{prob}% environment name
              {Problem}% Title text
  {enhanced, % tcolorbox styles
  attach boxed title to top left={xshift = 4mm, yshift=-2mm},
  colback=blue!5, colframe=black, colbacktitle=blue!3, coltitle=black,
  boxed title style={size=small,colframe=gray},
  fonttitle=\bfseries,
  separator sign none
  }%
  {} 
\newenvironment{problem}[1]{\begin{prob*}{#1}{}}{\end{prob*}}

\newtcbtheorem{exer}% environment name
              {Exercise}% Title text
  {enhanced, % tcolorbox styles
  attach boxed title to top left={xshift = 4mm, yshift=-2mm},
  colback=blue!5, colframe=black, colbacktitle=blue!3, coltitle=black,
  boxed title style={size=small,colframe=gray},
  fonttitle=\bfseries,
  separator sign none
  }%
  {} 
\newenvironment{exercise}[1]{\begin{exer*}{#1}{}}{\end{exer*}}

%%%%%%%%%%%%%%%%%%%%%%%%%%%%%%%%%%%%%%%%%%%%%%%%%%%%%%%%%%%%%%%%%%%%%%
% THEOREMS/LEMMAS/ETC.         																			  %
%%%%%%%%%%%%%%%%%%%%%%%%%%%%%%%%%%%%%%%%%%%%%%%%%%%%%%%%%%%%%%%%%%%%%%
\newtheorem{thm}{Theorem}
\newtheorem*{thm-non}{Theorem}
\newtheorem{lemma}[thm]{Lemma}
\newtheorem{definition}[thm]{Definition}
\newtheorem{remark}[thm]{Remark}
\newtheorem{corollary}[thm]{Corollary}
\newtheorem{proposition}[thm]{Proposition}
% % \newtheorem*{example*}[Theorem]{Example}
% \newtheorem*{example}[thm]{Example}

\theoremstyle{remark}
\newtheorem*{example}{Example}
% \newtheorem*{example*}[thm-non]{Example}

%%%%%%%%%%%%%%%%%%%%%%%%%%%%%%%%%%%%%%%%%%%%%%%%%%%%%%%%%%%%%%%%%%%%%%
% SECTION NUMBERING																				           %
%%%%%%%%%%%%%%%%%%%%%%%%%%%%%%%%%%%%%%%%%%%%%%%%%%%%%%%%%%%%%%%%%%%%%%
\renewcommand\thesection{\Alph{section}:}

\makeatletter
\newcommand{\cbigoplus}{\DOTSB\cbigoplus@\slimits@}
\newcommand{\cbigoplus@}{\mathop{\widehat{\bigoplus}}}
\makeatother

%%%%%%%%%%%%%%%%%%%%%%%%%%%%%%%%%%%%%%%%%%%%%%%%%%%%%%%%%%%%%%%%%%%%%%
% Additional math commands																		           %
%%%%%%%%%%%%%%%%%%%%%%%%%%%%%%%%%%%%%%%%%%%%%%%%%%%%%%%%%%%%%%%%%%%%%%
% \newcommand{\E}{\mathbb{E}}
% \newcommand{\Prb}{\mathbb{P}}
% \newcommand{\R}{\mathbb{R}}
% \newcommand{\C}{\mathbb{C}}
% \newcommand{\I}{\mathbb{I}}
% \newcommand{\F}{\mathbb{F}}
% \newcommand{\vecc}{\mathbf{c}}
% \newcommand{\Z}{\mathbb{Z}}
% \newcommand{\nul}{\text{null }}
% \newcommand{\range}{\text{range }}
% \newcommand{\Span}{\text{span}}
% \newcommand{\tr}{\text{tr}}
% \newcommand{\diag}{\text{diag}}
% \newcommand{\limn}{\lim_{n \to \infty}}
% \newcommand{\zo}{\overline{z}}
% \newcommand{\wo}{\overline{w}}
% \newcommand{\ao}{\overline{a}}
% \newcommand{\vo}{\overline{v}}
% \newcommand{\glo}{\overline{\lambda}}
% \newcommand{\Var}{\text{Var}}
% \newcommand{\1}{\mathbbm{1}}
% \newcommand{\Pm}{\mathbbm{P}}

% Mathbold abbrevation
\newcommand{\ab}{\mathbf{a}}
\newcommand{\bb}{\mathbf{b}}
\newcommand{\cb}{\mathbf{c}}
\newcommand{\db}{\mathbf{d}}
\newcommand{\eb}{\mathbf{e}}
\newcommand{\fb}{\mathbf{f}}
\newcommand{\gb}{\mathbf{g}}
\newcommand{\hb}{\mathbf{h}}
\newcommand{\ib}{\mathbf{i}}
\newcommand{\jb}{\mathbf{j}}
\newcommand{\kb}{\mathbf{k}}
\newcommand{\lb}{\mathbf{l}}
\newcommand{\mb}{\mathbf{m}}
\newcommand{\nb}{\mathbf{n}}
\newcommand{\ob}{\mathbf{o}}
\newcommand{\pb}{\mathbf{p}}
\newcommand{\qb}{\mathbf{q}}
\newcommand{\rb}{\mathbf{r}}
% \newcommand{\sb}{\mathbf{s}} # already defined
\newcommand{\tb}{\mathbf{t}}
\newcommand{\ub}{\mathbf{u}}
\newcommand{\vb}{\mathbf{v}}
\newcommand{\wb}{\mathbf{w}}
\newcommand{\xb}{\mathbf{x}}
\newcommand{\yb}{\mathbf{y}}
\newcommand{\zb}{\mathbf{z}}
\newcommand{\Ab}{\mathbf{A}}
\newcommand{\Bb}{\mathbf{B}}
\newcommand{\Cb}{\mathbf{C}}
\newcommand{\Db}{\mathbf{D}}
\newcommand{\Eb}{\mathbf{E}}
\newcommand{\Fb}{\mathbf{F}}
\newcommand{\Gb}{\mathbf{G}}
\newcommand{\Hb}{\mathbf{H}}
\newcommand{\Ib}{\mathbf{I}}
\newcommand{\Jb}{\mathbf{J}}
\newcommand{\Kb}{\mathbf{K}}
\newcommand{\Lb}{\mathbf{L}}
\newcommand{\Mb}{\mathbf{M}}
\newcommand{\Nb}{\mathbf{N}}
\newcommand{\Ob}{\mathbf{O}}
\newcommand{\Pb}{\mathbf{P}}
\newcommand{\Qb}{\mathbf{Q}}
\newcommand{\Rb}{\mathbf{R}}
\newcommand{\Sb}{\mathbf{S}}
\newcommand{\Tb}{\mathbf{T}}
\newcommand{\Ub}{\mathbf{U}}
\newcommand{\Vb}{\mathbf{V}}
\newcommand{\Wb}{\mathbf{W}}
\newcommand{\Xb}{\mathbf{X}}
\newcommand{\Yb}{\mathbf{Y}}
\newcommand{\Zb}{\mathbf{Z}}


% MathCal abbrevation
\newcommand{\acal}{\mathcal{a}}
\newcommand{\bcal}{\mathcal{b}}
\newcommand{\ccal}{\mathcal{c}}
\newcommand{\dcal}{\mathcal{d}}
\newcommand{\ecal}{\mathcal{e}}
\newcommand{\fcal}{\mathcal{f}}
\newcommand{\gcal}{\mathcal{g}}
\newcommand{\hcal}{\mathcal{h}}
\newcommand{\ical}{\mathcal{i}}
\newcommand{\jcal}{\mathcal{j}}
\newcommand{\kcal}{\mathcal{k}}
\newcommand{\lcal}{\mathcal{l}}
\newcommand{\mcal}{\mathcal{m}}
\newcommand{\ncal}{\mathcal{n}}
\newcommand{\ocal}{\mathcal{o}}
\newcommand{\pcal}{\mathcal{p}}
\newcommand{\qcal}{\mathcal{q}}
\newcommand{\rcal}{\mathcal{r}}
\newcommand{\scal}{\mathcal{s}}
\newcommand{\tcal}{\mathcal{t}}
\newcommand{\ucal}{\mathcal{u}}
\newcommand{\vcal}{\mathcal{v}}
\newcommand{\wcal}{\mathcal{w}}
\newcommand{\xcal}{\mathcal{x}}
\newcommand{\ycal}{\mathcal{y}}
\newcommand{\zcal}{\mathcal{z}}
\newcommand{\Acal}{\mathcal{A}}
\newcommand{\Bcal}{\mathcal{B}}
\newcommand{\Ccal}{\mathcal{C}}
\newcommand{\Dcal}{\mathcal{D}}
\newcommand{\Ecal}{\mathcal{E}}
\newcommand{\Fcal}{\mathcal{F}}
\newcommand{\Gcal}{\mathcal{G}}
\newcommand{\Hcal}{\mathcal{H}}
\newcommand{\Ical}{\mathcal{I}}
\newcommand{\Jcal}{\mathcal{J}}
\newcommand{\Kcal}{\mathcal{K}}
\newcommand{\Lcal}{\mathcal{L}}
\newcommand{\Mcal}{\mathcal{M}}
\newcommand{\Ncal}{\mathcal{N}}
\newcommand{\Ocal}{\mathcal{O}}
\newcommand{\Pcal}{\mathcal{P}}
\newcommand{\Qcal}{\mathcal{Q}}
\newcommand{\Rcal}{\mathcal{R}}
\newcommand{\Scal}{\mathcal{S}}
\newcommand{\Tcal}{\mathcal{T}}
\newcommand{\Ucal}{\mathcal{U}}
\newcommand{\Vcal}{\mathcal{V}}
\newcommand{\Wcal}{\mathcal{W}}
\newcommand{\Xcal}{\mathcal{X}}
\newcommand{\Ycal}{\mathcal{Y}}
\newcommand{\Zcal}{\mathcal{Z}}


% MathCal abbrevation
\newcommand{\ac}{\mathcal{a}}
\newcommand{\bc}{\mathcal{b}}
% \newcommand{\cc}{\mathcal{c}}
\newcommand{\dc}{\mathcal{d}}
\newcommand{\ec}{\mathcal{e}}
\newcommand{\fc}{\mathcal{f}}
\newcommand{\gc}{\mathcal{g}}
\newcommand{\hc}{\mathcal{h}}
\newcommand{\ic}{\mathcal{i}}
\newcommand{\jc}{\mathcal{j}}
\newcommand{\kc}{\mathcal{k}}
\newcommand{\lc}{\mathcal{l}}
\newcommand{\mc}{\mathcal{m}}
\newcommand{\nc}{\mathcal{n}}
\newcommand{\oc}{\mathcal{o}}
\newcommand{\pc}{\mathcal{p}}
\newcommand{\qc}{\mathcal{q}}
\newcommand{\rc}{\mathcal{r}}
% \newcommand{\sc}{\mathcal{s}}
\newcommand{\tc}{\mathcal{t}}
\newcommand{\uc}{\mathcal{u}}
\newcommand{\vc}{\mathcal{v}}
\newcommand{\wc}{\mathcal{w}}
\newcommand{\xc}{\mathcal{x}}
\newcommand{\yc}{\mathcal{y}}
\newcommand{\zc}{\mathcal{z}}
\newcommand{\Ac}{\mathcal{A}}
\newcommand{\Bc}{\mathcal{B}}
\newcommand{\Cc}{\mathcal{C}}
\newcommand{\Dc}{\mathcal{D}}
\newcommand{\Ec}{\mathcal{E}}
\newcommand{\Fc}{\mathcal{F}}
\newcommand{\Gc}{\mathcal{G}}
\newcommand{\Hc}{\mathcal{H}}
\newcommand{\Ic}{\mathcal{I}}
\newcommand{\Jc}{\mathcal{J}}
\newcommand{\Kc}{\mathcal{K}}
\newcommand{\Lc}{\mathcal{L}}
\newcommand{\Mc}{\mathcal{M}}
\newcommand{\Nc}{\mathcal{N}}
\newcommand{\Oc}{\mathcal{O}}
\newcommand{\Pc}{\mathcal{P}}
\newcommand{\Qc}{\mathcal{Q}}
\newcommand{\Rc}{\mathcal{R}}
\newcommand{\Sc}{\mathcal{S}}
\newcommand{\Tc}{\mathcal{T}}
\newcommand{\Uc}{\mathcal{U}}
\newcommand{\Vc}{\mathcal{V}}
\newcommand{\Wc}{\mathcal{W}}
\newcommand{\Xc}{\mathcal{X}}
\newcommand{\Yc}{\mathcal{Y}}
\newcommand{\Zc}{\mathcal{Z}}

% Mathhat abbrevation
\newcommand{\ah}{\hat{a}}
\newcommand{\bh}{\hat{b}}
\newcommand{\ch}{\hat{c}}
% \newcommand{\dh}{\hat{d}}
\newcommand{\eh}{\hat{e}}
\newcommand{\fh}{\hat{f}}
\newcommand{\gh}{\hat{g}}
\newcommand{\hh}{\hat{h}}
\newcommand{\ih}{\hat{i}}
\newcommand{\jh}{\hat{j}}
\newcommand{\kh}{\hat{k}}
\newcommand{\lh}{\hat{l}}
\newcommand{\mh}{\hat{m}}
\newcommand{\nh}{\hat{n}}
\newcommand{\oh}{\hat{o}}
\newcommand{\ph}{\hat{p}}
\newcommand{\qh}{\hat{q}}
\newcommand{\rh}{\hat{r}}
\newcommand{\sh}{\hat{s}}
% \newcommand{\th}{\hat{t}}
\newcommand{\uh}{\hat{u}}
\newcommand{\vh}{\hat{v}}
\newcommand{\wh}{\hat{w}}
\newcommand{\xh}{\hat{x}}
\newcommand{\yh}{\hat{y}}
\newcommand{\zh}{\hat{z}}
\newcommand{\Ah}{\hat{A}}
\newcommand{\Bh}{\hat{B}}
\newcommand{\Ch}{\hat{C}}
\newcommand{\Dh}{\hat{D}}
\newcommand{\Eh}{\hat{E}}
\newcommand{\Fh}{\hat{F}}
\newcommand{\Gh}{\hat{G}}
\newcommand{\Hh}{\hat{H}}
\newcommand{\Ih}{\hat{I}}
\newcommand{\Jh}{\hat{J}}
\newcommand{\Kh}{\hat{K}}
\newcommand{\Lh}{\hat{L}}
\newcommand{\Mh}{\hat{M}}
\newcommand{\Nh}{\hat{N}}
\newcommand{\Oh}{\hat{O}}
\newcommand{\Ph}{\hat{P}}
\newcommand{\Qh}{\hat{Q}}
\newcommand{\Rh}{\hat{R}}
\newcommand{\Sh}{\hat{S}}
\newcommand{\Th}{\hat{T}}
\newcommand{\Uh}{\hat{U}}
\newcommand{\Vh}{\hat{V}}
\newcommand{\Wh}{\hat{W}}
\newcommand{\Xh}{\hat{X}}
\newcommand{\Yh}{\hat{Y}}
\newcommand{\Zh}{\hat{Z}}


% Mathbar abbrevation
\newcommand{\abar}{\bar{a}}
\newcommand{\bbar}{\bar{b}}
\newcommand{\cbar}{\bar{c}}
\newcommand{\dbar}{\bar{d}}
\newcommand{\ebar}{\bar{e}}
\newcommand{\fbar}{\bar{f}}
\newcommand{\gbar}{\bar{g}}
% \newcommand{\hbar}{\bar{h}}
\newcommand{\ibar}{\bar{i}}
\newcommand{\jbar}{\bar{j}}
\newcommand{\kbar}{\bar{k}}
\newcommand{\lbar}{\bar{l}}
\newcommand{\mbar}{\bar{m}}
\newcommand{\nbar}{\bar{n}}
\newcommand{\obar}{\bar{o}}
\newcommand{\pbar}{\bar{p}}
\newcommand{\qbar}{\bar{q}}
\newcommand{\rbar}{\bar{r}}
\newcommand{\sbar}{\bar{s}}
\newcommand{\tbar}{\bar{t}}
\newcommand{\ubar}{\bar{u}}
\newcommand{\vbar}{\bar{v}}
\newcommand{\wbar}{\bar{w}}
\newcommand{\xbar}{\bar{x}}
\newcommand{\ybar}{\bar{y}}
\newcommand{\zbar}{\bar{z}}
\newcommand{\Abar}{\bar{A}}
\newcommand{\Bbar}{\bar{B}}
\newcommand{\Cbar}{\bar{C}}
\newcommand{\Dbar}{\bar{D}}
\newcommand{\Ebar}{\bar{E}}
\newcommand{\Fbar}{\bar{F}}
\newcommand{\Gbar}{\bar{G}}
\newcommand{\Hbar}{\bar{H}}
\newcommand{\Ibar}{\bar{I}}
\newcommand{\Jbar}{\bar{J}}
\newcommand{\Kbar}{\bar{K}}
\newcommand{\Lbar}{\bar{L}}
\newcommand{\Mbar}{\bar{M}}
\newcommand{\Nbar}{\bar{N}}
\newcommand{\Obar}{\bar{O}}
\newcommand{\Pbar}{\bar{P}}
\newcommand{\Qbar}{\bar{Q}}
\newcommand{\Rbar}{\bar{R}}
\newcommand{\Sbar}{\bar{S}}
\newcommand{\Tbar}{\bar{T}}
\newcommand{\Ubar}{\bar{U}}
\newcommand{\Vbar}{\bar{V}}
\newcommand{\Wbar}{\bar{W}}
\newcommand{\Xbar}{\bar{X}}
\newcommand{\Ybar}{\bar{Y}}
\newcommand{\Zbar}{\bar{Z}}


% Mathtilde abbrevation
\newcommand{\at}{\tilde{a}}
\newcommand{\bt}{\tilde{b}}
\newcommand{\ct}{\tilde{c}}
\newcommand{\dt}{\tilde{d}}
\newcommand{\et}{\tilde{e}}
\newcommand{\ft}{\tilde{f}}
\newcommand{\gt}{\tilde{g}}
% \newcommand{\ht}{\tilde{h}}
% \newcommand{\it}{\tilde{i}}
\newcommand{\jt}{\tilde{j}}
\newcommand{\kt}{\tilde{k}}
\newcommand{\lt}{\tilde{l}}
\newcommand{\mt}{\tilde{m}}
\newcommand{\nt}{\tilde{n}}
\newcommand{\ot}{\tilde{o}}
\newcommand{\pt}{\tilde{p}}
\newcommand{\qt}{\tilde{q}}
\newcommand{\rt}{\tilde{r}}
\newcommand{\st}{\tilde{s}}
% \newcommand{\tt}{\tilde{t}}
\newcommand{\ut}{\tilde{u}}
\newcommand{\vt}{\tilde{v}}
\newcommand{\wt}{\tilde{w}}
\newcommand{\xt}{\tilde{x}}
\newcommand{\yt}{\tilde{y}}
\newcommand{\zt}{\tilde{z}}
\newcommand{\At}{\tilde{A}}
\newcommand{\Bt}{\tilde{B}}
\newcommand{\Ct}{\tilde{C}}
\newcommand{\Dt}{\tilde{D}}
\newcommand{\Et}{\tilde{E}}
\newcommand{\Ft}{\tilde{F}}
\newcommand{\Gt}{\tilde{G}}
\newcommand{\Ht}{\tilde{H}}
\newcommand{\It}{\tilde{I}}
\newcommand{\Jt}{\tilde{J}}
\newcommand{\Kt}{\tilde{K}}
\newcommand{\Lt}{\tilde{L}}
\newcommand{\Mt}{\tilde{M}}
\newcommand{\Nt}{\tilde{N}}
\newcommand{\Ot}{\tilde{O}}
\newcommand{\Pt}{\tilde{P}}
\newcommand{\Qt}{\tilde{Q}}
\newcommand{\Rt}{\tilde{R}}
\newcommand{\St}{\tilde{S}}
\newcommand{\Tt}{\tilde{T}}
\newcommand{\Ut}{\tilde{U}}
\newcommand{\Vt}{\tilde{V}}
\newcommand{\Wt}{\tilde{W}}
\newcommand{\Xt}{\tilde{X}}
\newcommand{\Yt}{\tilde{Y}}
\newcommand{\Zt}{\tilde{Z}}


% MathTilde abbrevation
\newcommand{\aT}{\Tilde{a}}
\newcommand{\bT}{\Tilde{b}}
\newcommand{\cT}{\Tilde{c}}
\newcommand{\dT}{\Tilde{d}}
\newcommand{\eT}{\Tilde{e}}
\newcommand{\fT}{\Tilde{f}}
\newcommand{\gT}{\Tilde{g}}
\newcommand{\hT}{\Tilde{h}}
\newcommand{\iT}{\Tilde{i}}
\newcommand{\jT}{\Tilde{j}}
\newcommand{\kT}{\Tilde{k}}
\newcommand{\lT}{\Tilde{l}}
\newcommand{\mT}{\Tilde{m}}
\newcommand{\nT}{\Tilde{n}}
\newcommand{\oT}{\Tilde{o}}
\newcommand{\pT}{\Tilde{p}}
\newcommand{\qT}{\Tilde{q}}
\newcommand{\rT}{\Tilde{r}}
\newcommand{\sT}{\Tilde{s}}
\newcommand{\tT}{\Tilde{t}}
\newcommand{\uT}{\Tilde{u}}
\newcommand{\vT}{\Tilde{v}}
\newcommand{\wT}{\Tilde{w}}
\newcommand{\xT}{\Tilde{x}}
\newcommand{\yT}{\Tilde{y}}
\newcommand{\zT}{\Tilde{z}}
\newcommand{\AT}{\Tilde{A}}
\newcommand{\BT}{\Tilde{B}}
\newcommand{\CT}{\Tilde{C}}
\newcommand{\DT}{\Tilde{D}}
\newcommand{\ET}{\Tilde{E}}
\newcommand{\FT}{\Tilde{F}}
\newcommand{\GT}{\Tilde{G}}
\newcommand{\HT}{\Tilde{H}}
\newcommand{\IT}{\Tilde{I}}
\newcommand{\JT}{\Tilde{J}}
% \newcommand{\KT}{\Tilde{K}}
\newcommand{\LT}{\Tilde{L}}
\newcommand{\MT}{\Tilde{M}}
% \newcommand{\NT}{\Tilde{N}}
\newcommand{\OT}{\Tilde{O}}
% \newcommand{\PT}{\Tilde{P}}
\newcommand{\QT}{\Tilde{Q}}
\newcommand{\RT}{\Tilde{R}}
\newcommand{\ST}{\Tilde{S}}
\newcommand{\TT}{\Tilde{T}}
\newcommand{\UT}{\Tilde{U}}
\newcommand{\VT}{\Tilde{V}}
\newcommand{\WT}{\Tilde{W}}
\newcommand{\XT}{\Tilde{X}}
\newcommand{\YT}{\Tilde{Y}}
\newcommand{\ZT}{\Tilde{Z}}



%Nested Case




% Mathbold + hat abbrevation
\newcommand{\abh}{\mathbf{\hat{a}}}
\newcommand{\bbh}{\mathbf{\hat{b}}}
\newcommand{\cbh}{\mathbf{\hat{c}}}
\newcommand{\dbh}{\mathbf{\hat{d}}}
\newcommand{\ebh}{\mathbf{\hat{e}}}
\newcommand{\fbh}{\mathbf{\hat{f}}}
\newcommand{\gbh}{\mathbf{\hat{g}}}
\newcommand{\hbh}{\mathbf{\hat{h}}}
\newcommand{\ibh}{\mathbf{\hat{i}}}
\newcommand{\jbh}{\mathbf{\hat{j}}}
\newcommand{\kbh}{\mathbf{\hat{k}}}
\newcommand{\lbh}{\mathbf{\hat{l}}}
\newcommand{\mbh}{\mathbf{\hat{m}}}
\newcommand{\nbh}{\mathbf{\hat{n}}}
\newcommand{\obh}{\mathbf{\hat{o}}}
\newcommand{\pbh}{\mathbf{\hat{p}}}
\newcommand{\qbh}{\mathbf{\hat{q}}}
\newcommand{\rbh}{\mathbf{\hat{r}}}
\newcommand{\sbh}{\mathbf{\hat{s}}}
\newcommand{\tbh}{\mathbf{\hat{t}}}
\newcommand{\ubh}{\mathbf{\hat{u}}}
\newcommand{\vbh}{\mathbf{\hat{v}}}
\newcommand{\wbh}{\mathbf{\hat{w}}}
\newcommand{\xbh}{\mathbf{\hat{x}}}
\newcommand{\ybh}{\mathbf{\hat{y}}}
\newcommand{\zbh}{\mathbf{\hat{z}}}
\newcommand{\Abh}{\mathbf{\hat{A}}}
\newcommand{\Bbh}{\mathbf{\hat{B}}}
\newcommand{\Cbh}{\mathbf{\hat{C}}}
\newcommand{\Dbh}{\mathbf{\hat{D}}}
\newcommand{\Ebh}{\mathbf{\hat{E}}}
\newcommand{\Fbh}{\mathbf{\hat{F}}}
\newcommand{\Gbh}{\mathbf{\hat{G}}}
\newcommand{\Hbh}{\mathbf{\hat{H}}}
\newcommand{\Ibh}{\mathbf{\hat{I}}}
\newcommand{\Jbh}{\mathbf{\hat{J}}}
\newcommand{\Kbh}{\mathbf{\hat{K}}}
\newcommand{\Lbh}{\mathbf{\hat{L}}}
\newcommand{\Mbh}{\mathbf{\hat{M}}}
\newcommand{\Nbh}{\mathbf{\hat{N}}}
\newcommand{\Obh}{\mathbf{\hat{O}}}
\newcommand{\Pbh}{\mathbf{\hat{P}}}
\newcommand{\Qbh}{\mathbf{\hat{Q}}}
\newcommand{\Rbh}{\mathbf{\hat{R}}}
\newcommand{\Sbh}{\mathbf{\hat{S}}}
\newcommand{\Tbh}{\mathbf{\hat{T}}}
\newcommand{\Ubh}{\mathbf{\hat{U}}}
\newcommand{\Vbh}{\mathbf{\hat{V}}}
\newcommand{\Wbh}{\mathbf{\hat{W}}}
\newcommand{\Xbh}{\mathbf{\hat{X}}}
\newcommand{\Ybh}{\mathbf{\hat{Y}}}
\newcommand{\Zbh}{\mathbf{\hat{Z}}}


% Mathbold + Tilde abbrevation
\newcommand{\abT}{\mathbf{\Tilde{a}}}
\newcommand{\bbT}{\mathbf{\Tilde{b}}}
\newcommand{\cbT}{\mathbf{\Tilde{c}}}
\newcommand{\dbT}{\mathbf{\Tilde{d}}}
\newcommand{\ebT}{\mathbf{\Tilde{e}}}
\newcommand{\fbT}{\mathbf{\Tilde{f}}}
\newcommand{\gbT}{\mathbf{\Tilde{g}}}
\newcommand{\hbT}{\mathbf{\Tilde{h}}}
\newcommand{\ibT}{\mathbf{\Tilde{i}}}
\newcommand{\jbT}{\mathbf{\Tilde{j}}}
\newcommand{\kbT}{\mathbf{\Tilde{k}}}
\newcommand{\lbT}{\mathbf{\Tilde{l}}}
\newcommand{\mbT}{\mathbf{\Tilde{m}}}
\newcommand{\nbT}{\mathbf{\Tilde{n}}}
\newcommand{\obT}{\mathbf{\Tilde{o}}}
\newcommand{\pbT}{\mathbf{\Tilde{p}}}
\newcommand{\qbT}{\mathbf{\Tilde{q}}}
\newcommand{\rbT}{\mathbf{\Tilde{r}}}
\newcommand{\sbT}{\mathbf{\Tilde{s}}}
\newcommand{\tbT}{\mathbf{\Tilde{t}}}
\newcommand{\ubT}{\mathbf{\Tilde{u}}}
\newcommand{\vbT}{\mathbf{\Tilde{v}}}
\newcommand{\wbT}{\mathbf{\Tilde{w}}}
\newcommand{\xbT}{\mathbf{\Tilde{x}}}
\newcommand{\ybT}{\mathbf{\Tilde{y}}}
\newcommand{\zbT}{\mathbf{\Tilde{z}}}
\newcommand{\AbT}{\mathbf{\Tilde{A}}}
\newcommand{\BbT}{\mathbf{\Tilde{B}}}
\newcommand{\CbT}{\mathbf{\Tilde{C}}}
\newcommand{\DbT}{\mathbf{\Tilde{D}}}
\newcommand{\EbT}{\mathbf{\Tilde{E}}}
\newcommand{\FbT}{\mathbf{\Tilde{F}}}
\newcommand{\GbT}{\mathbf{\Tilde{G}}}
\newcommand{\HbT}{\mathbf{\Tilde{H}}}
\newcommand{\IbT}{\mathbf{\Tilde{I}}}
\newcommand{\JbT}{\mathbf{\Tilde{J}}}
\newcommand{\KbT}{\mathbf{\Tilde{K}}}
\newcommand{\LbT}{\mathbf{\Tilde{L}}}
\newcommand{\MbT}{\mathbf{\Tilde{M}}}
\newcommand{\NbT}{\mathbf{\Tilde{N}}}
\newcommand{\ObT}{\mathbf{\Tilde{O}}}
\newcommand{\PbT}{\mathbf{\Tilde{P}}}
\newcommand{\QbT}{\mathbf{\Tilde{Q}}}
\newcommand{\RbT}{\mathbf{\Tilde{R}}}
\newcommand{\SbT}{\mathbf{\Tilde{S}}}
\newcommand{\TbT}{\mathbf{\Tilde{T}}}
\newcommand{\UbT}{\mathbf{\Tilde{U}}}
\newcommand{\VbT}{\mathbf{\Tilde{V}}}
\newcommand{\WbT}{\mathbf{\Tilde{W}}}
\newcommand{\XbT}{\mathbf{\Tilde{X}}}
\newcommand{\YbT}{\mathbf{\Tilde{Y}}}
\newcommand{\ZbT}{\mathbf{\Tilde{Z}}}


% Mathbold + tilde abbrevation
\newcommand{\abt}{\mathbf{\tilde{a}}}
\newcommand{\bbt}{\mathbf{\tilde{b}}}
\newcommand{\cbt}{\mathbf{\tilde{c}}}
\newcommand{\dbt}{\mathbf{\tilde{d}}}
\newcommand{\ebt}{\mathbf{\tilde{e}}}
\newcommand{\fbt}{\mathbf{\tilde{f}}}
\newcommand{\gbt}{\mathbf{\tilde{g}}}
\newcommand{\hbt}{\mathbf{\tilde{h}}}
\newcommand{\ibt}{\mathbf{\tilde{i}}}
\newcommand{\jbt}{\mathbf{\tilde{j}}}
\newcommand{\kbt}{\mathbf{\tilde{k}}}
\newcommand{\lbt}{\mathbf{\tilde{l}}}
\newcommand{\mbt}{\mathbf{\tilde{m}}}
\newcommand{\nbt}{\mathbf{\tilde{n}}}
\newcommand{\obt}{\mathbf{\tilde{o}}}
\newcommand{\pbt}{\mathbf{\tilde{p}}}
\newcommand{\qbt}{\mathbf{\tilde{q}}}
\newcommand{\rbt}{\mathbf{\tilde{r}}}
\newcommand{\sbt}{\mathbf{\tilde{s}}}
\newcommand{\tbt}{\mathbf{\tilde{t}}}
\newcommand{\ubt}{\mathbf{\tilde{u}}}
\newcommand{\vbt}{\mathbf{\tilde{v}}}
\newcommand{\wbt}{\mathbf{\tilde{w}}}
\newcommand{\xbt}{\mathbf{\tilde{x}}}
\newcommand{\ybt}{\mathbf{\tilde{y}}}
\newcommand{\zbt}{\mathbf{\tilde{z}}}
\newcommand{\Abt}{\mathbf{\tilde{A}}}
\newcommand{\Bbt}{\mathbf{\tilde{B}}}
\newcommand{\Cbt}{\mathbf{\tilde{C}}}
\newcommand{\Dbt}{\mathbf{\tilde{D}}}
\newcommand{\Ebt}{\mathbf{\tilde{E}}}
\newcommand{\Fbt}{\mathbf{\tilde{F}}}
\newcommand{\Gbt}{\mathbf{\tilde{G}}}
\newcommand{\Hbt}{\mathbf{\tilde{H}}}
\newcommand{\Ibt}{\mathbf{\tilde{I}}}
\newcommand{\Jbt}{\mathbf{\tilde{J}}}
\newcommand{\Kbt}{\mathbf{\tilde{K}}}
\newcommand{\Lbt}{\mathbf{\tilde{L}}}
\newcommand{\Mbt}{\mathbf{\tilde{M}}}
\newcommand{\Nbt}{\mathbf{\tilde{N}}}
\newcommand{\Obt}{\mathbf{\tilde{O}}}
\newcommand{\Pbt}{\mathbf{\tilde{P}}}
\newcommand{\Qbt}{\mathbf{\tilde{Q}}}
\newcommand{\Rbt}{\mathbf{\tilde{R}}}
\newcommand{\Sbt}{\mathbf{\tilde{S}}}
\newcommand{\Tbt}{\mathbf{\tilde{T}}}
\newcommand{\Ubt}{\mathbf{\tilde{U}}}
\newcommand{\Vbt}{\mathbf{\tilde{V}}}
\newcommand{\Wbt}{\mathbf{\tilde{W}}}
\newcommand{\Xbt}{\mathbf{\tilde{X}}}
\newcommand{\Ybt}{\mathbf{\tilde{Y}}}
\newcommand{\Zbt}{\mathbf{\tilde{Z}}}


% Mathbold + bar abbrevation
\newcommand{\abb}{\mathbf{\bar{a}}}
\newcommand{\bbb}{\mathbf{\bar{b}}}
\newcommand{\cbb}{\mathbf{\bar{c}}}
\newcommand{\dbb}{\mathbf{\bar{d}}}
\newcommand{\ebb}{\mathbf{\bar{e}}}
\newcommand{\fbb}{\mathbf{\bar{f}}}
\newcommand{\gbb}{\mathbf{\bar{g}}}
\newcommand{\hbb}{\mathbf{\bar{h}}}
\newcommand{\ibb}{\mathbf{\bar{i}}}
\newcommand{\jbb}{\mathbf{\bar{j}}}
\newcommand{\kbb}{\mathbf{\bar{k}}}
\newcommand{\lbb}{\mathbf{\bar{l}}}
\newcommand{\mbb}{\mathbf{\bar{m}}}
\newcommand{\nbb}{\mathbf{\bar{n}}}
\newcommand{\obb}{\mathbf{\bar{o}}}
\newcommand{\pbb}{\mathbf{\bar{p}}}
\newcommand{\qbb}{\mathbf{\bar{q}}}
\newcommand{\rbb}{\mathbf{\bar{r}}}
\newcommand{\sbb}{\mathbf{\bar{s}}}
\newcommand{\tbb}{\mathbf{\bar{t}}}
\newcommand{\ubb}{\mathbf{\bar{u}}}
\newcommand{\vbb}{\mathbf{\bar{v}}}
\newcommand{\wbb}{\mathbf{\bar{w}}}
\newcommand{\xbb}{\mathbf{\bar{x}}}
\newcommand{\ybb}{\mathbf{\bar{y}}}
\newcommand{\zbb}{\mathbf{\bar{z}}}
\newcommand{\Abb}{\mathbf{\bar{A}}}
% \newcommand{\Bbb}{\mathbf{\bar{B}}}
\newcommand{\Cbb}{\mathbf{\bar{C}}}
\newcommand{\Dbb}{\mathbf{\bar{D}}}
\newcommand{\Ebb}{\mathbf{\bar{E}}}
\newcommand{\Fbb}{\mathbf{\bar{F}}}
\newcommand{\Gbb}{\mathbf{\bar{G}}}
\newcommand{\Hbb}{\mathbf{\bar{H}}}
\newcommand{\Ibb}{\mathbf{\bar{I}}}
\newcommand{\Jbb}{\mathbf{\bar{J}}}
\newcommand{\Kbb}{\mathbf{\bar{K}}}
\newcommand{\Lbb}{\mathbf{\bar{L}}}
\newcommand{\Mbb}{\mathbf{\bar{M}}}
\newcommand{\Nbb}{\mathbf{\bar{N}}}
\newcommand{\Obb}{\mathbf{\bar{O}}}
\newcommand{\Pbb}{\mathbf{\bar{P}}}
\newcommand{\Qbb}{\mathbf{\bar{Q}}}
\newcommand{\Rbb}{\mathbf{\bar{R}}}
\newcommand{\Sbb}{\mathbf{\bar{S}}}
\newcommand{\Tbb}{\mathbf{\bar{T}}}
\newcommand{\Ubb}{\mathbf{\bar{U}}}
\newcommand{\Vbb}{\mathbf{\bar{V}}}
\newcommand{\Wbb}{\mathbf{\bar{W}}}
\newcommand{\Xbb}{\mathbf{\bar{X}}}
\newcommand{\Ybb}{\mathbf{\bar{Y}}}
\newcommand{\Zbb}{\mathbf{\bar{Z}}}


\newcommand{\E}{\mathbb{E}}
\newcommand{\Prb}{\mathbb{P}}
\newcommand{\R}{\mathbb{R}}
\newcommand{\C}{\mathbb{C}}
\newcommand{\I}{\mathbb{I}}
\newcommand{\F}{\mathbb{F}}
\newcommand{\vecc}{\mathbf{c}}
\newcommand{\Z}{\mathbb{Z}}
\newcommand{\nul}{\text{null }}
\newcommand{\range}{\text{range }}
\newcommand{\Span}{\text{span}}
\newcommand{\tr}{\text{tr}}
\newcommand{\diag}{\text{diag}}
\newcommand{\limn}{\lim_{n \to \infty}}
\newcommand{\zo}{\overline{z}}
\newcommand{\wo}{\overline{w}}
\newcommand{\ao}{\overline{a}}
\newcommand{\vo}{\overline{v}}
\newcommand{\glo}{\overline{\lambda}}
\newcommand{\Var}{\text{Var}}
\newcommand{\1}{\mathbbm{1}}
\newcommand{\Pm}{\mathbbm{P}}
\usepackage{hyperref}


\title{\vspace{-2em}Chapter 3: Linear Maps}
\author{\emph{Linear Algebra Done Right}, by Sheldon Axler}
\date{}

\begin{document}
\maketitle 
\tableofcontents

\newpage 
\section*{3A: Vector Space of Linear Maps}
\addcontentsline{toc}{section}{3A: Vector Space of Linear Maps}

\begin{definition}[Linear Map]
    A \emph{linear map} from \(V\) to \(W\) is a function \(T \colon V \to W\) with the following 
    properties:
    \begin{itemize}
        \item \textbf{Additivity}: \(T(u + v) = Tu + Tv\) for all \(u, v \in V\).
        \item \textbf{Homogeneity}: \(T(\lambda v) = \lambda T(v)\) for all \(\lambda \in \F\) and \(v \in V\).
    \end{itemize}
\end{definition}

Notation: \(\Lcal(V, W), \Lcal(V)\)

\begin{itemize}
    \item The set of linear maps from \(V\) to \(W\) is denoted by \(\Lcal (V, W)\).
    \item The set of linear maps from \(V\) to \(V\) is denoted by \(\Lcal (V)\). 
\end{itemize}


\begin{lemma}[linear map basis lemma]
    Suppose \(v_1, \ldots, v_n\) is a basis of \(V\) and \(w_1, \ldots, w_n \in W\). Then 
    there exists a unique linear map \(T \colon V \to W\) such that 
    \[T v_k = w_k\]
    for each \(k = 1, \ldots, n\).
\end{lemma}

\begin{definition}[additional and scalar multiplication on \(\Lcal(V, W)\)]
    Suppose \(S, T \in \Lcal(V, W)\) and \(\lambda \in \F\). The sum \(S+ T\) and the 
    product \(\lambda T\) are the linear maps from \(V\) to \(W\) defined by 
    \[(S+T)(v) = Sv + Tv \text{  and  } (\lambda T)(v) = \lambda (Tv)\]
    for all \(v \in V\).
\end{definition}

\begin{remark}
    \(\Lcal(V, W)\) is a vector space.
\end{remark}

\begin{definition}[product of linear maps]
    If \(T \in \Lcal(U, V) \text{ and } S \in \Lcal(V, W) \), then the \emph{product} \(ST \in \Lcal(U, W)\)
    is defined by 
    \[(ST)(u) = S(Tu)\]
    for all \(u \in U\).
\end{definition}

\begin{remark}[algebraic properties of product of linear maps]
We have associativity, identity and distributive properties whenever such properties are defined. 
\end{remark}

\begin{thm}[linear maps take 0 to 0]
    Suppose \(T\) is a linear map from \(V\) to \(W\). Then \(T(0) = 0\). 
\end{thm}

\newpage 
\addcontentsline{toc}{subsection}{3A Problem Sets}
\begin{problem}{1}
    Suppose \(b, c \in \R\). Define \(T \colon \R^3 \to \R^2\) by 
    \[T(x,y,z) = (2x - 4y + 3z + b, 6x + cxyz)\]
    Show that \(T\) is linear if and only if \(b = c = 0\). 
\end{problem}

\begin{proof}
\(T\) is linear \(\Leftrightarrow\) \(T(x, y, z = 0)\)
\(\Leftrightarrow\) \(b = 0\)

In addition, let's only consider the second coordinate, then we have 

\[T(x_1 + x_2, y_1 + y_2, z_1 + z_2)_2 = 6(x_1 + x_2) + c(x_1 + x_2)(y_1 + y_2)(z_1 + z_2)\]
which only equals \(T(x_1, y_1, z_1) + T(x_2, y_2, z_2)\) if \(c = 0\). 
\end{proof}

\begin{problem}{3}
    Suppose that \(T \in \Lcal(\F^n, \F^m)\). Show that there exist scalars 
    \(A_{j,k} \in \F\) for \(j = 1, \ldots, m\) and \(k=1, \ldots,n\) such that 
    \[T(x_1, \ldots, x_n) = (A_{1,1}x_1 + \cdots+A_{1,n} x_n, \ldots, A_{m, 1} x_1 + \cdots + A_{m,n}x_n)\]
    for every \((x_1, \ldots, x_n) \in \F^n\). 
\end{problem}

\begin{proof}
Let \(\{u_1, \ldots, u_n\}\) denote the standard basis of \(\F^n\). We have that 
\[T u_i = (A_{1,i}, \ldots, A_{m, i})\]

Take arbitrary \((x_1, \ldots, x_n) \in \F^n\), we have that 
\[T x_i u_i = x_i (A_{1,i}, \ldots, A_{m, i})\]

thus we have that 

\[T(x_1, \cdots, x_n) = (A_{1,1} x_1 + \cdots + A_{1,n} x_n, \cdots, A_{m, 1}x_1 + \cdots + A_{m, n} x_n)\]

\end{proof}

\begin{problem}{4}
    Suppose \(T \in \Lcal(V, W)\) and \(v_1, \ldots, v_m\) is a list of vectors in \(V\) such that 
    \(T v_1, \ldots, T v_m\) is a linearly independent list in \(W\). Prove that \(v_1, \ldots, v_m\)
    is linearly independent. 
\end{problem}

\begin{proof}
This means that the only solution to \(\sum_{i=1}^{m} a_i T(v_i) = \sum_{i=1}^{m} T(a_i v_i) = 0\)
is all \(a_i = 0\). However, as we know \(T(0) = 0\) so \(a_i v_i = 0\) and thus we've proved the claim. 
\end{proof}


\begin{problem}{7}
    Show that every linear map from a one-dimensional vector space to itself is multiplication 
    by some scalar. More precisely, prove that if \(\dim V = 1\) and \(T \in \Lcal(V)\), then 
    there exists \(\lambda \in \F\) such that \(T v = \lambda v\) for all \(v \in V\). 
\end{problem}

\begin{proof}
Since \(\dim V = 1\), every \(v \in V\) can be expressed as \(\lambda v'\) for some other \(v' \in V\). 
As \(Tv \in V\), we have \(Tv = \beta v\). Then \(T v = T \lambda v' = \beta \lambda v' = \beta v\)
\end{proof}

\begin{problem}{8}
    Give an example of a function \(\phi \colon \R^2 \to \R\) such that 
    \[\phi(av) = a\phi(v)\]
    for all \(a \in \R\) and all \(v \in \R^2\) but \(\phi\) is not linear. 
\end{problem}

\begin{proof}
Consider \(f(x_1, x_2) = x_1^2 / x_2\) if \(x_2 \neq 0\) o.w. 0, then \(f(a(x_1, x_2)) = a x_1^2 = a f(x_1, x_2)\). 
However, \(f((x_1 + y_1 , x_2 + y_2)) = (x_1 + y_1)^2 / (x_2 + y_2) \neq f(x_1, x_2) + f(y_1, y_2)\). 
\end{proof}

\begin{problem}{11}
    Suppose \(V\) is finite-dimensional and \(T \in \Lcal(V)\). Prove that \(T\) is a scalar multiple 
    of the identity if and only if \(S T = TS\) for all \(S \in \Lcal(V)\). 
\end{problem}

\begin{proof}
\(\Rightarrow\) We can express \(T = \lambda I\). Then \(S (\lambda I) = \lambda S = \lambda I S = TS\).

\(\Leftarrow\) Let \(v_1, \ldots, v_m\) be a basis of \(V\). Pick \(S_i\) such that \(S_i (\sum_{i=1}^{m}a_i v_i)
= a_i v_i\), which is clearly a linear operator. Then we have that 

\begin{align*}
    S_i T(v) &= T S_i (v) \\ 
    S_i \sum_{j=1}^{m} b_j v_j & = T(a_i v_i) \\ 
    b_i v_i &= a_i T(v_i)
\end{align*}

This shows that for all \(v_i\), there exists \(\lambda_i\) such that \(T(v_i) = \lambda_i v_i\). Next 
we show that such \(\lambda_i\) does not depend on \(i\). Construct \(S_{ij}\) subtly such that 
\(S_{ij}\sum_{k=1}^{n} a_k v_k = a_j v_i + a_i v_j\). 

Then we have that 

\begin{align*}
    S_{ij} T v &= T S_{ij} v \\ 
    S_{ij} (\sum_{k=1}^{n}\lambda_k a_k v_k) &= T(a_j v_i + a_i v_j) \\ 
    \lambda_j a_j v_i + \lambda_i a_i v_j &= \lambda_i a_j v_i + \lambda_j a_i v_j 
\end{align*}

This shows that \(\lambda_i = \lambda_j\) for all \(i,j\) and thus we've shown that \(T = \lambda I\)
for some \(\lambda\). 

\begin{problem}{12}
    Suppose \(U\) is a subspace of \(V\) with \(U \neq V\). Suppose \(S \in \Lcal(U, W)\)
and \(S \neq 0\). Define \(T \colon V \to W\) by 

\[Tv = \begin{cases}
    Sv & \text{if } v \in U \\ 
    0 & \text{if } v \in V \text{ and } v \notin U 
\end{cases}\]

Prove that \(T\) is not a linear map on \(V\). 
\end{problem}


\begin{proof}
Take \(u \in U\) such that \(u \notin \text{Null}(S)\). Take \(v \in V \backslash U\), then 
\(u + v \in V \backslash U\). This means that 
\[T(u + v) = S(u + v) = 0 \neq T(u) + T(v) = Su\]
\end{proof}

\begin{problem}{13}
    Suppose \(V\) is finite-dimensional. Prove that every linear map on a subspace of \(V\) can 
    be extended to a linear map on \(V\). In other words, show that if \(U\) is a subspace of 
    \(V\) and \(S \in \Lcal(U, W)\), then there exists \(T \in \Lcal(V, W)\) such that 
    \(Tu = Su\) for all \(u \in U\).
\end{problem}

\begin{proof}

Note that there exists subspace \(P\) s.t. \(V = P \oplus U\). For all \(v \in V\),
\(v = p + u\) for some \(p, u\). Then define \(T(v) = Su + p\). Clearly we have that 
\(T(u) = Su \) for all \(u \in U\). It now only suffices to prove \(T\) is a linear map.
Homogeneity is trivial to show. For addivity, \(T(v_1 + v_2) = T(u_1 + u_2 + p_1 + p_2)
= S(u_1 + u_2) + p_1 + p_2 = (S u_1 + p_1) + (S u_2 + p_2) = Tv_1 + T v_2\).

\end{proof}


\begin{problem}{14}
    Suppose \(V\)  is finite-dimensional with \(\dim V > 0\), and suppose \(W\)
    is infinite-dimensional. Prove that \(\Lcal (V, W)\) is infinite-dimensional. 
\end{problem}

\begin{proof}
Recall the definition of infinite-dimension from ch2 P17:

\begin{center}
    \(V\) is infinite-dimensional if and only if there is a sequence 
    \(v_1, v_2, \ldots\) of vectors in \(V\) such that \(v_1, \ldots, v_m\)
    is linearly independent for every positive integer \(m\). 
\end{center}

\(W\) being infinite-dimensional implies this. Denote the sequence by \(w_1, w_2, \ldots\)
 Let \(v_1, \ldots, v_m\) be the basis for \(V\). 
Define a sequence of linear operators as follows: \(T_k \in \Lcal(V, W)\) such that \(T_k (v) = v_k\). 
Then we have that for every positive integer \(m\), the solution for the following equation 
is all \(a_i = 0\). 
\[a_1T_1(v) + \cdots + a_m T_m(v) = a_1 v_1 + \cdots + a_m v_m = 0\]

This shows that \(\Lcal(V, W)\) is infinite-diemsnional.


\end{proof}

\begin{problem}{15}
    Suppose \(v_1, \ldots, v_m\) is a linearly dependent list of vectors in \(V\). Suppose also that 
    \(W \neq \{0\}\). Prove that there exist \(w_1, \ldots, w_m \in W\) such that no 
    \(T \in \Lcal(V, W)\) satifies \(Tv_k = w_k\) for each \(k = 1, \ldots, m\). 
\end{problem}

\begin{proof}
Take \(w_1, \ldots, w_m\) to be linearly independent list of vectors in \(W\). Then we have that 

\(Ta_1 v_1 + \cdots T a_m v_m = a_1 w_1 + \cdots a_m w_m = 0\)

The only solution is that \(a_i = 0\) for all \(i\), but this contradicts that \(v_1, \ldots, v_m\)
is linearly dependent. 

\end{proof}



\begin{problem}{16}
    Suppose \(V\) is finite-dimensional with \(\dim V > 1\). Prove that there exist \(S, T \in \Lcal(V)\)
    such that \(ST \neq TS\). 
\end{problem}

\begin{proof}
Let \(v_1, \ldots, v_m\) be the basis of \(V\). Define \(S (v) = S(\sum_{i=1}^{m}a_i v_i)
= \sum_{i=1}^{m} a_{m - i} v_i\) and \(T(v) = a_1 v_1\). Then we have that 
\[ST v = S a_1v_1 = a_1 v_1\]

but 
\[T S v = T \sum_{i=1}^{m} a_{m - i} v_i = a_m v_1\]
\end{proof}

\begin{problem}{17}
    Suppose \(V\) is finite-dimensional. Show that the only two-sided ideas of \(\Lcal(V)\)
    are \(\{0\}\) and \(\Lcal(V)\), where we define that a subspace \(\Ecal\) of \(\Lcal(V)\)
    is called two-sided ideal if \(T E \in \Ecal\) and \(ET \in \Ecal\) for all \(E \in \Ecal\)
    and all \(T \in \Lcal(V)\). 
\end{problem}

\begin{proof}
It's easy to verify that \(\{0\}\) and \(\Lcal(V)\) are two-sided ideal of \(\Lcal(V)\). Suppose for 
the sake of contradiction that such \(\Ecal\) exists. Let \(e_1, \ldots, e_m\) be its basis and let 
\(e_1, \ldots e_m, e_{m+1}, \ldots, e_n\) be the basis for \(V\). Define \(T(v) = 
T(\sum_{i=1}^{n}a_i e_i) = \sum_{i=1}^{n}a_i e_{n-i}\) to be a linear map in \(V\) (ez to verify), and 
\(E_j (u) = E_j(\sum_{k=1}^{m}b_k e_k) = b_j e_j\). We have reached the contradicting example such 
that \(TE_j (v) = a_j e_{n - j} \in V \backslash \Ecal\). 

\end{proof}

\end{proof}

\newpage 


\section*{3B: Null Spaces and Ranges}
\addcontentsline{toc}{section}{3B: Null Spaces and Ranges}

\begin{definition}[null space]
    For \(T \in \Lcal(V, W)\), the \emph{null space} of \(T\), denoted by \(\text{null } T\), is the subset of 
    \(V\) consisting of those vectors that \(T\) maps to 0:
    \[\text{null } T = \{v \in V \colon Tv = 0\}\]
\end{definition}

\begin{corollary}
    Suppose \(T \in \Lcal(V, W)\), then \(\text{null } T\) is a subspace of \(V\). 
\end{corollary}

\begin{definition}[injective]
    A function \(T \colon V \to W\) is called \textbf{injective} if \(Tu = Tv\) implies \(u = v\). 
\end{definition}

\begin{thm}
    Let \(T \in \Lcal(V, W)\). Then \(T\) is injecive if and only if \(\text{null } T = \{0\}\). 
\end{thm}

\begin{definition}[range]
    For \(T \in \Lcal(V, W)\), the \textbf{range} of \(T\) is the subset of \(W\) consisting of those 
    vectors that are equal to \(T v\) for some \(v \in V\):
    \[\text{range} T = \{Tv \colon v \in V\}\]
\end{definition}

\begin{corollary}
    Suppose \(T \in \Lcal(V, W)\), then \(\text{range} T\) is a subspace of \(W\). 
\end{corollary}

\begin{definition}[surjective]
    A function \(T \colon V \to W\) is called \textbf{surjective} if its range equals \(W\). 
\end{definition}


\begin{thm}[fundamental theorem of linear map]
    Suppose \(V\) is finite-dimensional and \(T \in \Lcal(V, W)\). Then range \(T\) is finite-dimensional 
    and 
    \[\dim V = \dim \text{null } T + \dim \text{range } T\]
\end{thm}

\begin{proof}
Let \(u_1, \ldots, u_m\) be a basis of null \(T\) and let \(u_1, \ldots, u_m, v_1, \ldots, v_n\)
be a basis of \(V\). It now suffices to prove \(\dim\) range \(T = n\). Let \(v \in T\), then 
\[v = \sum_{i=1}^{m}a_i u_i + \sum_{j=1}^{n}b_j v_j\]
We apply \(T\) on both sides to get that 
\[Tv = \sum_{j=1}^{n}b_j T v_j \]

which shows that \(Tv_1, \ldots, T v_n\) spans range \(T\) and thus it's finite-dimensional. To show 
they are linearly independent, we have 
\[\sum_{j=1}^{n} b_j T v_j =  T \sum_{j=1}^{n} b_j v_j = 0\]

The only solution is that all \(b_j = 0\). 

\end{proof}

\begin{corollary}[linear map to a lower-dimen space is not injective]
    Suppose \(V\) and \(W\) are finite-dimensional vector spaces such that 
    \(\dim V > \dim W\). Then no linear map from \(V\) to \(W\) is injective. 
\end{corollary}

\begin{corollary}[linear map to a higher-dimen space is not surjective]
    Suppose \(V\) and \(W\) are finite-dimensional vector spaces such that 
    \(\dim V < \dim W\). Then no linear map from \(V\) to \(W\) is surjective. 
\end{corollary}

\noindent
\paragraph{Application} Consider the system of linear equation defined by the map \(T \colon \F^n \to \F^m\):
\[T(x_1, \ldots, x_n) = (\sum_{k=1}^{n}A_{1, k}x_k, \ldots, \sum_{k=1}^{n}A_{m, k}x_k)\]

\begin{corollary}[homogeneous systems of linear equations]
    A homogeneous system of linear equations with more variables than equations has nonzero solutions.
\end{corollary}

\begin{corollary}[inhomogeneous system fo linear equations]
    An inhomogeneous system of linear equations with more equations than variables has no 
    solution for some choice of the constant terms. 
\end{corollary}

\addcontentsline{toc}{subsection}{3B Problem Sets}
\begin{problem}{1}
    Give an example of a linear map \(T\) with dim null \(T\) = 3 and dim range \(T = 2\). 
\end{problem}

\begin{proof}
Consider \(T(x_1, x_2, x_3, x_4, x_5) = (0,0,0,x_4,x_5)\). 

It's easy to verify that \(\dim\) null \(T = 3\) and applying the theorem of linear map solves the problem.
\end{proof}


\begin{problem}{2}
    Suppose \(S, T \in \Lcal(V)\) are such that range \(S \subseteq \) null \(T\). Prove that 
    \((ST)^2 = 0\).
\end{problem}

\begin{proof}
This means that take \(x \in V\), \(T (S(x)) = 0 \). We have \((ST)^2 = STST = 0\). 
\end{proof}

\begin{problem}{3}
    Suppose \(v_1, \ldots, v_m\) is a list of vectors in \(V\). Define 
    \(T \in \Lcal (\F^m, V)\) by 
    \[T(z_1, \ldots, z_m) = z_1 v_1 + \cdots + z_m v_m\]
    (a) what property of \(T\) corresponds to \(v_1, \ldots, v_m\) spanning 
    \(V\)?

    (b) what property of \(T\) corresponds to \(v_1, \ldots, v_m\) being linearly 
    independent?
\end{problem}


\begin{proof}
(a) surjective (range = \(V\)) 

(b) injective (\(z_1, \ldots, z_m \) is identically zero if and only if 
\(z_1 v_1 + \cdots + z_m v_m = 0\))
\end{proof}


\begin{problem}{4}
    Show that \(\{T \in \Lcal(\R^5, \R^4) \colon \dim \text{null } T > 2\}\)
    is not a subspace of \(\Lcal(\R^5, \R^4)\).
\end{problem}


\begin{proof}
% dim range (\(T\)) = 5 - dim null (\(T\)) \(< 3\).
Try to come up with a counterexample. Consider \(f(e_1, e_2, e_3, e_4, e_5)
= (e_1, 0,0, e_4)\) and \(g(e_1, e_2, e_3, e_4, e_5) = (0, e_2, e_3, 0)\). Then 
we have that null \(f = \{0,e_2,e_3,0,e_5\}\) and null \(g = \{e_1, 0, 0, e_4, e_5\}\)
so both in \(\{T \in \Lcal(\R^5, \R^4) \colon \text{dim null } T > 2\}\). However,
\(f + g (e_1, e_2, e_3, e_4, e_5) = \{(e_1,e_2,e_3,e_4)\}\) and this means 
their dim null = 5 - 4 = 1 < 3. 

\end{proof}


\begin{problem}{7}
    Suppose \(V\) and \(W\) are finite-dimensional and \(2 \leq \dim V 
    \leq \dim W\). Show that \(\{T \in \Lcal(V, W) \colon T \text{ is 
    not injective}\}\) is not a subspace of \(\Lcal(V, W)\). 
\end{problem}


\begin{proof}
\(T\) iss not injective means that \(\dim V > \dim W \), which contradicts the 
assumption in the question. 
\end{proof}


\begin{problem}{9}
    Suppose \(T \in \Lcal(V, W)\) is injective and \(v_1, \ldots, v_n\) is 
    linearly independent in \(V\). Prove that \(Tv_1, \ldots, Tv_n\) is 
    linearly independent in \(W\).
\end{problem}

\begin{proof}
\[\sum_{i=1}^{n}z_iv_i = 0 \Leftrightarrow \sum_{i=1}^{n} z_iT(v_i) = 0\]
\end{proof}

\begin{problem}{10}
    Suppose \(v_1, \ldots, v_n\) spans \(V\) and \(T \in \Lcal(V, W)\). Show 
    that \(Tv_1, \ldots, T v_n\) spans range \(T\). 
\end{problem}

\begin{proof}
Take \(v\) in \(V\), then we know \(v = \sum_{i=1}^{n} a_i v_i\). Take \(w in\) range \(T\).
Then we know there exists \(v \in V\) such that \(w = T(v)\) and 
thus \(w = T(\sum_{i=1}^{n} a_i v_i) = \sum_{i=1}^{n}a_i T(v_i)\). 
\end{proof}


\begin{problem}{11}
    Suppose that \(V\) is finite-dimensional and that \(T \in \Lcal(V, W)\). Prove 
    that there exists a subspace \(\Ucal\) of \(V\) such that 
    \[\Ucal \cap \text{null } T = \{0\} \text{   and   } \text{range } T 
    = \{T u \colon u \in \Ucal\}\]
\end{problem}


\begin{proof}
Since we know null \(T\) is a subspace of \(V\), then there exists \(\Ucal\) 
such that \(\Ucal \oplus \text{null }T = V\). We can define \(\Ucal\) to be such 
case. To finish the proof, let \(v = u + t, u \in \Ucal, t \in null T\), then \(range T 
= \{T(u + t) = T u \colon u \in \Ucal\}\). 
\end{proof}


\begin{problem}{16}
    Suppose \(V\) and \(W\) are both finite-dimensional. Prove that there exists an injective 
    linear map from \(V\) to \(W\) if and only if \(\dim V \leq \dim W\).
\end{problem}

\begin{proof}
\(\Rightarrow\) \(T \in \Lcal(V, W)\) is injective. Then this means that \(\dim V = \dim \text{range } T
\leq \dim W\). 

\(\Leftarrow\) By assumption, we can define a linear map \(T\) such that 
\(T v_i = w_i\) where \(v_i, w_i\) are the respective basis of \(V\) and \(W\). 
Then we have that 
\[\sum_{i=1}^{n} a_i T v_i = \sum_{i=1}^{n} a_i w_i = 0 \Leftrightarrow a_i \text{ are identically zero} \]

This means that null \(T\) is \(\{0\}\) and thus it is injective. 
\end{proof}


\begin{problem}{17}
    Suppose \(V\) and \(W\) are both finite-dimensional. Prove that there exists 
    a surjective linear map from \(V\) onto \(W\) if and only if \(\dim V \geq \dim W\).
\end{problem}

\begin{proof}
\(\Rightarrow\) \(\dim V \geq \text{range } T = \dim W\). 

\(\Leftarrow\) By assumption we can define a linear map \(T\) such that 
\(T v_i = w_i\) for \( 1 \leq i \leq \dim W\) and \(T v_i = 0\) for \(\dim W \leq i \dim V\). 
This mean that take \(w \in W\), then \(w = \sum_{i=1}^{\dim W}a_i w_i\). At the 
same time, we know that for all \(w_i\), there exists \(v_i\) s.t. \(T v_i = w_i\), 
therefore we have \(w = \sum_{i=1}^{\dim W} a_i Tv_i\) and thus \(T\) is surjective.   
\end{proof}


\begin{problem}{18}
    Suppose \(V\) and \(W\) are finite-dimensional and that \(\Ucal\) is a 
    subspace of \(V\). Prove that there exists \(T \in \Lcal(V, W)\) such 
    that null \(T = \Uc\) if and only if \(\dim \Uc \geq \dim V - \dim W\). 
\end{problem}

\begin{proof}
We know \(\dim V = \text{null } T + \text{range } T\) and range \(T \leq \dim W\).
Therefore, \(\dim \Uc = \dim \text{null } T \geq \dim V - \dim W\).  

\end{proof}


\begin{problem}{19}
    Suppose \(W\) is finite-dimensional and \(T \in \Lc(V, W)\). Prove that 
    \(T\) is injective if and only if there exists \(S \in \Lc(W, V)\) such that 
    \(ST\) is the identity operator on \(V\).
\end{problem}

\begin{proof}
\(T\) is injective \(\Leftrightarrow\) \(T(v)\) is unique for \(v \in V\). 
\(\Leftrightarrow\) We can define \(S \colon S Tv = v\). 
\end{proof}

\begin{problem}{21}
    Suppose \(V\) is finite-dimensional, \(T \in \Lc(V, W)\), and \(\Uc\) is 
    a subspace of \(W\). Prove that \(\Tc =  \{ v \in V \colon Tv \in \Uc \}\) is a subspace 
    of \(V\) and 
    \[\dim \Tc = \dim \text{null } T + \dim (\Uc \cap \text{range } T)\]
\end{problem}

\begin{proof}
To prove subspace, we can simply follow by definition. We can define 
\(S \in \Lc(\Tc, \Ucal)\) such that \(S v = T v\) for all \(v \in \Tc\). Then 
we have that range(S) \(\in \Uc  \cap\) range(\(T\)) and that null \(T\) = 
null \(S\). Hence we have proved the claim. 
\end{proof}


\begin{problem}{22}
    Suppose \(\Uc\) and \(V\) are finite-dimensional vector spaces and \(S 
    \in \Lc(V, W)\) and \(T \in \Lc(\Uc, V)\). Prove that 
    \[\dim \nul ST \leq \dim \nul S + \dim \nul T\]. 
\end{problem}

\begin{proof}
We can write that 
\[\nul ST = \{u \in \Uc \colon T(u) \in \nul (S)\}\]

By the previous question we know that 
\[\dim \nul ST = \dim \nul T + \dim \nul S \cap \range T \leq \dim \nul T + \dim \nul S\]
\end{proof}

\begin{problem}{23}
    Suppose \(\Uc\) and \(V\) are finite-dimensional vector spaces and \(S \in \Lc(V, W)\)
    and \(T \in \Lc(\Uc, V)\). Prove that 
    \[\dim \range ST \leq \min \{\dim \range S, \dim \range T\}\].
\end{problem}

\begin{proof}
First note that \(\dim \range ST \leq \dim \range S\). This because \(\range ST \subseteq \range S\). 

To prove \(\dim \range ST \leq \dim \range T\). We have that \(\dim \Uc = \dim \nul T 
+ \dim \range T = \dim \nul ST + \dim \range ST\). Since we know \(\dim \nul T \geq \dim \nul 
ST\), \(\dim \range ST \leq \dim \range T\). 
\end{proof}


\begin{problem}{24}
    (a) Suppose \(\dim V = 5\) and \(S, T \in \Lc(V)\) are such 
    that \(ST = 0\). Prove that \(\dim \range TS \leq 2\). 

    (b) Give an example of \(S, T \in \Lc(\F^5)\) with \(ST=0\) and \(\dim \range TS = 2\).
    
\end{problem}


\begin{proof}
(a) \(\dim \nul ST = 5 \leq \dim \nul T + \dim \nul S\)

We also know that  \(5 = \dim \nul T + \dim \range T = \dim \range S + \dim \nul S\) 
and thus we have that \(\dim \range T + \dim \range S \leq 5\). This implies that \(\min 
\{\dim \range T, \dim \range S\} \leq 2\). Hence by applying P23 we finish the proof. 

(b) Consider \(T(x_1, x_2, x_3, x_4, x_5) = (x_3, x_4, 0, 0, 0), S(x_1, x_2, x_3, 
x_4, x_5) = (0,0,x_3,x_4, x_5)\). Then we have that \(ST(x_1, x_2, x_3,x_4,x_5)
= S(x_3, x_4, 0,0,0) = (0,0,0,0,0)\) while \(TS(x_1, x_2, x_3, x_4, x_5)
= T(0,0,x_3,x_4, x_5) = (x_3, x_4, 0,0,0)\).

\end{proof}

\begin{problem}{25}
    Suppose that \(W\) is finite-dimensional and \(S, T \in \Lc(V, W)\). Prove that 
    \(\nul S \subseteq \nul T\) if and only if there exists \(E \in \Lc(W)\)
    such that \(T = ES\).
\end{problem}

\begin{proof}
\(\nul S \subseteq \nul T \Rightarrow \dim \range T \leq \dim \range S \Rightarrow\)
let \(s_1, \ldots, s_n\) be basis of \(\range S\) and \(t_1, \ldots, t_m\) be basis of \(\range T\)
where \(m \leq n\). We can always define \(E \in \Lc (W)\) such that \(E(s_i) = t_i\) for all 
\(1 \leq i \leq m\) and 0 otherwise. 

The other direction is trivial. 
\end{proof}

\begin{problem}{26}
    Suppose that \(V\) is finite-dimensional and \(S, T \in \Lc(V, W)\). Prove that 
    \(\range S \subseteq \range T\) if and only if there exists \(E \in \Lc(V)\) 
    such that \(S = TE\). 
\end{problem}

\begin{proof}

    \(\range S \subseteq \range T \Rightarrow\) Take \(v_1, \ldots, v_m\) to be the basis 
    of \(V\), define the linear map \(S v_i = s_i\) for \(s_i \in \range S \subseteq \range T\).
    Then there exists \(u_1, \ldots, u_m\) such that \(T u_i = s_i\) for all \(i\). Then we 
    can define the linear map \(E v_i = u_i\) for all \(i\) such that \(S = TE\).  

    The other direction is trivial. 
\end{proof}


\begin{problem}{27}
Suppose \(P \in \Lc(V)\) and \(P^2 = P\). Prove that \(V = \nul P \oplus \range P\).
\end{problem}

\begin{proof}
Take \(v \in \nul (P) \cap \range (P)\). Then this means that \(P(v) = 0\) and there exists 
\(u\) s.t. \(v = P(u)\). At the same time, \(P(v) = P(P(u)) = P(u) = 0 =v\). So we have 
that \(\nul P \cap \range P = \{0\}\). Since \(P\) is defined on \(\Lc(P)\), we have that 
\(V = \nul P \oplus \range P\). 
\end{proof}

% \begin{problem}{30}
%     Suppose \(\varphi \in \Lc(V, \F)\) and \( \varphi \neq 0\). Suppose \(u \in V\) is not in \(\nul \varphi\). 
%     Prove that 
%     \[V = \nul \varphi \oplus \{a u \colon a \in \F\}\]
% \end{problem}


% \begin{proof}

% \end{proof}


% \begin{problem}{31}
%     Suppose \(V\) is finite-dimensional, \(X\) is a 
% \end{problem}


\newpage

\section*{3C: Matrices}
\addcontentsline{toc}{section}{3C: Matrices}

\begin{definition}[matrix, \(\Ab_{j,k}\)]
    Suppose \(m\) and \(n\) are nonnegative integers. An  m-by-n matrix \emph{matrix}
    \(\Ab\) is a rectangular array of elements of \(\F\) with \(m\) rows and \(n\) columns:
    \[\Ab = \begin{pmatrix}
        \Ab_{1, 1} & \cdots & \Ab_{1, n} \\ 
        \vdots & & \vdots \\ 
        \Ab_{m, 1} & \cdots & \Ab_{m, n}
    \end{pmatrix}\]
where the notation \(\Ab_{j, k}\) denotes the entry in row \(j\) and column \(k\). 
\end{definition}


\begin{definition}[matrix of a linear map, \(\Mc(T)\)]
    Suppose \(T \in \Lc(V, W)\) and \(v_1, \ldots, v_n\) is a basis of \(V\) and \(w_1, \ldots, 
    w_m\) is a basis of \(W\). The \emph{matrix of} \(T\) with respect to these bases is the 
    m-by-n matrix \(\Mc(T)\) whose entries \(\Ab_{j,k}\) are defined by 
    \[T v_k = \Ab_{1, k} w_1 + \cdots + \Ab_{m, k} w_m\]. 

    If the bases \(v_1, \ldots, v_n\) and \(w_1, \ldots, w_m\) are not clear from the context, then 
    the notation \(\Mc(T, (v_1, \ldots, v_n), (w_1, \ldots, w_m))\) is used. 
\end{definition}

\begin{remark}
    The \(k\)-th column of \(\Mc(T)\) consists of the scalars needed to write \(T v_k\) 
    as a linear combination of \(w_1, \ldots, w_m\): 
    \[T v_k = \sum_{j=1}^{m} \Ab_{j, k} w_j\]
\end{remark}

\begin{remark}
    If \(T\) is a linear map from \(n\)-dimensional vector space to an \(m\)-dimensional vector 
    space, then \(\Mc(T)\) is an m-by-n matrix. 
\end{remark}

\begin{corollary}[Matrix addition and scalar multiplication]
    Suppose \(S, T \in \Lc(V, W)\) and \(\lambda \in \F\). Then \(\Mc(\lambda S + T) =\lambda \Mc(S) + \Mc(T)\). 
\end{corollary}

For \(m\) and \(n\) positive integers, the set of all m-by-n matrices with entries in \(F\)
is denoted by \(\F^{m,n}\). 

\begin{thm}
    Suppose \(m\) and \(n\) are positive integers. \(\F^{m, n}\) is a vector space of dimension 
    \(mn\). 
\end{thm}

\begin{definition}[matrix multiplication]
    Suppose \(\Ab\) is an m-by-n matrix and \(\Bb\) is an n-by-p matrix. Then \(AB\) is defined 
    to be the m-by-p matrix whose entry in row \(j\), column \(k\), is given by the equation
    \[(\Ab \Bb)_{j,k} = \sum_{r=1}^{n} \Ab_{j, r} \Bb_{r, k}\]
    In words, the entry in row \(j\), column \(k\) of \(\Ab \Bb\) is computed by taking row \(j\)
    of \(\Ab\) and column \(k\) of \(\Bb\).  
\end{definition}

\textbf{Motivations}. Let \(v_1, \ldots, v_n\) to be the basis of \(V\), \(w_1, \ldots, w_m\)
to be the basis of \(W\), and \(u_1, \ldots, u_p\) to be the basis of \(U\). Consider linear 
maps \(T \colon U \to V\) and \(S \colon V \to W\). Suppose \(\Mc(T) = \Ab\) and \(\Mc(S) = \Bb\). 
For \(1 \leq k \leq p\), we have 

\begin{align*}
    (ST) u_k 
    &= S(\sum_{r=1}^{n}\Bb_{r, k} v_r) \\ 
    &=\sum_{r=1}^{n} \Bb_{r, k} (S v_r) \\ 
    &= \sum_{r=1}^{n} \Bb_{r,k} \sum_{j=1}^{m} \Ab_{j, r} w_j \\ 
    &= \sum_{j=1}^{m} (\sum_{r=1}^{n} \Ab_{j, r} \Bb_{r, k}) w_j
\end{align*}

\begin{thm}[matrix of product of linear maps]
    If \(T \in \Lc(U, V)\) and \(S \in \Lc(V, W)\), then \(\Mc(ST) = \Mc(S)\Mc(T)\).
\end{thm}


\begin{definition}[\(\Ab_{j, \cdot}, \Ab_{\cdot, k}\)]
    Suppose \(\Ab\) is an m-by-n matrix 
    \begin{itemize}
        \item If \(1 \leq j \leq m\), then \(\Ab_{j, \cdot}\) denotes the 1-by-n matrix 
        consisting of row \(j\) of \(\Ab\). 
        \item If \(1 \leq k \leq n\), then \(\Ab_{\cdot, k}\) denotes the m-by-1 matrix consisting 
        of column \(k\) of \(\Ab\). 
    \end{itemize}
\end{definition}

\begin{corollary}[entry of matrix product equals row times column]
    Suppose \(\Ab\) is an m-by-n matrix and \(\Bb\) is an n-by-p matrix. Then 
    \[(\Ab \Bb)_{j, k} = \Ab_{j, \cdot} \Bb_{\cdot, k}\]
    if \(1 \leq j \leq m\) and \(1 \leq k \leq p\). In other words, the entry in row \(j\), 
    column \(k\), of \(\Ab \Bb\) equals (row \(j\) of \(\Ab\)) times (column \(k\) of \(\Bb\)). 
\end{corollary}

\begin{corollary}[column of matrix product equals matrix times column]
Suppose \(\Ab\) is an m-by-n matrix and \(\Bb\) is an n-by-p matrix. Then 
\[(\Ab \Bb)_{\cdot, k} = \Ab \Bb_{\cdot, k}\]
if \(1 \leq k \leq p\). In other words, column \(k\) of \(\Ab \Bb\) equals \(\Ab\) times 
column \(k\) of \(\Bb\). 
\end{corollary}

\begin{corollary}[linear combination of columns]
    Suppose \(\Ab\) is an m-by-n matrix and \(\bb = \begin{pmatrix}
        b_1 \\ 
        \vdots \\ 
        b_n
    \end{pmatrix}\) is an n-by-1 matrix. Then 
    \[\Ab \bb = \bb_1 \Ab_{\cdot, 1} + \cdots + \bb_n \Ab_{\cdot, n}.\]
    In other words, \(\Ab \bb\) is a linear combination of the columns of \(\Ab\), with the scalars 
    that multiply the columns coming from \(\bb\). 
\end{corollary}

\begin{thm}[matrix multiplication as linear combination of columns]
    Suppose \(C\) is an m-by-c matrix and \(R\) is a c-by-n matrix. 

    \begin{itemize}
        \item If \(k \in \{1, \ldots, n\}\), then column \(k\) of \(CR\) is a linear combination 
        of the columns of \(C\), with the coefficients of this linear combination coming from 
        columns \(k\) of \(R\). 
        \item If \(j \in \{1, \ldots, m\}\), then row \(j\) of \(CR\) is a linear combination of 
        the rows of \(R\), with the coefficients of this linear combination coming from row 
        \(j\) of \(C\). 
    \end{itemize}
\end{thm}


\begin{definition}[column rank, row rank]
    Suppose \(\Ab\) is an m-by-n matrix with entries in \(\F\). 
    \begin{itemize}
        \item The column rank of \(\Ab\) is the dimension of the span of the columns of \(\Ab\)
        in \(\F^{m, 1}\). 
        \item The row rank of \(\Ab\) is the dimension of the span of the rows of \(\Ab\) in \(\F^{1, n}\). 
    \end{itemize}
\end{definition}

\begin{definition}[transpose, \(\Ab^\top\)]
    The transpose of a matrix \(\Ab\), denoted by \(\Ab^\top\), is the matrix obtained from \(\Ab\)
    by interchanging rows and columns. Specificially, if \(\Ab\) is an m-by-n matrix, then 
    \(\Ab^\top\) is the n-by-m matrix whose entries are given by the equation 
    \[(\Ab^\top)_{k, j} = \Ab_{j, k}\]
\end{definition}

\begin{lemma}[column-row factorization]
    Suppose \(\Ab\) is an m-by-n matrix with entries in \(\F\) and column rank \(c \geq 1\). Then 
    there exists an m-by-c matrix \(\Cb\) and a c-by-n matrix \(\Rb\), both with entries in \(\F\), 
    such that \(\Ab = \Cb \Rb\). 
\end{lemma}

\begin{proof}
Each column of \(\Ab\) is an m-by-1 matrix. The list \(\Ab_{\cdot, 1}, \ldots, \Ab_{\cdot, n}\)
of columns of \(\Ab\) can be reduced to a basis of the span of the columns of \(\Ab\). This 
basis has length \(c\). The \(c\) columns in this basis can be put together to form an m-by-c 
matrix \(\Cb\). 

If \(k \in \{1, \ldots, n\}\), then column \(k\) of \(\Ab\) is a linear combination 
of the columns of \(\Cb\). Make the coefficients of this linear combination into column 
\(k\) of a c-by-n matrix that we call \(\Rb\). Then \(\Ab = \Cb \Rb\). 
\end{proof}

\begin{thm}[column rank equals row rank]
    Suppose \(\Ab \in \F^{m, n}\). Then the column rank of \(\Ab\) equals the row rank of \(\Ab\). 
\end{thm}

\begin{proof}
Let \(c\) denote the column rank of \(\Ab\). Let \(\Ab = \Cb \Rb\) be the column-row factorization 
of \(\Ab\) given by the proof before, where \(\Cb\) is m-by-c and \(\Rb\) is c-by-n. Then the 
column-row factorization lemma tells us that every row of \(\Ab\) is a linear combination of 
the rows of \(\Rb\). Because \(\Rb\) has \(c\) rows, this implies that the row of \(\Ab\) is 
less than or equal to the column rank \(c\) of \(\Ab\).

To prove the other direction, we can do the same thing to \(\Ab^\top\) and then we can get that 
column rank of \(\Ab\) = row rank of \(\Ab^\top \leq \) column rank of \(\Ab^\top\) = row 
rank of \(\Ab\) which we proved above. 
\end{proof}

\newpage 
\addcontentsline{toc}{subsection}{3C Problem Sets}
\begin{problem}{1}
    Suppose \(T \in \Lc (V, W)\). Show that with respect to each choice of bases of \(V\) and 
    \(W\), the matrix of \(T\) has at least \(\dim \range T\) nonzero entries. 
\end{problem}

\begin{proof}
Let \(v_1, \ldots, v_n\) be the basis of \(V\) and \(w_1, \ldots, w_m\) be the basis of \(W\). 
Suppose there exists a matrix \(\Ab = \Mc(T)\) of \(T\) has less than \(\dim \range T\) nonzero 
entries.  This means that \(\Ab\) has at most \(\dim \range T - 1\) nonzero columns so this implies 
that \(\dim \range T < \dim \range - 1\) which forms a contradiction. 
\end{proof}

\begin{problem}{2}
    Suppose \(V\) and \(W\) are finite-dimensional and \(T \in \Lc(V, W)\). Prove that 
    \(\dim \range T = 1\) if and only if there exists a basis of \(V\) and a basis of \(W\)
    such that with respect to these bases, all entries of \(\Mc(T)\) equal 1. 
\end{problem}

\begin{proof}


\(\Rightarrow\) We solve this problem through careful construction of the basis. That is, 
we will construct basis \(v_1, \ldots, v_n\) of \(V\) and \(w_1, \ldots, w_m\) of \(W\)
such that \(T v_i = w_1 + \cdots + w_m\) for all \(i\). We can achieve this again mainly 
because \(\dim \range T  = 1\). let \(u_1, \ldots, u_m\) be a set of arbitrary basis of 
\(W\). Take \(w \in \range T\) so we have \(w = \sum_{i=1}^{m} a_i u_i\). We consider all 
the coefficients \(a_i\) as follows: take an index set \(I\) such that for all \(i \in I, 
a_i \neq 0\) and we have \(|I| = c\). Take arbitrary \(j \in I\) and we construct the 
basis as follows: let \(w_j = (a_j - (m - r))u_j, w_k = a_k u_k \) for all \(k \in I, k \neq j\), 
and \(w_k = u_k + u_j\) for all \(k \notin I\). Then we have that \(w = \sum_{l=1}^{m}w_l\). 
Since we know \(w \in \range T\), then there exists \(v_1 \in V\) s.t. \(T(v_1) = w\). Let 
\(v_2, \ldots, v_n\) be the basis of \(\nul T\). Since we know \(\dim V = \nul T + 1\) so 
\(v_1, v_2, \ldots, v_n\) consistutes a basis of \(V\). In this way, we successfully constructs
the basis such that all entries of \(\Mc(T)\) equal 1. 


\(\Leftarrow\) There exists basis \(v_1, \ldots, v_n\) of \(V\) and 
\(w_1, \ldots, w_m\) of \(W\) such that 
all entries of \(\Mc(T)\) equal 1. This means that \(T v_1 = \cdots = T v_n = w_1 + \cdots + w_m\). 
So we first have that \(\dim \range T \neq 0\). To prove \(\dim \range T = 1\), take arbitrary 
\(u_1, u_2 \in \range T\). Then we know that \(u_1 = T(\sum_{i=1}^{n} a_i v_i) = \sum_{i=1}^{n}a_i (w_1 + \cdots + w_n)\)
and similarly \(u_2 = \sum_{i=1}^{n}b_i (w_1 + \cdots + w_n)\). Here we have that 
\[u_1 =(\sum_i^n a_i) / (\sum_i^n b_i) u_2 \]
and thus \(\dim \range T\) has to be 1-dimensional since every two arbitrary vector is simply a 
scalar multiple of each other. 
\end{proof}

\begin{problem}{3}
    Suppose \(v_1, \ldots, v_n\) is a basis of \(V\) and \(w_1, \ldots, w_m\) is a basis of \(W\). 

    Show that if \(S, T \in \Lc(V, W)\) and \(\lambda \in \F\), then \(\Mc(\lambda S +T)
    = \lambda \Mc(S) + \Mc(T)\). 
\end{problem}

\begin{proof}
\begin{align*}
    (\lambda S + T) v_k 
    &= \lambda S(v_k) + T(v_k) \\ 
    &= \lambda \sum_{i=1}^{m} \Ab_{i, k} w_i + \sum_{i=1}^{m} \Bb_{i, k} w_i 
\end{align*}
\end{proof}


\begin{problem}{5}
    Suppose \(V\) and \(W\) are finite-dimensional and \(T \in \Lc(V, W)\). Prove that 
    there exists a basis of \(V\) and a basis of \(W\) such that with respect to these bases, 
    all entries of \(\Mc(T)\) are 0 except that the entries in row \(k\), column \(k\) equal 1 
    if \(1 \leq k \leq \dim \range T\). 
\end{problem}

\begin{proof}
Let \(\dim V = n, \dim W = m, \dim \range T = k\). The main idea of the proof is to construct 
basis \(v_1, \ldots, v_n\) and \(w_1, \ldots, w_m\) such that \(Tv_i = w_i\) for all \(1 \leq i \leq k\)
and zero otherwise. (So the constructed \(T\) satisfies the requirement). 

We proceed with obtaining the basis \(w_1, \ldots, w_k\) of \(\range T\) and extend this 
\(w_1, \ldots, w_k, \ldots, w_m\) to the basis of \(W\). We know for all \(1 \leq i \leq k\),
there exists \(v_i \in V\) s.t. \(T v_i = w_i\). We claim that \(v_1, \ldots, v_k\) are linearly 
independent. To prove this, see \(\sum_{j=1}^{k}a_j v_j = 0 \Rightarrow \sum_{j=1}^{k}a_jT(v_j)
= \sum_{j=1}^{k}a_j w_j = 0\). Similar to the proof in the fundamental theorem of linear map, 
we extend the basis to \(V\) through considering the null space. We further claim that 
(let \(K = \text{span }(v_1, \ldots, v_k)\)) \(V = K \oplus \nul T\) (which is easy to prove). Hence, 
extending the basis from \(K\) with the basis from \(\nul T\) completes the proof.  
\end{proof}



\begin{problem}{6}
    Suppose \(v_1, \ldots, v_m\) is a basis of \(V\) and \(W\) is finite-dimensional. Suppose 
    \(T \in \Lc(V, W)\). Prove that there exists a basis \(w_1, \ldots, w_n\) of \(W\) such 
    that all entries in the first column of \(\Mc(T)\) [with respect to these bases] are 0 
    except for possibly a 1 in the first row, first column. 
\end{problem}


\begin{proof}
Let \(w_1 = T v_1\) and extend \(w_1\) to basis of \(W\). Then we automatically obtains \(T\) 
that gets the desired property. 
\end{proof}


\begin{problem}{7}
    Suppose \(w_1, \ldots, w_n\) is a basis of \(W\) and \(V\) is finite-dimensional. Suppose 
    \(T \in \Lc(V, W)\). Prove that there exists a basis \(v_1, \ldots, v_m\) of \(V\) such 
    that all entries in the first of \(\Mc(T)\) [wrt. these bases] are 0 except for possibly 
    a 1 in the first row, first column. 
\end{problem}

\begin{proof}

Take arbitrary \(u_1 \in V\) and we have that \(T u_1 = \sum_{i=1}^{n} a_{i, 1} w_i\). Take 
\(v_1 = u_1 / a_{1, 1}\), then \(T v_1 = w_1 + \sum_{j=2}^{n} b_{j, 1} w_j\). We can extend 
\(v_1, u_2, \ldots, u_m\) to be the basis of \(V\). Then consider 
\[T u_j = \sum_{k=1}^{n}b_{k, j} w_k\]

Consider to let \(b_{1, j} = 0\) for all \( 2 \leq j \leq n\). To do this, let \(v_j = u_j 
- b_{1, j} v_1\), then we have that \(T(v_j) = T(u_j - b_{1, j} v_1) = \sum_{k=2}^{n} c_{k, j} w_k\). 
It now left to verify that \(v_1, v_2, \ldots, v_m\) is the basis. 
\begin{align*}
    q_1 v_1 + \cdots + q_m v_m &= 0 \\ 
    q_1 v_1 + q_2(u_2 - b_{1, 2} v_1) + \cdots + q_m (u_m - b_{1, m} v_1) &= 0 \\ 
    (q_1 - (q_2 b_{1, 2} + \cdots + q_m b_{1, m})) v_1 + q_2 u_2 + \cdots + q_m u_m &= 0 
\end{align*}

The only solution is that \(q_1 - (q_2 b_{1,2} + \cdots + q_m b_{1, m}) = q_2 = \cdots = q_m = 0\)
and thus we have completed the proof. 
\end{proof}



\begin{problem}{8}
    Suppose \(\Ab\) is an m-by-n matrix and \(\Bb\) is an n-by-p matrix. Prove that 
    \[(\Ab \Bb)_{j, \cdot} = \Ab_{j, \cdot} \Bb\]
    for each \(1 \leq j \leq m \). In other words, show that row \(j\) of \(\Ab \Bb\) equals 
    (row \(j\) of \(\Ab\)) times \(\Bb\). 
\end{problem}

\begin{proof}
We know that 
\[(\Ab \Bb)_{j, k} = \sum_{i=1}^{n}\Ab_{j, i} \Bb_{i, k}\]

For instance \((\Ab \Bb)_{j, 1} = \sum_{i=1}^{n}\Ab_{j, i} \Bb_{i, 1} = \Ab_{j, \cdot} \Bb_{\cdot, 1}\). 
Similarly, \((\Ab \Bb)_{j, k} = \Ab_{j, \cdot} \Bb_{\cdot, k}\) and thus by treating this we have 
\[(\Ab \Bb)_{j, \cdot} = \Ab_{j, \cdot} \Bb\] 
\end{proof}

\begin{problem}{9}
    Suppose \(\ab = (a_1, \cdots, a_n)\) is a 1-by-n matrix and \(\Bb\) is an n-by-p matrix. 
    Prove that 
    \[\ab \Bb = a_1 \Bb_{1, \cdot} + \cdots + a_n \Bb_{n, \cdot}\]
    In other words, show that \(\ab \Bb\) is a linear combination of the rows of \(\Bb\), 
    with the scalars that multiply the rows coming from \(\ab\).
\end{problem}


\begin{proof}
By definiion of matrix multiplicaion, the entry in the column \(k\) of the l.h.s. equals 
\[(\ab \Bb)_{1, k} = a_1 \Bb_{1, k} + \cdots + a_n \Bb_{n, k}\]
which equals the right side. 
\end{proof}


\begin{problem}{11}
    Prove that the distributive property holds for matrix addition and multiplication. In 
    other words, suppose \(\Ab, \Bb, \Cb, \Db, \Eb, \Fb\) aare matrices whose sizes are such that 
    \(\Ab(\Bb + \Cb)\) and \((\Db + \Eb)\Fb\) make sense. Explain this and prove that 
    \[\Ab(\Bb + \Cb) = \Ab \Bb + \Ab \Cb \ \ \ \ (\Db + \Eb)\Fb = \Db \Fb + \Eb \Fb\]
\end{problem}

\begin{proof}
We let \(\Ab, \Db, \Eb\) to be m-by-n matrix and \(\Bb, \Cb, \Fb\) to be n-by-p matrix. Then 
all the matrix multiplication make sense. We proceed to prove the equality:

\begin{align*}
    (\Ab (\Bb + \Cb))_{j, k} 
    &= \sum_{i=1}^{n} \Ab_{j, i} (\Bb + \Cb)_{i, k} \\ 
    &= \sum_{i=1}^{n} \Ab_{j, i} (\Bb_{i, k} + \Cb_{i, k}) \\ 
    &= \sum_{i=1}^{n} (\Ab_{j, i} \Bb_{i, k}) + (\Ab_{j, i} \Cb_{i, k}) \\ 
    &= \Ab \Bb + \Ab \Cb 
\end{align*}


For the other one, we have that 

\begin{align*}
    ((\Db + \Eb)\Fb)_{j, k} 
    &= \sum_{i=1}^{n} (\Db + \Eb)_{j, i} \Fb_{i, k} \\ 
    &= \sum_i^n (\Db_{j, i} \Fb_{i, k}) + (\Eb_{j, i} \Fb_{i, k}) \\ 
    &= \Db \Fb + \Eb \Fb 
\end{align*}

\end{proof}


\begin{problem}{13}
    Suppose \(\Ab\) is an n-by-n matrix and \(1 \leq j, k \leq n\). Show that the entry 
    in row \(j\), column \(k\) of \(\Ab^3\) (which is defined to mean \(\Ab \Ab \Ab\)) is 
    \[\sum_{p=1}^{n} \sum_{r=1}^{n} \Ab_{j, p} \Ab_{p, r} \Ab_{r, k}\]
\end{problem}


\begin{proof}

\begin{align*}
    (\Ab (\Ab \Ab))_{j, k }
    &= \sum_p^n \Ab_{j, p} (\Ab \Ab)_{p, k} \\ 
    &= \sum_p^n \Ab_{j, p} \sum_{r=1}^{n} \Ab_{p, r} \Ab_{r, k} \\ 
    &= \sum_{p}^{n} \sum_r^n \Ab_{j, p} \Ab_{p, r} \Ab{r, k}
\end{align*}

\end{proof}


\begin{problem}{14}
    Suppose \(m\) and \(n\) are postiive integers. Prove that the function \(\Ab \mapsto \Ab^\top\)
    is a linear map from \(\Fb^{m, n}\) to \(\Fb^{n, m}\).  
\end{problem}

\begin{proof}
Let \(T\) denote the linear map such that \(T(\Ab) = \Ab^\top\). Then we have that 
\[T(\lambda \Ab + \Bb) = (\lambda \Ab + \Bb)^\top = \lambda \Ab^\top + \Bb^\top = \lambda T(\Ab) + T(\Bb)\]
\end{proof}

\begin{problem}{15}
    Prove that if \(\Ab\) is an m-by-n matrix and \(\Cb\) is an n-by-p matrix, then 
    \[(\Ab \Cb)^\top = \Cb^\top \Ab^\top\] 
\end{problem}

\begin{proof}
\begin{align*}
    ((\Ab \Cb)^\top)_{j, k}
    &= (\Ab \Cb)_{k, j} \\ 
    &= \sum_{i=1}^{n} \Ab_{k, i} \Cb_{i, j}
\end{align*}

On the other hand, 
\begin{align*}
    (\Cb^\top \Ab^\top)_{j, k}
    &= \sum_{i=1}^{n} \Cb^\top_{j, i} \Ab_{i, k}^\top \\ 
    &= \sum_{i=1}^{n} \Ab_{k, i} \Cb_{i, j}
\end{align*}
\end{proof}


\begin{problem}{16}
    Suppose \(\Ab\) is an m-by-n matrix with \(\Ab \neq 0\). Prove that the rank of \(\Ab\)
    is 1 if and only if there exist \((c_1, \ldots, c_m) \in \F^m\) and 
    \((d_1, \ldots, d_n) \in \F^n\) such that \(\Ab_{j, k} = c_j d_k\)
    for every \(j = 1, \ldots, m\) and every \(k = 1, \ldots, n\). 
\end{problem}

\begin{proof}
\(\Rightarrow\) If rank of \(\Ab\) is 1, then this means that the span of 
the columns of \(\Ab\) is 1-dimensional, take \(\Ab_{\cdot, 1} \in \F^m\) from the span. 
(Note that here we assume \(\Ab_{\cdot, 1} \neq 0\) ow. take the first nonzero one)
Then we can let \((c_1, \ldots, c_m) = \text{vec}(\Ab_{\cdot, 1})\). and let 
For every other columns, they are the scalar multiple of the the first one, so 
there exists \(d_2, \ldots, d_n\) s.t. \(\Ab_{j, k}  = c_j d_k\) (we take 
\(d_1 = 1\)). 


\(\Leftarrow\) If we denote \( \vecc = (c_1, \ldots, c_m)\), then 
\[\Ab =  \begin{pmatrix}
    d_1 \vecc & \cdots & d_n \vecc
\end{pmatrix}\]
so each column is a scalar multiple of the other and thus \(\Ab\) has rank 1. 
\end{proof}

\begin{problem}{17}
    Suppose \(T \in \Lc(V)\), and \(u_1, \ldots, u_n\) and \(v_1, \ldots, v_n\) are bases of \(V\). 
    Prove that the following are equivalent:

    (a) \(T\) is injective. \\ 
    (b) The column of \(\Mc(T)\) are linearly independent in \(\F^{n, 1}\) \\ 
    (c) The columns of \(\Mc(T)\) span \(\F^{n, 1}\). \\ 
    (d) The rows of \(\Mc(T)\) span \(\F^{1 , n}\). \\ 
    (e) The rows of \(\Mc(T)\) are linearly independent in \(\F^{1, n}\). \\ 
    Here \(\Mc(T)\) means \(\Mc(T, (u_1, \ldots, u_n), (v_1, \ldots, v_n))\). 
\end{problem}

\begin{proof}

    Let \(\Ab\) represents the matrix of \(\Mc(T)\). 

\((a) \Rightarrow (b) \colon\) \(a_1 \Ab_{\cdot, 1} + \cdots + a_n \Ab_{\cdot, n} = 0 \). We also 
know that \(T v_i = \sum_{j=1}^{n}\Ab_{j, i} u_j\). Then we have that 
\[T(a_1 v_1 + \cdots + a_n v_n) = \sum_{i=1}^{n} a_i T(v_i) = \sum_{i=1}^{n}a_i \sum_{j=1}^{n}\Ab_{j, i} u_j 
= \sum_{j=1}^{n} (\sum_{i=1}^n a_i \Ab_{j, i}) u_j = 0 \]

To prove all \(a_i\) is zero 
we note that \(a_1 v_1 + \cdots + a_n v_n \in \nul T\) and since \(T\) is injective, the only possible case 
is that all \(a_i = 0\) and thus wee prove the claim. 


\((b) \Rightarrow (c)\) The columns of \(\Mc(T)\) forms a basis and thus span the space. 

\((c) \Rightarrow (d)\) Column rank of \(\Mc(T)\) is \(n\) and thus row rank is also \(n\) and thus the rows 
span \(\F^{1, n}\). 

\((d) \Rightarrow (e)\) Same as above. 

\((e) \Rightarrow (a)\) The only solution to \(a_1 \Ab_{1, \cdot} + \cdots + a_n \Ab_{n, \cdot} = 0\) is all 
\(a_i = 0\). Suppose for the sake of contradiction that \(T\) is not injective so there exists nonzero 
\( v \in \nul T\). We know that \(v = \sum_{i=1}^{n} a_i v_i\) and that \(T v_i = \sum_{j=1}^{n} \Ab_{j, i} u_j\). Then 
\[T v = \sum_{i=1}^{n} a_i T(v_i) = \sum_{i=1}^{n} a_i \sum_{j=1}^{n} \Ab_{j, i} u_j = \sum_{i=1}^{n}
(\sum_{j=1}^{n} a_i \Ab_{j, i})u_j = 0 \]

and thus \(\sum_{j=1}^{n} a_i \Ab_{j, i} = 0\) for all \(j\). We know not all \(a_i = 0\) since \(v\) is 
non-trivial and thus the solution contraidicts that the rows are all linearly independent. Proof completed.

\end{proof}




\newpage 

\section*{3D: Invertibility and Isomorphisms}
\addcontentsline{toc}{section}{3D: Invertibility and Isomorphisms}

\begin{definition}[invertible, inverse]
    \begin{itemize}
        \item A linear map \(T \in \Lc(V, W)\) is called \emph{invertible} if there exists a linear map 
        \(S \in \Lc(W, V)\) such that \(ST\) equals the identity operator on \(V\) and \(TS\) equals 
        the identity operator on \(W\). 
        \item A linear map \(S \in \Lc(W, V)\) satisfying \(ST = I\) and \(TS = I\) is called an 
        \emph{inverse} of \(T\). 
    \end{itemize}
\end{definition}

\begin{thm}
    An invertible linear map has a unique inverse. 
\end{thm}

If \(T\) is invertible, then its inverse is denoted by \(T^{-1}\). 


\begin{thm}
    A linear map is invertible if and only if it is injective and surjective. 
\end{thm}

\begin{proof}
\(\Rightarrow\) We have an invertible linear map \(T\) that \(T u = T v\). Then we have 
\[u = T^{-1} Tu = T^{-1} Tv = v\]

To show \(T\) is surjective, we have that for any \(w \in W, w = T(T^{-1}w)\). 

\(\Leftarrow\) Define \(S\) such that for each \(w \in W\), \(S(w)\) is the unique element 
s.t. \(T(S(w)) = w\). (we can do this due to injectivity and surjectivity). Then we have that 
\(T(ST)v = (TS)Tv = Tv\) and thus \(ST v = v\) and thus \(ST = I\). So we suffices the identity 
operator condition. It's easy to show that \(S\) is a linear map.

\end{proof}


\begin{thm}
    Suppose that \(V\) and \(W\) are finite-dimensional vector spaces, \(\dim V = \dim W\), 
    and \(T \in \Lc(V, W)\). Then 

    \[T \text{ is invertible } \Leftrightarrow T \text{ is injective } \Leftrightarrow T \text{ is surjective.}\]
\end{thm}

\begin{corollary}
    Suppose \(V\) and \(W\) are finite-dimensional vector spaces of the same dimension, \(S \in 
    \Lc(W, V)\), and \(T \in \Lc(V, W)\). Then \(ST = I\) if and only if \(TS = I\). 
\end{corollary}


\begin{definition}[Isomorphism]
    An \textbf{Isomorphism} is an invertible linear map. \\ 
    Two vector spaces are called \textbf{isomorphic} if there is an isomorphism from one vector space 
    onto the other one. 
\end{definition}


\begin{thm}[dimension shows whether vector spaces are isomorphic]
    Two finite-dimensional vector spaces over \(\F\) are isomorphic if and only if they have 
    the same dimension. 
\end{thm}


\begin{thm}
    Suppose \(v_1, \ldots, v_n\) is a basis of \(V\) and \(w_1, \ldots, w_m\) is abasis of \(W\). 
    Then \(\Mc\) is an isomorphism between \(\Lc(V, W)\) and \(\F^{m, n}\). 
\end{thm}


\begin{corollary}
    Suppose \(V\) and \(W\) are finite-dimensional. Then \(\Lc(V, W)\) is finite-dimensional and 
    \[\dim \Lc(V, W) = (\dim V) (\dim W)\]
\end{corollary}

\begin{definition}[matrix of a vector, \(\Mc(v)\)]
    Suppose \(v \in V\) and \(v_1, \ldots, v_n\) is a basis of \(V\). Then \textbf{matrix of \(v\)} 
    with respect to the basis is the n-by-1 matrix 
    \[\Mc(v) = \begin{pmatrix}
        b_1 \\ 
        \vdots \\ 
        b_n
    \end{pmatrix},\]
    where \(b_1, \ldots, b_n\) are the scalars such that 
    \[v = b_1 v_1 + \cdots + b_n v_n\] 
\end{definition}


\begin{remark}
    The matrix \(\Mc(v)\) of a vector \(v \in V\) depends on the basis \(v_1, \ldots, v_n\) and \(v\). We 
    can think of elements of \(V\) as relabeled to be n-by-1 matrices, i.e. \(V \mapsto \F^{n , 1}\).
\end{remark}

\begin{corollary}
    Suppose \(T \in \Lc(V, W)\) and \(v_1, \ldots, v_n\) is a basis of \(V\) and \(w_1, \ldots, w_m\)
    is a basis of \(W\). Let \(1 \leq k \leq n\). Then \(k\)th column of \(\Mc(T)\), which is denoted 
    by \(\Mc(T)_{\cdot, k}\), equals \(\Mc(T v_k)\)
\end{corollary}


\begin{thm}[linear maps act like matrix multiplication]
    Suppose \(T \in \Lc(V, W)\) and \(v \in V\). Suppose \(v_1, \ldots, v_n\) is a basis of \(V\)
    and \(w_1, \ldots, w_m\) is a basis of \(W\). Then 
    \[\Mc(Tv) = \Mc(T) \Mc(v)\]
\end{thm}

\begin{proof}
\begin{align*}
    \Mc(Tv)
    &= b_1 \Mc(T v_1) + \cdots + b_n \Mc(T v_n) \\ 
    &= b_1 \Mc(T)_{\cdot, 1} + \cdots + b_n \Mc(T)_{\cdot, n} \\ 
    &= \Mc(T) \Mc(v)
\end{align*}
\end{proof}

\begin{remark}
    Each m-by-n matrix \(\Ab\) induces a linear map from \(\F^{n, 1}\) to \(\F^{m, 1}\), namely the matrix 
    multiplication function that takes \(x \in \F^{n, 1}\) to \(\Ab x \in \F^{m, 1}\). We can think of 
    every linear map (from a finite vector space to another) as a matrix multiplication map 
    after suitable relabeling via the isomorphisms given by \(\Mc\). Specifically, if \(T \in \Lc(V, W)\)
    and we identify \(v \in V\) with \(\Mc(v) \in \F^{n, 1}\), then the result above says that we 
    can identify \(Tv\) with \(\Mc(T) \Mc(v)\). 
\end{remark}

\begin{thm}[dimension of \(\range T\) equals column rank of \(\Mc(T)\)]
    Suppose \(V\) and \(W\) are finite-dimensional and \(T \in \Lc(V, W)\). Then \(\dim \range T\)
    equals the column rank of \(\Mc(T)\). 
\end{thm}



\begin{thm}[change-of-basis-formula]
    Suppose \(T \in \Lc(V)\). Suppose \(u_1, \ldots, u_n\) and \(v_1, \ldots, v_n\) are bases of 
    \(V\). Let 
    \[\Ab = \Mc(T, (u_1, \ldots, u_n)) \text{  and  } \Bb = \Mc(T, (v_1, \ldots, v_n))\]
    and \(\Cb = \Mc(I, (u_1, \ldots, u_n), (v_1, \ldots, v_n))\). Then 
    \[\Ab = \Cb^{-1} \Bb \Cb.\]
\end{thm}

\begin{proof}

\end{proof}
\addcontentsline{toc}{subsection}{3D Problem Sets}
\begin{problem}{1}
    Suppose \(T \in \Lc(V, W)\) is invertible. Show that \(T^{-1}\) is invertible and 
    \[(T^{-1})^{-1} = T.\]
\end{problem}

\begin{proof}

    \(T^{-1}\) is invertible because there exists \(T\) such that \(T T^{-1} = T^{-1} T  = I\). 

\[T^{-1} T = TT^{-1} = I\]

so \((T^{-1})^{-1} = T\).
\end{proof}


\begin{problem}{2}
    Suppose \(T \in \mathcal{L}(U, V)\) and \(S \in \Lc(V, W)\) are both invertible linear maps. 
    Prove that \(ST \in \Lc(U, W)\) is invertible and that \((ST)^{-1} = T^{-1}S^{-1}\).
\end{problem}

\begin{proof}
\((ST)(T^{-1}S^{-1}) = S(T T^{-1})S^{-1} = I = T^{-1}S^{-1}S T\).
\end{proof}

\begin{problem}{3}
    Suppose \(V\) is finite-dimensional and \(T \in \Lc(V)\). Prove that the following are equivalent.

    (a) \(T\) is invertible. \\ 
    (b) \(T v_1, \ldots, T v_n\) is a basis of \(V\) for every basis \(v_1, \ldots, v_n\) of \(V\). \\ 
    (c) \(T v_1, \ldots, T v_n\) is a basis of \(V\) for some basis \(v_1, \ldots, v_n\) of \(V\). 
\end{problem}

\begin{proof}
\((a) \Rightarrow (b)\) It only suffice to prove linear independence. We can show this 
\[a_1 Tv_1 + \cdots + a_n T v_n = 0 \Longleftrightarrow a_1 v_1 + \cdots + a_n v_n = 0\]

since \(T\) is injective and thus the only solution is all \(a_i\) are identically zero. 

\((b) \Rightarrow (c)\) Trivial. 

\((c) \Rightarrow (a)\) By the linear map lemma, there exists \(S \in \Lc(V)\) such that 
\(S(T v_i) = v_i\) for all \(i\). Such \(S\) is the inverse of \(T\) (one can verify) and thus 
\(T\) is invertible. 
\end{proof}


\begin{problem}{5}
    Suppose \(V\) is finite-dimensional, \(U\) is a subspace of \(V\), and \(S \in \Lc(U, V)\). Prove 
    that there exists an invertible linear map \(T\) from \(V\) to itself such that 
    \(T u = Su\) for every \(u \in U\) if and only if \(S\) is injective. 
\end{problem}

\begin{proof}
\(\Rightarrow\) Since \(T\) is invertible, we can define \(T^{-1}\) restricted to \(U\), where 
\(S^{-1} u = T^{-1}u \) for all \(u \in U\). Then \(S\) has an inverse and thus is injective. 

\(\Leftarrow\) Let \(v_1, \ldots v_m\) be the basis of \(U\) and \(v_1, \ldots, v_m, x_1 \ldots, v_n\) 
be the extended basis of \(V\). Define \(T v_i = S v_i\) and \(T x_i = x_i\). Then \(T\) is injective 
and thus invertible. 
\end{proof}

\begin{problem}{6}
    Suppose that \(W\) is finite-dimensional and \(S, T \in \Lc(V, W)\). Prove that 
    \(\nul S = \nul T\) if and only if there exists an invertible \(E \in \Lc(W)\) such that 
    \(S = ET\).  
\end{problem}

\begin{proof}
\(\Rightarrow\) Suppose \(V\) is finite-dimensional for simplicity (one can do more work for 
relaxing this assumption). 
 Let \(V = \nul S \oplus C = \nul T \oplus C\) for some \(C\). Then we know that 
\(T|_C \colon C \to \range (T(C))\) and \(S|_C \colon C \to \range(S(C))\) is invertible.  Then 
there exists an invertible map \(\Eh \colon \range(T(C)) \to \range(S(C))\) as \(\dim \range T(C) 
= \range (S(C)) = \dim V\). Extending this map to \(W\) solves the problem (take bases from \(\range T(C)\)
and extend to \(W\) with ``identity'' map).

\(\Leftarrow\) Take \(v \in \nul T\), then \(S(v) = E(Tv) = E(0)  = 0\) and thus \(v \in \nul S\). Conversely,
take \(v \in \nul S\), then \(T(v) = E^{-1}S(v) = 0\) so \(v \in \nul T\). Thus \(\nul S = \nul T\). 
\end{proof}

\begin{problem}{7}
    Suppose that \(V\) is finite-dimensional and \(S, T \in \Lc(V, W)\). Prove that \(\range S = \range T\)
    if and only if there exists an invertible \(E \in \Lc(V)\) such that \(S = TE\). 
\end{problem}

\begin{proof}
\(\Rightarrow\) By 3B P26, we know that there exists \(\ET \in \Lc(V)\) s.t. \(S = T \ET\) as 
\(\range S \subseteq \range T\). \(\range S = \range T\) implies that \(\dim \nul S = \dim \range T\). 
Then we can define an invertible map \(\Ebar\) from \(\nul S\) to \(\nul T\). We can define the map \(E\) as follows:
Denote \(V = \nul S \oplus P\), then \(E(v) = \ET(v_1) + \Ebar(v_2)\) for \(v = v_1 + v_2\), where 
\(v_1 \in P, v_2 \in \nul S\) and now \(\ET = \ET |_P\). Then we have that \(TE(v) = T(\ET(v_1) + \Ebar(v_2)) = S(v_1)
+ S(v_2) = S(v)\). Now it left to verify that our proposed \(E\) is invertible. 

To show this, we mainly aim at proving \(\range \Et \cap \range \Ebar = \{0\}\), as if we have this, we 
know that \(E(v) = \ET(v_1) + \Ebar(v_2) = 0 \Rightarrow \ET(v_1) = \Ebar(v_2) = 0\). We know \(\Ebar\) is 
invertible so \(v_2 = 0\). To see \(\ET\) is injective, we can see that for any \(v_1 \in \nul \ET\), we 
have \(S v_1 = T \ET v_1 = T 0 = 0\) and thus \(v_1 \in \nul S \cap P = 0\), so \(v_1 = 0\) and thus 
\(nul (E) = \{0\}\), \(E\) is injective and therefore invertible. 

To complete the proof, let \(u \in \range \ET \cap \range \Ebar\), then we know there exists 
\(v_1 \in P, v_2 \in \nul S\) s.t. \(u = \ET(v_1) = \Ebar(v_2)\). We also know that \(u \in \nul T\)
as \(\range \Ebar = \nul T\). Now we have that \(S (v_1) = T \ET(v_1) = T \Ebar(v_2) = 0\) and thus 
\(v_1 \in \nul S\). So \(v_1 \in  P \cap \nul S = \{0\}\) and thus we have that \(u = \{0\}\), finishing 
the proof.

% then we know \(u \in \nul T\) as 
% \(\range \Ebar = \nul T\). So we have that there is \(x \in \nul S, Sx = T \ET x = T \Ebar x = T u = 0.\)
% So we know that \(x \in \nul T\)

% This means that \(\dim \nul S = \dim \nul T\) and thus we can define an invertible map 
% \(\Eh\) from \(\nul S\) to \(\nul T\). 


\(\Leftarrow\) Take any \(s \in \range S\), then there exists \(E(s)\) s.t. \(T(E(s)) = s\). Conversely, 
take any \(t \in \range T\), then there exists \(v \in V\) s.t. \(t = Tu = TE E^{-1}v = S (E^{-1}v)\). 
Thus \(\range S = \range T\). 

\end{proof}


\begin{problem}{8}
    Suppose \(V\) and \(W\) are finite-dimensional and \(S, T \in \Lc(V, W)\). Prove that 
    there exist invertible \(E_1 \in \Lc(V)\) and \(E_2 \in \Lc(W)\) such that 
    \(S = E_2 T E_1\) if and only if \(\dim \nul S = \dim \nul T\). 
\end{problem}

\begin{proof}
\(\Rightarrow\) We know from 3B exercises that \(\dim \range ST \leq \min (\dim \range S, \dim \range T)\). 
Applying  here gets that \(\dim \range S \leq \dim \range T\). Consider that \(T = E^{-1}_2 S E_1^{-1}\), 
then we can also derive \(\dim \range T \leq \dim \range S\) and thus \(\dim \range S = \dim \range T\). 
By the fundamental theorem of linear map, we can get that \(\dim \nul S = \dim \nul T\). 

\(\Leftarrow\) We know \(\dim \nul S = \dim \nul T\), so there exists an isomorphism \(\ET_1 \colon 
\nul S \mapsto \nul T\). Extending \(\ET_1 \colon \nul S \mapsto V\) stil preserves its injectivity. Therefore 
by P5 we have an invertible \(E_1 \colon V \mapsto V\) such that \(E_1 u = \ET_1 u\) for all \(u \in \nul S\). 
Now we intend to show that \(\nul S = \nul T E_1\), as proving so would imply there exists an invertible 
\(E_2 \in \Lc(W)\) such that \(S=E_2 T E_1\) by P6. To see this, take \(s \in \nul S\), then 
\(TE_1(s) = T(\ET_1 s) = 0\) and thus \(s \in \nul TE_1\). To see the other direction, take 
\(t \in \nul T E_1\), then we know \(E_1 t \in \nul T = \range \ET_1\). Thus there exists \(v \in \nul S\)
s.t. \(E_1 t = \ET_1 v = E_1 v\). As \(E_1\) is invertible, \(t = v \in \nul S\) and thus we have shown 
\(\nul S = \nul TE_1\), completing the proof.   
\end{proof}


\begin{problem}{9}
    Suppose \(V\) is finite-dimensional and \(T \colon V \to W\) is a surjective linear map of \(V\)
    onto \(W\). Prove that there is a subspace \(U\) of \(V\) such that \(T |_U\) is an isomorphism 
    of \(U\) onto \(W\).
\end{problem}

\begin{proof}
\(T\) being surjective means that \(\dim V \geq \dim W\). Take any subspace of \(V\) that has 
\(\dim W\), then there is an isomorphism between this subspace and \(W\). 
\end{proof}

\begin{problem}{10}
    Suppose \(V\) and \(W\) are finite-dimensional and \(U\) is a subspace of \(V\). Let 
    \[\Ec = \{T \in \Lc(V, W) \colon U \subseteq \nul T\}.\]

    (a) Show that \(\Ec\) is a subspace of \(\Lc(V, W)\). \\ 
    (b) Find a formula for \(\dim \Ec\) in terms of \(\dim V, \dim W,\) and \(\dim U\). 
\end{problem}

\begin{proof}
(a) Take \(T_1, T_2 \in \Ec\) and \(\lambda \in \F\), then consider \(\nul(\lambda T_1 + T_2)\). Let 
\(u \in U\), then \(\lambda T_1 + T_2 (u) = \lambda T_1 (u) + T_2 (u) = 0 + 0 = 0\) and thus we have 
\(\lambda T_1 + T_2  \in \Ec\) and thus \(\Ec\) is a subspace of \(\Lc(V, W)\). 

(b) Following the hint, define \(\Phi \colon \Lc(V, W) \to \Lc(U, W)\) by \(\Phi(T) = T |_U\). Then 
\(\Phi \in \Lc(\Lc(V, W), \Lc(U, W))\). We try to show the range and null space of \(\Phi\). 

We first 
claim \(\nul \Phi = \Ec\). To see this, take \(T \in \nul \Phi\), then we know \(\Phi(T) 
= T |_U = 0 \in \Lc(U, W)\), meaning that \(U\) is the subset of \(\nul T\), so \(\nul \Phi \subseteq \Ec \). 
Conversely, take \(T \in \Ec\), then we have that for all \(u \in U\), \(Tu = 0\), then we have that 
\(T |_U = 0\). This shows that \(\Ec \subseteq \nul \Phi\). 

Next we claim \(\range \Phi = \Lc(U, W)\). Take any \(S \in \Lc(U, W)\), then we can naturally always extend 
\(S\) to \(T \in \Lc(V, W)\) by setting \(T u = S u\) for all \(u \in U\). The converse direction is trivial. 

Thus we have that \(\dim \Ec = \dim \nul \Phi\) and \(\dim \range \Phi = \dim \Lc(U, W) = \dim U \dim W\)
and \(\dim \Lc(V, W) = \dim V \dim W = \dim \nul \Phi + \dim \range \Phi\), thus we derive that 
\[\dim V \dim W = \dim \Ec + \dim U \dim W.\]

\end{proof}



\begin{problem}{14}
    Prove or give a counterexample: If \(V\) is finite-dimensional and \(R, S, T \in \Lc(V)\) are such 
    that \(RST\) is surjective, then \(S\) is injective. 
\end{problem}


\begin{proof}
We know that \[\dim \range RST = \dim \range V \leq \min \{\dim \range R, \dim \range S, \dim \range T\} \]
This means that \(\dim RST \leq \dim \range S\).
If \(\dim \nul S > 0\), then \(\dim \range S < \dim V\) and thus 
contradicts the assumption that \(RST\) is surjective. 
\end{proof}

\begin{problem}{15}
    Suppose \(T \in \Lc(V)\) and \(v_1, \ldots, v_m\) is a list in \(V\) such that \(T v_1, \ldots, T v_m\)
    spans \(V\). Prove that \(v_1, \ldots, v_m\) spans \(V\). 
\end{problem}

\begin{proof}
We can reduce the list to basis and use previous conclusions. Since \(T v_1, \ldots, T v_m\) spans \(V\), 
we can reduce the list to \(T v_1, \ldots, T v_n\) such that the list is the basis of \(V\). By previous 
results, \(v_1, \ldots, v_n\) is also linearly independent and its dimension equals the \(\dim V\) and thus 
is the basis of \(V\). It thus spans \(V\). Extending the list to \(v_1, \ldots, v_m\) also spans \(V\). 

\end{proof}


\begin{problem}{16}
    Prove that every linear map from \(\F^{n, 1}\) to \(\F^{m, 1}\) is given by a matrix multiplication. 
    In other words, prove that if \(T \in \Lc(\F^{n, 1}, \F^{m, 1})\), then there exists an m-by-n matrix 
    \(\Ab\) such that \(Tx =\Ab x\) for every \(x \in \F^{n, 1}\).
\end{problem}

\begin{proof}
We can simply consider the standard basis of \(\F^{n, 1}\) and \(\F^{m, 1}\). Then Let \(\Ab\) be the matrix 
of \(T\) wrt these bases. Then we have that by definition, 
\[T x = \Mc(Tx) = \Mc(T) \Mc(x) = \Ab x\]
\end{proof}

\begin{problem}{17}
    Suppose \(V\) is finite-dimensional and \(S \in \Lc(V)\). Define \(\Ac \in \Lc(\Lc(V))\) by 
    \[\Ac(T) = ST\] 
    for \(T \in \Lc(V)\). 

    (a) Show that \(\dim \nul \Ac = \dim V \dim \nul S\). \\ 
    (b) Show that \(\dim \range \Ac = \dim V \dim \range S\). 
\end{problem}

\begin{proof}
(a) We prove this by showing \(\nul \Ac = \Lc(V,\nul S)\). Take \(A \in \nul \Ac\), then we know that 
\(\Ac (A) = SA = 0\) and thus \(Av \in \nul S\) for all \(v \in V\). For the other direction, take 
\(A \in \Lc(V, \nul S)\), then \(SA(v) = \Ac(v) = 0\) and thus \(A \in \nul \Ac\). 

(b) We know \(\dim (\Lc(V)) = \dim \nul \Ac + \dim \range \Ac\). Hence 
\[\dim \range \Ac = \dim V (\dim V - \dim \nul S) = \dim V \dim \range S.\]
\end{proof}


\begin{problem}{18}
    Show that \(V\) and \(\Lc(\F, V)\) are isomorphic vector spaces. 
\end{problem}

\begin{proof}
For simplicity, assume \(V\) is finite-dimensional, then \(\dim V = \dim \F \dim V = \dim \Lc(\F, V)\) and 
thus they are isomorpic. One might do a more careful construction for infinite-dimensional \(V\). 
\end{proof}


\begin{problem}{19}
    Suppose \(V\) is finite-dimensional and \(T \in \Lc(V)\). Prove that \(T\) has the same matrix with 
    respect to every basis of \(V\) if and only if \(T\) is a scalar multiple of the identity operator.
\end{problem}

\begin{proof}
\(\Rightarrow\) We know \(T v_k = \sum_{j=1}^{n} A_{j, k} v_k = \sum_{j=1}^{n} 2 A_{j, k} v_k
= T (2v_k)\). This means that \(A_{j, k} = 0\) for \(j \neq k\). When \(j = k\), we have that 
\(T v_k = A_{k, k} v_i\) for some arbitrary ordering of the same basis. Then this means that 
\(A_{i, i} = A_{j, j}\) for all \(i, j\). Thus, \(T\) is a scalar multiple of the identity operator. 

\(\Leftarrow\) This follows by the definition of the identity operator. 
\end{proof}

\newpage 

\section*{3E: Products and Quotients of Vector Spaces}
\addcontentsline{toc}{section}{3E: Products and Quotients of Vector Spaces}

\begin{definition}[product of vector spaces]
    Suppose \(V_1, \ldots, V_m\) are vector spaces over \(\F\). 
    \begin{itemize}
        \item The \textbf{product} \(V_1 \times \cdots \times V_m\) is defined by 
        \[V_1 \times \cdots \times V_m = \{(v_1, \ldots, v_m) \colon 
        v_1 \in V_1, \ldots, v_m \in V_m\}.\]
        \item Addition on \(V_1 \times \cdots \times V_m\) is defined by 
        \[(u_1, \ldots, u_m) + (v_1, \ldots, v_m) = (u_1 + v_1, \ldots, u_m + v_m).\]
        \item Scalar multiplication on \(V_1 \times \cdots \times V_m\) is defined by 
        \[\lambda(v_1, \ldots, v_m) = (\lambda v_1, \ldots, \lambda v_m).\]
    \end{itemize}
\end{definition}

\begin{thm}[product of vector spaces is a vector space]
    Suppose \(V_1, \ldots, V_m\) are vector spaces over \(\F\). Then \(V_1 \times \cdots \times V_m\)
    is a vector space over \(\F\). 
\end{thm}


\begin{thm}[dimension of a product is the sum of dimensions]
    Suppose \(V_1, \ldots, V_m\) are finite-dimensional vector spaces. Then \(V_1 \times \cdots \times 
    V_m\) is finite-dimensional and 
    \[\dim (V_1 \times \cdots \times V_m) = \dim V_1 + \cdots + \dim V_m.\]
\end{thm}


\begin{lemma}[products and direct sums]
    Suppose that \(V_1, \ldots, V_m\) are subspaces of \(V\). Define a linear map \(\Gamma \colon 
    V_1 \times \cdots \times V_m \to V_1 + \cdots + V_m \) by
    \[\Gamma (v_1, \ldots, v_m) = v_1 + \cdots + v_m.\]
    Then \(V_1 + \cdots + V_m\) is a direct sum if and only if \(\Gamma\) is injective. 
\end{lemma}

\begin{thm}[a sum is a direct sum if and only if dimensions add up]
    Suppose \(V\) is finite-dimensional and \(V_1, \ldots, V_m\) are subspaces of \(V\). Then 
    \(V_1 + \cdots + V_m\) is a direct sum if and only if 
    \[\dim(V_1 + \cdots + V_m) = \dim V_1 + \cdots + \dim V_m.\]
\end{thm}


\begin{definition}[\(v + U\), translation]
    Suppose \(v \in V\) and \(U \subset V\). Then \(v + U\) is the subset of \(V\) defined by 
    \[v + U = \{v + u \colon  u \in U\}.\]

    The set \(v + U\) is said to be a \textbf{translate} of \(U\).
\end{definition}


\begin{definition}[\textbf{quotient space}, \(V/U\)]
    Suppose \(U\) is a subspace of \(V\). Then the \textbf{quotient space} \(V / U\) is the set of 
    all translates of \(U\). Thus 
    \[V / U = \{v + U \colon v \in V\}.\]
\end{definition}

\begin{remark}
    If \(U = \{(x, 2x) \in \R^2 \colon x \in R\}\), then \(\R^2 / U\) is the set of all lines in 
    \(\R^2\) that have slope 2. 
\end{remark}


\begin{lemma}[two translates of a subspace are equal or disjoint]
    Suppose \(U\) is a subspace of \(V\) and \(v, w \in V\). Then 

    \[v - w \in U \Leftrightarrow v + U = w + U \Leftrightarrow (v + U) \cap (w + U) = \emptyset\]
\end{lemma}

\begin{definition}[addition and scalar multiplication on \(V / U\)]
    Suppose \(U\) is a subspace of \(V\). Then \textbf{addition} and \textbf{scalar multiplication} 
    are defined on \(V / U\) by 
    \begin{align*}
        (v + U) + (w + U) &= (v+w) + U \\ 
        \lambda (v+U) = (\lambda v) + U 
    \end{align*}
    for all \(v, w \in V\) and all \(\lambda \in \F\). 
\end{definition}


\begin{thm}[quotient space is a vector space]
    Suppose \(U\) is a subspace of \(V\). Then \(V / U\), with the operations of addition and 
    scalar multiplication as defined above, is a vector space. 
\end{thm}

\begin{definition}[quotient map]
    Suppose \(U\) is a subspace of \(V\). The \textbf{quotient map} \(\pi \colon 
    V \to V / U\) is the linear map defined by 
    \[\pi(v) = v + U\]
    for each \(v \in V\). 
\end{definition}


\begin{thm}[dimension of quotient space]
    Suppose \(V\) is finite-dimensional and \(U\) is a subspace of \(V\). Then 
    \[\dim V/U = \dim V - \dim U\]
\end{thm}


\begin{definition}
    Suppose \(T \in \Lc(V, W)\). Define \(\TT \colon V / (\nul T) \to W\) by 
    \[\TT (v + \nul T) = Tv.\]
\end{definition}


Another interpretation of quotient space is on \textbf{congruence of subspaces}. The definition goes 
as follows:

If \(Y\) is a subspace of \(X\), then two vectors \(x_1, x_2 \in X\) are \textbf{congrugent modulo} \(Y\),
denoted \(x_1 \cong x_2 (\mod Y)\) if \(x_1 - x_2 \in Y\).

The quotient space \(X / Y\) denotes the set of equivalence classes in \(X\), modulo \(Y\) by defining that 
\[\{x\} + \{z\} \coloneqq \{x+ z\} \ \ \ a\{x\} \coloneqq \{ax\}\]

So here one can think of the zero vector of \(v + U\) defined as the equivalence class that contains that 
zero vector \(0 \in V\): 
\[0 + U = \{0 + u \colon u \in U\} = U.\]
This means that the zero element in the quotient space \(V / U\) is simply the subspace \(U\), and this 
equivalence class contains all vectors differ from 0 by an element of \(U\), meaning it is precisely 
\(U\) itself, so \(\pi(v) = 0\) if and only if \(v \in U\). 

\begin{remark}
    The output of \(\pi(v)\) is the equivalence class \(v + U\), which is the set of all vectors in \(V\)
    that are equivalent to \(v\) under the subspace \(U\).
\end{remark}

For \(\TT\), the input to it is an equivalent class \(v + \nul (T)\) in the quotient space \(V / \nul T\). 
It represents all vectors in \(V\) that differ from \(v\) by a vector in \(\nul T\). Its output is 
\(T v\), which is an element in \(W\). 

The null space is 
\[\nul (\Tt) = \{v + \nul(T) \colon \Tt(v + \nul (T)) = 0\}. \]
Since we know that \(\Tt(v + \nul T) = Tv\), we have that 
\[\Tt(v + \nul T) = 0 \ \Leftrightarrow \ Tv = 0.\]

so \(v \in \nul T\) and thus \(\nul \Tt = \{\nul T\}\). We have that \(\range \Tt = \range T\).

\begin{thm}
    Suppose \(T \in \Lc(V, W)\). Then \\ 
    (a) \(\Tt \circ \pi = T\), where \(\pi\) is the quotient map of \(V\) onto \(V / \nul T\); \\ 
    (b) \(\Tt\) is injectve;\\ 
    (c) \(\range \Tt = \range T\);\\ 
    (d) \(V / \nul T\) and \(\range T\) are isomorphic vector spaces. 
\end{thm}

\begin{remark}
    One might think of the domain of \(\Tt\) as the non-trivial domain of \(T\) and the range as the normal 
    image of \(T\). 
\end{remark}

\newpage 
\addcontentsline{toc}{subsection}{3E Problem Sets}

\begin{problem}{1}
    Suppose \(T\) is function from \(V\) to \(W\). The \emph{graph} of \(T\) is the subset of \(V \times W\)
    defined by 
    \[\text{graph of } T = \{(v, Tv) \in V \times W \colon v \in V\}.\]
    Prove that \(T\) is a linear map if and only if the graph of \(T\) is a subspace of \(V \times W\). 
\end{problem}

\begin{proof}
The two properties closely relate to each other:
\[(v_1, Tv_1) + (v_2, Tv_2) \in V \times W \colon v \in V \Leftrightarrow Tv_1 + Tv_2 = T(v_1 + v_2)\]
and 
\[(\lambda v, T \lambda v) \in V \times W \colon v \in V \Leftrightarrow T \lambda v = \lambda T v\]
\end{proof}

\begin{problem}{2}
    Suppose that \(V_1, \ldots, V_m\) are vector spaces such that \(V_1 \times \cdots \times V_m\) 
    is finite-dimensional. Prove that \(V_k\) is finite-dimensional for each \(k = 1,\ldots,m\).
\end{problem}

\begin{proof}
    Recall the definition of infinite-dimension from ch2 P17:

    \begin{center}
        \(V\) is infinite-dimensional if and only if there is a sequence 
        \(v_1, v_2, \ldots\) of vectors in \(V\) such that \(v_1, \ldots, v_m\)
        is linearly independent for every positive integer \(m\). 
    \end{center}

    Suppose for the sake of contradiction that there exists \(V_k\) to be infinite-dimensional. Then 
    this means that there is a sequence \(v_{k, 1}, v_{k, 2}, \ldots\) of vectors in \(V_k\) such 
    that \(v_{k, 1}, \ldots, v_{k, m}\) is linearly independent for every positive integer \(m\). 
    We can build such sequence for the product space as well. More specifically, consider the sequence 
    \(0, \ldots, v_{k, j}, \ldots, 0\) where we fill 0 for each of other \(V_j\)'s and only leave the 
    sequence from \(V_k\). In this case, we can see that the product space is infinite-dimensional, 
    contradicting the claim. 
\end{proof}

\begin{problem}{3}
    Suppose \(V_1, \ldots, V_m\) are vector spaces. Prove that \(\Lc(V_1 \times \cdots \times V_m, W)\)
    and \(\Lc(V_1, W) \times \cdots \times \Lc(V_m, W)\) are isomorphic vector spaces. 
\end{problem}

\begin{proof}

We define a permutation invariant set function \(S \colon \Lc(V_1, W) \times \cdots \times \Lc(V_m, W)
\to \Lc(V_1 \times \cdots \times V_m, W)\) such that 
\[S(T_1, \ldots, T_m) = T_1 + \cdots + T_m\]

We will show that \(S\) is an isomorphism (it's linear). For injectivity, \(\nul S\) means that 
for any arbitrary input \(T_1 + \cdots + T_m = 0\), which implies that \(T_i = 0\) for all \(i\). 
For surjectivity, take \(\phi \in \Lc(V_1 \times \cdots \times V_m, W)\), we can define 
\(T_i(v_i) = \phi((0, \ldots, v_i, \ldots, 0))\) where \(i\) means for the space \(V_i\). Then we can 
naturally get that \(\phi((v_1, \ldots, v_m))= \phi(\sum_{i=1}^{m}e_i^\top v_i) = \sum_{i=1}^{m} T_i (v_i) \)
and thus we finish the proof.

\end{proof}

\begin{problem}{4}
    Suppose \(W_1, \ldots, W_m\) are vector spaces. Prove that \(\Lc(V, W_1 \times \cdots \times W_m)\)
    and \(\Lc(V, W_1) \times \cdots \times \Lc(V, W_m)\) are isomorphic vector spaces. 
\end{problem}

We can simply define a concatenate operator \(S \colon \Lc(V, W_1) \times \cdots \times 
\Lc(V, W_m) \to \Lc(V, W_1 \times \cdots \times W_m)\) such that 
\[S(T_1, \ldots, T_m) = (T_1, \ldots, T_m)\]
We prove this linear map is an isomorphism. The injective property can be proved using the same logic as 
above. For surjectivity, take \(\phi \in Lc(V, W_1 \times \cdots \times W_m)\), then we can naturally 
obtain \(T_k = \pi_k \phi\) where \(\pi_k\) means projecting the \(k\)-th component of the input. Note that 
\(T_k \in \Lc(V, W_k)\) and therefore we finish the proof. 


\begin{problem}{5}
    For \(m\) a positive integer, define \(V^m\) by 
    \[V^m = \underbrace{V \times \cdots \times V}_{m \text{ times}}.\]
    Prove that \(V^m\) and \(\Lc(\F^m, V)\) are isomorphic vector spaces. 
\end{problem}

\begin{proof}
\[\dim (V^m) = m \dim V = \dim(\Lc(\F^m, V)).\]
\end{proof}

\begin{problem}{6}
    Suppose that \(v, x\) are vectors in \(V\) and that \(U, W\) are subspaces of \(V\)
    such that \(v + U = x + W\). Prove that \(U = W\).  
\end{problem}

\begin{proof}
Note that \(v = x + w\) for some \(w \in W\) if we take \(u = 0 \in U\) and similarly 
\(x = v + u\) for some \(u \in U\). Hence \(v - x \in W\) and \(x - v \in U\). Then we 
have that take \(u \in U, u = (x + w) - v = (x - v) + w \in W\). Similarly, take 
\(w \in W, w = (v + u) - x = (v - x) + u \in U\). Hence \(W = U\). 
\end{proof}

\begin{problem}{9}
    Prove that a nonempty subset \(A\) of \(V\) is a translate of some subspace of \(V\) if and only 
    if \(\lambda v + (1 - \lambda) w \in A\) for all \(v, w \in A\) and all \(\lambda \in \F\). 
\end{problem}

\begin{proof}
\(\Rightarrow\) Suppose \(A = x + U\) for some subspace \(U \subseteq V\). Then take 
\(a_1 = x + u_1 \in A\) and \(a_2 = x + u_2 \in A\). We have that 
\[\lambda a_1 + (1 - \lambda) a_2 = \lambda x + \lambda u_1 + (1 - \lambda)x + (1 - \lambda)u_2 
= x + (\lambda u_1 + (1 - \lambda)u_2) \in A\]

\(\Leftarrow\) take any \(x \in A\) and define the subspace such that 
\[U \coloneq (-x) + A\]
first \(0 = -x + x \in U\). Next, take \(v_1 - x\) and \(v_2 - x \in U\). We have that 
\(v_1 - x + v_2 - x = (-x) + (2v_1 - x)/2 + (2v_2 - x)/2 \in A\) as \(2v_i - x \in A\)
by taking \(\lambda = 2\). Finally, \(\lambda (-x + v) = (-x) + (\lambda v + (1 - \lambda) x) \in A\). 

\end{proof}

\begin{problem}{10}
    Suppose \(A_1 = v + U_1\) and \(A_2 = w + U_2\) for some \(v, w \in V\) and some subspaces 
    \(U_1, U_2\) of \(V\). Prove that the intersection \(A_1 \cap A_2\) is either a translate of 
    some subspace of \(V\) or is the empty set. 
\end{problem}

\begin{proof}
If \(A_1 \cap A_2\) is not empty, then \(S = A_1 \cap A_2\) is a subspace such that \(v + U_1|_S = w + U_2|_S\),
so we know that \(U_1 |_S = U_2 |_S\) by Problem 6. Hence it is a translate of some subspace of \(V\). Otherwise,
it is the empty set. 
\end{proof}

\begin{problem}{11}
    Suppose \(U = \{(x_1, x_2, \ldots) \in \F^\infty \colon x_k \neq 0 \text{ for only finitely many }k\}.\).\\ 
    (a) Show that \(U\) is a subspace of \(\F^\infty\). \\ 
    (b) Prove that \(\F^\infty / U\) is infinite-dimensional. 
\end{problem}

\begin{proof}
(a) all zero, where \(x_k \neq 0\) for zero \(k\)'s (finitely many) is an element of \(U\). Next, take \((x_1, x_2, \ldots)\)
and \((y_1, y_2, \ldots) \in U\). Then \((x_1 + y_1, x_2 + y_2, \ldots)\) will have finitely many nonzero
since each of them has only finitely many nonzero entries. Same holds for \(\lambda(x_1, x_2, \ldots)\). 

(b) Let's consider a ``standard basis'' \(\{v_i\}_{i=1}^\infty\) that only has 0 on \(i\)-th spot and all 1 otherwise. 
We can see that for all \(m\), \(v_1, \ldots, v_m\) is linearly independent. And so does \(v_1 + U, \ldots, 
v_m + U\) since each \(v_i \notin U\), 

\end{proof}

\begin{problem}{12}
    Suppose \(v_1, \ldots, v_m \in V\). Let 
    \[A = \{\lambda_1 v_1 + \cdots + \lambda_m v_m \colon \lambda_1, \ldots, \lambda_m \in F \text{ and }
    \lambda_1 + \cdots + \lambda_m = 1\}.\]
    (a) Prove that \(A\) is a translate of some subspace of \(V\). \\ 
    (b) Prove that if \(B\) is a translate of some subspace of \(V\) and \(\{v_1, \ldots, v_m\} \subseteq B\), 
    then \(A \subseteq B\). \\ 
    (c) Prove that \(A\) is a translate of some subspace of \(V\) of dimension less than \(m\). 
\end{problem}

\begin{proof}
(a) We try to show this through using the conclusion from Problem 9. Let \(a_1 = \sum_{i=1}^{m}\alpha_i v_i\)
and \(a_2 = \sum_{i=1}^{m} \gamma_i v_i\) s.t. \(\sum_{i=1}^{m} \alpha_i = \sum_{i=1}^m \gamma_i = 1\). We have that 
\begin{align*}
    \lambda a_1 + (1 - \lambda) a_2 
    &= \sum_{i=1}^{m} (\lambda \alpha_i + (1 - \lambda) \gamma_i ) v_i 
\end{align*}
We can show that 
\[\sum_{i=1}^{m} (\lambda \alpha_i + (1 - \lambda) \gamma_i ) 
= \lambda \sum_{i=1}^{m} \alpha_i + (1 - \lambda) \sum_{i=1}^{m}\gamma_i = 1\]

Thus \(\lambda a_1 + (1 - \lambda) a_2 \in A\) and thus \(A\) is a translate of some subspace of \(V\). 

(b) Let \(B = w + Y\) for some subspace \(Y \subseteq V\). Then we have 
that \(v_k = w + y_k\). Take \(x \in A\), then we know that 
\[x = \sum_{i=1}^{m} \lambda_i v_i = \sum_{i=1}^{m} 
\lambda_i (w + y_i) = w + \sum_{i=1}^{m} \lambda_i y_i \in B\]

(c) Write \(A = w + U\). Denote \(B = \text{span} \{v_1, \ldots, v_m\}\). 
Then by (b) we know that \(A \subseteq B\). We will show that \(\dim U 
< m\). The statement is trivial if \(v_1, \ldots, v_m\) is not linearly 
independent. Let's consider the case that it's indeed linearly independent. 
Here we claim that \(w \notin U\) and thus \(\dim U < m\). To see this, 
the only way to write \(\sum_{i=1}^{m} \lambda_i v_i = 0\) is to let all 
\(\lambda_i = 0\), but in this case the vectors will not be in \(A\), so 
\(0 \notin A\). Suppose for the sake 
of contradiction that \(w \in U\). 
This implies that \(0 = w + (-w) \in w + U = A\), forming a contradiction. 
Hence we finish the proof. 

\end{proof}


\begin{problem}{13}
    Suppose \(U\) is a subspace of \(V\) such that \(V / U\) is finite-dimensional. 
    Prove that \(V\) is isomorphic to \(U \times (V / U)\). 
\end{problem}

\begin{proof}
\[\dim V = \dim U + (\dim V - \dim U) = \dim U + \dim (V / U).\]
\end{proof}

\begin{problem}{14}
    Suppose \(U\) and \(W\) are two subspaces of \(V\) and \(V = U 
    \oplus W\). Suppose \(w_1, \ldots, w_m\) is a basis of \(W\). 
    Prove that \(w_1 + U, \ldots, w_m + U\) is a basis of 
    \(V / U\).  
\end{problem}

\begin{proof}
First we show that the list is linearly independent.  We consider the following system:

\[0 + U = \lambda_1(w_1 + U) + \cdots + \lambda_m (w_m + U) = (\lambda_1 w_1 
+ \cdots + \lambda_m w_m) + U\]

This means that \( \lambda_1 w_1 
+ \cdots + \lambda_m w_m \in U\) and since we know that \(w_1, \ldots, w_m \notin U\), we have 
that the only solution is \(\lambda_i = 0\) for all \(i\). 

Next we show that the list spans \(V / U\). Take arbitrary \(v + U \in V / U\), then we have that 
\(v = u + w\) for some \(w = \sum_{i=1}^{m} \lambda_i w_i\). Then we have that 
\[v + U = \sum_{i=1}^{m} \lambda_i w_i + U = \sum_{i=1}^{m} \lambda_i (w_i + U).\]

Thus the list spans the quotient space. 

\end{proof}


\begin{problem}{15}
    Suppose \(U\) is a subspace of \(V\) and \(v_1 + U, \ldots, v_m + U\) is a basis of 
    \(V /U\) and \(u_1, \ldots, u_n\) is a basis of \(U\). Prove that 
    \(v_1, \ldots, v_m, u_1, \ldots, u_n\) is a basis of \(V\). 
\end{problem}

\begin{proof}
We first know that 
\[\dim V = \dim V / U + \dim U.\]

So it only suffices to prove the list either spans \(V\) or is linearly independent. Take 
any \(v \in V\). Then we have that 
\[v + U = \lambda_1 v_1 + \cdots + \lambda_m v_m + U.\]

Thus \(v - \sum_{i=1}^{m}\lambda_i v_i \in U\). So 
\[v = \sum_{i=1}^{m} \lambda_i v_i + \sum_{j=1}^{n} \alpha_j u_i.\]

Hence we finish the proof. 
\end{proof}


\begin{problem}{16}
    Suppose \(\varphi \in \Lc(V, \F)\) and  \(\varphi \neq 0\). Prove that 
    \(\dim V / (\nul \varphi) = 1\). 
\end{problem}

\begin{proof}
% We know that 
% \[V / \nul \varphi = \{v + u \colon v \in V, u \in \nul \varphi\}.\]

We know that \(\dim \range \varphi  = 1 \) and thus 
\[\dim V / \nul \varphi = \dim V  - \dim \nul \varphi = \dim \range \varphi = 1.\]

\end{proof}


\begin{problem}{17}
    Suppose that \(U\) is a subspace of \(V\) such that \(\dim V / U = 1\). Prove that there exists 
    \(\varphi \in \Lc(V, \F)\) such that \(\nul \varphi = U\).
\end{problem}

\begin{proof}
We make construction as follows: (1) there exists a natural isomorphism \(T \colon V / U \to 
\F\). We further define \(S \colon V \to V / U\) by \(S(v) = v + U\). Then we show the map 
\[\varphi = TS\]
satifies the requirement. 

First take \(u \in U\). Then we have that 
\[\varphi(u) = TS(u) = T(u + U) = T(0) = 0\]

and thus \(u \in \nul \varphi\). 

Conversely take \(u \in \nul \varphi\), then 
\[0 = \varphi(u) = T(S(u)) = T^{-1} T(S(u)) = S(u).\]
Thus \(u \in \nul S = U\). We've finished the proof. 
\end{proof}


\begin{problem}{18}
    Suppose \(U\) is a subspace of \(V\) such that \(V / U\) is finite-dimensional. 

    (a) Show that if \(W\) is a finite-dimensional subspace of \(V\) and \(V = U + W\), 
    then \(\dim W \geq \dim V / U\). 

    (b) Prove that there exists a finite-dimensional subspace \(W\) of \(V\) such that 
    \(\dim W = \dim V / U\) and \(V = U \oplus W\).
\end{problem}

\begin{proof}
(a) We know that 
\[\dim W = \dim V + \dim (U \cap V) - \dim U \geq \dim V - \dim U = \dim V / U.\]

(b) We will construct the space \(W\) through some manipulation with the basis. Since we 
are given \(V/ U\), we start with the basis \(w_1 + U, \ldots, w_m + U\) of that. Note that 
\(\{w_1, \ldots, w_m\}\) is linearly independent as the only solution is all-zero. We can now 
define 
\[W = \text{span}\{w_1, \ldots, w_m\}.\]

Then we have that \(\dim W = \dim V / U\) and now it suffices to verify \(V = U \oplus W\). To 
see this, one only needs to show \(W \cap U = \{0\}\) as we've proved that \(V = W + U\) in 
previous questions (of similar construction). Take arbitrary \(v \in W \cap U\). Then we have that 
\[v = \sum_{i=1}^{m} \lambda_i w_i \]

and that 
\[v + U = \sum_{i=1}^{m} \lambda_i w_i + U = 0 + U\]

The only solution is all \(\lambda_i = 0\). 
\end{proof}

\begin{problem}{19}
    Suppose \(T \in \Lc(V, W)\) and \(U\) is a subspace of \(V\). Let \(\pi\) denote the quotient 
    map from \(V\) onto \(V / U\). Prove that there exists \(S \in \Lc(V/U, W)\) such that 
    \(T = S \circ \pi\) if and only if \(U \subseteq \nul T.\)
\end{problem}

\begin{proof}
\(\Rightarrow\) Take \(u \in U\), then we have that 
\[T(u) = S \pi(u) = S(u + U) = S(0) = 0\]
and thus \(u \in \nul T\). 

\(\Leftarrow\) We define \(S \colon V / U \to W\) by 
\[S(v + U) = T(v)\]
It now only suffices to show that this map is valid for the equivalence mapping. Take 
\(v_1 + U = v_2 + U\). Then we have that \(v_1 - v_2 \in U \subseteq \nul T\) and thus \(T(v_1 - v_2)
= 0 = T v_1 - T v_2\) and thus \(S(v_1 + U) = T v_1 = Tv_2 = S(v_2 + U)\). This map is clearly 
linear and thus we have \(T = S \circ \pi\).  
\end{proof}


\newpage 

\section*{3F: Duality}
\addcontentsline{toc}{section}{3F: Duality}

\begin{definition}[linear functional]
    A \textbf{linear functional} on \(V\) is a linear map from \(V\) to \(\F\). In other words, 
    a linear functional is an element of \(\Lc(V, \F)\). 
\end{definition}

\begin{definition}[dual space, \(V'\)]
    The \textbf{dual space} of \(V\), denoted by \(V'\), is the vector space of all linear 
    functionals on \(V\). In other words, \(V' = \Lc(V, \F)\). 
\end{definition}

\begin{lemma}
    Suppose \(V\) is finite-dimensional. Then \(V'\) is also finite-dimensional and 
    \[\dim V' = \dim V.\]
\end{lemma}

\begin{definition}[dual basis]
    If \(v_1, \ldots, v_n\) is a basis of \(V\), then the \textbf{dual basis} of 
    \(v_1, \ldots, v_n\) is the list \(\varphi_1, \ldots, \varphi_n\) of elements 
    of \(V'\), where each \(\varphi_j\) is the linear functional on \(V\) such that 
    \[\varphi_j(v_k) = \begin{cases}
        1 &\text{if } k = j, \\ 
        0 &\text{if } k \neq j.
    \end{cases}\]
\end{definition}

\begin{thm}[dual basis gives coefficients for linear combination]
    Suppose \(v_1, \ldots, v_n\) is a basis of \(V\) and \(\varphi_1, \ldots, \varphi_n\)
    is the dual basis. Then 
    \[v = \varphi_1(v) v_1 + \cdots + \varphi_n(v) v_n\]
    for each \(v \in V\). 
\end{thm}

\begin{thm}[dual basis is a basis of the dual space]
    Suppose \(V\) is finite-dimensional. Then the dual basis of a basis of \(V\) is a basis 
    of \(V'\). 
\end{thm}

\begin{definition}[dual map, \(T'\)]
    Suppose \(T \in \Lc(V, W)\). The \textbf{dual map} of \(T\) is the linear map 
    \(T' \in \Lc(W', V')\) defined for each \(\varphi \in W'\) by 
    \[T'(\varphi) = \varphi \circ T.\]
\end{definition}

\begin{corollary}[algebraic properties of dual map]
    Suppose \(T \in \Lc(V, W)\). Then \\ 
    (a) \((S + T)' = S' + T'\) for all \(S \in \Lc(V, W)\); \\ 
    (b) \((\lambda T)' = \lambda T'\) for all \(\lambda \in \F\); \\ 
    (c) \((ST)' = T'S'\) for all \(S \in \Lc(W, U)\). 
\end{corollary}

The goal of this section is to describe \(\nul T'\) and \(\range T'\) in terms of 
\(\range T\) and \(\nul T\). 

\begin{definition}[annihilator, \(U^0\)]
    For \(U \subseteq V\), the \textbf{annihilator} of \(U\), denoted by \(U^0\), is 
    defined by 
    \[U^0 = \{\varphi \in V' \colon \varphi(u) = 0 \text{ for all } u \in U\}.\]
\end{definition}

\begin{remark}
    \(U^0\) is a subspace of \(V'\). 
\end{remark}

\begin{thm}[dimension of the annihilator]
    Suppose \(V\) is finite-dimensional and \(U\) is a subspace of \(V\). Then 
    \[\dim U^0 = \dim V - \dim U.\]
\end{thm}


\begin{lemma}
    Suppose \(V\) is finite-dimensional and \(U\) is a subspace of \(V\). Then \\ 
    (a) \(U^0 = 0 \Leftrightarrow U = V\); \\ 
    (b) \(U^0 = V' \Leftrightarrow U = \{0\}\). 
\end{lemma}

\begin{lemma}[the null space of \(T'\)]
    Suppose \(V\) and \(W\) are finite-dimensional and \(T \in \Lc(V, W)\). Then \\ 
    (a) \(\nul T' = (\range T)^0\); \\ 
    (b) \(\dim \nul T' = \dim \nul T + \dim W - \dim V\).
\end{lemma}

\begin{thm}[\(T\) surjective is equivalent to \(T'\) injective]
    Suppose \(V\) and \(W\) are finite-dimensional and \(T \in \Lc(V, W)\). Then 
    \[T \text{ is surjective } \Leftrightarrow T' \text{ is injective}.\]
\end{thm}

\begin{lemma}[the range of \(T'\)]
    Suppose \(V\) and \(W\) are finite-dimensional and \(T \in \Lc(V, W)\). Then \\ 
    (a) \(\dim \range T' = \dim \range T\); \\ 
    (b) \(\range T' = (\nul T)^0\).
\end{lemma}

\begin{thm}[\(T\) surjective is equivalent to \(T'\) injective]
    Suppose \(V\) and \(W\) are finite-dimensional and \(T \in \Lc(V, W)\). Then 
    \[T \text{ is surjective } \Leftrightarrow T' \text{ is injective}.\]
\end{thm}


\begin{thm}[matrix of \(T'\) is transpose of matrix of \(T\)]
    Suppose \(V\) and \(W\) are finite-dimensional and \(T \in \Lc(V, W)\). Then 
    \[\Mc(T') = (\Mc(T))^\top\]
\end{thm}

\begin{proof}
Let \(v_1, \ldots, v_n\) be basis of \(V\) and \(w_1, \ldots, w_m\) be basis of \(W\). Let 
\(\varphi_1, \ldots, \varphi_n\) be the dual basis of \(V'\) and \(\psi_1, \ldots, \psi_m\) 
be the dual basis of \(W'\). 

Let \(A = \Mc(T)\) and \(C = \Mc(T')\). Suppose \(1 \leq j \leq m\) and \(1 \leq k \leq n\). From 
the definition of \(\Mc(T')\) we have 
\[T'(\psi_j) = \sum_{r=1}^{n} C_{r, j} \varphi_r\]

Then applying both sides of the equation above to \(v_k\) gives that 
\begin{align*}
    (\psi_j \circ T) (v_k) 
    & = \sum_{r=1}^{n} C_{r, j} \varphi_r (v_k) \\ 
    & = C_{k, j}
\end{align*}
At the same time, we have that 

\begin{align*}
    (\psi_j \circ T) (v_k)
    &= \psi_j (T v_k) \\ 
    &= \psi_j(\sum_{r=1}^{m}A_{r, k} w_r) \\ 
    &= \sum_{r=1}^{m} A_{r, k} \psi_j (w_r) \\ 
    &= A_{j, k}
\end{align*}

Here we have that \(C_{k, j} = A_{j, k}\) and thus \(C = A^\top\). Hence, \(\Mc(T') = (\Mc(T))^\top\), 
as desired. 

\end{proof}


We can use duality to provide an alternative proof for the rank of matrix:

\begin{thm}[column rank equals row rank]
    Suppose \(\Ab \in \F^{m, n}\). Then the column rank of \(\Ab\) equals the row 
    rank of \(\Ab\). 
\end{thm}


\newpage 
\addcontentsline{toc}{subsection}{3F Problem Sets}
\begin{problem}{3}
    Suppose \(V\) is finite-dimensional and \(v \in V\) with \(v \neq 0\). Prove that 
    there exists \(\varphi \in V'\) such that \(\varphi(v) = 1\). 
\end{problem}

\begin{proof}
The main requirement is to ensure that the constructed \(\varphi\) is valid. Note that 
there exists  subspace \(W \subseteq V\) such that \(V = W \oplus \text{span}\{v\}\). Thus 
we can define \(\varphi(u) = i\) where \(u = w + iv\) for \(u \in V, w \in W, i \in \F\). Here 
\(\varphi\) is a valid linear map and \(\varphi(v) = 1\). 
\end{proof}

\begin{problem}{6}
    Suppose \(\varphi, \beta \in V'\). Prove that \(\nul \varphi \subseteq \nul \beta\) 
    if and only if there exists \(c \in \F\) such that \(\beta = c \varphi\). 
\end{problem}

\begin{proof}
\(\Rightarrow\) If \(\varphi = 0\), then it trivially holds. If \(\varphi \neq 0 \), then 
we can first take \(v_0 \in V\) s.t. \(\psi(v_0) \neq 0 \), then define \(c = \beta(v_0) / \varphi(v_0)\), 
then we have that \(\beta(v) = \beta(v_0 + u) = \beta(v_0) = c \varphi(v_0)\) for some \(u \in \nul \beta\). 

\(\Leftarrow\) Take \(v \in \nul \varphi\), then \(\beta(v) = c \varphi(v) = 0\). Thus 
\(\nul \varphi \subseteq \nul \beta\). 
\end{proof}


\begin{problem}{8}
    Suppose \(v_1, \ldots, v_n\) is a basis of \(V\) and \(\varphi_1, \ldots, \varphi_n\) is 
    the dual basis of \(V'\). Define \(\Gamma \colon V \to \F^n\) and \(\Lambda \colon 
    \F^n \to V\) by 
    \[\Gamma (v) = ( \varphi_1(v), \ldots, \varphi_n(v) ) \text{  and  } \Lambda(a_1, 
    \ldots, a_n) = a_1 v_1 + \cdots + a_n v_n.\]

    Explain why \(\Gamma\) and \(\Lambda\) are inverses of each other. 
\end{problem}

\begin{proof}
Take \(a_1, \ldots, a_n \in \F^n\), then 

\begin{align*}
    \Gamma(\Lambda(a_1, \ldots, a_n)) 
    &= \Gamma(a_1 v_1 + \cdots + a_n v_n)    \\ 
    &= (\varphi_1(a_1 v_1), \ldots, \varphi_n(a_n v_n)) \\ 
    &= (a_1, \ldots, a_n)
\end{align*}

Similarly, take \(v = a_1 v_1 + \cdots + a_n v_n\), we can get that 
\[\Lambda(\Gamma(v)) = v\]

\end{proof}

\begin{problem}{9}
    Suppose \(m\) is a positive integer. Show that the dual basis of the basis 
    \(1, x, \ldots, x^m\) of \(\Pc_m(\R)\) is \(\varphi_0, \varphi_1, \ldots, \varphi_m\), where 
    \[\varphi_k(p) = \frac{p^{(k)}(0)}{k!}.\]
\end{problem}

\begin{proof}
We consider different cases. If \(j = k\), then 
\[\varphi_k(x^j) = \frac{(x^j)^{(k)}}{k!} = \frac{k!}{k!} = 1\]

If \(j < k\), then 
\[\varphi_k(x^j) = \frac{(x^j)^{(k)}}{k!} = \frac{0^{j-k}   j!/(j-k)!}{k!} = 0\]

If \(j > k\), then 
\[\varphi_k(x^j) = \frac{(x^j)^{(k)}}{k!} = \frac{0}{k!} = 0\]

Hence, we have that 
\[\varphi_k(x_j) = \begin{cases}
    1 & \text{if } j=k \\ 
    0 & \text{o.w.}
\end{cases}\]
\end{proof}


\begin{problem}{11} 
    Suppose \(v_1, \ldots, v_n\) is a basis of \(V\) and \(\varphi_1, \ldots, \varphi_n\) is 
    the corresponding dual basis of \(V'\). Prove that 
    \[\psi = \psi(v_1) \varphi_1 + \cdots + \psi(v_n) \varphi_n .\]
\end{problem}

\begin{proof}
By property of dual basis, we have that 
\[\psi = a_1 \varphi_1 + \cdots + a_n \varphi_n\]

Apply \(v_k\) on both sides give that 
\begin{align*}
    \psi(v_k) 
    & = a_1 \varphi_1 (v_k) + \cdots + a_n \varphi_n (v_k) \\ 
    & = a_k
\end{align*}
Substitue this back gives the desired equality. 
\end{proof}


\begin{problem}{13}
    Show that the dual map of the identity operator on \(V\) is the identity operator on \(V'\). 
\end{problem}

\begin{proof}
Take arbitrary \(f \in V'\), then we have that 
\[I'(f)(v) = f \circ I(v) = f(v)\]
for all \(v \in V\). 
\end{proof}

\begin{problem}{16}
    Suppose \(W\) is finite-dimensional and \(T \in \Lc(V, W)\). Prove that 
    \[T' = 0 \Leftrightarrow T = 0.\]
\end{problem}

\begin{proof}
This is obvious from \(\dim \range T' = \dim \range T\). 
\end{proof}

\begin{problem}{19}
    Suppose \(U \subseteq V\). Explain why 
    \[U^0 = \{\varphi \in V' \colon U \subseteq \nul \varphi\}\]
\end{problem}

\begin{proof}
This follows from definition. 
\end{proof}

\begin{problem}{20}
    Suppose \(V\) is finite-dimensional and \(U\) is a subspace of \(V\). Show that 
    \[U = \{v \in V \colon \varphi(v) = 0 \text{ for every } \varphi \in U^0\}.\]
\end{problem}

\begin{proof}

    Denote \(K = \{v \in V \colon \varphi(v) = 0 \text{ for every } \varphi \in U^0\}\).

Take \(u \in U\) and \(\varphi \in U^0\), then by definition \(\varphi(u) = 0\), and thus 
\(u \in K\). 

Conversely, take \(\varphi \in U^0\). Suppose for the sake of contradiction 
that there exist \(v \notin U\) but \(v \in K\), then \(\varphi(v) \neq 0\), contradicting that 
\(v \in K\), thus completing the proof. 

\end{proof}

\begin{problem}{21}
    Suppose \(V\) is finite-dimensional and \(U\) and \(W\) are subspaces of \(V\). \\ 
    (a) Prove that \(W^0 \subseteq U^0\) if and only if \(U \subseteq W\). \\ 
    (b) Prove that \(W^0 = U^0\) if and only if \(U = W\).  
\end{problem}


\begin{proof}
(a) \(\Rightarrow\) We know that there exists \(\psi \in V'\) such that \(\nul \psi = W\). Then 
\(\psi \in W^0\) and thus \(\psi \in U^0\). This means that \(U \subseteq W = \nul \psi \).

\(\Leftarrow\) Take \(\psi \in W^0\). So we know \(\psi(w) = 0\) for all \(w \in W\). Suppose 
for the sake of contradiction that \(\psi \notin U^0\), then there exists \(u \in U\) s.t. 
\(\psi(u) \neq 0\). However, as \(u \in W\), we have reached a contradiction. 


(b) \(W^0 = U^0 \Leftrightarrow W_0 \subseteq U_0 \text{ and } U_0 \subseteq W_0 
\Leftrightarrow U \subseteq W \text{ and } W \subseteq W \Leftrightarrow U = W\).
\end{proof}



\begin{problem}{22}
    Suppose \(V\) is finite-dimensional and \(U\) and \(W\) are subspaces of \(V\). 

    (a) Show that \((U + W)^0 = U^0 \cap W^0\). \\ 
    (b) Show that \((U \cap W)^0 = U^0 + W^0.\)
\end{problem}

\begin{proof}
Note that we have 
\begin{align*}
    (U + W)^0&= \{\varphi \in V' \colon \varphi(v) = 0 \text{ for every } v \in U+W\} \\ 
    (U \cap W)^0 &= \{\varphi \in V' \colon \varphi(v) = 0 \text{ for every } v \in U \cap W\}
\end{align*}

(a) Take \(\varphi \in (U+W)^0\), then we know that \(\varphi(u) = 0\) and \(\varphi(w) = 0\)
for all \(u \in U, w \in W\). Thus \((U+W)^0 \subseteq U^0 \cap W^0\). Conversely, take 
\(\varphi \in U^0 \cap W^0\), then we know that for every \(v = u + w, \varphi(v)
= \varphi(u) + \varphi(w) = 0\), therefore \(U^0 \cap W^0 \subseteq (U+W)^0\). 

(b) Take \(\varphi_1 \in U^0, \varphi_2 \in W^0\), then we have that 
\(\varphi_1 + \varphi_2 (v) = \varphi_1(v) + \varphi_2(v) = 0\) for \(v \in U \cap W\), therefore 
\(U^0 + W^0 \subseteq (U \cap W)^0\). It suffices now to show that the dimension equal each other.

\begin{align*}
    \dim ((U\cap W)^0) = \dim (V) - \dim (U \cap W)
\end{align*}

Note that 
\begin{align*}
    \dim (U^0 + W^0)
    &=\dim(U^0) + \dim (W^0) - \dim (U^0 \cap W^0) \\ 
    &= (\dim(V) - \dim(U)) + (\dim(V) - \dim(W)) - \dim ((U+W)^0) \\ 
    &= 2 \dim(V) - (\dim(U) + \dim(W)) - (\dim(V) - \dim(U+W)) \\ 
    &= \dim(V) - (\dim(U) + \dim(W) - \dim(U+W)) \\ 
    &=\dim(V) - \dim(U \cap W)
\end{align*}

Thus we have completed the proof. 
\end{proof}


\begin{problem}{23}
    Suppose \(V\) is finite-dimensional and \(\varphi_1, \ldots, \varphi_m \in V'\). Prove 
    that the following three sets are equal to each other.

    (a) span(\(\varphi_1, \ldots, \varphi_m\)). \\ 
    (b) \(((\nul \varphi_1) \cap \cdots \cap (\nul \varphi_m))^0 \coloneq A\) \\ 
    (c) \(\{\varphi \in V' \colon (\nul \varphi_1) \cap \cdots \cap (\nul \varphi_m) \subseteq \nul \varphi\}
    \coloneq B\) 
\end{problem}

\begin{proof}
\((a) \Rightarrow (b)\) take \(\varphi = \sum_{i=1}^{m}a_i \varphi_i \in \text{span}(\varphi_1, \ldots, \varphi_m)\). 
Then we know that for every \(v \in (\nul \varphi_1) \cap \cdots \cap (\nul \varphi_m)\), 
\(\varphi(v) = \sum_{i=1}^{m}a_i \varphi_i(v) = 0\). 

\((b) \Rightarrow (c)\)  Take \(\varphi \in A\), then by P19 this holds. 

\((c) \Rightarrow (a)\) Take \(\varphi \in B\), then we know by P3 there exists \(c \in \F\)
such that \(\varphi = c \varphi_i\) for all \(1 \leq i \leq m\). This means that 
\(\varphi \in \text{span}(\varphi_1, \ldots, \varphi_m)\).
\end{proof}

\begin{problem}{24}
    Suppose \(V\) is finite-dimensional and \(v_1, \ldots, v_m \in V\). Define a linear map 
    \(\Gamma \colon V' \to \F^m\) by \(\Gamma(\varphi) = (\varphi(v_1), \ldots, 
    \varphi(v_m))\). 

    (a) Prove that \(v_1, \ldots, v_m\) spans \(V\) if and only if \(\Gamma\) is injective. \\ 
    (b) Prove that \(v_1, \ldots, v_m\) is linearly independent if and only if \(\Gamma\) is 
    surjective. 
\end{problem}

\begin{proof}
(a) \(\Rightarrow\) Take \(v = \sum_{i=1}^{m} a_i v_i\) and \(\varphi \in \nul \Gamma\). We 
aim to prove that \(\varphi = 0\). We have that \(\Gamma(\varphi) 
= (\varphi(v_1), \ldots, \varphi(v_m)) = 0\) so \(\varphi(v_1) = \cdots = \varphi(v_m) = 0\). 
So for all \(v \in V\), \(\varphi(v) = \sum_{i=1}^{m} a_i \varphi(v_i) = 0\). Thus \(\varphi\)
is the zero map and thus \(\Gamma\) is injective. 

\(\Leftarrow\) Suppose for the sake of contradiction that \(v_1, \ldots, v_m\) does not 
span \(V\), then this means there exists some subspace \(W\) such that \(\text{span}(v_1, \ldots, v_m)
\oplus W = V\). Then there exists nonzero \(\varphi \in V'\) such that \(\varphi(v_k) = 0\) for all  
\(k\) (one can set the basis in \(W\) to be nonzero mapping). Then this means that \(\varphi \in 
\nul \Gamma \neq 0\), contradicting that \(\Gamma\) is injective.  

% Suppose \(v_1, \ldots, v_m\) does not span \(V\). Then there exists 
% \(v_{m+1} \neq 0\) and \(a_{m+1} \neq 0\) and \(v \in V\) such that \(v = \sum_{i=1}^{m+1}a_i v_i\). Since 
% \(\Gamma\) is injective, \(\nul \Gamma = \{0\}\). Let \(\varphi(v) = \sum_{i=1}^{m+1}a_i \varphi(v_i)
% = 0\). We also know that \(\Gamma(\varphi) = (\varphi(v_1), \ldots, \varphi(v_m)) = 0\) and thus 
% combining this two terms give that \(a_{m+1} v_{m+1} = 0\). This means that \(v = \sum_{i=1}^{m}a_i v_i \in 
% \text{span}(v_1, \ldots, v_m)\), contradicting the assumption. 

(b) \(\Rightarrow\)  Let \(K\) be the subspace spanned by the linearly independent list of vectors 
\(v_1, \ldots, v_m\). Then by the linear map lemma, for all \((a_1, \ldots, a_m) \in \F^m\), there 
exists a unique mapping \(T' \colon K \to \F\) such that \(T' v_i = a_i\). Extending \(T'\) to \(V\)
ensures that \(\Gamma\) is surjective. 

\(\Leftarrow\) Suppose for the sake of contradiction that \(v_1, \ldots, v_m\) are not linearly 
independent, then we know there exists nonzero \(a_i's\) such that \(a_1 v_1 + \cdots + a_m 
v_m = 0\). Applying any linear function \(\varphi \in V'\) to this linear combination yields that 
\[\varphi(\sum_{i=1}^{m}a_iv_i) = \varphi(0) = 0\]

This equals that 
\[\sum_{i=1}^{m}a_i \varphi(v_i) = \sum_{i=1}^{m}a_i \Gamma(\varphi)_i = 0\]

Since there is nonzero \(a_i\)'s, this means that the image of \(\Gamma\) is a strict subspace 
of \(\F^m\), therefore contradicting the hypothesis that it's surjective.

\end{proof}

\begin{problem}{25}
    Suppose \(V\) is finite-dimensional and \(\varphi_1, \ldots, \varphi_m \in V'\). Define 
    a linear map \(\Gamma \colon V \to \F^m\) by \(\Gamma(v)=(\varphi_1(v), \ldots, \varphi_m(v))\). 

    (a) Prove that \(\varphi_1, \ldots, \varphi_m\) spans \(V'\) if and only if \(\Gamma\) is injective. \\ 
    (b) Prove that \(\varphi_1, \ldots, \varphi_m\) is linearly independent 
    if and only if \(\Gamma\) is surjective. 
\end{problem}

\begin{proof}
(a) \(\Rightarrow\) Take any \(\varphi \in V'\), then \(\varphi = \sum_{i=1}^{m} a_i \varphi_i\). Take any 
\(v \in V\) and let \(\Gamma(v) = 0\). Then this means that 
\[\Gamma(v) = (\varphi_1(v), \ldots, \varphi_m(v)) = 0\]
So we have that \(\varphi_i (v) = 0\) for all \(i\). By the spanning assumption, we have that 
for any \(\varphi \in V', \varphi(v) = \sum_{i=1}^{m} a_i \varphi_i(v) = 0\). Thus \(v = 0\) 
and then \(\nul (\Gamma) = \{0\}\). 

\(\Leftarrow\) Suppose \(\varphi_1, \ldots, \varphi_m\) does not span \(V'\), then this means there 
exists some subspace \(W\) such that \(\text{span}(\varphi_1, \ldots, \varphi_m)\oplus W = V'\). Let 
\(\varphi_1, \ldots, \varphi_n\) be the basis of \(\text{span}(\varphi_1, \ldots, \varphi_m)\), 
\(\varphi_{n+1}, \ldots, \varphi_{l}\) to be the basis of \(W\). Let \(v_1, \ldots, v_n, 
v_{n+1}, \ldots, v_{l}\) be the corresponding basis for \(V\). Then there exists nonzero \(v \in V\)
such that \(\varphi_k(v) = 0\) for all \(k\) as \(v = \sum_{i=1}^{n} \varphi_i(v) v_i + 
\sum_{j=n+1}^{l} \varphi_j(v) v_j\), where one can obtain nonzero \(\varphi_j(v)\) for some 
\(n+1 \leq j \leq l\) but zero \(\varphi_i(v)\) for all \(1 \leq i \leq n\). Such \(v \in 
\nul \Gamma\), contradicting the hypothesis that \(\Gamma\) is injective.

% Suppose \(\varphi_1, \ldots, \varphi_m\) does not span \(V'\), then there exists 
% nonzero \(v_{m+1}, a_{m+1}\) such that \(v = \sum_{i=1}^{m} a_i v_i\). Since \(\Gamma\) is injective,
% we know \(\nul \Gamma = \{0\}\), and thus \(\Gamma(v)\)

(b) \(\Rightarrow\) Let \(K\) be the subspace spanned by the linearly independent list of vectors 
\(\varphi_1, \ldots, \varphi_m\). Then for all \((a_1, \ldots, a_m) \in \F^m\), take 
\(v = \sum_{i=1}^{m}a_i v_i\) for the corresponding basis \(v_1, \ldots, v_m\). Then  we have that 
\(\varphi_i(v) = a_i\) so \(\Gamma\) is surjective. 

\(\Leftarrow\) Suppose for the sake of contradiction that \(\varphi_1, \ldots, \varphi_m\) are not linearly 
independent, then we know there exists nonzero \(a_i's\) such that \(a_1 \varphi_1 + \cdots + a_m 
\varphi_m = 0\). Applying any vector \(v \in V\) to this linear functional yields that 
\[\sum_{i=1}^{m}a_i \varphi_i (v) = 0(v) = 0\]

This equals that 
\[\sum_{i=1}^{m}a_i \varphi_i (v) = \sum_{i=1}^{m}a_i \Gamma(v)_i = 0\]

Since there is nonzero \(a_i\)'s, this means that the image of \(\Gamma\) is a strict subspace 
of \(\F^m\), therefore contradicting the hypothesis that it's surjective.


\end{proof}

\begin{problem}{26}
    Suppose \(V\) is finite-dimensional and \(\Omega\) is a subspace of \(V'\). Prove that 
    \[\Omega = \{v \in V \colon \varphi(v) = 0 \text{ for every } \varphi \in \Omega\}^0\]
\end{problem}

\begin{proof}
Denote \(U = \{v \in V \colon \varphi(v) = 0 \text{ for every } \varphi \in \Omega\}\). 
So we have that 

\[U^0 = \{\psi\in V' \colon \psi(v) = 0 \text{ for every } v \in U\}\]


\(\Rightarrow\) Take \(\varphi \in \Omega\) and \(v \in U\), then by definition \(\varphi(v) = 0\)
and thus \(\varphi \in U^0\). 

\(\Leftarrow\) Take \(\psi \in U^0\), suppose \(\psi\notin \Omega\). Take \(v \in U\), then we know 
that for every \(\varphi \in \Omega, \varphi(v) = 0\). Since \(\psi \notin \Omega\), there 
exists \(v, \text{ s.t. }\psi(v) \neq 0\). However, this contradicts that \(\psi(v) = 0\) for all 
\(v \in U\), the proof is completed.  
\end{proof}

\begin{problem}{28}
    Suppose \(V\) is finite-dimensional and \(\varphi_1, \ldots, \varphi_m\) is a linearly independent 
    list in \(V'\). Prove that 
    \[\dim((\nul \varphi_i) \cap \cdots \cap (\nul \varphi_m)) = (\dim V) - m\]
\end{problem}

\begin{proof}
Note that \((\bigcap_{i=1}^m \nul \varphi_i)^0 = \text{span}(\varphi_1, \ldots, \varphi_m)\) by P23. Then 
we have that \(\dim \bigcap_{i=1}^m \nul \varphi_i = \dim V - \text{span}(\varphi_1, \ldots, \varphi_m)  
= \dim V - m\). 
\end{proof}

\begin{problem}{30}
    Suppose \(V\) is finite-dimensional and \(\varphi_1, \ldots, \varphi_n\) is a basis of \(V'\). 
    Show that there exists a basis of \(V\) whose dual basis is \(\varphi_1, \ldots, \varphi_n\).
\end{problem}

\begin{proof}
We know that \(\dim V' = n\) and that \(\dim \nul \varphi_i + 1 (\dim \range \varphi_i) = n\). Thus 
we can construct that \(V = \nul \varphi_i \oplus U_i\) for all \(1 \leq i \leq n\). Here 
\(\dim U_i = 1\). Here we extract all \(v_i's\) from those \(U_i\). By linear map lemma, there exists 
\(v_i \in U_i\) such that \(\varphi_i (v_i)= 1\). We claim that \(v_1, \ldots, v_n\) is the 
corresponding basis of \(V\). 

We first verify that it is the ``corresponding basis''. Consider \(\varphi_j(v_i)\) for \(j \neq i\). 
Suppose for contradiction that there exists nonzero \(w \in U_i \cap U_k\). Since we know all 
\(U_i\) are 1-d, \(U_i = U_k\). Here we have that \(\nul \varphi_i = \nul \varphi_k\), then this 
means that \(\varphi_i = c \varphi_k\) for some \(c \in \F\), but this contradicts the assumption 
that \(\varphi_1, \ldots, \varphi_n\) is the basis of \(V'\). 

Next we verify it is basis. It only suffices to verify that the list is linearly independent. To 
see this, \(0 = \sum_{i=1}^{m} a_i v_i\). Apply \(\varphi_i\) on both side yield that 
\(0 = \varphi_i(0) = \varphi(\sum_{i=1}^{m} a_i v_i) = a_i\). Thus all coefficients are zero.
\end{proof}


\begin{problem}{32}
    The \emph{double dual} space of \(V\), denoted by \(V''\), is defined to be the dual space of 
    \(V'\). In other words, \(V'' = (V')'\). Define \(\Lambda \colon V \to V''\) by 
    \[(\Lambda v)(\varphi) = \varphi(v)\]
    for each \(v \in V\) and \(\varphi \in V'\).

    (a) Show that \(\Lambda\) is a linear map from \(V\) to \(V''\). \\ 
    (b) Show that if \(T \in \Lc(V)\), then \(T'' \circ \Lambda = \Lambda \circ T\), where 
    \(T'' = (T')'\). \\ 
    (c) Show that if \(V\) is finite-dimensional, then \(\Lambda\) is an isomorphism from \(V\)
    onto \(V''\). 
\end{problem}


\begin{proof}
(a) First, \(\Lambda(0)(\varphi) = \varphi(0) = 0\). Next, take \(v_1, v_2 \in V\), then 
\(\Lambda(\lambda v_1 + v_2) (\varphi) = \lambda\varphi(v_1) + \varphi(v_2) = 
\lambda\Lambda(v_1) + \Lambda(v_2)\).

(b) By definition, the double dual map \(T'' \colon V'' \to W''\) is defined as 
\(T''(\psi) = \psi(T'(\varphi))\) for \(\psi \in V''\) and for all \(\varphi \in W'\). This 
further equals that \(T''(\psi) = \psi \varphi \circ T(v)\) for \(v \in V\). 

\begin{align*}
    T'' \circ (\Lambda v)(\varphi)
    &= (\Lambda v) \circ T' (\varphi) \\ 
    &= (\Lambda v) \varphi \circ T  \\ 
    &= \Lambda \circ T 
\end{align*}


(c) First we have that \(\dim V = \dim V' = \dim V''\). Note that if we take \(v \in \nul \Lambda\), 
then this means for all linear functional \(\varphi \in V'\), we have that \(\Lambda v (\varphi)
= \varphi(v) = 0\). So \(v = 0\) and thus \(\Lambda\) is injective and therefore an isomorphism.
\end{proof}

\end{document}

